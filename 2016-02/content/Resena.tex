\documentclass[spanish]{ijsra}
\def\IJSRAidentifier{\currfilebase} %<---- don’t change this!
%-------Title | Email | Keywords | Abstract-------------
\def\shorttitle{Rese\~{n}a III Congreso Internacional de J\'{o}venes}
\def\maintitle{Rese\~{n}a III Congreso Internacional de J\'{o}venes Investigadores del Mundo Antiguo}
\def\cmail{cijimamurcia@gmail.com}
\def\keywords{Conference proceedings}
%\def\keywordname{}%<--- redefine the name “Keywords“ in needed language
\def\abstract{}
%--------Author’s names------------
\def\authorone{Consuelo Isabel Caravaca Guerrero}
\def\authortwo{\mbox{D\'{a}maris L\'{o}pez Mu\~{n}oz}}
%-------Biographical information-------------
\def\bioone{Consuelo Isabel Caravaca Guerrero es Licenciada en Historia por la Universidad de Murcia (2013), M\'{a}ster en Formaci\'{o}n del Profesorado (2014) y M\'{a}ster en Historia y Patrimonio Hist\'{o}rico (2016). Ha participado en las excavaciones de Presentaci\'{o}n Legal (Cerro del Pino, Portm\'{a}n), Sima de las Palomas (Dolores de Pacheco), La Boja (Mula) y Cueva Negra (Caravaca). Especializada en el mundo egipcio, ha realizado su tesis de m\'{a}ster sobre los retratos funerarios del Fayum. Tambi\'{e}n ha investigado y realizado ponencias a nivel nacional sobre la celebraci\'{o}n del \textit{Heb Sed} en el Templo de Karnak, la herencia egipcia en la concepción de la imagen cristiana, o el culto a la diosa Sekhmet.}
\def\biotwo{D\'{a}maris L\'{o}pez Mu\~{n}oz es Graduada en Historia por la Universidad de Murcia (2015), y M\'{a}ster en Historia y Patrimonio Hist\'{o}rico (2016). Ha participado en las excavaciones de B\'{\i}lbilis y Valdeherrera (Zaragoza), Cabezo peque\~{n}o del Esta\~{n}o (Guardamar del Segura), Villa romana de Los Villaricos (Mula, Murcia), Necr\'{o}polis de Silla del Papa (Tarifa, C\'{a}diz). Especializada en el mundo romano, ha realizado su tesis de m\'{a}ster sobre los ciclos din\'{a}sticos imperiales en Hispania Citerior, a trav\'{e}s de la escultura y epigraf\'{\i}a presente en foros y teatros. Tambi\'{e}n ha investigado y realizado ponencias a nivel internacional sobre emperatrices romanas y el culto imperial en Hispania.}
%------University/Institution--------------
\def\affilone{Universidad de Murcia}
%\def\affiltwo{}%<---- comment or delete if there is no second author.
%\def\affilthree{}%<---- comment or delete if there is no third author.
%\def\affilfour{}%<---- comment or delete if there is no fourth author.
%\def\affilfive{}%<---- comment or delete if there is no fifth author.
%--------Mapping of authors to affiliations------------
%% authorone:--> * <--- copy/paste that symbol to \affiloneauthor etc. below
%% authortwo:--> † <--- copy/paste that symbol to \affiloneauthor etc. below
%% authorthree:--> ‡ <--- copy/paste that symbol to \affiloneauthor etc. below
%% authorfour: --> § <--- copy/paste that symbol to \affiloneauthor etc. below
%% authorfive: --> ¶ <--- copy/paste that symbol to \affiloneauthor etc. below
%-------------------------------------------------------------------------
%\def\affiloneauthor{}%<---- paste the symbol of the authors into {}
%\def\affiltwoauthor{}%<---- paste the symbol of the authors into {}
%\def\affilthreeauthor{}%<---- paste the symbol of the authors into {}
%\def\affilfourauthor{}%<---- paste the symbol of the authors into {}
%\def\affilfiveauthor{}%<---- paste the symbol of the authors into {}

%\begin{filecontents}{\IJSRAidentifier.bib}
%Bibliography-data HERE
%\end{filecontents}
\begin{document}
\begin{otherlanguage}{spanish}
\IJSRAopening
%-------
\lettrine{C}{omo} desde hace tres a\~{n}os, el III Congreso Internacional de J\'{o}venes Investigadores del Mundo Antiguo, organizado por el Centro de Estudios del Pr\'{o}ximo Oriente y la Antig\"{u}edad Tard\'{\i}a (CEPOAT), tuvo lugar en la Facultad de Letras de la Universidad de Murcia. Al igual que en el resto de ediciones, el encuentro que tuvo lugar los d\'{\i}as 7 y 8 de abril supuso un espacio de aprovechamiento, discusi\'{o}n e intercambio de conocimientos y perspectivas hist\'{o}ricas entre j\'{o}venes y consagrados historiadores. El acontecimiento tuvo lugar en el Hemiciclo de la Universidad de Murcia. La Facultad de Letras se convirti\'{o} durante dos d\'{\i}as en la sede sapiencial de reuni\'{o}n de los participantes y asistentes, donde expusieron las nuevas teor\'{\i}as, estudios y visiones sobre distintos temas como la arqueolog\'{\i}a, el arte, la historiograf\'{\i}a, la filolog\'{\i}a cl\'{a}sica y dem\'{a}s ciencias afines vinculadas a la Historia Antigua. El total de personas asistentes al congreso rond\'{o} las 125 personas y 65 ponentes entre los dos d\'{\i}as.

El congreso se abri\'{o} con la conferencia de la Dra. Helena Jim\'{e}nez Vial\'{a}s, investigadora de la Universidad de Toulouse y profesora honoraria de la Universidad Aut\'{o}noma de Madrid. Su conferencia abord\'{o} el estudio de las ciudades antiguas del Estrecho de Gibraltar, \textit{Gadir, Carteia} y \textit{Baelo Claudia}, a trav\'{e}s de un estudio comparado de las fuentes cl\'{a}sicas y el empleo de los Sistemas de Informaci\'{o}n Geogr\'{a}fica (SIG), para establecer una secuencia de ocupaci\'{o}n territorial de las distintas culturas fenicia, p\'{u}nica y romana del extremo meridional de la Pen\'{\i}nsula Ib\'{e}rica. 

Las sesiones, organizadas por \'{a}reas geogr\'{a}ficas, comenzaron con las ponencias sobre Egipto y Pr\'{o}ximo Oriente, donde pudimos contar con sesiones tan variadas como el arte, acontecimientos pol\'{\i}ticos, tendencias historiogr\'{a}ficas o cuestiones de g\'{e}nero, a trav\'{e}s de las ponencias de investigadores procedentes de diversas universidades, especialmente de la Universidad Complutense de Madrid, la Aut\'{o}noma de Madrid o las de Ja\'{e}n y Vigo. Se comenz\'{o} con la comunicaci\'{o}n de Iria Souto de la Universidad de Vigo, en torno al periodo amarniense y la reforma religiosa del fara\'{o}n Akenat\'{o}n durante la dinast\'{\i}a XVIII a trav\'{e}s de un estudio desde el arte, la sociedad o la econom\'{\i}a.

Tras cerrar la mesa de Egipto e inaugurando la mesa de Pr\'{o}ximo Oriente tuvo lugar la conferencia a cargo de los editores de la Revista \textit{Panta Rei}, en donde explicaron la historia y labor realizada desde la revista e invitaron a los asistentes a participar en ella en la pr\'{o}xima edici\'{o}n. Durante esta mesa se trataron diversos temas relacionados con los amuletos egipcios de la colecci\'{o}n Mathew Beyens, as\'{\i} como las tendencias historiogr\'{a}ficas y nuevas perspectivas para el estudio de las relaciones interculturales en el Pr\'{o}ximo Oriente Antiguo, la concepci\'{o}n del dios mesopot\'{a}mico Marduk o sobre las distintas hip\'{o}tesis del origen del hoy dios de la religi\'{o}n jud\'{\i}a, \textit{YHWH}, apoy\'{a}ndose en las principales evidencias literarias y arqueol\'{o}gicas.

La primera mesa de la tarde estuvo relacionada con la H\'{e}lade y el mundo griego en la Antig\"{u}edad. Un tema novedoso fue el expuesto por Luis Calero, doctorando de la Universidad Aut\'{o}noma de Madrid,  que trat\'{o} de acercase a las practicas musicales de la Grecia arcaica a trav\'{e}s de un estudio filol\'{o}gico de la l\'{\i}rica griega. Otros temas abordados en esta sesi\'{o}n estuvieron relacionados con la pederastia en la sociedad espartana, la organizaci\'{o}n y espacio en los misterios de Eleusis, un an\'{a}lisis comparativo iconogr\'{a}fico de las representaciones femeninas aladas, a trav\'{e}s de la numism\'{a}tica griega o la base arqueol\'{o}gica del nacionalismo ateniense, entre otras. 

Tras un breve receso, la segunda mesa de la tarde se centr\'{o} en el \'{a}mbito ib\'{e}rico que estuvo centrado en el estudio cer\'{a}mico como los \textit{kalathoi} y perspectivas de g\'{e}nero. Patricia Rosell de la Universidad de Alicante trat\'{o} el tema de la mujer ibera a trav\'{e}s de su vestimenta, calzado, elementos de orfebrer\'{\i}a, joyer\'{\i}a y adornos del cabello en la iconograf\'{\i}a encontrada en las fuentes arqueol\'{o}gicas, para interpretar su posici\'{o}n en la sociedad ib\'{e}rica. Gema Negrillo de la Universidad de Granada realiz\'{o} un an\'{a}lisis de la imagen de las amazonas en la cer\'{a}mica de figuras rojas \'{a}ticas desde el campo de la arqueolog\'{\i}a funeraria.  Por \'{u}ltimo la sesi\'{o}n de la tarde finaliz\'{o} con preguntas y debate entre los participantes y asistentes.
 
El segundo d\'{\i}a del congreso, la primera mesa de la ma\~{n}ana estuvo centrada en el \'{a}mbito hispano romano con ponencias donde se analiz\'{o} el culto imperial en las capitales de Hispania a trav\'{e}s de la escultura, la epigraf\'{\i}a y la numism\'{a}tica impartida por D\'{a}maris L\'{o}pez o conocimos las formas de h\'{a}bitat romano a trav\'{e}s de la aplicaci\'{o}n de la Teor\'{\i}a de Christaller cuyas conclusiones expuestas por Alejandro Ru\'{\i}z y Mar\'{\i}a Jos\'{e} Jim\'{e}nez, dieron como resultado distintas categor\'{\i}as de poblamiento romano en la Regi\'{o}n de Murcia. 

La mesa continu\'{o} con distintas ponencias sobre el mundo romano entre las que se analiz\'{o} la orientaci\'{o}n de las construcciones empleadas en \textit{Carthago Nova} gracias a la t\'{e}cnica de la \textit{varatio}, triangulaci\'{o}n y la comparaci\'{o}n con las fuentes literarias antiguas. Otras ponencias estaban basadas en distintos temas como las nuevas perspectivas de investigaci\'{o}n y estado de la cuesti\'{o}n sobre el yacimiento de la Illeta dels Banyets, ajuares egipcios hallados en contextos funerarios de necr\'{o}polis romanas, pasando por temas de filosof\'{\i}a y derecho griego y romano. La \'{u}ltima sesi\'{o}n de la jornada estuvo relacionada con perspectivas de g\'{e}nero en la Hispania romana. 

Durante la sesi\'{o}n de la tarde abord\'{o} temas relacionados con la religi\'{o}n romana y paleocristiana, como es el caso de la ponencia del doctor Jorge Cuesta versada sobre las persecuciones cristianas, cuestiones transcendentales como la proximidad del fin del mundo y la llegada del anticristo a trav\'{e}s de la visi\'{o}n de las fuentes literarias de la parte occidental del Imperio. 
La \'{u}ltima sesi\'{o}n del congreso trat\'{o} temas como las invasiones b\'{a}rbaras, la visi\'{o}n de los emperadores romanos a trav\'{e}s de Orosio y la figura del comes Bonifacio. La \'{u}ltima comunicaci\'{o}n de la tarde fue realizada por el doctorando de la Universidad de Murcia Jos\'{e} \'{A}ngel Castillo, donde expuso las principales trasformaciones pol\'{\i}ticas desarrolladas a ra\'{\i}z de la rebeli\'{o}n arriana durante el gobierno de Recaredo. 

El congreso fue clausurado por Jos\'{e} Javier Mart\'{\i}nez Garc\'{\i}a, egipt\'{o}logo e investigador del Centro de Estudios de Pr\'{o}ximo Oriente y la Antig\"{u}edad Tard\'{\i}a.  


%\IJSRAsection{small headline}

%\IJSRAseparator


\IJSRAclosing
\end{otherlanguage}
\end{document}