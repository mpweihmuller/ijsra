\openingarticle
\def\ppages{\pagerange{Scollan:firstpage}{Scollan:lastpage}}
\def\shorttitle{Domestication and Infectious Diseases}
\def\maintitle{Domestication and Infectious Diseases: Evidence For Human Cultural  Niche Construction}
\def\shortauthor{Margaret Scollan}
\def\authormail{m.e.scollan@durham.ac.uk}
\def\affiliation{University of Durham}
\def\thanknote{\footnote{Margaret Scollan recently earned her MSc in Archaeological Science with Distinction from Durham University. 
She received her BS in Biology \emph{cum laude} from Boston College. 
Her Master’s dissertation research focused on analysing the mobility and dietary trends of high-status British Iron Age individuals via multi-isotope analysis. 
Her general research interests focus on the intersection of biology and archaeology, including but not limited to the study of disease evolution, 
human-pathogen interaction, and using archaeological techniques to influence the development of modern-day disease treatments. She is currently a Research Associate at the Regeneron Genetics Center, USA.}}
%--------------------------------------------------------------
\mychapter{\maintitle}
\begin{center}
	{\Large\scshape\shortauthor\thanknote}\\[1em]
	\email \\
	\affiliation
\end{center}
\vspace{3em}
\midarticle
%--------------------------------------------------------------
\label{Scollan:firstpage}

	\begin{myabstract}
	 Niche construction\marginnote{Abstract\\ (in Spanish see below)} is the modification of an organism’s environment that results in the evolutionary (biological) and/or cultural (anthropological) change of the organism. This paper explores the rise of pathogenic disease in Southwest Asia during the Neolithic period ($\sim$10000 to 4500--2000\BC) through the lens of niche construction. 
Humans constructed a new environmental niche when they domesticated animals, which resulted in the rise and spread of various deadly pathogens due to increase in human population size and animal-human proximity. In order to combat this spread of disease, humans responded by making cultural changes to prevent disease, such as new sanitation measures. The human population also began to evolve in order to counteract disease. Co-evolutionary dynamics developed between humans, animals, and pathogens, which are ongoing to this day.

\keywords[Keywords]{Niche construction, Co-evolutionary dynamics, Domestication, Infectious disease, Neolithic.}		
	\end{myabstract}
	


%	\section{Introduction}
%	\subsection{Niche construction theory}
	
\lettrine[nindent=0em,lines=3]{N}{iche} \marginnote{Niche construction theory} construction theory (NCT) highlights the ability of organisms to change environments through their activities and choices, thus acting as co-directors in their own and other species’ evolution \parencite[2]{Odling-Smee_2003}. 
Unlike natural selection and genetic inheritance theories, which emphasize that the environment introduces pressures to which organisms eventually adapt via genetic mutation, niche construction theory stresses that organisms play an active role in adapting to their environments \parencite[304]{Laland_2010a}. 
For example, animals can create nests, dams, paths, and webs, plants can chemically alter their environments, and humans can wear clothes and build fires and shelters. 
NCT is sometimes referred to as “triple-inheritance,” because it involves genetic, cultural and ecological inheritance \parencites[643]{Odling-Smee_1996}[251]{Odling-Smee_2003}.

	Niche construction can introduce selective pressures into an ecosystem to which future generations are exposed \parencite[135]{Laland_2000}. 
This ecological inheritance means that generations of organisms inherit both ancestral genes and modified environments. Resulting selective pressures not only affect the organism making the changes, but also other organisms that live in that environment. Niche construction is a major form of connectivity between biota, and is a source of co-evolutionary interactions 
\parencite[201]{Odling-Smee_2003}. 
In traditional co-evolutionary theory, selection on one species triggers selection on the other without either organism triggering the change. This is illustrated in the statement: as human genes evolve to counteract infecting pathogens, pathogens evolve means by which to overcome these defenses. 
In this example, both groups undergo natural selection, with neither having an active role in how the genes evolve. However, niche construction theory emphasizes that the trigger can be due to an organism’s agency. For example, as humans develop vaccines to boost immunity to pathogens, pathogens evolve to overcome these defenses. In this example, humans use their agency to develop protection from disease. 

	Humans are particularly effective niche constructors because of their ability to learn and culturally transmit knowledge and skills in order to devise efficient problem solutions \parencite{Richerson_2005}. 
Archaeologists and anthropologists have long acknowledged human agency and its relationship with human environments. NCT provides a framework for bringing together anthropological and archaeological perspectives on human culture and society with biological perspectives on the selection of human genes. 
In certain instances, NCT can explain the lack of direct relationships between human allele frequencies and selective environments. For example, by developing fire and clothing, humans adapted to colder environments without having a significant change in gene expression \parencite[140--141]{Laland_2010b}. 
In other cases, it explains how selective pressures on various genes could be the result of cultural activities. For example, the spread of dairy farming resulted in the selection of the allele responsible for lactose absorption \parencites[3738]{Burger_2007}{Feldman_1989}{Holden_1997}.

%	\subsection{Research aims}
This\marginnote{Research aims} paper aims to describe the domestication of animals, particularly in Southwest Asia, during the Neolithic ($\sim$10000 to 4500--2000\BC) 
and its subsequent effects on human culture and genetics through the framework of niche construction theory (Fig. \ref{fig:ScollanFig1ab}). 
NCT is an ideal framework for exploring the effects of animal domestication because domestication is neither easily defined as biological mutualism nor cultural phenomenon. 
Instead, it is a combination of biological symbiotic relationships 
and the human capacity to effect behavioural change through learning \parencite[111--115]{Zeder_2006a}. 
The domestication of animals initiated a major change in the human environment when large groups of animals began to live in close proximity to humans. 
A major result of domestication was the introduction of new pathogens into the human population. 
These pathogens and their associated diseases set new gene-culture co-evolutionary dynamics in motion, resulting in genetic changes in humans, animals and pathogens: for example, the ongoing “arms race” of resistance and anti-resistance between people and microbes \parencite{Boni_2005}. 
Furthermore, human learning and observation led to new behaviours to prevent, counteract, and/or contain disease.

	%% this is just a demonstration figure
	\begin{figure}[!htb]
		\includegraphics[width=\linewidth]{figures/ScollanFig1ab.jpg}
		\caption{a Steps of niche construction, by author. Initial behavior leads to a change in an environmental niche, resulting in selective pressure for the biological or cultural change of an organism. 
In an unstable niche, changes result in further alterations of environment and the cycle continues \parencite{Rowley-Conwy_2011}. 
\newline b Representation of niche construction initiated with the domestication of animals.}
		\label{fig:ScollanFig1ab} %best is to use the name of the file
	\end{figure}

\section{Animal domestication}
It is argued that domestication is one of the most important events in human history, 
as it had far-reaching consequences into food production, 
the rise of civilizations, and global demography \parencite[700]{Diamond_2002}. 
As humans adopted sedentary, farming lifestyles, they began to domesticate groups of animals in settlement centres, resulting in an unprecedented proximity between large groups of animals and humans. 
Why humans began to domesticate animals is a complicated question, and the answers vary from the influence of changing climates to differences in resource availability 
\parencite[111--115]{Zeder_2006a}. 
\textcite[189--190]{Smith_2007} argues that macroevolutionary forces did not directly cause domestication, but rather caused humans to intensify their niche construction efforts. Animal domestication itself likely occurred as part of broader environmental altering projects based on knowledge of local flora and fauna 
\parencite[111--115]{Zeder_2006a}. 
These actions were not spontaneous or anomalous, but rather purposeful decisions to manage or alter an environment. 
As specific reasons for domestication differed around the globe,
 NCT provides a general universal context for domestication, 
relevant across a diverse range of geographies and cultures \parencite[1993--195]{Smith_2007}. 

Differing environmental factors and animal characteristics resulted in domestication. 
In recent scholarship, archaeologists have shifted from morphological 
and genetic markers as primary indicators of domestication in favour of a broader approach including investigations of palaeopathology and palaeodiet
\parencites{Makarewicz_2012}{Pearson_2007} along with more traditional methodologies such as osteological measuring and analysis of kill-off patterns. 
Domestication occurred at different times in different places, but the following discussion will provide a general sense of its occurrence at the onset of the Neolithic in Southwest Asia. 
Increasing amounts of human refuse attracted wild dogs to settlements about \num{12000} years ago, resulting in closer contact with humans and a reduction in aggression. 
In the archaeological record, reduced aggression is marked by shorter snouts, 
and crowding and size reduction in teeth \parencite[341, 343--344]{Moray_1994}. 
Large stores of food and waste may have also attracted disease-bearing rodents \parencite[2087]{Linseele_2007}. 
When wild cats began to feed on the rats and mice plaguing agricultural settlements, 
humans encouraged their presence in order to protect stored food \parencite[2087]{Linseele_2007}.
Like dog domestication, cat domestication is also evident through a reduction in size and aggression. 

 However, the most likely candidates for domestication were large terrestrial herbivores and omnivores \parencite[702]{Diamond_2002}. 
Management of sheep, cattle, and goats began in the Levant around 9000--7000\BC. 
Domestication was most possible when animals lived in fixed-membership, 
male-hierarchal herds, were diurnal, were non-territorial and non-migratory, 
and had the potential to breed in captivity \parencites[702]{Diamond_2002}[856]{Rowley-Conwy_2011}. 
This explains why animals such as gazelle and deer were incompatible with domestication. 
Human herd management is evident in shifts in the age and sex composition of faunal assemblages. 
For example, by \num{9000} years ago, goat herds in the Zagros mountains were 
characterized in the hallmark demographics of meat production: 
a majority of adult breeding females with a few adult males \parencites[201]{Zeder_2006b}[187]{Zeder_2006c}.  

%\section{Disease}
%\subsection{Disease and population size in the pre-Neolithic}
Infectious\marginnote{Disease and population size in the pre-Neolithic} pathogens require 
a population of a certain size in order to survive and spread \parencites[113]{Armelagos_2005}{Armelagos_1984}{McNeill_1984}. 
A group of hunter-gatherers was not likely to increase over 100 people \parencite[272]{Armelagos_1970}, 
and thus, was not capable of sustaining infectious pathogens. 
Palaeolithic diseases were relatively benign as to best utilize their small host population \parencite[37]{Pearce-Duvet_2006}, 
and the parasites associated with them can be detected in the archaeological 
record through the study of human coprolites and pelvic soil from burials, 
among other contexts \parencite{Mitchell_2013}. 
Parasites can be characterised into two classes: heirloom species and souvenir species. 
Heirloom species are those parasites which evolved with anthropoids and continued to infect Homo sapiens,
such as head and body lice, pinworms, and yaws \parencites[114]{Armelagos_2005}[47--48]{Cockburn_1971}. 
Souvenir species are those that humans accidentally picked up during the course of daily activity through insect or animal bites or consumption of contaminated meat, 
but whose primary hosts were not human. These include sleeping sickness, 
tetanus, scrub typhus, trichinosis, and schistosomiasis \parencite[15]{Armelagos_1991}. 
Thus, humans were not getting sick due to construction of novel niches, 
but instead through accidental entry into a generalised multi-host zoonotic system.

It was formerly believed that the low population size during the Palaeolithic and Mesolithic was due to high mortality from infectious disease, but a large body of data has now debunked this theory \parencites{Angel_1984}[13]{Armelagos_1991}[60]{Armelagos_1975}[127]{Armelagos_1990}. 
Fertility and mortality rates must be balanced in order for a 
population to remain a small size \parencite[113]{Armelagos_2005}. 
Hunter-gatherer populations had a low fertility rate for several reasons. 
Firstly, hunter-gatherer children needed to be carried for 
approximately the first four years of life, which forced 
women to space births every four years \parencite[325--326]{Lee_1980}. 
Furthermore, a diet of wild foods and a mobile lifestyle 
resulted in low body fat levels that could have limited fertility \parencite{Wilmsen_1989}. 
Additionally, \textcite{Dunbar_1993} theorised that neocortex size and the evolution of 
language affected the ability of peoples to maintain relationships between groups of a certain size, 
suggesting that low population size may also have been due to an inability to maintain groups larger than approximately 100 individuals. 

%\subsection{Disease and population size in the Neolithic}
The \marginnote{Disease and population size in the Neolithic} rise of infectious disease in the Neolithic depended on two separate results of domestication: the creation of much denser human populations and the increase in frequency of disease transmission \parencite[703--704]{Diamond_2002}. 
When humans domesticated animals and encouraged growth of herds, this along with plant domestication resulted in a larger food source capable of supporting a growing population. Sedentary lifestyle meant that women could reduce the amount of time between 
child-bearing and were more likely to have body fat levels 
necessary for reproduction \parencites[333]{Lee_1980}{Wilmsen_1989}. 
Even though the rise of infectious disease caused greater mortality, 
disease struck mostly the very old and very young, or the non-reproducing 
members of the population. Those that survived to adulthood most likely
 had developed immunity to common pathogens, and thus the reproducing 
part of the population remained relatively stable \parencite[18]{Armelagos_1991}. 
Increasing the fertility rate over the mortality rate resulted in population expansion. 
For example, it is estimated the Neolithic settlements of Catalhöyük, Turkey 
had a population at least over \num{1000} by 7000\BC which could have grown to approximately \num{10000}, 
and the population of Talianki, Ukraine was approximately \num{15000}--\num{30000} by 3800\BC \parencites{Hodder_2011}{Kohl_2009}. 

Human groups thus became an adequate size to host infectious disease, regardless of the effect of disease on mortality levels. 
For example, phylogeny studies of the deadliest strain of malaria, Plasmodium falciparum, and of the microbe responsible for human tuberculosis, 
Mycobacterium tuberculosis, indicate that both pathogens had specialised for human hosts before the emergence of domestication \parencite[373--374]{Pearce-Duvet_2006}. 
However, urbanisation resulted in the conditions for them to spread and sustain disease transmission. The oldest archeologically identified case of tuberculosis was recorded at the Neolithic site of Atlit-Yam, Israel; 
dating to approximately 7000\BC, researchers claim it was one of the first villages in Southwest Asia with evidence of agriculture and animal domestication \parencite{Hershkovitz_2008}. 
Shigella or dysentery also originated as a human disease prior to the Neolithic, 
but did not become a major issue until crowded environments resulted in contaminated 
drinking water \parencite[119]{Diamond_1987}. 

Before humans had gathered into large groups, some herd animals already had the population size necessary to sustain bacteria and viruses. Disease transmission became more likely as these herds were kept nearer to areas of habitation \parencite[273]{Armelagos_1970}. 
This close contact with animals resulted in them becoming both direct sources of disease and indirect sources of wildlife diseases (Fig \ref{fig:ScollanFig2}) \parencites[370]{Pearce-Duvet_2006}[446]{Daszak_2000}. 
Contact with milk, hair, animal dust, or faeces could result in anthrax, Q fever, 
brucellosis, tuberculosis, and Toxoplasma gondii \parencites[273]{Armelagos_1970}[2750]{Sibley_2009}). 
Furthermore, studies have demonstrated that the specialised human pathogens responsible for measles, 
pertussis, smallpox, and tuberculosis cause similar immune responses when experimentally 
infected in mammalian models \parencites{Norrby_1985}{Arico_1987}, 
suggesting that human and animal diseases are similar in nature.

	\begin{figure}[!htb]
		\includegraphics[width=\linewidth]{figures/ScollanFig2.jpg}
		\caption{Number of diseases that humans share with domestic animals, by author. Note that overlaps exist in the data, since several types of pathogens afflict multiple species. 
Also, note that as relative degree of intimacy (contact) with animal increases, so too do shared pathogens. This data supports the idea that many humans pathogens were transmitted from close contact with domestic animals \parencites{McNeill_1984}{Hull_1963}.}
		\label{fig:ScollanFig2} %best is to use the name of the file
	\end{figure}
Phylogenetic sequencing has further improved the understanding of the origins of infectious disease. For example, measles derived from rinderpest, which afflicts ruminants, and is related to canine distemper virus 
\parencite{Westover_2001}. Although the pathogen causing human pertussis, Bordetella pertussis, is a specialised human strain, the disease can also be caused by B. parapertussis or B. bronchiseptica, 
which are two generalised strains affecting sheep, horses, dogs, cats and rabbits\parencite[259]{Porter_1994}. 
Although the origins of smallpox are still debated, 
one possible explanation of this specialised human pathogen 
is that it derived from a similar rodent strain of the virus that was 
indirectly transmitted to humans from infected cows \parencites{Pulford_2002}[115--116]{Gubser_2002}. 

Human populations would have been additionally vulnerable to these new infectious diseases due to poor nutrition and health. 
In the late Palaeolithic, humans further broadened their diets to include more small
 game and various plants in order to reduce the risk of a relatively unpredictable 
food supply \parencites[6993--6994]{Stiner_2001}[25--26]{Stiner_2002}. 
Eating different foods protected from reliability on a single food source,
thus limiting the possibility of starvation or malnutrition. 
The rise of agriculture and domestication reduced the types of food available, 
limiting most diets to staple crops and livestock.
In addition to greater 
likelihood of starvation if crop failure occurred, major starches, 
such as maize, rice and wheat, offer little vitamins or nutrients \parencite{Larsen_1995}. 
Poor health is evident from skeletal assemblages. 
Average height decreased by about five inches for males and females from the 
Palaeolithic to Neolithic, and enamel defects indicative of malnutrition, anaemia,
 bone lesions, and degenerative conditions like arthritis increased \parencite{Armelagos_1984}. 
Since malnutrition contributes to disease susceptibility \parencites[119]{Diamond_1987}{Cohen_2000}, 
infectious diseases were even more likely to affect the growing human population during the Neolithic.

%\section{Effects of animal domestication} 
%\subsection{Co-evolutionary dynamics between humans, animals and microbes}
Genes selected\marginnote{Effects of animal domestication\\Co-evolutionary dynamics between humans, animals and microbes} in the Palaeolithic would have provided little protection against new infectious 
diseases \parencite[115]{Armelagos_2005}, 
so gene adaptations and evolution of immune system mechanisms would have been necessary. 
As the population was larger than ever before, there was greater genetic heterogeneity, 
which resulted in the greater probability of developing novel mutant genes to protect 
from disease \parencite[273]{Armelagos_1970}. 
As contact between pathogens and humans increased and continued, 
the virulence of the pathogens decreased as individuals developed 
immunity and populations developed resistance \parencites{Dubos_1975}{Motulsky_1960}. 
Population resistance could have occurred relatively quickly, as the 
rate of gene selection for humans can be as little as 500 years or approximately 25 generations \parencite[253--255]{Kingsolver_2001}. 
There is evidence that individuals were developing increased immunity to diseases. 
Unlike a population of hunter-gatherer Natufians (10500--8300\BC, $\sim$200n) from the southern Levant, a population of pre-pottery Neolithic skeletons (8300--5500\BC, $\sim$200n) from the same region had a significantly greater amount of bone lesions, suggesting not only a greater amount of inflammatory disease, but also an immune response to it (Fig. \ref{fig:ScollanFig3}) \parencite[128]{Eshed_2010}. 
Lesions would not have had time to develop if an individual was 
killed quickly from a disease, and thus they suggest a degree of disease resistance \parencite{Wood_1992}. 
As individuals acquired immunity, the population as a whole began to acquire resistance.

	\begin{figure}[!htb]
		\includegraphics[width=\linewidth]{figures/ScollanFig3.jpg}
		\caption{Prevalence of inflammatory bone lesions in pre-Neolithic and Neolithic populations, by author. Difference is significant (one-tailed Fisher’s exact test, $df=1, P=0.033$). Data and calculations sourced from \cite{Eshed_2010}.}
		\label{fig:ScollanFig3} %best is to use the name of the file
	\end{figure}
The classic genetic examples of adaptation to disease are sickle cell trait, thalassemia, and glucose-6-phosphate dehydrogenase (G6PD) deficiency, which all protect against malaria. 
Plasmodium infect and reproduce in red blood cells, and each of these genetic mutations alters the environment of the red blood cell, making it unsuitable for pathogen development \parencites[235]{Balter_2005}{Durham_1991}. However, these traits are usually the most beneficial when a person is heterozygous for them, or has one mutated allele to compliment one normal allele in a homologous chromosome pair. 
Homozygosity, or two mutated alleles, results in other health issues, such as anaemia and sickle cell disease. Another example of human gene evolution in response to disease is a mutation of the CCR5 gene. Shown to protect against modern-day AIDS, this mutation may have been selected in populations due to its ability to protect from smallpox \parencite[235]{Balter_2005}. 
Mutations in MHC molecules, which are involved with the immune system’s recognition of pathogens, may have also led to protection from malaria and other infectious diseases \parencite[2038]{Hill_2001}. 
The diseases that arose from animal domestication did not only alter human genomes and populations, but also affected the evolutionary history of the animals and disease-causing microbes. 
For example, taenid worms infecting humans derived from those that had felids and hyenas as definitive hosts and bovids as intermediate hosts \parencite[785--786]{Hoberg_2001}. 
Phylogenetics demonstrates that after the domestication of animals, close contact with humans resulted in cattle and pigs becoming intermediate hosts of taenid worms \parencites[376]{Pearce-Duvet_2006}{Schmidt_1989}. Furthermore, evidence demonstrates that pathogens may be able to shift rapidly between generalist and specialist strategies when host conditions are in flux due to the rapid generational turnover of 
microbes \parencite[539]{Wolinsky_1996}. 
Generalised animal pathogens evolved to become specialised for human hosts when they came into contact with the large, vulnerable human population. Furthermore, as human populations developed resistance, pathogens evolved to combat these resistant genes, resulting in the on-going relationship between human resistance and bacterial anti-resistance. 
Thus, the domestication of animals resulted in a co-evolutionary dynamic between humans, animals, and pathogens.

%\subsection{Human cultural solutions}
Adaptive lag\marginnote{Human cultural solutions} is a term in ecology that refers to the time lapse between a change in selection pressure and the evolutionary response that results \parencite[98]{Laland_2006}. 
Cultural niche construction delays adaptive lag by presenting solutions to problems through behavioural modifications. 
The rise of infectious diseases altered the niche originally created when humans began to domesticate animals, and in turn, humans created cultural solutions for the disease problem. 
In the archaeological record, sewer systems, wells, and latrines are the most easily identifiable forms of evidence to suggest sanitary measures.
The Indian late Neolithic site of Mohenjo-Daro (2500\BC) had an advanced sewage system of covered ducts and drainage canals \parencite[121]{Mughal_2011} 
and there is evidence of drainage systems and primitive “toilets” 
in the Orkney settlement of Skara Brae (3100--2500\BC) as well \parencite{Gordon-Childe_1983}. 
Such evidence suggests that humans had determined the benefits of separating wastes from the living environment.

Although archaeological evidence of other disease solutions from the Neolithic period is scarce, 
a discussion of solutions later in history may suggest measures taken by the populations 
first facing infectious disease. As \textcite[3]{Douglas_1966} states, 
no one can be sure when the ideas of pollution and impurity first developed. 
There are many laws of sanitation and quarantine in the Bible. In Deuteronomy (23:12-14), 
it is advised to wash oneself after touching pus or blood and to deposit waste outside of the main living area.
Also, quarantine of individuals with leprosy or other skin afflictions is advised (Leviticus, 13:1-59).
Perhaps earlier peoples also determined that avoidance of a diseased person or animals may be protective. Overall, it is likely that through observation of the effects of new pathogens, humans began to develop solutions in order to combat or prevent rise and spread of disease.

%\section{Conclusion}
The \marginnote{Conclusion}aim of this paper is to use niche construction theory to explore how animal domestication in Southwest Asia during the Neolithic period brought about a new disease-scape. Niche construction theory emphasizes that organisms can change their environments and co-direct their own and other species’ evolution. This differs from conventional co-evolutionary theory, which stresses that the environment changes the organisms, not the other way around. Throughout human history, we have altered our environments to suite our needs, leading to selective pressures that changed the gene pools of both our population and the populations of other species.

Reasons for animal domestication differed between regions and periods. In general, domestication was not spontaneous, but rather the outcome of a series of purposeful decisions regarding environmental management. 
NCT allowed for a complete analysis of animal domestication as it accounts for both the biological change of domesticated animals and the cultural change that occurred when domesticates became a part of the human environment. 
The proximity of animals to humans introduced new pathogens into the human population. These pathogens were able to spread due to the growing size of the human population, which had grown due to increased food sources and a sedentary lifestyle. Furthermore, factors such as animal herds having been big enough to host pathogens prior to domestication and poor nutrition accounted for additional vulnerability of human populations during the Neolithic.

The rise of infectious disease in human populations set off a series of co-evolutionary interactions between humans, animal, and microbes. Humans that survived to adulthood may have developed immunity to certain diseases. As the human population was larger than ever before, greater genetic heterogeneity resulted in the development of population resistance to disease. Furthermore, the genetic landscapes of animals and pathogens were altered as well. Domesticated animals became new intermediary hosts of different pathogens such as taenid worms, and pathogens themselves evolved to infect different hosts or specialised for human hosts. In addition to evolutionary changes, humans attempted to prevent or contain disease through cultural constructs such as various sanitation measures. 
Figure \ref{fig:ScollanFig1ab}b provides a summary of the argument of this paper, 
depicting that animal domestication led to the rise of infectious disease, which in turn led to evolutionary and cultural changes in human populations.

NCT brings together both biological phenomena and cultural practice in its explanations of events. Because of this, it would be a valuable tool in explaining various archaeological contexts, as human history is as much a story about the choices of people as it is a story of changing genes and populations. 
This theory could be used as a foundation to explain other archaeological queries regarding disease throughout time, such as the social and biological effects of the medieval Black Death, the devastation of Native American tribes upon European arrival to the Americas, the \nth{19} century cholera outbreaks, or even the modern-day AIDS pandemic. Humans are still evolving, adapting, and creating new niches, and with every new niche, comes a new set of effects that will alter our genetic and cultural future.	
\myseparator
	%	ABSTRACT SPANISH
	%----------------------------------------------------------------------------------------
	\begin{myabstract}
\foreignlanguage{spanish}{
	La construcción\marginnote{Abstract (Spanish)} de nichos es la modificación del ambiente de un organismo que resulta en el cambio evolutivo (biológico) y/o cultural (antropológico) del organismo. En este este artículo se explora el aumento de enfermedades patógenas en el Suroeste de Asia durante el Neolítico ($\sim$10000 to 4500--2000 a.\,C.) a través de la perspectiva de la construcción de nichos. Los humanos construyeron un nuevo nicho ecológico cuando domesticaron a los animales, lo cual dio como resultado el aumento y dispersión de varios patógenos mortíferos, debido al incremento de la población humana y la proximidad entre hombres y animales. Para combatir y prevenir la dispersión de enfermedades, los humanos han respondido mediante cambios culturales, como la aplicación de nuevas medidas sanitarias. De igual manera, la población humana también comenzó a evolucionar para contrarrestar las enfermedades, desarrollándose dinámicas co-evolutivas entre seres humanos, animales y patógenos, las cuales continúan presentes en la actualidad.}

	\keywords[Palabras claves]{Construcción de nichos, dinámicas co-evolutivas, domesticación, enfermedades infecciosas, Neolítico}
		
	\end{myabstract}
	

\printbibliography[heading=subbibnumbered] 
\label{Scollan:lastpage}
\closingarticle
