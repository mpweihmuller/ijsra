\openingarticle
\def\ppages{\pagerange{presentation:firstpage}{presentation:lastpage}}
\def\shorttitle{IJSRA Issue Presentation}
\def\maintitle{Presentation of the first issue of the \emph{International Journal of Student Research in Archaeology}}
\def\shortauthor{Gonzalo Linares Matás}
\def\authormail{gonzalo.linaresmatas@st-hughs.ox.ac.uk}
\def\affiliation{St. Hugh's College, University of Oxford}
\def\thanknote{\footnote{\href{https://oxford.academia.edu/GonzaloLinaresMat\%C3\%A1s}{Gonzalo Linares Matás} is a second-year undergraduate studying the BA in Archaeology \& Anthropology at St. Hugh’s College, University of Oxford.}}
%--------------------------------------------------------------
\mychapter{Presentation of the first issue of the\\ \emph{International Journal of Student Research in Archaeology}}
\begin{center}
	{\Large\scshape\shortauthor}\\[1em]
	\email \\
	\affiliation
\end{center}
\vspace{3em}
\midarticle
%--------------------------------------------------------------
\label{presentation:firstpage}

\lettrine[slope=4pt,findent=-3pt,lines=3]{A}{rchaeologists} are relevant social agents, and they should aim to demonstrably pre\-sent our work as interesting and useful. 
We have the social responsibility to make knowledge available for the public engage with our audience, and to reach people outside the discipline. The practical side of archaeology can be one potential scenario. 
As \emph{Lilla Vonk} notes, the possibility of interacting and experiencing the past through and engagement with heritage generates fascination and activates imagination, positively impacting the wellbeing of dementia and arguably other mental health patients. 
Our discipline can provide an invaluable source of motivation for continuing healthcare in a more enjoyable and meaningful way. Another possibility for public engagement is through the presentation of both research and interpretation. 
\emph{Antonio Sánchez}, in his study of Roman \textit{viae} in Hispania, convincingly argues that archaeology is not limited to excavation and recording. 
The public dimension of heritage ownership demands that it should be known and respected by everyone, and museums can become a platform for this shared exploration of the significance of archaeological remains.
The conservation and curation of museum pieces thus become a paramount dimension of archaeological practice due to their relevance in public presentation and heritage display as embodied materiality of historical memory.
\emph{Wael Gabo Elgat} reports the scientific methodology behind the treatment of Khedive Ismail’s antique gun at the NMM-Saladin Citadel in Egypt after suffering a certain degree of decay.
		
The study of the past has a clear socio-political dimension in the present, both in the construction of interpretive narratives and in the process of public reception. \emph{Arba Bekteshi} investigates the implicit intersection between archaeological interpretation and the political context in which this process takes place. Certain political regimes utilise archaeology in order to construct and present and actively manipulated vision on the past, with propagandistic purposes. Nationalist narratives tend to portray an idealised notion of a by-gone past which paradoxically embodies the anxieties and political desires of grandeur and supremacy imagined for the future of the community, and these perspectives are illustrated by the case of Illyrian numismatics in interpretations of Albanian archaeology during the communist period. Arba advocates for increasing awareness of the cultural baggage of these practices for the future of the discipline, which methodologically and theoretically should continue its move forward from a culture-historical approach to a more nuanced, post-colonial and interpretive academic landscape. In relation to the reception of archaeological knowledge, \emph{Ariane Maggio} argues that identity is socially-constructed concept based on personal experiences throughout our lives. She explores how the scientific research and knowledge about biological notions, such as genetic inheritance can have a public impact in transforming or reinforcing our perspectives of relatedness as a species, decisive in the dynamics of positional relationships of humans within our socio-cultural worlds.
		
Human life and social organisation is not dependent on any single factor, so simple narratives about the origin and collapse of societies are, by definition, incomplete. Neither are humans doomed to fail in difficult natural environments, nor are they completely isolated from their influence. \emph{Dylan Davis} presents in this paper a complex explanatory framework on the tetra-dimensional influence of climate. He considers the interrelation of multiple factors in the interpretation of the causes of the flourish and decline of civilisations, applied to the case of the Roman state. Instead of arguing for a single, definitive cause, he argues that environmental, economic, and socio-political contexts are presented as mutually influential in cultural change, supported by a wealth of diverse data, from palaeoenvironmental diagrams to archaeological interpretations of shipwreck patterns and the differential trade interactions between Roman communities through time.
		 
The times where a Western white archaeologist engaged single-handedly, without assessing the influence of his own background, in the interpretation of archaeological remains are now part of the history of the discipline. The practice of archaeological research now is decidedly more inclusive and recognises the multivocality of the evidence. \emph{Lucy Northwood} reconsiders how particularities of past local experiences need to be taken into account through a broader characterisation of their cultural context. Her inclusion of Hindu religiosity, local narratives and a critical phenomenological approach in order to achieve a more nuanced interpretation of the embodied relation between humans and their landscapes in India. \emph{Rick Takkou} and \emph{Sonja Dobrosky} explore the usefulness of alternative approaches to archaeological practice in challenging contexts. In their surveys, they apply post-colonial methodologies to the prehistory of the classical world, incorporating the voices and experiences of the local community of Zakynthos into the processes of knowledge production, in a more socially-engaged conceptualisation of archaeological research.
		 
Archaeological interpretation is constantly influenced by theoretical developments in other disciplines. In addition to its long-standing relationship with anthropology, archaeology has borrowed and applied concepts from both natural sciences and humanities. \emph{Margaret Scollan} illustrates the application of niche-construction theory for understanding the interaction between humans and their living environment. The domestication of animals and the sedentarisation of human communities represented an innovative context which fostered the development of pathogenic agents. Humans had to adapt against these pathogens by developing cultural responses to alleviate their impact. Solutions derive in new problems, so human communities and the elements of their environment are entangled in the engagement to create alternative niches in which to thrive, in an interrelated, co-adaptive process. \emph{Vivian van Heekeren} assesses the integrative potential of Panofsky’s multi-layered approach to explore the symbolism and the meaning of the elements present in the Memento Mori mosaic from Pompeii. These interdisciplinary perspectives can contribute to further enriching archaeological interpretation.
		 
A fundamental dimension of the practice of archaeological interpretation is to reconsider previous knowledge on the basis of the evidence available. Several papers deal with reassessing claims or misconstrued narratives about the past. Colonial encounters fostered the reconsideration of notions of identity and racial purity, which I perceive them to be cultural classificatory schemes to regulate and stratify the internal hierarchies in social relations within multicultural environments. \emph{Kelton Sheridan} specifically deals with the question about the relevance of the notion of hybridity recently posed by S. Silliman. The author argues that the notion of hybridity, not as a condition but as a process, is still relevant as a concept, insofar it reminds us humans of the reality of our inherent genetic intermixing, making irrelevant claims supporting discrimination based on supposed biological or racial grounds.
		 
Another important type of reassessment encompasses a more nuanced understanding of the past. \emph{Helen Rayer} challenges the traditional academic conceptualisation of the functionality of settlements in Britannia, avoiding a strong dichotomisation between civil and military in Roman times. She argues that there is an intrinsic relationship between both aspects in the nature of the organisation of social relationships within the context of the Roman presence in Britain. \emph{Sam Hughes} explores how the shape of Irish Iron Age swords represents an adaptation of ‘global’ trends in military weaponry to localised warfare needs. He reminds that archaeological objects were functional items of daily life, rather than just typological markers in seriation sequences.
		 
Archaeologists sometimes tend to perceive ‘cultures’ as discrete entities, and relationships between them tend to be restricted to material exchanges. \emph{Bertie Norman} presents an ambitious comparative project exploring the influences in Homeric epic narratives of contemporary cultures, proposing the notion that the Homeric underworld could be based on real experiences and geographies. These texts were written down in the Archaic Period, but are arguably deeply rooted in Bronze Age conceptualisations of warfare and worldviews of Eastern Mediterranean societies.
		 
Hypotheses and interpretations about the past through material remains should be assessed for their plausibility. 
In this issue, three authors specifically deal with methodologies for hypo"-thesis"-testing. Contemporary intensive archaeological research is characterised by the gathering of large amounts of data, which, need to be processed and interpreted. \emph{Alix Thoeming} considers computational and mathematical procedures to render patterns understandable. She argues that programs such as the Naïve Bayes Classifier can complement intuition, knowledge and experience in the interpretation of correlations in the archaeological record. Alix applied it to the variability of decorative designs in Viking-age stone runes in order to discern whether any temporal pattern can be discerned, given that they cannot be dated through other conventional techniques. She also highlighted the usefulness of statistical analyses for testing and discarding hypotheses, or to confirm the inexistence of patterns which may have otherwise been used for seriation purposes. \emph{Camilo Barcia García} incorporates advanced spatial methodologies, including GIS, to test hypotheses about human behaviour and the management of social space in the Palaeolithic, through the case study of the Lower Gallery of La Garma (Cantabria, Spain). Another way of testing hypotheses is through experimentation. \emph{Amanda Gaggioli} incorporates here an additional line of evidence in support of the archaeological interpretation of functionality of early stone points in the Americas. Experimental archaeology is increasingly recognised as a source for broadening our understanding of the intricacies of the technological processes involved in the production of tools by past human communities.
		 
		 We are delighted to announce that in this issue we have included two interviews with key academic archaeologists. We are grateful to Prof. Brian Fagan (UC-Santa Barbara) and Prof. Rosemary Joyce (UC-Berkeley) for accepting our invitation. We have also included in this issue reviews of archaeological conferences which are specifically oriented towards students, such as \emph{ASA} or \emph{NASC}. We are very grateful to the conference coordinators for their enthusiasm in collaborating with us. We have also included three other conference reviews by students (\emph{ASAPA}, \emph{NZAA}, \emph{WESIPS}). These conference reviews contribute to the exploration of the international dimension of archaeological research and the opportunities available for students to get involved in the further diffusion of their research.
		 
		 In the dialogue with \emph{Prof. Fagan}, he reminds how important is understanding the historical dimension of archaeology as a discipline, bearing in mind that colonialism and the expansion of the global empires did foster research throughout the world, setting an agenda which is increasingly adapting to a post-colonial theoretical landscape. He argues that academia, particularly the US system, is too focused with testing and grading, which is “mindless nonsense and counterproductive”, but, in his view, the “total obsession in academic archaeology” is the notion of ‘publish or perish’. I think that this drive encourages shorter research projects, which may not fully explore the addressed questions. Furthermore, Prof. Fagan argues that publishing and the diffusion of archaeological knowledge should not be restricted to senior folk; “quite the contrary”. Students should contribute to the dialogues for the discipline to retain its vitality. “Without new ideas, fresh faces, and innovative perspectives, we are nothing”. This public outreach is paramount, because, although “it’s fatal to assume that everyone is interested in archaeology”, we “cannot afford to be an ivory-tower discipline divorced from the real world”. Archaeology has a fundamental contribution to make “about cultural and biological diversity, and about the nature of being human”. What is more, I think that through the long-term perspective derived from the nature of its evidence, archaeology provides a clear insight into the notion that things were different in the past, and that they can be different in the future. 
		 
		 \emph{Prof.\,Joyce} explains how important is to engage students into the active learning of useful research skills, which can be applied to professional practice. Teaching therefore could be adapted to the motivations of the learners, who would explore and discuss those aspects in which they are interested the most. Moreover, with the anxiety for unpublished data in research and the inherent fieldwork limitations in terms of money and time, Prof. Joyce reminds how useful research sources can museum collections be for students. Museums also have a key role to play in public engagement with archaeological knowledge, and can become places in which taken-for-granteds can be challenged and transformed, although these institutions should acknowledge that the information they present is “an interpretation, not a timeless truth”. Prof. Joyce notes that archaeological publishing needs to be updated to the context and needs of the twenty-first century: archaeologists should “create a network of knowledge production”, quicker and focused in advancing our understanding, rather than delaying publication until having the big statement in pursuit of individual recognition. We students should develop our careers as generous scholars, mindful about the consequences that our interpretations about histories have for past and present societies, explaining what are we doing through archaeological research and interact with local communities.
		 
\SetBlockThreshold{1} 
 \blockquote{\textit{“I think we show a great failure of perspective to worry about [the abstract global past] when so many people have been killed, wounded, and displaced. How would I like to solve this? Stop the wars. Stop strategic invasions. Stop drone strikes. Stop the international traffic in arms. (\ldots) Spend less of our time pursuing big history, and more writing the textured histories of the local, the everyday, and everyone.”} – Prof. Rosemary Joyce.}	 
		 Prof. Joyce reminds that archaeology shows that societies can fail to meet challenges, and that “we can kill ourselves through insistence that doing the same thing will work”. I believe that archaeology can become a constant reminder that we can change and improve the current socio-historical context we are living in.

		I would like to thank everyone who has engaged in all the different stages of this innovative project, particularly authors and reviewers, for their support, participation and patience. We are looking forward to continuing our role in the future, increasing the international presence of student archaeological research and practice. 

		

	\label{presentation:lastpage}
\closingarticle
