\documentclass{ijsra}
\def\IJSRAidentifier{\currfilebase}%<<<< DO NOT change this line
\def\shorttitle{Elephant Imagery on Greco-Bactrian and Indo-Greek Coinage}
\def\maintitle{Local Contexts to Elephant Imagery on Greco-Bactrian and Indo-Greek Coinage}
\def\shortauthor{Replace with author name}
\def\authormail{Replace with author email}
\def\affiliation{Replace with author affiliation}
\def\thanknote{Author biography}
\def\keywords{Hellenistic, Greco-Bactrian, Indo-Greek, Numismatics, Royal Ideology, Elephants.}
%\def\keywordname{}
\begin{filecontents}{\IJSRAidentifier.bib}
Bibliography-files
\end{filecontents}

\begin{document}
\IJSRAopening%<<<< DO NOT change this line
	{\Large\scshape
	\shortauthor}%
	\footnote\thanknote%
	\\[1em]
	\email\\
	\affiliation
\IJSRAmid%<<<< DO NOT change this line

\begin{IJSRAabstract}
The monarchs of the Greco-Bactrian and Indo-Greek kingdoms in Hellenistic Bactria and India are known mainly through coins. This article studies the political significance of elephant motifs on Greco-Bactrian and Indo-Greek coinage. The study involves content and context analysis of the iconography in order to gain an understanding of the meaning and function of the elephant imagery. Through the analysis, the paper argues that Greco-Bactrian and Indo-Greek kings used elephant symbolism to express royal power in a cross-cultural context. Hellenocentric approaches that overlook the local context to royal power in Hellenistic kingdoms are challenged. This helps provide a clearer understanding of how Hellenistic kings established and maintained control over kingdoms with multiethnic populations.
\end{IJSRAabstract}

\lettrine[nindent=0em,lines=3]{T}{he} Persian satrapy of Bactria, in present-day Afghanistan, Uzbekistan, and Tajikistan, was conquered by Alexander the Great in 327 \BC Afterwards, Bactria was incorporated into the Seleucid Empire, established by Alexander’s general and successor in Asia, Seleucus Nicator. In the mid-third century \BC, Bactria became an independent Hellenistic kingdom under the Greco-Bactrian kings, who expanded their territory in the early second century \BC to include ancient north-west India (now mainly Afghanistan, Pakistan, and parts of India). 

Remaining article continues..

\IJSRAclosing%<<<< DO NOT change this line
\end{document}