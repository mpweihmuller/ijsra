\openingarticle
\def\ppages{\pagerange{Vonk:firstpage}{Vonk:lastpage}}
\def\shorttitle{Archaeology of Dementia Care}
\def\maintitle{Archaeology’s Potential for Active Engagement in Dementia Care}
\def\shortauthor{Lilla Vonk}
\def\authormail{lillavonk@gmail.com}
\def\affiliation{Leiden University}
\def\thanknote{\footnote{Lilla Vonk  is currently a Master student at Leiden University in the research track of Archaeological heritage in a globalizing world. She holds a BA cum laude in Ancient cultures from VU University in Amsterdam and a BA (Hons) in Aesthetics, Art and Culture from Université Paris I Panthéon-Sorbonne. Her research interests are articulated around the social benefits of archaeology and heritage. Her current research focuses on the potential for archaeology-based assistance in dementia care in the Netherlands.}}
%--------------------------------------------------------------
\mychapter{\maintitle}
\begin{center}
	{\Large\scshape\shortauthor\thanknote}\\[1em]
	\email \\
	\affiliation
\end{center}
\vspace{3em}
\midarticle
%--------------------------------------------------------------
\label{Vonk:firstpage}

%----------------------------------------------------------------------------------------
\begin{myabstract}
Dementia\marginnote{Abstract\\ (In Dutch see below)} is prevalent among the elderly population 
of Europe, and cases of dementia are expected to increase rapidly in the coming years. 
While dementia has severe psychological impact and social consequences for individuals, 
it has primarily been studied from a neuro-medical viewpoint. Understandings of the
psycho-social implications of the syndrome and consequences for wellbeing and quality of
life are topics that have begun to emerge only in the previous two decades.
The involvement of disciplines other than those stemming from the neurological and 
medical fields, can enrich the lifestyle and wellbeing of dementia patients.
This paper argues that archaeology can make a valuable contribution to European dementia care.
It sets out a theoretical argument that builds on previous initiatives involving archaeology
and heritage within a health care context. I argue that specific characteristics of
archaeology make it suitable for such an involvement. I conclude that engaging in 
archaeology-based activities could be beneficial for the well-being of people with dementia.
	
\keywords[Keywords]{Archaeology, heritage, health care, dementia, wellbeing, quality of life.}	
\end{myabstract}
	
%----------------------------------------------------------------------------------------
	%	KEYWORDS
%----------------------------------------------------------------------------------------

\lettrine[nindent=0em,lines=3]{C}{ultural} heritage and archaeology are attributed with great social potential, which should be developed and tapped into over the coming years.
In a recent publication, the European Commission referred to cultural heritage as “a significant force for \nth{21} century Europe \ldots 
It is being discovered by both governments and citizens as a means of improving economic performance,
people’s lives and living environments” \parencite[5]{Commission_2015}. 
For the archaeological sector, social relevance is particularly gaining in importance. 
Social relevance is a crucial and integral part  justifying archaeological practice as well 
as determining its significance within current societies \parencite{Boom_inpress}.  
Furthermore, the Faro Convention \parencite{Europe_2005}  underlines the importance
of public involvement with cultural heritage, connecting social and cultural development 
with  an improvement in quality of life \parencite[5--7]{Europe_2005}.	
		
This last point is a particularly interesting avenue to pursue. 
While the Faro Convention stresses the significance of cultural heritage 
for society at large, there are groups within European society that could benefit more, 
from an involvement with cultural heritage. I would like to argue here that the social
potential of cultural heritage, and specifically archaeology, has particular benefits for 
persons affected by dementia. The premise of this paper is that the cultural heritage sector
could be a particularly powerful contributor to dementia care, 
and that it is time to gain insight into the ways in which, the cultural sector and
archaeology, can actively contribute to developments within dementia care in the near future. 
	
One of the consequences of dementia, which will be discussed further on, 
is the occurrence of behavioural changes in individuals that can take various psycho-social
and physical shapes (such as frustration, aggression, boredom and wandering).
Traditionally, care for individuals with dementia has, perhaps justifiably,
primarily focused on dealing with, what was viewed as, negative behaviour and consequences 
in ways that can be characterized as symptom management; mood and behaviour were to be
controlled, if needed by using medication \parencites[541]{Kitwood_1993}[541]{Kitwood_1993}.
The tendency of using antipsychotic medication to counteract negative behaviour has been
shifting over the past ten years, but there remains considerable room for improvement
 \parencite[39--44]{Banerjee_2009}.	
	
As far as scientific research is concerned, dementia has for a long time been approached 
from a purely medico-scientific viewpoint. Much research has focused on the prevention and
neurological progress of dementia, but not on the important social and psychological
implications and consequences of this condition. Meanwhile, there is a large and increasing
group of people suffering from dementia, who are dealing with decreased quality of life and 
a poor general sense of well-being as a result of their condition and the way it is handled.
This is an urgent matter, but different approaches, hailing from different fields other than
the medical and neurological ones, provide promising ways to tackle this problem.

In this article, I will explore the possibility of an involvement from the archaeological 
and heritage sector in health care, and later more specifically in dementia care. 
Firstly, I will elaborate on the increasing prevalence of dementia and the implications of 
the condition in regards to quality of life. Subsequently, I will set out my argument 
building on a recent initiative that brought a heritage-based activity to different 
health care settings.
		
%\section{The demographic aging of Europe}
	
In\marginnote{The demographic aging of Europe} order for the cultural heritage sector 
to connect favourably with society, it is crucial to keep societal trends in mind. 
Currently, Europe’s elderly population is steadily increasing \parencite[8]{Grut_2013}.
Demographic trends over the recent decades show the younger generation is decreasing, 
due to a declining number of children per family, whereas the life expectancy for the 
existing and elderly population is growing \parencite[8]{Grut_2013}. 
Faced with these aging populations, European societies are beginning to re-think 
their existing economic and societal structures in answer to the challenges that 
accompany this demographic development \parencite[9]{Grut_2013}. 
One particular challenge is that as the elderly population increases, so do the cases 
of dementia. In 2009, it was estimated that in Western Europe 6.98 million people are 
living with dementia, a number which is expected to increase to 13.44 million in 2050 \parencite[8]{International_2009}. 		
In response to these current demographic trends, the EU has stated that life-long 
learning is an important strategic tool to improve social cohesion and economic 
development \parencite[12]{Commission_2010}. Another important aspect and potential 
challenge for aging populations is their quality of life. Several initiatives 
throughout Europe have made use of cultural heritage to stimulate elderly people’s 
mental health and wellbeing by providing learning and participatory activities \parencite[8]{Grut_2013}. 
However, these initiatives mainly focus on the segment of 
the elderly population that is not affected by mental health problems, such as dementia.
 Initiatives that do target individuals with this condition while addressing their quality 
of life through a cultural heritage approach are not widespread. 
This is unfortunate since, especially for this segment of the population, quality of life 
and the improvement thereof is an important issue. Moreover, heritage approaches to 
this topic show a distinctive potential which deserves further attention.		
		
%	\section{Quality of life}
	
Here,\marginnote{Quality of life} it is useful to elaborate a bit further on the concept 
of quality of life. Quality of life is an index for the general well-being of both 
individuals and societies. It consists of several sub-indicators, such as health, 
living environment, education, leisure and social interaction. Within health care, 
the perceived quality of life is expressed in the so-called health-related quality of life.
This health-related assessment of quality of life specifically considers how disease or 
disability affects the well-being of an individual \parencite{prevention_2011}.  
The indicators are divided into several domains. These domains vary from physical to 
social and environmental markers. The latter comprises the opportunities and participation 
in (cultural) recreation and leisure. There are currently multiple instruments in use 
to assess quality of life, which are often tailored to specific groups such as elderly 
people with dementia. Indicators can vary depending on the assessment instrument applied. 
No specific instrument is central to this paper, and quality of life will therefore refer 
to general categories that recur in multiple instruments, such as social contact 
and opportunities for participation in recreational activities and leisure, 
than to points within these categories that are more specific to particular instruments.
		
%	\section {Dementia and implications for quality of life}
	
Dementia\marginnote{Dementia and implications for quality of life} is an umbrella term 
for conditions which entail the progressive loss of cognitive functions, severe enough 
to interfere with a person’s daily functioning. Dementia itself is not designated as a 
disease; but rather a cluster of symptoms that can accompany a disease such as 
Alzheimer’s or Parkinson’s disease. When caused by a disease, the process is considered 
to be irreversible and no clinical cure exists.  There are different types of dementia, 
and while each type has a set of specific characteristics, the aforementioned general
pathology remains similar \parencite{Europe_2013}. Old age is an important risk factor, 
and the vast majority of dementia sufferers are individuals aged over 65. 
Dementia can be distinguished from so-called age-associated memory impairment by the severity 
of cognitive decline. Age-associated memory impairment is considered as only very mild 
cognitive decline, which manifests itself in memory deficits -- such as the forgetting of 
names or the placement of familiar objects. However, this does not affect an individual’s 
ability to participate in social engagements.  In comparison, cases of mild dementia do 
affect this ability, as dementia causes moderate cognitive decline, resulting in decreased 
concentration, gaps of memory concerning one’s personal history, and a decreased ability to 
travel and handle finances. As dementia progresses individuals increasingly lose their 
independence, and eventually become entirely dependent on others to take care of them \parencite{Reisberg_1983}.
	
The progression of dementia has serious implications for the quality of life of 
individuals affected by the condition. It causes an increasing rate of dependency 
while at the same time, placing individuals at greater risk of becoming socially isolated. 
As the condition progresses, an individual might not be able to pursue the hobbies or 
activities that interested them previously, which can lead to boredom as well as physical, 
social and cognitive under-stimulation and isolation \parencite{Society_2013}. 
Dementia has thus come to be described as a condition that seemingly strips away 
personhood, agency and social contacts, including intimate relationships as well as 
social confidence, causing apathy or depression in many cases \parencite[274,284]{Kitwood_1992}. 
Another important factor is the way dementia is handled and viewed, 
not only by those suffering from it but also by those in their immediate surroundings. 
Decline and loss of abilities are often strongly emphasized, which only reinforces the 
feeling of impairment. The consequence of these diverse factors is an increasing and
significant loss of quality of life and well-being.		
	
For a long time however quality of life has not been central to dementia research, 
which rather focused on the medical and neurological aspects of the condition instead 
of the persons suffering from dementia. This changed in \citeyear{Kitwood_1992} when \citeauthor{Kitwood_1992} published a novel article presenting a person-centred approach to dementia.
\citeauthor{Kitwood_1992}  suggested that while neurological degeneration is a determinant for an individual’s performance; 
personal psychology and social environments are also equally important influences on the progress of dementia. Most importantly, 
\citeauthor{Kitwood_1992}  advocated that social interactions could be the key to improving the condition of affected persons, and may even halt the progress of their condition. 
He connected social activity to so-called “rementia”, the regaining of formerly lost cognitive capabilities \parencite[271,280]{Kitwood_1992}. 
Since then, concepts like quality of life and well-being in the context of dementia have generated more attention, interestingly also from the cultural 
heritage and archaeology sector.	
	
%	\section{Involvement of cultural heritage and archaeology with health and wellbeing}
	
Some\marginnote{Involvement of cultural heritage and archaeology with health and well-being}
initiatives have already begun to explore the role cultural heritage could potentially 
play to improve the well-being of patients in the health sector. 
I will now discuss one of these initiatives in order to illustrate the potential 
of a heritage-based approach within health care. This specific project has shown 
encouraging preliminary results, which could prove particularly insightful in the 
context of dementia care. Furthermore, the research was conducted in several different 
care situations: in a hospital offering acute care, a psychiatric hospital, an elderly 
care home and within two neurological rehabilitation units, suggesting such an 
approach would be applicable within varying contexts \parencite[231--232]{AnderE_2013}.	
	
%	\section{Heritage in Hospitals}
	
The\marginnote{Heritage in Hospitals} project in question is called “Heritage in Hospitals”,
an initiative that took place in the United Kingdom. The program was developed in 
2008 and the results and outcomes, originating from data gathered from over 
250 participants, were published in 2013 \parencite{AnderE_2013}. 
Project organisers brought a box of artefacts, loaned from university museums, 
to an audience that had been excluded from museums due to hospitalization or 
long-term stays in care homes. The artefacts came from archaeological, zoological, 
geological and art collections and were all relatively small objects (a necessary 
criterion for transport) that were fit to be handled. Depending on the context of 
the institution, either one-to-one (in the hospital and two rehabilitation centres) 
or group (in the elderly care home and the psychiatric hospital) sessions were carried 
out, wherein participants, alongside museum professionals, handled artefacts.	
This project sought to gain an understanding of the therapeutic effects of a so-called 
“museum intervention” in health care contexts \parencite[230--232]{AnderE_2013}. 
Grounded theory was used to collect and analyse the data: an impact-assessment of the 
sessions was established by coding 51 transcribed session-recordings, and the coded data 
was subsequently tied to an instrument used to measure indicators of 
well-being \parencite[237--239]{AnderE_2013}.The key outcomes of this project have been
described as “engagement processes” and “expressions of wellbeing” 
\parencite[234]{AnderE_2013}. The measures taken after the intervention when compared to 
the baseline measure (a measurement taken prior to an intervention), indicate that 
participants showed significant improvements in  levels of well-being, expressed 
in increased positive emotions and decreased negative emotions. 
Participants stated they felt happier and healthier after the intervention. 
Due to the high level of engagement participants experienced, the intervention 
was successful in distracting patients from the hospital surroundings. 
	
In the context of this project, engagement was used as a concept to describe certain
interactions and behaviours, indicating focus, involvement and motivation, 
during the sessions. Over the course of several sessions, participants 
linked the artefacts to personally owned objects, as well as previous knowledge 
and experience, while using words indicating interest, surprise, and even fascination 
and amazement \parencite[234]{AnderE_2013}. The session facilitated multiple facets of 
object engagement, such as engagement through sensory elements (touch, vision), 
learning about the artefacts, personal recollection or connection to the artefacts, 
and sense of privilege. The strong presence of engagement is interesting in this context, 
as many hospitalized participants were dealing with issues of anxiety, pain, uncertainty,
boredom, lack of stimulation, loss of identity, and depression at the time the sessions 
were conducted. After the sessions, most participants indicated having an improved sense 
of well-being, which was expressed through improved mood, decreased anxiety and increased
confidence \parencite[234--235]{AnderE_2013}. 
The outcomes concerning well-being were primarily articulated around distraction and
stimulation, which were argued to be important factors for well-being in the context of
hospitals and care homes. Indicators for well-being were the expression of positive 
emotions, and having a regained sense of vitality and energy; the production of new 
knowledge, as well as generating interest and desire to learn; and tapping into personal
memories and recollections, which was connected to a renewed sense of identity \parencite[235--236]{AnderE_2013}.
									
This research further argued that the use of heritage artefacts, obtained from museum
collections, was of fundamental importance to the increase in positive emotions experienced by participants. The reason being that cultural heritage artefacts are attributed with a certain status, since they are considered important artefacts, be it for aesthetic, material,
historical or other reasons, and thus such objects are particularly well-suited for sparking interest, wonder and fascination. In addition to this, heritage objects on display in museums are (usually) not meant to be touched by visitors, which can induce a feeling of privilege in participants who are allowed to do so \parencite[240]{AnderE_2013}. 
Furthermore, the explicit references made by participants to the specific artefacts that were handled, can be considered an indicator that the positive results were not solely due to the attention and social interaction that the project brought to the participants.	
	
“Heritage in Hospitals” certainly shows interesting and promising results, although more research is required to look into how long such positive effects can last. In the context of dementia care, this initiative shows some particularly compelling results. The participants of this project struggled with issues that are well known to affect persons with dementia, notably boredom, lack of stimulation, loss of identity and depression. 
Some of these issues can induce a state of lethargy which can seem hard to overcome. This project however showed that handling heritage objects can take individuals out of this state, if perhaps only temporarily. In addition to this, the heritage objects were shown to be strong vehicles for reminiscence, which carries a particular importance within dementia care and which will be clarified further on in this paper.	
	
%	\section {The potential of archaeology}
	
	“Heritage in Hospitals”\marginnote{The potential of archaeology} shows that the involvement of cultural heritage in the health care sector can have positive and promising outcomes. Notably the aspect of active engagement, closely linked to well-being, proved to be an important outcome, was regarded as highly positive by the participants \parencite[235--236]{AnderE_2013}. For this reason, it is particularly interesting to look further into the potential of fields within the cultural heritage sector that could facilitate high levels of engagement. One such field is archaeology. The discipline of archaeology has a strong empirical character in addition to a vast theoretical foundation, which may make it a perfect instrument for the activation of both a physical and an intellectual level of engagement.	
	
%	\section{Archaeo-appeal}
	
	Archaeology\marginnote{Archaeo-appeal} has over the decades had a strong presence within Western popular culture. While professional archaeologists do not always seem thrilled with popular representations of archaeology and archaeologists, the enormous popularity of fictional accounts such as the Indiana Jones franchise, illustrates that archaeology appeals strongly to the general public. The immense popularity and notoriety that archaeology has garnered within pop culture and the general public, is largely due to such representations and stories, based on early adventurer-archaeologists, whose impressive finds have been widely reported in the Western media. Instead of condemning popular representations for their possible inaccuracies from a scientific point of view, it could prove more valuable to look deeper into the associations and meanings archaeology has gathered through pop culture, and to rather see how these attributes could be capitalised in initiatives of social relevance.	
	
	According to Cornelius Holtorf, archaeology oozes what he refers to as “archaeo-appeal”, a certain magic which  is conveyed through the experience of archaeological practice and the imagining of the past \parencite[156]{Holtorf_2005}. The association of “magic” is established and reinforced through highly romanticised representations of archaeology as treasure hunting, and the discovery of ancient, mysterious and sacred objects. As \textcite[156]{Holtorf_2005} notes, within the public eye archaeology has come to be associated with motifs that are very popular in pop culture in general. Archaeology embodies and combines several of these motifs, such as the use of advanced technology, unique discovery, nostalgia for ancient worlds, and exotic locations. \textcite[157]{Holtorf_2005} argues that participation, or a simulated participation, in archaeology could be a powerful experience that can be both entertaining and educational. Within this experience, the public could live a range of these preconceived metaphors and engage in the experience, or the dream, of investigating the past in order to come closer to it in the present \parencite[156--157]{Holtorf_2005}. The latter element seems not only essential to professional archaeologists but also to the general public.	
	
	It is fundamental to consider archaeology in the context of pop culture, as the majority of the general public, who are not qualified as professional archaeologists, build their conceptions of archaeology from this imagery. In order to reach these audiences successfully, it is important to keep this cultural frame of reference in mind. Furthermore, this imagery could prove crucial for audiences such as persons affected by dementia and especially those dealing with associated negative cognitive effects such as boredom, under-stimulation and depression, since archaeology seems to possess a certain “magic” evoking wonder and amazement. These two themes are also apparent in the coded data of the “Heritage in Hospitals” initiative \parencite[235]{AnderE_2013}.						 
	
%	\section{The archaeological imagination}
	
	The\marginnote{The archaeological imagination} notion of metaphors associated with archaeology is important. Michael Shanks, an archaeologist who is engaged in exploring archaeology’s role in modern society, states that metaphors are widely used to represent conceptions of the work of archaeologists and to shape perceptions of the past \parencite[25, 64]{Shanks_2012a}. 
Like \citeauthor{Holtorf_2005}, \citeauthor{Shanks_2012a} observes the resonation of metaphors in the modern representation of “the” archaeologist, who he describes as a romantic figure, engaged in digging and discovering what was lost and forgotten and piecing together the clues that tell us about the past of distant civilizations. He views such metaphors as expressions of the so-called “archaeological imagination”, which he refers to as a creative impulse which exists at the heart of archaeology and the meanings that are attributed to archaeology through cultural reception \parencite[25]{Shanks_2012a}. This imagination includes a range of attitudes towards remains, traces, memory, time and history. It is through the archaeological imagination that fragments of the past can be “reanimated”; the life-world behind ruins can be recreated and the people behind artefacts can spring to life once more \parencite[9, 25]{Shanks_2012a}.
	
	\textcite[17]{Shanks_2012a} considers this type of imagination in an unrestricted manner: not only scientific engagements with the past operate through this imaginative faculty, but so do museums, re-enactors at fairs and non-professional archaeology enthusiasts. For him, archaeological practice is not restricted to science, as it contains this highly creative component that encompasses experiential and subjective aspects of human experience \parencite[17]{Shanks_2012a}. \textcite[17--18]{Shanks_2012a} argues that archaeology is best defined simply as working on remains of the past, which allows a myriad of engagements to be viewed as archaeology. This notion of imagination is a creative understanding of archaeology, which also takes into account creative understandings of life today, and the possibilities of change and innovation. In this understanding, archaeology is viewed as truly creative: through the aspect of imagination, interventions within individual realities can be made, connecting perceptions and cultural imagery with experiences to enhance human life. Such an understanding of archaeology truly extends its scope towards all kinds of different spheres, with which archaeology, when understood as a strictly scientific discipline, is not in contact. It opens up exciting possibilities for engagements with archaeology outside of academic settings, such as in health care contexts.
	 		
%	\section{Archaeology as an enjoyable activity}
	
	It has\marginnote{Archaeology as an enjoyable activity} been noted that persons affected by mild to moderate dementia can strongly benefit from so-called Cognitive Stimulation Therapy. This type of therapy consists of interventions that offer enjoyable activities aimed at the stimulation of several cognitive faculties such as thinking, concentration, and memory. These activities usually take place in a social setting of small groups \parencite{Woods_2012}. Enjoyable activities are usually defined as meaningful activities, which have a deeper impact than activities that are primarily focused on passing time. Many such programs have been found to improve moods and to reduce so-called “disruptive behaviour” and thus positively impact quality of life \parencite[124]{Teri_1992}. Therefore, engagement in enjoyable activities is extremely important for effective therapy. 
	
Progressing dementia can result in the inability of a person to perform certain activities and pursue certain interests. Activities that were once highly enjoyable can turn into frustrating experiences. This does not mean however that engagement in activities cannot be achieved at all, especially in the context of mild to moderate dementia. What it does mean is that activities must be suitable and take this context into account. Therefore, there is a strong emphasis on activities that provide meaning and which are enjoyable and rewarding. When creating activities that aim to be enjoyable for people with dementia, there are no fixed criteria. However, activities that are currently offered tend to focus on creating meaning in the present by building on past interests and activities \parencite{Association_2015}. As a result of the person-centred approach that is taken to dementia care, activities that relate to past interests are encouraged. The underlying idea is that persons with dementia are more likely to be engaged, and thus interested, in activities they previously practiced. It is however safe to assume that the majority of these persons are neither professional nor amateur archaeology enthusiasts. 		
	
	But this is precisely where popular culture comes in: while archaeology may not have been actively practiced as a profession, hobby or interest by participants in the past, it is highly likely that they have been introduced to archaeology by pop culture. Furthermore, the character of these associations is arguably connected to powerful sensations of amazement, which could make archaeology particularly well-suited as an activity that could offer an “out of the ordinary” experience built on vivid imagery. In fact, it could even be an asset if participants were not previously active within archaeology, as this could increase factors of novelty and discovery, which could reinforce a feeling of excitement. In addition to this, archaeology features a set of characteristics which could provide more meaningful engagements. 	
	
%	\section {Characteristics of archaeology}
	
	Archaeology\marginnote{Characteristics of archaeology} engages with the past. In doing so, it combines narrative, expressed through interpretations of the past and story-telling, with a physical and active component, which can be experienced by participating in excavations and through the tactile dimension of handling finds. Moreover, archaeology is a social practice. These characteristic elements of archaeology form an excellent basis for activities specifically designed for persons with dementia.	
		
	Excavating can be a strenuous physical activity. As such, it could offer the same benefits as physical activities and different forms of exercise that are currently commonly offered in dementia care. The Alzheimer’s Society (2015) states that physical activity as part of a care program can have a positive impact on well-being, both physically and mentally. Activities can be characterized as “physical” if they cause an increased heart rate and deeper breathing. This includes sport activities, but also more day-to-day activities such as gardening. Aside from the benefits that are generally acquired through physical activities (such as a reduced risk for heart disease and high blood pressure, improved bone strength and suppleness), people with dementia could also strongly benefit from cognitive improvement that is associated with physical activity. Improving sleep could also be highly beneficial, as would other associated benefits such as the improvement of confidence, self-esteem and mood \parencite{Association_2015}.
	
	Depending on the physical and cognitive condition of each individual, there are multiple activities that could engage people in the practice of an archaeological excavation.  There are excavations that let the public participate in actual excavations, or that have a special place on-site where simulated excavation can take place. Some recent examples in The Netherlands are the “Nijmegen graaft!” (Nijmegen digs!) project that took place in 2014 \parencite{Unknown_2014} and the field days open to public participation organized by the municipality Oss in 2013 \parencite{Unknown_2013}. The project in Nijmegen included special workshops for children to introduce them to archaeological work and to allow them to make a contribution to the excavation. Adults were also invited to work alongside archaeologists. In Oss, the municipality invited local residents of Horzak to participate in an excavation taking place in their neighbourhood. These projects show that involving members of the general public in archaeological excavations is feasible and are moreover well-received by the public. The project in Nijmegen was declared a great success by the organizers and the excavation site in Oss was so well-visited that it led to an exhibition about public participation in the city hall of Oss.
	
	However, a sudden change of environment and the physical displacement needed to get to the site of excavation might make the participation in real field digs less suitable for those in the later stages of dementia. But despite these constraints, it is still possible to be actively involved with archaeology. In activity programs designed for children, archaeologists often make use of sandboxes to create simulated digs. Sandboxes can be places outdoors or indoors (with a few additional precautions). Moreover, it is also possible to execute the same idea, but on a smaller scale, in terraria for instance, as the “Tales from the Sea” project illustrates \parencite{Cutler_2013}. This project, initiated by Bournemouth University Dementia Institute, involved public participants by letting them dig for artefacts in terraria filled with sand. Additionally, sensory stimulation was also encouraged by object handling sessions, and by incorporating sounds and smells reminiscent of the sea into the sessions. Tactile stimulation is considered a form of cognitive development, and has been connected to improvements regarding general mood \parencite[471--472, 474--475]{Baker_2003}. Participating in simulated digs reinforced tactile stimulation, by letting people sift through the sand and uncover objects with different textures, materials, shapes and sizes which could be touched (such as sand, pottery, beads). Drawing, or taking pictures, of such objects could possibly further reinforce this stimulation.
	
	An added feature of excavations is their social character, and \textcite[271, 278, 281]{Kitwood_1992} stressed the importance of social interaction in the context of dementia. Excavations take place in team settings, where people work together and collaborate. This would translate particularly well in an activity directed at people with dementia, who often benefit from group activities since social contact is an extremely important aspect of quality of life and an individual’s wellbeing, and is often lacking as social networks tend to decrease as the condition progresses.  The activities linked to excavation are also diverse enough (e.g., digging, cleaning, photographing and drawing finds) to allow collaborative “teamwork” and offer possibilities of engagement tailored to different interests. Furthermore, the interpretation and contextualization of finds adds a level of intellectual engagement in which no physical activity is necessary and might be preferred by some participants. These elements make archaeology an activity that is suitable for a wide range of individuals with diverse interests and preferences.
	
	Moreover, archaeology’s emphasis on human life in the past may facilitate reminiscence. Reminiscence is a term used to denote the act of recollecting past experiences and events \parencites{Society_2015a}{Society_2015b}. 
	Reminiscence as an activity, or therapy, is frequently used in dementia care, as it emphasizes what a person still can do (remembering past events, for instance) instead of stressing the loss of cognitive functions. Furthermore, reminiscing on past activities and events that shaped individuals’ identities can have an affirmative and reassuring function for people with dementia \parencite{Unknown_2015}. For example, archaeology’s orientation on the past might trigger recollections of personal pasts, former interests in history in school settings or of visits to archaeological landmarks. Another important dimension is the strong connection archaeology has to the social topics it studies, such as the daily-life activities of past peoples. Handling finds that bear witness to daily life activities in the past, such as cooking pots and utensils, could possibly be ‘triggers’ for the personal past and memories, which in the context of dementia could be highly valuable, since it might enhance an individual’s enjoyment of an archaeology-based activity.	
	
%	\section{Discussion}
	
	I \marginnote{Discussion} have argued that archaeology could potentially be a strong contributor to dementia care. I specifically discussed the possibility of engaging in, and with, archaeology through activities that are aimed at being experienced as enjoyable and meaningful. Many potential benefits for participants with dementia have been listed, but it is also important to emphasize what can be gained from an involvement with the health care sector, or dementia care in particular, from an archaeological approach. The key is the aforementioned social potential attributed to this field. Shanks sums up a part of what the realization of this potential could signify when saying:
	
\begin{quote}
Because I care, and I believe that many others share such a care and concern to identify and facilitate the creation of experiences of the past and the present that make life richer. We can bear witness to lost and forgotten pasts. We can facilitate many more people’s creative involvement in making pasts their own.\footnote{\cite[19]{Shanks_2012b}.}
	\end{quote}				
	Given this potential to enrich the lives of people by bringing them in ‘contact’ with the past, an involvement in health care seems particularly fitting. By tapping into this, a powerful statement in terms of the social relevance of archaeology could be made. While a significant part of archaeology’s relevance for society lies in the advancement of knowledge and understanding of the past and its peoples, it is crucial to realize that the discipline could effectively make an impact on individuals’ lives in terms of improving well-being. This aspect is important to highlight in debates on archaeology’s social relevance, and it is worthwhile to look further into this potential asset. 	
		
	In such a context, it is equally important to acknowledge a fuller scope of what archaeology is and can be, for people engaging with it. As Shanks and Holtorf illustrate in their discussion of archaeological imagery, archaeology has in a way become a sort of public property and the public is increasingly engaging with archaeology as the public outreach of the field is growing. As Shanks put it: in a sense, we are all archaeologists now, and it is restrictive to understand archaeology solely as a scientific discipline \parencite[41--42]{Shanks_2012a}. Such an understanding of archaeology implies a harmonization between popular culture and science. This equally means that creative engagements with archaeology will play a bigger role in the discipline, but this is not necessarily damaging to the credibility of archaeology nor does it necessarily imply a “Disneyization” \parencite[139,157]{Holtorf_2005} of the field. Instead, the scope of archaeology could be enlarged and the field could gain a deeper embeddedness in modern day society. Nonetheless, this does not mean that all levels of archaeological engagement should be mingled. Rather, it means that the definition of engaging in archaeology could be extended and that both sides of the spectrum are not mutually exclusive; archaeology may be a scientific study of the past to some, and an enjoyable activity to others, but both groups engage in archaeology.
	
	Returning to the European Commission’s statement on cultural heritage as being capable of improving people’s lives \parencite[5]{Commission_2015}, I would argue that there are indeed indications that support this claim. In the future, it will be essential to look further into the links between cultural heritage, archaeology and well-being, in order to back up this claim with further empirical data exploring the ways engagements in cultural heritage and archaeology may correlate with an improvement in quality of life.  

	\myseparator
\begin{myabstract}
	\foreignlanguage{dutch}{
	Dementie\marginnote{Abstract (Netherlandic)} komt vaak voor bij ouderen in Europa, en naar verwachting zal het aantal gevallen van dementie sterk stijgen in de komende jaren. Er is tot op heden met name onderzoek gedaan naar dementie vanuit een neurologisch en medisch uitgangspunt, hoewel dementie ook ernstige psychologische impact en sociale consequenties heeft voor individuen.  Inzichten in de psychosociale implicaties van dit syndroom, en de gevolgen voor welzijn en levenskwaliteit, ontstonden in de afgelopen twee decennia. Een betrekking van disciplines met een niet-neurologische of medische achtergrond kan een verrijking vormen van de wijze waarop er met dementie, en de effecten hiervan op individuen, wordt omgegaan. Dit paper betoogt dat archeologie in dit opzicht een waardevolle bijdrage kan leveren aan Europese dementie zorg. Het zet een theoretisch argument uiteen dat voortbouwt op eerdere initiatieven omtrent archeologie en erfgoed in de context van gezondheidszorg. Het argument dat ik presenteer benadrukt specifieke kenmerken van archeologie die deze discipline geschikt maken voor een dergelijke bijdrage. Ik concludeer dat een deelname aan archeologische activiteiten een gunstige werking kan hebben op het welzijn van mensen met dementie.
	}
	
	\keywords[Trefwoorden]{Archeologie, erfgoed, gezondheidszorg, dementie, dementie zorg, welzijn, levenskwaliteit.}
	
\end{myabstract}
\printbibliography[heading=subbibnumbered] 
\label{Vonk:lastpage}
%------------------------------------------------------------------------------
\closingarticle
