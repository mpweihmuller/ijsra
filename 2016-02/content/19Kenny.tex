% !TEX
\documentclass[%
	%draft
	]{ijsra}
\def\IJSRAidentifier{\currfilebase} %<---- don’t change this!
%-------Title | Email | Keywords | Abstract-------------
\def\shorttitle{IQUA: Symposium Review}
\def\maintitle{Irish Quaternary Association (IQUA)\\ Symposium Review}
\def\cmail{patricia.kenny@ucdconnect.ie}
\def\keywords{Conference, IQUA, Irish Quaternary Association, Prehistory, Review}
%\def\keywordname{}%<--- redefine the name “Keywords“ in needed language
\def\abstract{}
%--------Author’s names------------
\def\authorone{Patricia Kenny}
%-------Biographical information-------------
\def\bioone{Patricia Kenny graduated from University College Dublin with a BSc in Archaeology and Geology in 2015. She has since graduated with an MA in Archaeology, also at UCD. Her research interests include Irish prehistory, Neolithic Europe, the study of megalithic monuments and geoarchaeology. Patricia is currently working at UCD School of Archaeology as a tutor, and is planning upon starting a PhD within the next year, examining the structural stones of passage tombs in Atlantic Europe.}
%------University/Institution--------------
\def\affilone{University College Dublin}

%\begin{filecontents}{\IJSRAidentifier.bib}
%Bibliography-data HERE
%\end{filecontents}
\begin{document}
\IJSRAopening%<---- don’t change this!
%-------
\lettrine{T}{he} 
%\IJSRAsection{small headline}
%\IJSRAseparator
Irish Quaternary Association held its autumn symposium in Beggars Bush, Dublin, on the 24th of November 2016. The title of the symposium was “Early Human Occupation of Ireland”, and the speakers covered a variety of topics relating to human occupation of Ireland from the Palaeolithic through to the Neolithic. There were eight speakers scheduled to speak; however, Dr. Ruth Carden was unfortunately unable to attend. The event provided an opportunity for researchers to discuss theories and models of human occupation of Ireland, and to explore areas that have potential for further research.

The event began with a speech from keynote speaker, Dr. Richard Jennings from Liverpool John Moores University. He reviewed the evidence for the potential of human occupation of Ireland in the Palaeolithic, providing a context for the famed bear patella from Alice and Gwendoline Cave, Co. Clare. This patella, discovered in 1902-3, was examined by Dowd and Carden in 2016. They discovered cut marks on the bone (dated to 12’800-12’600 BP), suggesting that human occupation of Ireland started much earlier than previously believed. This topic has always been a favourite of mine, and I was eager to hear what Jennings had to say.
Jennings first set the scene, discussing human occupation of Europe and Britain in the Palaeolithic, the fluctuating climate, and the growth and contraction of ice sheets and land bridges. He described the multiple occupations of Britain throughout the Palaeolithic, painting a stark picture of the harsh weather conditions endured by humans and related species. He finished this review by describing a group known as the “reindeer hunters”, characterised by their tools and preferred prey. It was these people, he argued, who may have had the opportunity to enter Ireland. 

Jennings then turned his attention to Ireland, specifically to a site in Co. Waterford, Ballynamintra Cave. Jennings has been conducting excavations at this site since 2014, searching for undisturbed Pleistocene sediments. In addition to this, working with Dr. Frank McDermott from University College Dublin, he has found and is examining speleothems which preserve climatic indicators within them. Using these records, he hopes to be able to build up a record of the climate in Ireland in the Palaeolithic, allowing an insight into whether humans could have lived here. 

Although Jennings has not yet found any concrete evidence for humans settling in Ireland in the Palaeolithic, the climate record that he is building up is invaluable, and it seems that this area has a lot of potential. His research provided a stimulating start to the day, and set the bar high for subsequent speakers. 

The next speaker, Prof. Daniel Bradley from Trinity College Dublin, was discussing an entirely different topic – ancient DNA analysis of Irish prehistoric populations. I have to admit; I was a little wary of this topic, as I know very little about it and expected that I would be quite lost. However, Bradley described the analysis and how it works in very clear terms, leaving no room for confusion. He explained how genetic researchers discover where prehistoric people came from, as well as describing the incredible advances made in this field in the last couple of decades. 

He then gave some interesting examples of how genetics can be used in studying the past, discussing the genetic evidence for migration of people to Ireland in the Neolithic and Bronze Age. He also described how genetic traits, such as hair colour, eye colour and lactose tolerance, can be reconstructed from ancient DNA. While the presentation itself was brilliant, the order of the speakers seemed a little unusual – Bradley’s research would have made more sense later in the day. 
After a brief coffee break, Dr. Ruth Carden was scheduled to speak. However, she was unable to attend, and so the break was followed by a review of the Irish flora during the Mesolithic by Dr. Bettina Stefanini. 

Stefanini’s review was an interesting discussion of which plants may have been exploited in the Irish Mesolithic. This presentation had the potential to become a mere literature review; however, Stefanini expertly reviewed literature and archaeological evidence for Mesolithic flora while adding her own observations and ideas. She drew on a variety of disciplines, including ethnographic accounts of plant use, particularly the historical use of plants in Ireland. Stefanini concluded by suggesting that Mesolithic people were likely to have relied upon a variety of plants, which may not survive in the archaeological record. She called for a more detailed analysis of the possible uses of plant material in the Mesolithic. 

The next speaker, Dr. Robin Edwards of Trinity College Dublin, took the conference in an entirely different direction. His research focussed upon models of land bridges between Ireland, Britain and Europe over time. While his presentation was interesting, it would have been more appropriate to have it directly after the keynote. The two topics were closely related, and discussed the same time period. Interestingly, the model put forward by Edwards conflicted with the model discussed by Jennings, which could have led to an interesting discussion. 

Edwards’ presentation covered the history of the field, describing how it has changed and developed over time, reviewed the problems with current models and finally putting forward the most recent model of land bridge development. Like Stefanini and Jennings, he ended by calling for more research into the area – a theme which was reflected across many of the presentations. 

The next presentation brought us back to the Mesolithic, which highlighted the flaws in the order of speakers. Dr. Kieran Westley from Ulster University, discussed the presence of submerged Mesolithic landscapes, how to find them and their potential for future investigations. He has used geophysical techniques to create images of the strata found below the seafloor and has successfully located several submerged landscapes. He then outlined how he conducts coring and excavation on the sites. This presentation was very interesting and is clearly an area with a lot of potential. However, it struck me that Westley did not seem to be taking full advantage of the sites. He did not mention any environmental analysis, which would be extremely beneficial in those landscapes. Like those before him, he ended by stressing the potential of this field for further research. 

This discussion of submerged landscapes was followed by Dr. Graeme Warren from University College Dublin. He reviewed current knowledge of the Irish Mesolithic and how it relates to its European counterparts. Once again, the focus of his presentation was highlighting areas which could be investigated in more detail, as well as drawing together interesting pieces of evidence which seem to have been overlooked thus far. He discussed the similarities and differences between the Irish and European Mesolithic, the likelihood of a shamanistic or animistic society in Mesolithic Ireland and the potential for niche construction and deliberate landscape modification in the Mesolithic. Warren’s presentation convincingly presented the Irish Mesolithic as an area with a lot of research potential, echoing the sentiments put forward by Stefanini and Westley.

Finally, Dr. Jessica Smyth of University College Dublin outlined current knowledge of the Irish Neolithic, drawing upon research she has carried out over the last decade. She reviewed the evidence for settlement in the Irish Neolithic, which seems to be similar to European settlement, although there are some differences. She also discussed current dating of the Neolithic, and how it has been improved in recent years. She discussed the importance of dairy activities in the Neolithic, revealed by her analysis of the lipid remains in Irish pottery. She suggests that the importance of dairy use in the Neolithic may have been a result of social norms, as many Neolithic people were lactose intolerant. Smyth expertly demonstrated how scientific analysis can be used to interpret possible social choices in prehistory. The presentation was an interesting snapshot of current understanding of the Irish Neolithic. However, once again I must wonder at the order of the speakers – Bradley’s discussion of genetics would have tied in with Smyth’s presentation very well. 
The final presentation was made by Thomas Deane, press officer for the faculty of engineering, mathematics and science from Trinity College Dublin. He discussed the ways in which scientists and researchers can contact the media. It was an interesting and practical end to the day, especially considering how important the media is in representing archaeology to the public. 

The IQUA symposium was a fantastic event, with interesting speakers who covered a variety of topics, techniques and specialities. I left with a head full of new information and ideas, and it is safe to say that I thoroughly enjoyed the event. If I were to suggest any improvements, I would suggest that they consider the order of speakers more carefully. It would have made more sense to me to have Dr. Robin Edwards earlier in the day, and Dr. Daniel Bradley towards the end. However, these are simply cosmetic issues, and overall, I found the symposium to be thought provoking, well organised and enjoyable. IQUA is hosting the International Quarternary Association (INQUA) Conference in Dublin in 2019, and based upon the 2016 symposium, it is guaranteed to be a well organised and enjoyable event. 

\IJSRAclosing%<---- don’t change this!
\end{document}