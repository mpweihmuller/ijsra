\documentclass[%
	%draft
	]{ijsra}
\def\IJSRAidentifier{\currfilebase} %<---- don’t change this!
%-------Title | Email | Keywords | Abstract-------------
\def\shorttitle{A History of Central Mexican Archaeology}
\def\maintitle{From Colonialism to Nationalism, the Indian to \emph{Indigenismo}:\\ 
A History of Central Mexican Archaeology}
\def\cmail{John@Doe.com}
\def\keywords{Mexico, Mexican archaeology, colonialism, nationalism, racism, \emph{indigenismo}, Indian, indigenous, assimilation, \emph{mestizo}, tourist archaeology}
%\def\keywordname{}%<--- redefine the name “Keywords“ in needed language
\def\abstract{Most published work on the history of Mexican archaeology has been in Spanish, and most publications in article format have focussed on developments during or since the 1900s. This paper presents a concise, English summary of the origins and development of the practice of archaeology alongside the changing political landscape of Mexico through from the 1500s to the present. This article explores: the colonial concept of the “Indian” and the initial conversion effort of the 1500s; the birth of antiquarianism and archaeology in the 1700s and 1800s respectively; the establishment of archaeology as a discipline with Manuel Gamio; the Revolution of the 1910s and the \emph{indigenismo} movement; and modern controversies around tourist archaeology, the status of living indigenous peoples in Mexico, and the relationship between archaeology and the Mexican government. “Archaeology” is defined here as the study of the (human) past through material remains, an “archaeologist” as one who practices archaeology, a “practice” as the application of certain ideas and methods, and a “discipline” as a branch of knowledge taught in a school system. It is argued that archaeology in central Mexico (the practice and discipline) developed alongside Mexican nationalism during periods when the pre-Hispanic past and particularly the Aztec empire was viewed positively, and thus developed in tandem with \emph{indigenismo} and the notion of Mexico as a \emph{mestizo nation}.}
%--------Author’s names------------
\def\authorone{Jon Doe}
\def\authortwo{Jon Hancock}%<---- comment or delete if you do not need a second author.
%\def\authorthree{}%<---- comment or delete if you do not need a third author.
%\def\authorfour{}%<---- comment or delete if you do not need a fourth author.
%\def\authorfive{}%<---- comment or delete if you do not need a fifth author.
%-------Biographical information-------------
\def\bioone{Jon Doe is doing his research about ...}
\def\biotwo{John Hancok on the other is doing something about ....}%<---- comment or delete if there is no second author.
%\def\biothree{}%<---- comment or delete if there is no third author.
%\def\biofour{}%<---- comment or delete if there is no fourth author.
%\def\biofive{}%<---- comment or delete if there is no fifth author.
%------University/Institution--------------
\def\affilone{Jon Doe’s university or institution}
\def\affiltwo{Jon Hancocks insitution}%<---- comment or delete if there is no second author.
%\def\affilthree{}%<---- comment or delete if there is no third author.
%\def\affilfour{}%<---- comment or delete if there is no fourth author.
%\def\affilfive{}%<---- comment or delete if there is no fifth author.
%--------Mapping of authors to affiliations------------
%% authorone:--> * <--- copy/paste that symbol to \affiloneauthor etc. below
%% authortwo:--> † <--- copy/paste that symbol to \affiloneauthor etc. below
%% authorthree:--> ‡ <--- copy/paste that symbol to \affiloneauthor etc. below
%% authorfour: --> § <--- copy/paste that symbol to \affiloneauthor etc. below
%% authorfive: --> ¶ <--- copy/paste that symbol to \affiloneauthor etc. below
%-------------------------------------------------------------------------
\def\affiloneauthor{*}%<---- paste the symbol of the authors into {}
\def\affiltwoauthor{†}%<---- paste the symbol of the authors into {}
%\def\affilthreeauthor{}%<---- paste the symbol of the authors into {}
%\def\affilfourauthor{}%<---- paste the symbol of the authors into {}
%\def\affilfiveauthor{}%<---- paste the symbol of the authors into {}

\begin{filecontents}{\IJSRAidentifier.bib}
@book{Bernal1980,
location =  {London},
author = {Bernal, I.},
publisher = {Thames and Hudson},
title = {A History of Mexican Archaeology: The Vanished Civilizations of Middle America},
year = {1980},
}


@article{Bonfil1977,
author = {Bonfil, B. G.},
journaltitle =  {Boletín Bibliográfico De Antropología Americana},
number = {48},
pages = {17-32},
title = {El Concepto De Indio En America: Una Categoria De La Situacion Colonial},
volume = {39},
year = {1977},
}


@book{Bueno2016,
location =  {Albuquerque, N.M.},
author = {Bueno, C.},
publisher = {University of New Mexico Press},
title = {The Pursuit of Ruins: Archaeology, History and the Making of Modern Mexico},
year = {2016},
}


@article{Caballero2008,
author = {Caballero López, P.},
journaltitle =  {Social Anthropology},
number = {3},
pages = {329-345},
title = {Which heritage for which heirs? The pre-Columbian past and the colonial legacy in the national history of Mexico},
volume = {16},
year = {2008},
}


@book{Cortes1971,
location =  {New York},
author = {Cortés, H.},
edition = {A. R. Pagden},
publisher = {Grossman Publishers},
title = {Letters from Mexico},
year = {1971},
}


@book{Diaz2005,
location =  {Tucson},
author = {Díaz Balsera, V.},
publisher = {The University of Arizona Press},
title = {The Pyramid under the Cross: Franciscan Discourses of Evangelization and the Nahua Christian Subject in Sixteenth-century Mexico},
year = {2005},
}


@book{Diaz1963,
location =  {Baltimore},
author = {Díaz del Castillo, B.},
edition = {J. M. Cohen},
publisher = {Penguin Books},
title = {The conquest of New Spain},
year = {1963},
}


@book{Escalante2013,
location =  {México},
author = {Escalante, P. G.},
publisher = {Fondo de Cultura Económica},
title = {Los códices mesoamericanos antes y después de la conquista española: historia de un lenguaje pictográfico},
year = {2013},
}


@article{Fowler1987,
author = {Fowler, D. D.},
journaltitle =  {American Antiquity},
number = {2},
pages = {229-248},
title = {Uses of the Past: Archaeology in the Service of the State},
volume = {52},
year = {1987},
}


@article{Gamio1942,
author = {Gamio, M.},
journaltitle =  {Boletín Bibliográfico De Antropología Americana},
number = {1/3},
pages = {35-42},
title = {Franz Boas en México},
volume = {6},
year = {1942},
}


@article{Gandara1981,
author = {Gándara, V. M.},
journaltitle =  {Boletín De Antropología Americana},
pages = {7-70},
title = {La vieja \enquote{nueva arqueología} (segunda parte)},
volume = {3},
year = {1981},
}


@article{Gandara1987,
author = {Gándara, V. M.},
journaltitle =  {Boletín De Antropología Americana},
pages = {5-13},
title = {Hacia una teoría de la observación en arqueología},
volume = {15},
year = {1987},
}


@techreport{Giles2015,
author = {Giles, R. V.},
title = {Destruyen restos arqueológicos para construir autopista en Morelos. [Website]},
year = {2015},
url = {http://www.cronica.com.mx/notas/2015/911229.html},
comment = {Accessed February 1st, 2016},
}


@article{Gomez2007,
author = {Gomez Goyzueta, F.},
journaltitle =  {Cuicuilco: Revista de la Escuela Nacional de Antropología e Historia},
number = {41},
pages = {217-242},
title = {Análisis del desarrollo disciplinar de la Arqueología Mexicana y su relación con el Patrimonio Arqueológico en la actualidad},
volume = {14},
year = {2007},
}


@article{Ortiz1996,
author = {Ortiz Elizondo, H. and Hernández Castillo, A.},
journaltitle =  {Political and Legal Anthropology Review (PoLAR)},
number = {1},
pages = {59-66},
title = {Constitutional Amendments and New Imaginings of the Nation: Legal Anthropological and Gendered Perspectives on \enquote{Multicultural Mexico}},
volume = {19},
year = {1996},
}


@article{Hernandez2002,
author = {Hernandez Castillo, R. A.},
journaltitle =  {Political and Legal Anthropology Review (PoLAR)},
number = {1},
pages = {90-109},
title = {Indigenous Law and Identity Politics in Mexico: Indigenous Men's and Women's Struggles for a Multicultural Nation},
volume = {25},
year = {2002},
}


@book{Howell2001,
location =  {Ithaca, New York},
author = {Howell, M. C and W. Prevenier},
publisher = {Cornell University Press},
title = {From Reliable Sources: An Introduction to Historical Methods},
year = {2001},
}


@book{Leon2002,
location =  {Norman},
author = {León Portilla, M.},
editor = {translated by Mauricio J. Mixco},
publisher = {The University of Oklahoma Press},
title = {Bernardino de Sahagún: First Anthropologist},
year = {2002},
}


@book{Leon1996,
location =  {Leiden, Netherlands},
author = {León, L. V.},
publisher = {Centre of Non-Western Studies},
title = {El Leviatán arqueológico: antropología de una tradición científica en México},
year = {1996},
}


@techreport{Lobjois2013,
author = {Lobjois, B. M. S.},
title = {La arqueología mexicana en tiempos porfirianos y revolucionarios},
year = {2013},
}


@incollection{Lopez2010,
location =  {Mexico City},
author = {López Aguilar, F.},
booktitle = {Patrimonio, identidad y complejidad social: enfoques interdisciplinarios},
pages = {395-411},
publisher = {Escuela Nacional de Antropología e Historia},
title = {La arqueología Mexicana: Reflexiones sobre la ética y las prácticas académicas},
year = {2010},
}


@incollection{Lopez2001,
location = {Mexico City},
author = {López Luján, L.},
booktitle = {Descubridores del pasado en Mesoamérica},
publisher = {Antiguo Colegio de San Ildefonso},
year = {2001},
}


@incollection{Lopez2006,
location =  {Mexico City},
author = {López Luján, L.},
booktitle = {Carl Nebel: Pintor viajero del siglo XIX},
editor = {A. Aguilar},
pages = {20-33},
publisher = {Editorial Artes de México},
title = {La arqueología mesoamericana en la obra de Nebel},
year = {2006},
}


@article{Matos1979,
author = {Matos Moctezuma, E.},
journaltitle =  {Nueva Antropología-Revista de Ciencias Sociales},
pages = {7-25},
title = {Las corrientes arqueológicas en México},
volume = {12},
year = {1979},
}


@incollection{Matos1995,
author = {Matos Moctezuma, E.},
booktitle = {The Great Temple of Tenochtitlan: center and periphery in the Aztec world},
editor = {J. Broda and D. Carrasco and E. Matos Moctezuma},
pages = {15-60},
publisher = {University of California Press},
location={Berkeley},
title = {The Templo Mayor of Tenochtitlan: History and Interpretation},
year = {1995},
}


@incollection{Patterson1995,
location =  {Santa Fe, New Mexico},
author = {Patterson, T. C.},
booktitle = {Making Alternative Histories: The Practice of Archaeology and History in Non-Western Settings},
editor = {P. R. Schmidt and T. C. Patterson},
pages = {69-84},
publisher = {School of American Research Press},
title = {Archaeology, History, Indigenismo, and the State in Peru and Mexico},
year = {1995},
}


@incollection{Sanders2000,
author = {Sanders, W. T.},
booktitle = {Patrimonio y conservación arqueológica},
editor = {C. Paillés María and H. Gómez Rueda and N. Castillo Tejero},
pages = {29-34},
publisher = {Inah-Conaculta},
title = {Cultural Heritage: Redefinition and Reevaluation},
year = {2000},
}


@book{Skidmore1989,
location =  {Oxford},
author = {Skidmore, T. E. and P. H. Smith},
publisher = {Oxford University Press},
title = {Modern Latin America, 2nd ed},
year = {1989},
}


@book{Swarthout2004,
location =  {New York},
author = {Swarthout, K. R.},
publisher = {Peter Lang Publishing Inc},
title = {Assimilating the Primitive: Parallel Dialogues on Racial Miscegenation in Revolutionary Mexico},
year = {2004},
}


@book{Todorov1984,
location =  {New York},
author = {Todorov, T.},
publisher = {Harper and Row Publishers},
title = {The Conquest of America: The Question of the Other, translated by Richard Howard},
year = {1984},
}


@book{Trigger2006,
location =  {New York},
author = {Trigger, B.},
publisher = {Cambridge University Press},
title = {A History of Archaeological Thought},
edition = {2},
year = {2006},
}


@incollection{Villela2010a,
location =  {Los Angeles},
author = {Villela, K. D.},
booktitle = {The Aztec Calendar Stone},
editor = {K. D. Villela and M. E. Miller,},
pages = {50-57},
publisher = {Getty Publications},
title = {Editor's Commentary},
year = {2010},
}


@incollection{Villela2010b,
location =  {Los Angeles},
author = {Villela, K. D.},
booktitle = {The Aztec Calendar Stone},
editor = {K. D. Villela and M. E. Miller,},
pages = {81-82},
publisher = {Getty Publications},
title = {Editor's Commentary},
year = {2010},
}


@article{Zermeno2002,
author = {Zermeño, G.},
journaltitle =  {Nepantla: Views from South},
number = {2},
pages = {315-331},
title = {Between Anthropology and History: Manuel Gamio and Mexican Anthropological Modernity (1916-1935)},
volume = {3},
year = {2002},
}
\end{filecontents}
\IJSRAopening
%-------
\lettrine{N}{ationalism} and the perceived importance of the pre-Hispanic past were closely linked to the development of archaeology in Mexico, and the colony of New Spain that predated it. The study of indigenous populations in what is now central Mexico began immediately after the fall of the Aztec capital of Tenochtitlan in 1521 (the “conquest”) with the work of friars like Bernardino de Sahagún and Bartolomé de Las Casas, with the aim of converting indigenous people to Catholicism. However, it was not until the late 1700s that artifacts and material remains began being studied directly for information about the past with Antonio de León y Gama, and not until the early 1800s, with the first wave of nationalist and indigenist sentiment that glorified the pre-Hispanic past in the years leading up the War of Independence, that archaeology became an established practice in the colony. Archaeology was a tool used to help create a nation with a unique identity out of the colony of New Spain, and the pre-Hispanic past garnered popular interest for the first time for providing Mexico with part of that unique identity. It was not until the Porfiriato (1880–1910) and the work of Leopoldo Batres that archaeological research and restoration work began to receive governmental funding, and not until after the Revolution (1910–1917) and the work of Manuel Gamio that archaeology became a trained and regulated discipline. With the Revolution, Mexico sought to unify itself culturally and socially beneath a proud \emph{mestizo} (mixed-racial of Spanish and indigenous) label, spawning the controversial \emph{indigenismo} movement that both spurred archaeological study of the pre-Hispanic past and encouraged the assimilation of living natives. Archaeology in the 1900s was initially driven by a nationalist pride that favoured “advanced” cultures of Mesoamerica and showy sites for tourism, but increasingly became scientifically, intellectually and theoretically driven, as the rights for living indigenous people also slowly improved.

This paper provides a concise English summary of these key periods and personages in the development of archaeology and its precursors in central Mexico, presenting a unified trajectory that places early colonial research alongside the better-known developments of the 1900s, and explores the relationship between the development of archaeology and the nationalist and indigenist movements of Mexico. 

\IJSRAsection{The 1500s—Christianity and the Noble Savage}

Although there were no antiquarians \footnote{Historians who studied the past through manuscripts and artefacts. Collectors of antiquities.} or archaeologists in Mexico before the 1700s, and although archaeology as an acknowledged practice did not exist till the 1800s, research on the indigenous people of central Mexico began a few short years after the fall of Tenochtitlan, with the arrival of twelve Franciscan missionaries in 1524. 
After Charles I of Spain (Charles V) became ruler of the Holy Roman Empire in 1519, Spain saw herself as the new divinely chosen country for spreading Christianity across the world through political domination, and Hernán Cortés, having given this same holy conversion mission as the purpose and validation for his campaign against the Aztecs, requested that the missionaries be sent \parencites[40]{Cortes1971}[168-169]{Diaz2005}. 
“The Twelve”, their number echoing the twelve apostles of the Bible, came with the intent of stomping out idolatry and sacrifice, and they learned Nahuatl and studied the customs of the Nahuas in order to do so \parencites[4]{Diaz2005}{Swarthout2004}. 

The initial responses to the Nahuas, and particularly the Aztecs or Mexica (the ethnic group of the Aztec capital of Tenochtitlan), were varied. Cortés and his soldier Díaz del Castillo both describe their awe upon entering Tenochtitlan and seeing the Templo Mayor and Moctezuma’s palace, as well as their deference for Mexica manners and accomplishments, yet Cortés recounts the natives’ ignorance of Christianity and their human sacrifices in horror, and calls the Mexica “barbarous” \parencites[35,105-109]{Cortes1971}[216]{Diaz1963}. 
This supposed “contradiction” of Mexica barbarity coupled with their accomplishments is still puzzled over today \parencite[this idea largely lingers in discussions of Aztec sacrifice, ex.][51]{Carrasco1999}, and at the time, Cortés’ views were not unique. 
Generally speaking, the indigenous people of the Americas were a puzzle and a mystery: they were perfect or primitive; cultureless or innocent; beast or child; stupid or gullible; cannibals and sodomites and all of the above \parencite[105]{Restall2003}.

The question of the capacities, value, and place in the world of these indigenous people was debated fiercely in Spain and the New World. The controversy around the natives’ “barbarity” was one with huge implications. At risk was the legitimization, morally and legally, of the invasion of Middle America, as well as the (proper) treatment of the so-called “Indians”—a word that only reinforced their liminal status as people, and has been argued to have reinforced their subjugation as well \parencites[29-30]{Swarthout2004}{Bonfil1977}. 
Myths of humanoid monsters—including cannibals, people with tails and people without heads—had existed in Europe for hundreds of years before Europe discovered North America, and some Europeans argued that the New World indigenous people fit those myths \parencite[103]{Restall2003}. 
They were either inhuman creatures who didn’t possess reason and were created as natural slaves by God, or were half or “imperfectly human” \parencite[150]{Todorov1984} 
because they lacked Christianity and “proper” clothing and weapons \parencite[103]{Restall2003}. 
On the opposite extreme, some argued that the indigenous were innocent and pure like Adam before sin, better suited to a utopia than Spanish society, and the Church argued that the indigenous people only needed guidance and could learn to be Christian \parencites[104]{Restall2003}[31]{Swarthout2004}.

Despite the mixed and often negative views of the natives, the Spanish did have a genuine concern for their welfare because of their interest in exploiting their labour. 
When Bartolomé de Las Casas (ca.1474–1566), a Dominican friar who was controversial for defending the natives and critiquing Cortés and the Spanish, claimed that the indigenous people were dying out because of Spanish brutality\footnote{Disease was actually the main cause for these extreme population declines \parencite[128]{Restall2003}.}, 
the Crown took it seriously \parencite[129]{Restall2003}. 
In 1537, Charles V agreed to Pope Paulo III’s declaration saying the natives were rational human beings with souls who could understand Catholicism and were thus not to be exploited, and in 1542 the New Laws outlawed any sort of Indian slavery \parencite[32]{Swarthout2004}.

That this ruling did not reflect an agreement on the status of the indigenous people is illustrated in the Valladolid Debate of 1550, arranged by the king of Spain between the scholars Juan Ginés de Sepúlveda (1494–1573) and Las Casas. Las Casas defended a positive and romanticized view of the indigenous people, arguing that they were morally sound as well as deserving of respect and compensation for their suffering. By contrast, Sepúlveda maintained that the natives were naturally inferior and were made by God to serve the Spanish, likening them to “wild beasts” who “hardly deserve the name of human beings” \parencite[107,132]{Restall2003}; 
the practice of human sacrifice was his strongest case for their inferiority, and Sepúlveda argued that the primitiveness of the Indians justified the violence of Spanish colonialism \parencites[32-33]{Swarthout2004}[186]{Todorov1984}. 
Like many at the time, Sepúlveda also called all of the indigenous “cannibals” in a circular argument: they were barbaric and were thus cannibals; they were cannibals and thus barbarians \parencite[107]{Restall2003}. 
The contradictory views of New World natives reflected in the Valladolid Debate formed the foundation for the dichotomous concept of the “noble savage” that was coined more than fifty years later \parencite[107]{Restall2003}.

Bartolomé de Las Casas was one of the many friars who studied the natives in an attempt to convert them, and his debate with Sepúlveda reflects the conflict he and some others had between empathizing with the natives and his original mission to study them only to convert them \parencite[201]{Todorov1984}, 
including Bernardino de Sahagún. Sahagún (1499–1590) was a contemporary friar who studied the Nahuas in Mexico much as a modern anthropologist would \parencite[9]{Leon2002}. 
Because texts like Las Casas’ and Sahagún’s portrayed the natives too sympathetically, the Spanish crown increasingly censored native study \parencite[162]{Diaz2005}. 
After 1527, permission was required for any foreigner to access writing on the natives, and after 1556 every book written on them had to be approved by the Council of Indies. Sahagún’s own writings were confiscated in 1577 by Phillip II, after which it was forbidden for writings on Nahua beliefs of culture to be printed or circulated \parencite[161-162]{Diaz2005}. 
Because of these laws, many histories and codices written in the late 1500s or in the 1600s were not published until after Mexican Independence, a key exception being Ixtliloxochitl’s (of Texcoca and Spanish descent) \emph{Relación}, commissioned by the viceroy of New Spain in the early 1600s. 

Amongst the Spanish, Crown censorship stopped the publication and open distribution of such texts even if it did not stop their writing, but it had very little impact on natives. With the “conquest” of 1521, very few indigenous people had been “conquered” by any sense of the word; by contrast, in 1523, Spanish control over even the Aztec capital of Tenochtitlan was precarious and still dependent on their hundreds of thousands of native allies, Spanish campaigns against indigenous people in what was then New Spain and later Mexico were ongoing all the way into the 1900s, and even the natives who were defeated by or allied with the Spanish retained a great deal of cultural and political autonomy \parencite[70-73]{Restall2003}. 
Histories were still being written by and about indigenous people into the 1600s and yet the idea is still prevalent that by the end of the 1500s the pre-Hispanic past (and the cultures that belonged to it) was seen as unimportant, “dead” and “buried forever” \parencite[17]{Matos1995}, 
an idea that is wrong even if referring to only the Spanish, and harmfully silences native voices and native resilience in the centuries following the Spanish “conquest”
\footnote{There are many reasons that the word “conquest” was initially used: by the conquistadores, to emphasize the completeness of their (incomplete) domination and minimize the native (ally) role in that domination in order to secure titles and \emph{encomiendas} from the Crown; by the Church, to exaggerate the swiftness and improbability of defeat in order to show that victory was divinely sanctioned (and morally justified); by the Spanish, to reflect the new and immediate legal status of New Spain and its people as being already the property of Spain even before the Spanish had seen most of it (and thus, noncompliant indigenous people were “traitors” and “rebels” and could be punished accordingly); and now, as a half-accidental product of history, where the self-glorification of conquistador \emph{probanzas} led to the same exaggerations in codices and histories and eventually solidified in Prescott’s popular history, where the “conquest” was an impossible feat of only a handful of exceptional Spanish men led by the hero Cortés, with no mention of other actors or factors \parencite[13,15,68]{Restall2003}. In addition to the word “conquest” being entirely inaccurate other than in referring to the fall of Tenochtitlan, the continuing use of the term “Spanish conquest” today to refer to the Spanish invasion and colonialism in Middle America is problematic for the following reasons: it helps to reinforce racist ideas of native inferiority and Spanish/European superiority; it contributes to misconceptions about the Spanish invasion and the place of indigenous people in the colonial period; and it silences the many indigenous groups who played critical roles in the fall of the Aztec Empire and in later campaigns by giving credit only to the Spanish. This is why the authors avoid referring to the “Spanish Conquest” in this essay, and more often refer to the fall of Tenochtilan in 1521, particular wars or campaigns, or the Spanish invasion of Middle America more broadly.}.

Since most indigenous groups of New Spain were not defeated in the 1500s but were still very much alive and independent, antiquarian or archaeological interest, had it existed, would only have applied to no longer extant cultures and ruins like that of Teotihuacan. However, in the 1500s there was no Spanish interest in studying the cultures of the New World for purposes other than conversion and assimilation. By contrast, the Church did their best to destroy the sites, monuments and manuscripts that pre-dated the conquest in order to wipe out reminders of Nahua beliefs \parencite[36]{Bernal1980}. 
Friar Durán suggested demolishing the newly built Spanish cathedral because it stood on Mexica constructions, and in a letter to his Franciscan order in 1531, Bishop Zumárraga described the destruction of five hundred temples and twenty thousand idols as part of the conversion effort \parencite[36,39]{Bernal1980}. 
Only the Maya were an exception: their architecture was admired from the start \parencite[42]{Bernal1980}. 

\IJSRAsection{The 1700s and 1800s—The Birth of Antiquarianism and Archaeology}

In the late 1600s and early 1700s, interest in the pre-Hispanic past grew for legal and political reasons, propelled both by \emph{criollos} (Americas-born Spaniards) fighting for more power within the Spanish caste system, and land dispute issues between different indigenous groups and with the Spanish, which became increasingly important in the 1700s as populations grew \parencites[49]{Bernal1980}[122]{Restall2003}. 
Land had never been an issue or a goal for the Spanish in their colonial endeavors—their interest lay in wealth, labour and religious conversion, not land or the “wholesale Hispanization” of indigenous peoples \parencite[75]{Restall2003}.

The 1700s also brought the first scholarly (non-indigenous) interest in the pre-Hispanic past, starting with the work of Carlos de Sigüenza y Góngora (1670–1750). A notable figure at the turn of the century, he was a famous \emph{criollo} antiquarian and a chair at the Mexican university. 
Sigüenza was one of the first scholars to see New Spain as a blend of Spaniard and “Indian”, with a past that extended back beyond Cortés’ or even Columbus’ arrival. He expressed these views long before they became widely accepted, and more than half a century before they became tied to the Independence movement \parencite[52]{Bernal1980}. 
Following his conviction that native histories were an important part of New Spain’s, Sigüenza was the first to compile a large number of manuscripts and artefacts of Spanish, colonial indigenous and pre-colonial origins \parencite[50-51]{Bernal1980}, 
including censored texts like Chimalpahin’s annals and Ixtliloxochitl’s \emph{Historia chichimeca} (named by Sigüenza) written in the early to mid-1600s. 

Other than a few descriptions of archaeological sites and some antiquarian collectors like Sigüenza, little in the way of pre-Hispanic research occurred before its sudden increase in the late 1700s, accompanying rising nationalism and Enlightenment thought \parencite[290]{Lopez2001}.
Although the Enlightenment took place earlier in Europe, 
ideas like Locke’s \emph{tabula rassa} and Rousseau’s portrayal of the noble savage contributed to a growing interest in the pre-Hispanic past \parencite[69,74]{Bernal1980}. 
Another trigger for this new wave of research in the late 1700s was the discovery of two Mexica monoliths in the main plaza in Mexico City in 1790: the Sun or Calendar Stone and the statue of Coatlicue \parencite[18]{Matos1995}. 
Antonio de León y Gama (1736–1802) was the first to publish a description of these monoliths in 1792 \parencite[50]{Villela2010a}.

Gama, who has been called the first Mexican archaeologist, was a leading intellectual of the late 1700s and a symbol of the Enlightenment in New Spain \parencites[80]{Bernal1980}[50]{Villela2010a}. 
Gama wrote a history of ancient Mexico (though it was not published because of ongoing censorship \parencite[51]{Villela2010a}), and was the first New Spaniard accredited with directly studying a monument for information about the past, 
recognizing that the \emph{Ollin} glyph on the Calendar stone referenced the four ages of the Mexica past, and correctly ordering the Mexica months \parencites[84]{Bernal1980}[51-52]{Villela2010a}. 
A notable contemporary archaeologist was José Antonio de Alzate y Ramírez (1737–1799), who was the first person to provide descriptions of archaeological sites outside of the main valleys and the Mayan region, and who led one of the first archaeological excavations in New Spain in 1777 \parencites[79-80]{Bernal1980}[290]{Lopez2001}[50]{Villela2010a}.

Gama’s ideas were built on by the traveller Alexander von Humboldt (1759–1859), who visited Mexico on his tour of the Americas. Humboldt published a book describing Mexican archaeological sites and monuments, spreading the use of the word “Aztec” to describe the Mexica ethnic group and the Triple Alliance Empire \parencites[100]{Bernal1980}[19]{Matos1995}[81-82]{Villela2010b}, 
a word based on their ancestors’ homeland of Aztlan. Though he harshly concluded that there was no aesthetic merit in any pre-Hispanic creations \parencite[23]{Lopez2006}, 
after studying Gama’s writings Humboldt did argue that the Mexica had an advanced understanding of math and astronomy and thus could not have been uncivilized savages \parencite[81-82]{Villela2010b}, an idea that persisted from early scholars like Sepúlveda.

In order for Humboldt to study the statue of Coatlicue, it had to be dug up from where it had been buried in the basement of the University of Mexico for protection and to prevent its use in the independence movement of the time \parencite[85]{Bernal1980}[10]{Matos1979}[20]{Matos1995}.
Following the invasion of Spain by Napoleon in 1808 and the placement of Napoleon’s brother on the Spanish throne, Mexico and other Spanish colonies revolted, and pre-Hispanic artefacts like the statue of Coatlicue were used as symbols of rebellion against Spain \parencites[20]{Matos1995}[31]{Skidmore1989}. 
The first rebellion, and the \textit{mestizo} and native army led by Hidalgo and Morelos, ended with the reinstatement of Ferdinand VII on the Spanish throne \parencite[31-32]{Skidmore1989}.

In the early 1800s and increasingly throughout the War of Independence (1820–1836) the pre-Hispanic past was used to legitimize this rebellion, and thus the rebellion was important to the development of the practice of archaeology. In this first wave of indigenism, the leaders of the independence movement saw the Spanish government as an oppressive foreign power and Mexicans as heirs to the victims of the Spanish invasion \parencite[336]{Caballero2008}, 
idealizing and romanticizing the pre-Hispanic past and particularly more “advanced” cultures like the Aztecs \parencite[223]{Fowler1987}; 
however, the Spanish and their role in the identity of (the colony and then) Mexico was controversial throughout the 1800s. On one end were those who saw Cortes as a villain, glorifying indigenous heroes\footnote{This focus on heroes was typical of nationalist movements around the world in the 1800s; however, this adoption of indigenous (etc) heroes was also the antithesis of the heroification of Cortes.} like the Aztec ruler Cuauhtemoc and those who supported natives like Las Casas, while on the other, some maintained that Cortes was the founding father of Mexico and that the pre-Hispanic past was not an important part of their identity \parencite[69]{Restall2003}. 
When the political importance of the pre-Hispanic past waned following Independence, an interest in it had nonetheless been established along with the practice of archaeology. During the 1800s, the first archaeological museum was founded, an encyclopaedia of the pre-Hispanic past and many previously censored texts were published, and the pre-Hispanic past was expressed in the fine arts. 

\IJSRAsection{The Porfiriato, “Indians”, and Mexico as a Mestizo Nation}

At the end of the 1800s, archaeology flourished. During the dictatorship of Porfirio Díaz (1880–1910), positivism\footnote{Positivism, from Auguste Comte in the 1830s, was based on the idea that human society, like the natural world, was governed by laws of causality that are possible to learn objectively through reason and empirical science \parencite[13]{Howell2001}.} increased archaeological interest for scientific reasons, and a new wave of nationalism increased it for political ones \parencite[77]{Patterson1995}. 
For the first time, the government funded archaeological expeditions, and Leopoldo Batres was instrumental in making this happen \parencite[149,159]{Bernal1980}. 
Batres (1852–1926) is best known for restoring the Temple of the Sun at Teotihuacan in the early 1900s, and the data from his many excavations (ex. at Mitla and Monte Albán) and his resulting publications are still referenced today; he has been critiqued for his poor technique \parencites[149]{Bernal1980}[78]{Bueno2016}, 
but archaeology had still not become a trained discipline, and archaeological methods had not been standardized. Although most archaeologists during the Porfiriato worked on museum collections, Batres and others at this time collected and increasingly published archaeological data from every part of what later became known as the cultural area of Mesoamerica \parencites[159]{Bernal1980}[77]{Patterson1995}[63]{Swarthout2004}. 
The importance of using this data also grew: anyone studying the pre-Hispanic past was now seen to need archaeological data in addition to textual data in order to do so, and positivist and scientific thought led to the discarding of old assumptions in favour of more rigorous empirical study \parencites[159]{Bernal1980}[64]{Swarthout2004}.

Just as it had in the early 1800s, this burgeoning of archaeology went hand in hand with nationalism; this time, with a move to create a unifying national history with which to strengthen Mexico as a country, nationally and internationally \parencites[92]{Bueno2016}[63]{Swarthout2004}. 
At the start of the Porfiriato, many Mexicans would still have identified more with their town, ethnicity and culture than any national identity, and to unite its people with a common origin and strengthen Mexico as its own nation with its own history, the idea of Mexico as a \textit{mestizo} nation, as a mix of Spanish and pre-Hispanic indigenous cultures, was increasingly put forward by the government; this was a driving force in the restoration of ancient ruins, the erection of monuments to national heroes, and the new funding for archaeological work at this time \parencites[6,92]{Bueno2016}[63,66]{Swarthout2004}. 
Among other changes, the 1897 Law of Monuments made archaeological ruins the property of the federal government, and for the first time, exploring, modifying or excavating them without governmental permission became a crime \parencites[81]{Bueno2016}[180]{Lobjois2013}.

However, despite this official embracing of pre-Hispanic indigenous peoples, there was still a stark contrast between their status and the status of living indigenous peoples because living indigenous people were not seen to be heirs to that past \parencite[337-338]{Caballero2008}. 
The racist concept of the “Indian” persisted, and was only strengthened by the positivism that helped historical research. 
While in the colonial period “Indians” were seen as “a backward, uncivilized Other” in need of assimilation \parencite[63]{Swarthout2004} and proven to be inferior by their “defeat” in the “Spanish Conquest”, now their subjugation was academically justified by observational “science” that identified them as inferior by physical attributes and circumstance \parencites[70]{Patterson1995}[64]{Swarthout2004}. 
Eduardo Matos Moctezuma argues \textcite[12,14]{Matos1979} that archaeology during the Porfiriato was used as a politico-cultural façade 
(“\emph{fachada politico-cultural}”) that helped to further the subjugation of living indigenous peoples while justifying the power of the new \textit{criollo} and \textit{mestizo} elites in Mexico following Independence. If everyone in Mexico was \textit{mestizo}, and neither indigenous people nor Spanish people existed anymore, then racism could not exist, poverty and subjugation were signs of inferiority and failure, and power was wielded by those who earned it.

There was another reason that this official embracing of the indigenous past in the national history was harmful to some indigenous peoples—it \textit{selectively} embraced past indigenous cultures. 
It glorified the Aztec Empire and the other more “advanced” cultures of the area like the Maya and Toltec, and ignored many indigenous groups entirely, including the non-sedentary groups of northern Mexico (among them, the Yaqui of northern Mexico were not even defeated until the 1900s \parencite[72]{Restall2003}) 
and many indigenous cultures that had historically been allies with the Spanish \parencite[42]{Bueno2016}. 
For these non-Aztec, non-state cultures, the suggestion that everyone in Mexico was (/is) part-Aztec, part-Spanish denied their cultures, silenced their voices, and glossed over key historical truths: that many indigenous groups had been happy to see the Aztecs fall because they had either been enemies of them (ex. the Tlaxcalla), or exploited by them. The Huejotzingo, for instance, were one of the conquistadores’ staunchest allies both in defeating the Aztecs and in other Spanish campaigns in the following decades \parencite[49]{Restall2003}. 
Even the Maya, glossed under this umbrella label, fought against each other, some on the side of the Spanish and some against, with independent polities existing into the 1900s \parencite[50,72]{Restall2003}. 
There was never, before or after the Spanish invasion or now, an acceptance by Mexican indigenous peoples that they were all of one culture and identity, much less that they were or are all “Aztec”. 

\IJSRAseparator{The Revolution and the \textit{Indigenismo} Movement}

These social changes, the new national history, and the adoption of the singular \textit{mestizo} or indigenous-Spanish Mexican identity solidified with the revolution that ended Díaz’s dictatorship \parencite[66-67]{Swarthout2004}. 
Unifying the country and enculturating (and Europeanizing) “Indians” had been a concern of the Mexican government(s) for some time \parencites[336]{Caballero2008}[75]{Restall2003}, 
and with the Revolution for the first time the impoverished lower classes were officially incorporated into the nation socially and politically; colonial-period injustices were acknowledged, social and economic reforms were promised, and Mexico sought to unify and democratize itself while minimizing foreign influences \parencites[66-67]{Swarthout2004}[276]{Trigger2006}. 
The Revolution solidified the new national history and a new favouring of the pre-Hispanic past over that of the Spanish colonial period \parencites[330]{Caballero2008}[276]{Trigger2006}, because the pre-Hispanic past was what gave Mexico its unique identity.

The glorification of the pre-Hispanic past and the re-writing of history fit into changes in Western history itself at that time, and the rise of “social history” in the 1900s which focussed on minorities and ordinary people rather than institutions and elites \parencite[14]{Howell2001}. 
However, the re-writing and re-interpreting of history went a step further in Mexico following the Revolution. Caballero López \textcite{Caballero2008} describes how the history of Milpa Alta, now one of the districts of Mexico City, illustrates this. The official history of Milpa Alta as written post-Revolution proudly and nostalgically identifies its origin in the Mexica Empire, describing the Mexica as the ancestors of all indigenous people. In striking contrast, the official narrative of the city’s history as written in an administrative document in 1690 describes the founding of Milpa Alta as the moment when its patron saint appeared to them shortly after Tenochtitlan fell. The origins they acknowledged in the colonial period were tied to the Spanish and to Catholicism, but after the Revolution they extended their history backward into the newly glorified Aztec past \parencite[331-333]{Caballero2008}. 
Many native communities and cultures described the past differently from Milpa Alta, but they all used history politically. Some Maya, to emphasize how long-lived and resilient their culture was, described the Spanish invasion as only one change of many in their much longer history, while some indigenous groups needed to emphasize their Christian piousness, their loyalty to the Spanish, or their legal claim to their territory \parencite[122]{Restall2003}.

This rewriting of history to include proud origins in the pre-Hispanic period on a country-wide scale was part of the \textit{indigenismo} movement, which celebrated the role of indigenous people in Mexico’s past while putting the position of poverty and the “Other” status of living indigenous people on the national agenda. Unlike during the War of Independence, the \textit{indigenismo} movement of the 1900s was not just about venerating pre-Hispanic natives, but also about dealing with the “problem” of Indians in a country of \textit{mestizos}, and though it directly faced long-building problems around native rights, its solution was assimilation \parencite[60]{Ortiz1996}. 
At once a discriminatory and a hopeful movement, it saw indigenous peoples as not quite Mexican, \textit{mestizo}, or possessing the admired qualities of their ancestors, but it also saw them as redeemable and able to achieve these goals. In the decades following the Revolution, some indigenous groups were forbidden from speaking indigenous languages or wearing traditional clothing in order to promote “equality” and to make everyone culturally the same \parencite[91,97]{Hernandez2002}.

The glorification of \textit{pre-Hispanic} natives and their cultures, by contrast, was now part of Mexico’s identity and pride, and as archaeology was called on to disprove ideas of pre-Hispanic native inferiority that went back to the 1500s \parencite[95]{Swarthout2004}, a great deal of money was put into the budding discipline. 
In 1910, a school for archaeology (the Escuela Internacional de Arqueología y Etnología Americana or EIAEA) was founded, though it closed a decade later from the turmoil of the ongoing revolution. At this time, the first systematic archaeological studies and academic training in Mexico began (and thus archaeology became a “discipline”), influenced by Franz Boas’ theory of historical particularism, in a generation that included Manuel Gamio, George Vaillant, Alfred Kidder, and Alfred Tozzer \parencites[330]{Caballero2008}[234]{Fowler1987}[277]{Trigger2006}.

Manuel Gamio (1883–1960), called the “Father of Mexican Anthropology”, saw himself as both belonging to an anthropological tradition going back to Sahagún and as beginning a “new anthropology” that was practical and scientific, and unlike Batres, he saw archaeology as a branch of anthropology rather than history \parencites[185]{Lobjois2013}[321]{Zermeno2002}. 
Gamio was a leading intellectual and the leading anthropologist of the Revolutionary and post-Revolutionary periods, and it was he who spread the notion of the greatness of the Aztec (and particularly the Mexica) widely \parencites[337]{Caballero2008}[77]{Patterson1995}, 
furthering the favouring of the Aztec over other pre-Hispanic groups that had already begun. After training under Boas in the United States \parencite[164]{Bernal1980}, 
in 1913 Gamio created the first prehistoric chronology for the Valley of Mexico from excavations at San Miguel Amantla \parencite[277]{Trigger2006}, 
in 1917 he founded the first organization for anthropological study (La Dirección de Antropología) \parencite[164,186]{Bernal1980}, 
from 1917–1922 he worked at Teotihuacan and turned the partially-restored ruins (thanks to Batres’ nationalist restoration project in 1910 \parencites[267]{Trigger2006}) 
into a public archaeological site and a tourist attraction \parencite[338]{Caballero2008}, 
and in 1927 he did ethnographic work on Mexican immigrants in the US, the first study of its kind since Sahagún \parencite[317]{Zermeno2002}. 
Like Sahagún however, Gamio saw his research as a tool for “curing” the Indians—of their “backwardness”, rather than of idolatry \parencite[323]{Zermeno2002}. 
He saw Mexico’s indigenous populations as degenerations of pre-Hispanic cultures that were holding Mexico back from progress, but he also saw the problem as solvable through modernization and education because of their respected heritage \parencite[338]{Caballero2008}[78]{Patterson1995}. 
Gamio was one of the first big proponents of these \textit{indigenismo} ideas, advocating for the assimilation of indigenous peoples, and ethnographic work that might hasten this assimilation, as a solution to the Indian “problem” and the need for national progress \parencite[60]{Ortiz1996}.

Alfonso Caso was another indigenista anthropologist and an important member of what some call the “Mexican school” of archaeology, working to show the pre-Hispanic past through his many excavations \parencite[Litvak 1978 in][13]{Matos1979}. 
In the 1930s, Caso excavated the major sites of Teotihuacan, Monte Alban, Tula and Chichen Itza \parencites[340]{Caballero2008}[234]{Fowler1987}, 
and founded the National Institute of Anthropology and History (INAH), in 1939, which still grants all licenses for Mexican excavations \parencites[79]{Patterson1995}[277]{Trigger2006}, 
though like Batres his work still suffered technically \parencite[15]{Matos1979}. 

\IJSRAsection{The Mid-1900s—Culture-History and “Mesoamerica”}

Archaeological technique improved in the 1930s and 1940s as Mexican archaeology became culture-historical, and focussed on creating unique cultural chronologies with little theory. Although the use of stratigraphy in excavations had become common with the founding of the EIAEA in 1911 and the work of Eduardo Noguera, stratigraphy only became an essential archaeological method during this period in order to create chronologies in place of the old descriptions of ruins and artefacts \parencite[161,172,188]{Bernal1980}. 
At this time, chronologies were established for the Valley of Oaxaca and the Central Plateau valleys \parencite[188]{Bernal1980}, and in 1938, the year before Caso founded INAH, The National School of Anthropology and History (ENAH)—which offered degrees in archaeology—was co-founded by Paul Kirchoff, a German-born Mexican. 

Paul Kirchoff coined the term “Mesoamerica” in 1943, fundamentally altering Mexican archaeology. “Mesoamerica” was originally intended to be an analytical tool based on a cultural area rather than the largely geographic one it became, and rose out of culture-history’s love of categorization and evolution-driven models \parencite[4809,4811]{Runggaldier2001}. 
It was initially based on contact-period cultures and similarities they shared in agriculture (including a reliance on maize), clothing, architecture, writing, calendars (a shared ritual and solar calendar), ritual, and military practices \parencite[4807-4810]{Runggaldier2001}. 
It was an etic term that did not reflect any shared perceived shared culture amongst the people, but problems also arose with the concept because of its application to both post-contact and very early (up to 10kya) contexts in the following decades, and because it was based on an idea of progress leading to civilization which put the cultures of central Mexico, and particularly the Aztec, Teotihuacano, and Toltec, as more advanced than the rest \parencite[4807-4809,4812]{Runggaldier2001}.

The introduction of the concept of Mesoamerica changed what was studied by archaeologists and impacted funding. Starting in the 1940s, systematic archaeological research expanded from the Basin of Mexico, Oaxaca and Maya regions to include the entire region of Mesoamerica (though not beyond, to the nomadic cultures of northern Mexico, for instance), and cultures were reinterpreted as being part of a larger Mesoamerican context and trajectory; regional projects working within the unifying idea of Mesoamerica only grew in the coming decades as “Mesoamerican archaeology” became its own sub-discipline \parencite[4810]{Runggaldier2001}. 
Starting with the introduction of the concept, archaeological projects that demonstrated the greatest achievements of Mesoamerican civilizations were also favoured financially \parencite[4812]{Runggaldier2001}.

One of the first scholars to begin using the concept of Mesoamerica was Pedro Armillas in his research on cultural boundaries. Armillas was critical of the focus of Mexican archaeology in the 1940s on creating chronologies out of ceramics, and he wanted archaeological research to work toward finding cross-cultural universals and to involve more theory and analysis \parencite[4810]{Runggaldier2001}. 
Armillas brought the theories of Gordon Childe—and through him Marx and Durkheim—to Mexico in the mid-1900s \parencite[17]{Matos1979}. 
Only then, around 1950, did the focus of archaeology begin to shift from creating chronologies to implementing more multi-disciplinary and interpretive approaches. The ENAH pushed this change and helped direct Mexican archaeology to become more scientific and address theoretical problems \parencite[19]{Matos1979}. 
By 1950, archaeology in Mexico had established itself as a field much as we know it now: since the start of the Porfiriato, it had transitioned from an unfunded practice of self-taught scholars to a discipline and paid profession with its own institute, laboratories, museums and trained practitioners who collaborated with specialists for excavations and analysis \parencite[188]{Bernal1980}.

\IJSRAsection{The Late 1900s to the present—Tourist archaeology, INAH, and Mexico’s identity}

From the Revolution until the 1970s (and to a lesser degree until the present), archaeological study was heavily influenced by the ability of sites to gain revenue from tourism, because of the ongoing link between archaeology and Mexico’s national identity. In order to create and present a past that would foster national unity, the pre-Hispanic past was popularized through museums and large, accessible and showy sites \parencite[277]{Trigger2006}, 
sites that, not coincidently, often showed the most “advanced” Mesoamerican cultures. After the Tlatelolco massacre in 1968 when hundreds of protestors advocating for democracy were shot down by police in Mexico City, a huge wave of anti-state sentiment was accompanied by critiques of archaeology \parencite[80]{Patterson1995}, 
as well as critiques of \textit{indigenismo} ideas \parencite[61]{Ortiz1996}. 
Among the criticisms of the discipline was a demand for archaeology to be driven by scientific questions and not by tourism \parencite[80]{Patterson1995}. 
The idea of doing archaeology for tourism was also critiqued by Eduardo Matos Moctezuma, who began excavations at the Templo Mayor in Mexico City in 1978 \parencite[80]{Patterson1995}. 
Now one of the most important sites in Mexico, the (re-)discovery of the Templo Mayor increased state-funded archaeology \parencite[80]{Patterson1995}; 
at the beginning, the site and its artefacts were used both for international publicity and for promoting that same syncretic past and national history that had been established with the Revolution \parencite[234]{Fowler1987}.

Another change in the 1970s was a broadening of who became archaeologists, which affected interpretation of the past. Between the Revolution and the 1970s, most indigenista anthropologists were state intellectuals. Because of this, cultural-evolutionary ideas, and the focus of most archaeological work on monumental architecture, cultural and social change in the past were viewed as being elite- and state-driven, and studies of the past focussed on elites and state institutions \parencites[220-230]{Gomez2007}[84]{Patterson1995}. 
In part this trend of who became archaeologists was affected by the continuing poverty of the indigenous, despite the new \textit{indigenismo} integration movement \parencite[84]{Patterson1995}. 

Increasingly in the 1970s, efforts were being made to situate Mexican archaeology within larger theoretical frameworks \parencite[19-25]{Matos1979}. 
\textcite{Matos1979} and \textcite{Gomez2007} both argue that though the excavation and survey techniques of Mexico prioritized mathematical and statistical data in the same way we do today from as early as the 1950s, by the 1970s these methods were still not accompanied by enough theory or reflexivity. 
\textcite[229]{Gomez2007} notes that Mexican archaeology was missing the problematization of Schiffer and Harris’ processual ideas, and 
\textcite{Gandara1981} argues that the indiscriminate and uncritical adoption of concepts from Marxism, ecology, and cultural and processual archaeology led Mexican archaeology to a "dogmatic falsificationism", which made contradicting culture historical ideas, and reflexive research, impossible. This dogmatic falsificationism further led Mexican archaeology to a “paradigmatic delay” (“\textit{retraso paradigmático}”) that Gándara argues is still setting back the discipline in Mexico today 
\parencites{Gandara1981}{Gandara1987}{Gomez2007}. As an example of this delay, \textcite[21]{Matos1979} points to the adoption of Childe’s and Althusser’s concepts by Mexican archaeologists in the late 1970s without them really understanding their ideas or who Childe and Althusser were responding to.

By the 1990s, \textcite{Leon1996} argued that Mexican archaeologists had coalesced into two groups: the “official” and the “academic”. “Official” archaeologists included researchers like Eduardo Matos Moctezuma who were concerned with the “how”, “when” and “why” of elites and state institutions (usually working on important Mesoamerican cultures), whereas the “academic” archaeologists, including researchers like Linda Manzanilla, were more interested in the everyday lives of commoners. 
This divide has, to a degree, existed in Mexican archaeology since the Porfiriato—then it was between museum archaeologists and site excavators and restorers like Batres. Batres’ work and marked ability to secure funding, as well as laws passed during the Porfiriato around archaeological sites and excavations, gave power that has never been lost to the National Institute of Anthropology and History (INAH), when it became INAH’s archaeological projects versus the university’s (UNAM) \parencite[79]{Bueno2016}.

Today, INAH still gives preference to “official” projects that invoke popular interest, while underfunding has led to many of the biggest “academic” projects in Mexico being sponsored and codirected by foreign archaeologists \parencite{Gomez2007}[277-278]{Trigger2006}. 
This can be seen in the 100+ archaeological sites that have been partially restored and open to the public in Mexico as of the last decade \parencite[277-278]{Trigger2006}, 
alongside the abandonment and/or destruction of less showy sites, like the archaeological zone in La Mezquiterain, a district of Tlaltizapán, Morelos \parencite{Giles2015}. 
This “tourist archaeology” and preference for showy monumental sites has inspired the criticism and frustration of many academics \parencites{Gomez2007}{Lopez2010}[30]{Sanders2000}. 

Tourist archaeology is an understandable legacy given the nationalist motivations behind the growth and development of archaeology in Mexico for centuries, including the construction of Mexico’s \textit{mestizo} identity and singular national history. The glorification of the past that initially encouraged the growth of archaeology has been argued by many to now be hindering Mexican archaeology—favouring the finding, restoration and conservation of monuments in the place of scientific research, a broadening of archaeological projects to include those that do not directly improve Mexico’s international image, and theoretical and methodological advancements \parencites{Gandara1981}{Gandara1987}{Gomez2007}{Lopez2010}.

Change has, however, slowly been coming for living indigenous people in Mexico. After decades of indigenous rights movements, Mexico reformed its Constitution in 1992 to acknowledge the rights (and existence) of indigenous peoples. A paragraph was added that stated that indigenous people in Mexico have cultural rights and the right to practice their own customs; it also said that the law should take into account indigenous methods of self-government in indigenous cases \parencite[59]{Ortiz1996}. 
Activist movements fighting for indigenous autonomy within Mexico are ongoing to this day, amongst a lingering racism toward indigenous people and the country’s ongoing anti-multicultural national identity \parencite[93-95]{Hernandez2002}. 

Mexico’s identity continues to be proudly rooted in the achievements of the Aztec empire. Mexicans are “a \textit{mestizo} people”, and Aztec symbols like the eagle on the flag and the Calendar Stone are used as symbols of the country \parencites[335]{Caballero2008}[234]{Fowler1987}[227]{Trigger2006}. 
The connections between Mexico and the Aztec Empire are made explicit in modern-day Mexico, but as illustrated in this essay, just like the silencing and social reforms of the official \textit{indigenismo} movement, the increasing favouring of the Aztecs over other indigenous groups, and the development of archaeology as a discipline, this view of history is a development of the 20th century \parencite[329]{Caballero2008}.

In 1964, a plaque was dedicated to President López Mateos in Mexico City in the place where the Aztecs made their last stand. It reads: “On 13 August 1521, Tlatelolco, heroically defended by Cuauhtemoc, fell into the power of Hernán Cortés. It was neither a triumph nor a defeat, but the painful birth of the \textit{mestizo} people that is Mexico today” \parencite[234]{Fowler1987}. 
These words show the appropriation of both a Spanish and Aztec history, and the transformation of the concept of \textit{mestizo} from the second-class mixed-racial minority of colonial Mexico to the revered cultural and national idea that includes all Mexicans today. The “heroics” of the emperor Cuauhtemoc reflect the idealization of the Aztec people and the importance that that indigenous blood has in the \textit{mestizo} mixture to this day, despite the issues still faced by the modern indigenous from the colonial legacy and the ongoing silencing of non-Aztec, non-“advanced”, and non-Mesoamerican indigenous peoples of Mexico and their role in its history. 

The transitions from the silenced or subjugated “Indian” to indigenous people with cultural rights, the savagery and naivety of the pre-Hispanic past to its glorification and romanticization, and the colony of New Spain to the country of Mexico, are changes that accompanied and still affect the development of archaeology in Mexico. The case of Milpa Alta shows how fundamentally history can be changed with changing politics and national values, and from that one can glean the effect that both the initial destruction and forgetting of natives in the 1500s and the current glorification of the Aztec can have on archaeological interpretation. 
Though the \textit{indigenismo} wave of tourist archaeology built up the discipline from the pre-Revolution days of self-taught descriptive work, it also favoured the cities and temples, the states and performances, the Aztec who shared the same capital and the large beautiful buildings of the Maya and Teotihuacanos. In Mexico, the view of the native and the view of his past are what inspired the development of archaeological study out of the first cultural studies of the 1500s, to the antiquarian and archaeological work of Sigüenza and Gama respectively, to the rise of a rigorous discipline with Gamio and the glorification of the past, and to the continuing growth of research-driven, scientific and theoretically informed “academic” archaeology in Mexico. To this day the place of the native and the pre-Hispanic past continues to affect the culture of Mexico and those who work there, and continues to influence licensing and funding for archaeological work, as well as the national and international presentation of Mexico itself.




%
% \IJSRAsection{small headline}
%
% \IJSRAseparator


\IJSRAclosing