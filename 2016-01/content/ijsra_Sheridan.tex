\openingarticle
\def\ppages{\pagerange{Sheridan:firstpage}{Sheridan:lastpage}}
\def\shorttitle{A Case for Hybridity}
\def\maintitle{A Case for Hybridity}
\def\shortauthor{Kelton Sheridan}
\def\authormail{kmsherid@syr.edu}
\def\affiliation{University of Syracuse, USA}
\def\thanknote{\footnote{Kelton Sheridan is a recent graduate of Syracuse University where she double majored in Anthropology and Art History. She has worked on projects in Barbados and Ireland. Her research interests include Preclassic Maya ceramics as well as the Colonial Maya and postcolonial theory. Recently, she has been working for Northeast Archaeology Research Center on CRM projects in Vermont.}}
%--------------------------------------------------------------
\mychapter{\maintitle}
\begin{center}
	{\Large\scshape\shortauthor \thanknote}\\[1em]
	\email \\
	\affiliation
\end{center}
\vspace{3em}
\midarticle
%--------------------------------------------------------------
\label{Sheridan:firstpage}
%----------------------------------------------------------------------
	\begin{myabstract}

In this\marginnote{Abstract} paper, I explore the use of hybridity as a concept in archaeology and anthropology, specifically drawing on Homi Bhabha’s formulations. Through examining different discourses on the subject, I analyse how authors have used or misused this term. I then deconstruct what Bhabha means by hybridity and argue that hybridity is still a very relevant and necessary concept because of the ways in which it affects people’s lives today. I discuss examples of contexts in which hybridity is more apparent and where putting the term to use would prove useful. Finally, I conclude the paper with an answer to the question, “Is it time to move away from hybridity?”

\keywords[Keywords]{hybridity, colonialism, Homi Bhabha, third space, mimicry, cultural affiliation, NAGPRA.}
	\end{myabstract}

\lettrine[nindent=0em,lines=3]{T}{he} concept of hybridity has a long and complex history. It began as a negative reference to the biological mixing of humans, but today, redefined by postcolonial theorist Homi Bhabha, hybridity is a complex and powerful term that, if used as intended by Bhabha, could change the ways in which individuals view and interact with each other. In order to demonstrate that this concept is worth closer examination, I discuss the criticisms surrounding Bhabha’s use of this approach. To further explain the complexity of hybridity, I analyse the different ways in which authors have misused the term and then demonstrate some of the contexts in which hybridity is more obvious. The emphasis on the word ‘more’ is important because, as this paper argues, hybridity is a constant process that is always in motion.

	By understanding the important contexts that have significant political and social ramifications for individuals today, the reader will be able to see the impact the use of hybridity could have. In a recent paper, \textcite {Silliman_2014}  suggested it might be time to move away from the notion of ‘hybridity’. I will conclude this article with an answer to Silliman’s question of whether or not the concept of ‘hybridity’ should be abandoned. 
	
	
%	\section {Discourse on Hybridity}
	Most\marginnote{Discourse on Hybridity} commonly, as borrowed from biology, hybridity is defined as a mixture of two or more essential entities that produce a new product. Hybridity as Homi Bhabha discusses it is much more complex and nuanced. However, his writings are often inaccessible and difficult to decipher, perhaps leading to unintended interpretations of his theory. While Bhabha discusses hybridity in specifically colonial encounters, his concept can be applied to many and perhaps all anthropological and archaeological contexts. Authors that use hybridity usually do not approach it as a process, but instead focus on the points of origins of the new hybrid subject \parencite[267]{Deagan_2013}. That is, they essentialise cultures or pick out the strands of each culture that may be associated with the new product. In doing so, these authors are further promoting the idea of ‘original’ and ‘pure’ entities that somehow mixed together and created something new, a hybrid. The problem with this is that hybrids are made up of already hybrid antecedents because, as I further argue, it is impossible to access an antecedent to hybridity. 
	
	Some authors interpret the interactions that went into creating the hybrid as reflective of social choices \parencite{Deagan_2013} and \parencite{Silliman_2009}. While in many cases this may be true, hybridisation is not always a choice but rather an unintended but inevitable outcome of cultural encounters and continual cultural reproduction. For example, Kathleen Deagan, in her chapter “Hybridity, Identity, and Archaeological Practice,” treats the Catholic and Native elements present in a Muchik burial found at Morrope, in the Lambayeque province of Peru, as discrete entities \parencite[267]{Deagan_2013}. She does not incorporate these two cultural traditions into the larger scale of a long-term process, reinforcing the notion that cultures are pure, discrete entities. In doing so, she does not take into account the complex histories of change within cultures. It is important to recognise that all cultures are constantly undergoing change and that no culture has remained the same since its inception. Furthermore, Deagan does not cite Bhabha. If one is going to use “hybridity,” one needs also to understand how it has evolved in the framework of postcolonial theory. The term hybrid has a different connotation today than it did when it was first employed in colonial and racial contexts. While the concept cannot escape the loaded baggage of its history, Bhabha’s reformulation makes the process of hybridisation central to postcolonial theory situations. 

	Stephen Silliman has suggested that archaeologists study hybrid objects in the short term in keeping with traditional archaeological training \parencite{Silliman_2014}. Since hybridisation is a constant process with no beginning or end, preparing archaeologists to understand and interpret change at sites over the long‐term would prove useful \parencite{Lightfoot_1995}. This would mean archaeologists need to examine a site, not just at a single period, but contextualised within a larger temporal framework. Accepting that hybridisation is a process that has always existed and will continue to exist is critical to changing perspectives on the concept of “culture.” 

	In his most recent article, Silliman restates that hybridity deserves a requiem since it is “poised to ultimately fail as anything truly useful for archaeologists” \parencite[14]{Silliman_2015}. He argues that the concept fails to recognise “cultural trajectories” by creating “durable hybrids”  \parencite[14]{Silliman_2015}. With this phrase, Silliman sums up the issues with academic uses of hybridity. However, it is not necessarily the term that is at fault, but rather the ways in which scholars have put it to use. Silliman does not seem to consider the impressive potential of the concept when utilised as Bhabha intends it because he is overthinking hybridity and looking for an end point rather than seeing hybridity as an ongoing process. Just as there is no beginning, there is no end. The hybrid may “well be a function of our own over-wrought classification schemes” \parencite[14]{Silliman_2015} because, as he himself notes, there is an “overreliance on origins instead of practices” \parencite[8]{Silliman_2015}and a desire to fix time in a continuing process of change. If used as Bhabha conceptualises it, hybridity could act as a general framework for understanding cultural change and for undermining power structures based on the imaginary and damaging idea of cultural purity. It is not a method of classification, but rather a notion of processual change. 

	Silliman states, “if everything ends up being hybridity, then it is unlikely that the term or concept will have any use to archaeology” \parencite[489]{Silliman_2013}. \textcite{Silliman_2015} raises questions such as: What is not a hybrid?, Is hybridity a quality or a state?, and When does something stop being a hybrid? While these questions are important to consider with respect to hybridity, the term does not necessarily become irrelevant to archaeological contexts because of the lack of an answer. Understanding that all cultures are products of entangled cultural interactions is critical to re-interpreting views that there are no “’pure’ antecedents” \parencite[489]{Silliman_2013}. Bhabha’s notion of hybridity can set up a framework for interpreting archaeological sites in a more processual way. Incorporating the notion that a culture is a combination of many factors could prove beneficial to archaeologists because it would eliminate possible essentialising interpretations. Essentialising cultures proves to be problematic because it “reduce[s] complex heterogeneous structures to a supposed inner truth or essence” \parencite[73]{Liebmann_2008}.

	When discussing Bhabha’s hybridity, the notion of mimicry is often explained as being a physical outcome of the concept of hybridity. Mimicry is the “desire for a reformed, recognisable Other” \parencite[122]{Bhabha_1994} , often in a colonial context. The colonised subject looks and acts the same as the colonisers, yet there is something fundamentally different that will never fully permit them to fit; they are “the same, but not quite”  \parencite[122]{Bhabha_1994}. 

	In her book \citetitle{McClintock_1995}, \citeauthor{McClintock_1995} discusses different colonial relations and their effects on individuals. In one section, McClintock critiques both Luce Irigaray’s writings on gender mimicry and Bhabha’s ideas on colonial ambivalence. She argues that both are somewhat limited in their discussions of each topic. Irigaray, for example, writes, “…women learn to mimic femininity as a social mask” \parencite[62]{McClintock_1995}. McClintock believes that in the process of her discussion, Irigaray reinforces the gender binaries she is attempting to break down. Additionally, McClintock claims that Irigaray neglects to analyse the way in which men themselves strategically use gender masquerade.   	She goes on to criticise Bhabha, saying he focuses only on mimicry with respect to men and fails to adequately explore the gendered interactions of mimicry \parencite[62]{McClintock_1995}. McClintock’s approach to hybridity reflects the belief that you cannot separate race, class, and gender because they are already irreducibly hybrid and intersectional. An anthropological focus exclusively on race would risk becoming an essentialist approach which, as \textcite [62]{McClintock_1995} points out, is the opposite of Bhabha’s conceptualisation of hybridity. 
	McClintock’s incorporation of gender and class factors into hybridity arguably breaks down essentialist views of individuals in favour of a more nuanced approach. Bhabha, in fact, does not pay much attention to these components. However, Bhabha’s concept of hybridity is designed to serve as a general, holistic approach to human interaction that is not divided into categories of race, class and gender \parencite{Bhabha_1994}. He is not essentialising because he does not discuss gender and class; instead, he is setting up a broad framework within which scholars can work by focusing on cultural encounters in general and dealing with culture as a whole. This framework is one where the notion of hybridity is commonly accepted and incorporated into cultural perspectives on a daily basis. 
	
	Hybridity, as used in postcolonial theory, is much different than its use in the colonial era. Robert Young, for example, discusses the history of the term in Colonial Desire, arguing that within the colonial context, the term hybridity was strongly tied to the biological interbreeding of cultures \parencite[6]{Young_1995}. During this era, different races were often considered to be different species with their own cultures and customs, leading to colonial anxieties concerning the intermixing of species and what that would do to the ‘pure’ white race \parencite[6]{Young_1995}. Authors speculated about what would become of the mixed offspring—Could they reproduce? 
Would they degrade through time \parencite[8]{Young_1995}?  
The idea of pure races continued into the twentieth century with the writing of Mein Kampf, perhaps the culmination of racial ignorance \parencite[8]{Young_1995}. With such a tumultuous history, it is understandable why authors would decide not to use the term. However, Bhabha’s re-structuration of the concept is important in its development towards the state it is in today. 
	
%	\section {Hybridity and Bhabha}
	
	Bhabha’s\marginnote{Hybridity and Bhabha} notion of hybridity is indeed complex, and he writes about it in messy and complicated ways. The hybrid is unquestionably created from cultural interactions and other cultural dynamics that are not always reflective of power relations. According to Bhabha, when one culture dominates another, a space of translation, or “third space,” opens up \parencite[53]{Bhabha_1994}. The hybrid emerges in this space. The concept of a third space is useful in conceptualising the hybrid because it reinforces that the hybrid is entirely different from either of its constituents and therefore must be discussed from a new perspective. Bhabha postulates that a hybrid is “neither one nor the other” \parencite[49]{Bhabha_1994} , meaning that once the hybrid is created, it can no longer be described in terms of the cultures from which it originated. It is something entirely new. For this reason, Bhabha emphasises studying the hybrid in terms of how it functions in this new space instead of focusing on its origins. Moving away from focusing on the contributors to the hybrid attempts to move past the essentialisation of what cannot be essential entities: namely culture, identity, and practice. 
	
	Deconstructing essentialist attitudes is the foundation of hybridity. Essentialism, or viewing something as pure and unchanging, is virtually impossible in terms of culture because “there is nothing essential or stable about cultural, ethnic, or natural traditions” \parencite[5]{VossAllen_2008}. Since the intermixing and changing of a culture is constant, argues Bhabha, there can be no essential, static culture \parencite[52]{Bhabha_1994}. Recognising that hybridisation is a continual process that is always operating will aid in rectifying the ways in which scholarly analysis homogenises cultural groups. Employing hybridity can help to elucidate the subtle differences between groups. Bhabha explains that the third space “displaces histories that constitute it and sets up new structures of authority” \parencite[211]{Rutherford_1990}. Bhabha focuses on what emerges from the hybrid space as opposed to what went into creating the hybrid. It seems as though the hybrid has been stigmatised in the past because of “the intervention of otherness” \parencite[211]{Rutherford_1990}, in which the hybrid is often fetishised and viewed negatively, leading to domination and exploitation.
	The concept of Bhabha’s hybridity is basic but, at the same time, complex in its nuances. In 1998, based on an essay that contained convoluted sentences, Bhabha was awarded second place in the Bad Writing Contest created by the journal Philosophy and Literature, which placed his work under the category of “demanding” \parencite{Smith_1999}. In “When Ideas Get Lost in Bad Writing”, New York Times writer Dinitia Smith proposed that perhaps scholars are “making themselves irrelevant” by using incomprehensible language and sentence structure in their writing.  Admittedly, it takes multiple read-throughs to understand Bhabha, which may deter some from reading his work entirely. He could be more straightforward in his presentation of challenging topics. 
	
	Defending his work, Bhabha explained in an interview that the parts of his work that are harder to understand are those in which he is “trying to think in a futuristic kind of way” \parencite[82]{Mitchell_1995}. He admitted that his writing is worse when he has not completely worked through an idea and is looking to spark a discussion. While this is an interesting deconstruction of traditional academic writing styles, his ideas get lost in the translation and interpretation of his work. 
	Bhabha also addressed the interpretation of his work. His assertions are not supposed to be simply accepted and followed exactly.  Instead, he produces “theoretical work [that] should in the fullest sense be open to translation” \parencite[82]{Mitchell_1995}. However, while individuals should be allowed to interpret his work freely, Bhabha has a responsibility to his readers to write more clearly, particularly given the potential usefulness of his ideas. Hybridity is not just a theoretical idea; the identification of a hybrid has real political and social repercussions today, as I discuss. As academics use the term ‘hybrid’ in their writings, these consequences must be at the forefront of their minds. At the same time, using Bhabha’s notion of hybridity will only be beneficial if people understand that it is not just some people that are hybrid, but all. 
	
%	\section{Contexts in which Hybridity Becomes Apparent}
	
	As  \textcite[52]{Bhabha_1994} \marginnote{Contexts in which Hybridity Becomes Apparent}
	 argues, hybridity is a constant, never‐ending process. But while it may be omnipresent, there are places where it is more apparent and, because of its political and social implications, where people are forced to engage with it, particularly in contexts where the hybrid has the potential to threaten the dominant culture with ambivalence and the ability to have feet in multiple worlds \parencite[77]{Anzaldua_1987}. Those identifying as hybrid create anxiety even today because they defy the traditional ideas of imposed, discrete categories. 
	 
	The space of translation becomes more obvious when exchanges of power are involved. Conversion of Indians was an important goal of the British colonial project in India \parencite{Bhabha_1994}. While some Indians complied with many of the British missionaries’ teachings and converted to Christianity, it was a new type of Christianity, one with Indian conditions. They refused to take the sacrament and questioned, “How can the word of God come from the flesh-eating mouths of the English?” \parencite[166]{Bhabha_1994}. In this case, Bhabha argues, the Indians were using their “powers of hybridity” \parencite[167]{Bhabha_1994} to maintain their religious ideologies in modified form. In this space, power relations become strained, creating an immense amount of anxiety amongst the colonisers. This is where mimicry becomes more apparent. Bhabha often views mimicry as having the power of subversion. The “mimic man”, as he calls it, can serve as a “class of interpreters” \parencite[124]{Bhabha_1994} between the colonised and the colonisers because they inhabit both worlds. In this third space, mimic men are a destabilising force that provokes anxiety on the part of the colonisers. The power of the hybrid is demonstrated here: the native population knows the British want them to become Christian, but however willing the natives are to accept this change, there are some beliefs they cannot give up. Their power of negotiation can be seen as a subversive act towards the coloniser. 
	
	In Borderlands: La Frontera, Gloria Anzaldúa presents the border between Mexico and Texas as a liminal zone where there are political ramifications for hybridity. Anzaldúa poetically and personally describes these borderlands as “the lifeblood of two worlds merging to form a third country” \parencite[3]{Anzaldua_1987}. This ‘third country’, similar to Bhabha’s third space, is where the hybrid emerges, along with the conflicts surrounding the feared power of the hybrid. The anxieties of the dominant group, in this case white America, become tangible as the border between Texas and Mexico becomes a literal warzone \parencite[11]{Anzaldua_1987}where people have to fight to survive. The ways in which American corporations interact with people living in these border towns contribute to the confusion surrounding their existence. 
	
	Anzaldúa’s book provides insights into what it means to identify as hybrid today. She brings to life Bhabha’s notion of the third space and its importance. Hybridity is not just a theory; it is a reality. Power relations are the key factors in these borderlands. Because white America does not have to fight to survive in these illegitimate places, this culture is less concerned about what happens within them. Since “the only ‘legitimate’ inhabitants are those in power, the whites and those who align themselves with whites,” \parencite[3-4]{Anzaldua_1987}, this creates difficulties for the hybrid population regarding questions of identity. Bhabha’s hybrid theory aids in understanding the realities of daily life for the hybrid population. The cultural practices that emerge from this space cannot be traditionally categorised because they are something entirely different, enabling people to adapt to in-between situations. Acknowledging this will prove to be more effective in understanding the new hybrid culture rather than trying to pick out the antecedent cultures that contributed to it. How the hybrid functions in the future is what matters. In the case of the Mexico-Texas borderlands, the people who identify as hybrids are stuck between acceptance in their own group and rejection in the greater America. Anzaldúa’s book is thus useful in grounding the theory of hybridity in actual reality.
	
	The mestizaje population, or the individuals that identify as descendants of Natives, Spanish, and Mexicans \parencite[5]{Anzaldua_1987}, may be widely acknowledged as existing, but since they are not American citizens, they have no legal or human rights in the context of the United States. It is because they have no rights that they are brutally taken advantage of. Major corporations exploit workers for long hours, minimal pay, and no benefits \parencite[12]{Anzaldua_1987}. Women are sexually assaulted, but, as with most of the other workers, they cannot speak English and therefore cannot look for help \parencite{Anzaldua_1987}. Mostly, these abuses go undocumented \parencite{Anzaldua_1987}.
	
	Anzaldúa’s call for more mediators \parencite[78]{Anzaldua_1987}  is exactly right. But while she is looking for ways to remedy the exploitation of those living in the third space, she is herself making it exclusive. As a queer female of mixed-race, she can “walk out of one culture and into another” \parencite[77]{Anzaldua_1987}, which allows her to relate to multiple cultures. But what does that mean for the rest of us who feel this issue is of the utmost importance and, while we may not have experienced it first-hand, want to fix the current situation? That being said, Anzaldúa highlights an important area that most of America is familiar with but does not have to deal with, where people easily fall through the cracks. This is where hybridity will lead to “a massive uprooting of dualistic thinking” \parencite[80]{Anzaldua_1987}, thus changing the core way we view others. In this context, the concept of hybridity would have to be understood by the larger dominant population in order to benefit the marginalised population. Ideally, if the greater part of America utilised the notion of hybridisation, the lives of those inhabiting the borderlands would radically change. 
	
%	\section{The Real Effects Hybridity Has Today}
	Whether\marginnote{The Real Effects Hybridity Has Today} called ‘hybridity’ or not, this idea plays an active role in people’s lives. While hybridisation has been a constant product of cultures encountering and interacting with one another, it has not always been unintentional. Active hybridisation was a tool used by colonists in the hopes of resetting the population to its default of Spanish or white \parencite[42]{Warren_1998}. By intermixing with indigenous people, the Spanish thought they could dilute or eliminate authentic indigenous cultural customs in favour of Spanish practices. As natives became biologically less ‘native,’ it was thought they would simultaneously become more culturally Spanish. 
	
	Today, authenticity is a key factor governments use to determine the legitimacy of an indigenous group.  To be seen as authentic, groups are forced to conform to traditional and often obsolete standards of what it means to be indigenous. Designating a group as ‘hybrid’ and therefore ‘untraditional’ or ‘inauthentic’ may result in a denial of rights. When hybrid becomes synonymous with inauthentic, it can be detrimental to native groups because they then, by definition, lack the continuity said to be a requirement of indigeneity. Some scholars avoid the term hybrid for this reason. Here, Bhabha’s notion of hybridity as a universal process becomes powerful because it helps break down notions of authenticity and purity. No group has stayed the same throughout time because all groups are hybrid and continuing to hybridise.
	

	Matthew Liebmann uses hybridity to address identities today as he works with resolving issues of repatriation. The Native American Graves Protection and Repatriation Act manages the repatriation of human remains, sacred objects, and cultural patrimony of objects to Native Americans in the United States. Repatriation can only occur if a tribe is recognised by the federal government and demonstrates a cultural affiliation with the items \parencite[74]{Liebmann_2008}. Cultural affiliation is defined as “a relationship of shared group identity which can be reasonably traced historically or prehistorically” between a present day Native group and an earlier group  \parencite[75]{Liebmann_2008}. This is a double edged-sword for indigenous identity. If a group admits they have changed in the past five centuries, they are denied recognition because they are not authentic. If they assert they have not changed, then the group is playing into the dominant group’s socially constructed notions of what it means to be Native, succumbing to yet another instance in which the “colonisers fix an identity to the colonised” \parencite[77]{Anzaldua_1987}. Forcing Native Americans to legitimise their existence in terms of what the dominant culture defines as indigenous is a double standard and demonstrates a lack of understanding of cultural change. The dominant culture’s demand that modern indigenous people portray themselves as unchanged before the arrival of the Spanish revitalises historical stereotypes. With the arrival of the Spanish, Native culture changed forever and became irreversibly less Native. Bhabha’s concept of hybridity, however, emphasises that all cultures have been changing throughout time. If both the dominant group and the indigenous group have changed culturally, the drive for contemporary Native groups to homogenise themselves with their past would be significantly reduced. As Bhabha puts it, “the first principle is always in the space of secondariness” \parencite[83]{Mitchell_1995}.
	
	In Guatemala, hybridity has been used to enact violence against people who identify as Maya \parencite[6]{Warren_1998}. During the 1980s, the government attempted to create a unified nation of Guatemala by suppressing any expression of Maya identity. Archaeology and its discourse surrounding the Maya became a weapon to target indigenous groups and support the notion that “true Maya culture consists of those features surviving from the precontact period” \parencite[13]{FischerBrown_1996}. In response, indigenous groups employed ‘strategic essentialism’ \parencite[11-13]{Spivak_1990} in order to gain political recognition, working within the structure that was marginalising them and using it to their advantage. Again, the utilisation of Bhabha’s hybridity here would be beneficial because it nullifies the idea that a pure race is possible, thus negating the hierarchical power structures that are in place.
	
%	\section {Is It Really Time To Move Away From Hybridity?}
	In a talk\marginnote{Is It Really Time To Move Away From Hybridity?} last October, Stephen Silliman asked the question, “Is it time to move away from hybridity?” \parencite{Silliman_2014}. He asserted that hybridity may void all uses in relation to culture because it is being used too freely. While Silliman is right that hybridity is being used too freely, the issue lies in scholars misunderstanding of how it can be used effectively. Deagan, Silliman, and Young all present cases in which the term ‘hybridity’ is not used in the way it was defined by Bhabha. Deagan fails to analyse Catholic and Native interactions as a process and neglects to interpret the site in terms of the third space \parencite[267]{Deagan_2013}. McClintock raises the idea that Bhabha himself has negated his own notion of hybridity by focusing on culture and failing to discuss hybridity in terms of gender and class. Bhabha has reduced something, which, according to McClintock, as a hybrid cannot be separated from both class and gender \parencite[62]{McClintock_1995}. While Silliman believes that hybridity “would never be applicable in cases of cultural continuity… [it can] only be used for cultural change” \parencite[489]{Silliman_2013}, the process of hybridisation is continuous because all cultures are constantly changing. 

	Silliman raised many concerns surrounding the use of the term hybridity, concluding his talk with the question of whether hybridity deserved a “revision or a requiem” \parencite{Silliman_2014}. To this I answer that Bhabha’s work deserves a second, or maybe third and fourth, reading. Silliman did not propose an alternative to hybridity. If we abandon this notion, what will fill its gap? As presented in the cases above, hybridity is a daily reality for individuals in Guatemala, at the border of Texas and Mexico, and those most impacted by NAGPRA. To treat cultural groups as essential entities rather than as hybrids obscures the fact that these individuals are constantly adjusting to new influences as they interact with a dominant culture that is itself changing.

	The absence of hybridity would potentially lead to the return of colonial binaries and traditional views of essentialism. The three cases cited above serve to demonstrate that Bhabha’s culture theory is based in day-to-day realities for many marginalised groups. Turning away from trying to understand hybridity is not doing justice to many individuals who have historically been subjected to prejudice.
	

	
\printbibliography[heading=subbibnumbered] 
\label{Sheridan:lastpage}
\closingarticle

