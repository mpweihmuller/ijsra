\documentclass{ijsra}
\def\IJSRAidentifier{\currfilebase}%<<<< DO NOT change this line
\def\shorttitle{Review of Grauer, A Companion to Paleopathology}
\def\maintitle{Review of \emph{Grauer, Anne L. (ed.) 2012: A Companion to Paleopathology. New York: Wiley-Blackwell}}
\def\shortauthor{Michael Rivera}
\def\authormail{mbcr2@cam.ac.uk}
\def\affiliation{Department of Archaeology and Anthropology, University of Cambridge}
\def\thanknote{Michael B. C. Rivera (\href{http://www.cambridge.academia.edu/MichaelRivera}{www.cambridge.academia.edu/MichaelRivera}) is a PhD student at the University of Cambridge conducting research on bioarchaeology and coastal adaptations in the prehistoric Baltics (between 5,400\BC–450\AD), investigating aspects of activity, diet and disease during cultural transitions. Michael is a Reviewer for the International Journal of Student Research in Archaeology and a co-organizer of the 4th Annual Student Archaeology Conference (the proceedings of which will be published in the next issue of the IJSRA)}
\def\keywords{}
%\def\keywordname{}
\begin{filecontents}{\IJSRAidentifier.bib}
@book{Ortner2003a, %< bibtex-key
    author = {Ortner, D. J.},
    title = {Identification of pathological conditions in human skeletal remains},
    publisher = {Academic Press},
    location = {San Diego},
    year = {2003},
    }
    
@incollection{Ortner2003b,
    author = {Ortner, D. J.},
    title = {Background data in paleopathology},
    pages = {37--44},
    editor = {Ortner, D. J.},
    booktitle = {Identification of Pathological Conditions in Human Skeletal Remains},
    publisher = {Academic Press},
    location = {San Diego},
    year = {2003},
    }

@incollection{Ortner2003c,
    author = {Ortner, D. J.},
    title = {Theoretical issues in paleopathology},
    pages = {109--118},
    editor = {Ortner, D. J.},
    booktitle = {Identification of Pathological Conditions in Human Skeletal Remains},
    publisher = {Academic Press},
    location = {San Diego},
    year = {2003},
}

@article{Ortner2011,
        author = {Ortner, D. J.},
                               title = {Human skeletal paleopathology},
   pages = {4--11},
   journaltitle = {International Journal of Paleopathology},
   volume = {1},
   year ={2011},    
    }

@incollection{Robb2000,
    author = {Robb, J.},
    title = {Analysing human skeletal data},
    pages = {475--490},
    editor = {Cox, M. and Mays, S.},
    booktitle = {Human Osteology: In Archaeology and Forensic Science},
    publisher = {Cambridge University Press},
    location = {New York},
    year = {2000},
    }

@book{Roberts2007,
    author = {Roberts, C. and Manchester, K.},
    title = {The archaeology of disease},
    publisher = {The History Press},
    location = {Stroud},
    year = {2007},
    }

@book{Waldron1994,
    author = {Waldron, T.},
    title = {Counting the dead: the epidemiology of skeletal populations},
    publisher = {Wiley},
    location = {New York},
    year = {1994},
    }

@book{Waldron2009,
    author = {Waldron, T.},
    title = {Palaeopathology},
    publisher = {Cambridge University Press},
    location = {New York},
    year = {2009},
    }

@book{Wells1964,
    author = {Wells, C.},
    title = {Bones, bodies and disease},
    publisher = {Thames and Hudson},
    location = {London},
    year = {1964},
    }

\end{filecontents}

\begin{document}
\IJSRAopening%<<<< DO NOT change this line
	{\Large\scshape
	\shortauthor}%
	\footnote\thanknote%
	\\[1em]
	\email\\
	\affiliation
\IJSRAmid%<<<< DO NOT change this line

%\begin{IJSRAabstract}
%Abstract
%\end{IJSRAabstract}

\lettrine[nindent=0em,lines=3]{P}{aleopathology} is an interdisciplinary science concerned with the origins and evolution of disease, and how diseases may manifest themselves on ancient human remains. 
For any budding archaeologists, this Companion exists as a useful resource for any students interested in learning how diagnoses of disease are made using bones and teeth and related back to past civilizations and environments.
This volume succeeds in complementing other published works such as \textit{Human Paleopathology} \parencite{Ortner2003a}, \textit{The Archaeology of Disease} \parencite{Roberts2007} and \textit{Palaeopathology} \parencite{Waldron2009}, which is a goal set out by the editor in her introductory chapter.
The book manages to offer fresh and unique perspectives on how approaches to studying ancient disease have become more interdisciplinary over the last 25 years, and how interpretations of well-known diseases have been re-evaluated in light of recent technological advances and the exciting results they yield.

This volume is edited by Dr. Anne L. Grauer, who serves as Professor and Chair of the Department of Anthropology at Loyola University Chicago.
She specializes in the study of human morbidity, mortality and ancient disease, and the social environments that influence disease prevalence.
Her expertise, combined with that of the group of specialists she has assembled as contributors, has made it possible for this Companion to cover considerably broad scope, addressing a range of theoretical, methodological and historical issues embedded in this field and surrounding its sister disciplines.
While comprehensive, this volume is also easily digestible to bioarcheologists and non-specialists alike, serving as a good primer for how paleopathological research can contribute a wealth of meaningful information to archeological investigations.
Chapters 2 and 3, focused on ethical issues and the ‘biocultural approach’ respectively, give good introductions into the historical evolution of the discipline as well.
They discuss the history of paleopathology, spanning from pre-1960s bioanthropology, with its problematic typological and racial classification of human remains, to today’s paradigmatic shifts towards more ethically conscious research focusing on the interaction between past humans and environments.

Three parts make up this book: the first of which offers broader insights into the history and potential future of our field, and how methods used in paleopathology have diversified in the last few decades.
The latter is due in no small part to the partnering up of paleopathology with other disciplines like molecular biology (chapter 5), ecological isotope studies (chapter 6), ancient DNA studies (chapter 8), and parasitology (chapter 10).
In particular, Chapter 4, titled “The Bioarchaeological Approach to Paleopathology”, tells readers how paleopathological investigation can lead to discoveries about the social roles of archeological people, the food availability and livability of the environments around them, their experiences of health and disease, their activity and migratory patterns, and their propensity to suffer from fractures or enjoy pre-modern forms of medical treatment.
In this chapter, Michele R. Buzon of Purdue University writes about how archeological investigations have often ignored or dismissed paleopathology as a helpful avenue of inquiry.
Dr. Buzon stresses the utility in having an osteological expert on archeological research teams for comprehensive data recording and more holistic interpretations of previous human experience, combining paleopathological information with evidence gathered by other subdisciplines.

The second part of the \textit{Companion} reviews various methods and technologies used in paleopathology, most notably how experts center their research design upon the theoretical framework known as ‘differential diagnosis’ (chapter 14), and collect, interpret, store and present their data in a useful manner (chapter 19).
Other chapters in Part II also discuss the advantages and shortcomings of age and sex estimation methods (chapter 15), and how newer techniques in histology (chapter 13) and radioimaging (chapter 18) have aided in recent archeological studies of past health. 
Diseases differentially affect morbidity and mortality within a population and the affected individuals’ abilities “to function effectively in the biocultural environment” (Ortner, chapter 14).
Therefore, students who will undoubtedly come across case studies and reports could benefit from these chapters, as they provide important emphases on the need to search for replicability and transparency in the methodology sections of papers, to recognize any technical and interpretational limitations in their differential diagnoses, and to seek ample discussions by the authors on the \textit{significance} of a disease in a past society.

Deeper insights into specific disease conditions are provided in the remaining chapters in Part III, including (but not limited to) signs of trauma and violence in the past (chapter 20), the study of dietary deficiency disorders like scurvy, rickets and anemia (chapter 22), tuberculoid infections (chapter 24), leprosy (chapter 25) and treponemal diseases (chapter 26), the quantification and qualification of activity-related changes to the skeleton (chapter 29), and what dental paleopathology can teach us about dietary health and subsistence patterns in the past (chapter 30).
Each of these chapters reviews the previous literature, explores the intricacies of their diagnoses, the technologies and protocols utilized to record the relevant lesions, and what genetic, molecular and clinical studies reveal about the evolution of these diseases.
They also touch on the significance of such diseases within greater narratives of human history (\textit{e.g.}, as in the onset of widespread agriculture at the start of the Holocene, or the fifteenth-century interactions between Old World and New World people, among other historical, socio-political and economic contexts).

One crucial aspect of a paleopathologist’s work is to combine different forms of evidence to design a differential diagnosis about disease, and to comment on its relevance to the greater archaeological setting.
While Chapter 14 (“Differential Diagnosis and Issues in Disease Classification”) illustrates well what constitutes useful descriptions of bony abnormalities, only a small sub-section of the chapter is devoted to how an archeologist might paint an overall picture of the level of health in a person or population.
For this, \textcite{Waldron1994}, \textcite{Robb2000} and \textcites{Ortner2003b}{Ortner2003c}{Ortner2011} are recommended readings as complements to this compendium. They discuss the objectives of paleopathology at length, and the theoretical issues that may afflict their successful integration into archeological and forensic investigations.

Common throughout these many authors’ contributions is the expressed need to account for the unique environmental and cultural contexts surrounding a particular disease expression, and to consider the inter-individual differences in disease experience among various ages, gender categories, and socio-political and economic standings within any given society.
As \textcite[18]{Wells1964} puts it, “the intricate relationship between a people’s way of life and the diseases they endure is the chief reason for the study of paleopathology.”
With the ‘globalization’ of scientific workers and information, and an ever-increasing flow of ideas through papers, conferences and internationally collaborative projects, this volume pertinently advocates for more holistic approaches to studying the past and greater consistency and comparability in the data recording procedures employed worldwide.
It also underscores the merits of appreciating regional histories and environments, an important message to all biological anthropologists studying modern and ancient humans, or with an interest in genetics, medicine or epidemiology.
The \textit{Companion} is also written for archeologists unaccustomed to working with human bone, but who wish to gain a better appreciation for the work put into archeological endeavors by osteologists.

\begin{IJSRAquote}{Mary Lucas Powell and Della Collins Cook\footnotemark}
The next generations of paleopathologists will discover ‘secrets of the past’ that we only dreamed of unlocking. As human beings, we are all infinitely capable of enlightenment, and the future of paleopathology is, in a sense, just beginning.
\end{IJSRAquote}
\footnotetext{Chapter 12 (“How Does the History of Paleopathology Predict Its Future?”)}
\IJSRAclosing%<<<< DO NOT change this line
\end{document}
