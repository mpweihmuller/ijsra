% !TEX root = ../auf-dem-forum.tex

\addchap{Zusammenfassung (deu/eng)}
% \pdfbookmark[1]{Zusammenfassung}{post_Zusammenfassung}
% \pdfbookmark[2]{Auf dem Forum}{post_AufdemForum}

\label{formalia:abstract}
\addsec{\thesisTitle}%\thesisTitleaddon
Auf dem antiken Forum kulminierte das öffentliche, soziale Leben,
welches geprägt war
von einer pragmatischen Auseinandersetzung mit der den Platz umgebenden Architektur,
so die Grundannahme  für diese Arbeit.
Diese  sozio"=kulturelle Funktion von Architekturen wird an (ausgewählten) Platzanlagen der italischen Halbinsel in der späten Republik und frühen Kaiserzeit untersucht,
indem die Pragmatik der architektonischen Platzgestaltung in den Vordergrund gerückt wird, \idest die Art und Weise, wie die  bauliche Struktur der Fora Einfluss auf die konkrete Benutzung des Raumes nahm.

%Methode
Als methodische Annäherung wird die von \citeauthor{Lefebvre_2000} postulierte Raumdreiteilung gewählt und mit den Ansätzen der Architektursoziologie erweitert.
Damit gliedert sich die Arbeit in die Bereiche
Repräsentationsraum (Nutzungen, \cref{part:repraesentationsraum}),
Raumrepräsentation (Konzeption, \cref{part:raumrepraesentation})
und Raumpraxis (Wahrnehmung, \cref{part:raumpraxis}).

%Platznutzungen
Um einen Eindruck an möglichen Platznutzungen zu gewinnen,
werden diese aus literarischen, epigrafischen und bildlichen Quellen ermittelt.
Dabei zeigen sich neben verschiedenen vereinzelt überlieferten Ereignissen
auch Ereignisse, die in unterschiedlichen Medien mehrfach überliefert sind.
Einen sehr prominenten Platz nehmen dabei Gladiatorenkämpfe auf Fora ein.
% Weitere Nutzungsszenarien ergeben sich wurde ebenfalls Markt abgehalten,
% Recht gesprochen,
% Reden gehalten,
% Abstimmungen vorgenommen.
Bei diesen Nutzungsszenarien waren verschiedene temporäre Bauten notwendig,
um auf individuelle und stadt"=verschiedene Bedürfnisse zu reagieren.

%Vitruv als steile These
Anhand einer Textpassage aus \citeauthor{Vitr}s \citetitle{Vitr} wird diskutiert,
ob in einer antiken Konzeption von Platzanlagen bereits auf die temporäre Ausgestaltung Rücksicht genommen wurde.
%Lochbefunde und Statik
Den archäologischen Beleg für eine temporäre Nutzung liefern Löcher,
die sich besonders auf italischen Fora erhalten haben.
Diese Befundgruppe steht im Zentrum der Arbeit und wird daher aus verschiedenen Richtungen beleuchtet.
Nach einer typologischen Einteilung und der
 Ermittlung des statischen Potenzials sind die Lochbefunde Grundlage für 3D"=Rekonstruktionen.
%Rekonstruktionen
Besonders für die Fora von Paestum, Cosa und Alba Fucens werden verschiedene temporäre Bauten und Nutzungsszenarien erschlossen,
die die Lochbefunde berücksichtigen.
Auf allen drei Fora sind Rekonstruktionen von \enquote*{Wahlbrücken} (\emph{pontes}), Podeste und Tribünen möglich.

%Wahrnehmungen
Im letzten Schritt werden die eruierten Holzarchitekturen auf eine \enquote*{raumgreifende Wahrnehmung} hin untersucht.
Fokussiert wird auf die Aspekte von Bewegung (Kinästhetik), Hören/Sehen (Akustik/Visualität) und  klimatisches Spüren (Taktilität).
Es zeigt sich dabei, dass die temporären Bauten mitunter an Orten rekonstruiert werden können,
die gegenüber permanenten oder traditionell gewählten Orten größere Hör- und Sichtbarkeitsbereiche  ermöglichen.
Das Forum erfährt dadurch alternative Nutzungsmöglichkeiten.

Die Erkenntnisse aus der Arbeit sind, dass Nutzungsstrukturen über einen sehr langen Zeitraum beibehalten wurden,
wobei die allmählich gewonnenen Erfahrungswerte zu einer verbesserten Statik der Bauten führten.
Änderungen im Befund der Lochstrukturen korrelieren mit gesellschaftlichen (nicht notwendigerweise politischen) Umstrukturierungen,
sodass neu geschaffene Architekturen Funktionen der Fora übernahmen.
Die Lochbefunde sind damit auch Zeugnis einer allgemein am Ende der Republik zu greifenden funktionalen Ausdifferenzierung von Architektur und Gesellschaft.

\addsec{On the Forum}

\begin{foreignlanguage}{english}
Public, social life in the ancient city culminated on the forum: that this was facilitated by a pragmatic appropriation of the built architectural space surrounding the forum forms the underlying foundation of this thesis. This socio"=cultural function of forum architecture is analyzed for (selected) cities of the Italian peninsula of the Late Republic and Early Imperial period. The focus is on the utilitarian dimension of the built structures of the public space – that is, how the architectural configuration of fora influenced the parameters of its use.

The methodological approach followed here is Lefebvre’s threefold definition of space.
Accordingly, the thesis is structured into three, analyzing the spatial imaginary and its use \pcref{part:repraesentationsraum},
the representations or theoretical conceptions of space \pcref{part:raumrepraesentation},
and its everyday practices and perceptions \pcref{part:raumpraxis}.

In order to determine what functions and uses are attested for public space in antiquity, literary, epigraphic and visual sources were taken into consideration. In addition to those activities attested to in the individual sources, several activities are corroborated by multiple sources, chief among them gladiator games. The variety of activities recorded shows that the construction of temporary structures was necessary in order to meet these different, specific spatial demands.
Using the evidence provided by \citeauthor{Vitr}s \citetitle{Vitr}, whether or not temporary constructions were considered during the conception of public spaces in antiquity is evaluated.

Pits and Structural Considerations:
The main archaeological evidence for the construction of temporary structures is provided by the discovery of series of systematically positioned pits on Italian fora. Here, the pits are classified and used as the foundation for structural calculations in order to reconstruct the various architectural possibilities as 3D models.

Reconstructions:
Probable temporary constructions are reconstructed based on the evidence of pits for the fora of Paestum, Cosa and Alba Fucens. On each of these fora, the reconstruction of \enquote*{electoral bridges} (\emph{pontes}), landings and stands are possible.

The impact on the sensory perception of these spaces as a result of the temporary wooden structures thus obtained are then analyzed, with a focus on movement, acoustics, sound, and climate. The acoustic and visual perception for those spaces where temporary structures can be reconstructed is improved in comparison with the permanent structures. The forum itself is thus characterized by differing strategies of use.

These strategies continued to be in use over a long period and, with time, the experience gained led to structural improvements of the temporary constructions. Changes to the pits can be correlated to social (and not necessarily political) reorganizations, whereby new buildings began to assume the functions of the fora. The pits therefore provide evidence for an increase in functional differentiation within architecture and society in the Late Republic.
\end{foreignlanguage}
