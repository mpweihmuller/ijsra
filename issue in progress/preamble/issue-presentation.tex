\def\IJSRAidentifier{\currfilebase} %<---- don’t change this!
%-------Title | Email | Keywords | Abstract-------------
\def\shorttitle{Issue Presentation}
\def\maintitle{Presentation of the fourth issue of IJSRA}
\def\cmail{gonzalo.linaresmatas@st-hughs.ox.ac.uk}
%--------Author’s names------------
\def\authorone{Gonzalo Linares Matás}
%-------Biographical information-------------
\def\bioone{\href{https://oxford.academia.edu/GonzaloLinaresMatás}{\authorone}   is reading for an MSt in Landscape Archaeology and Molecular Bioarchaeology at St Hugh’s College, University of Oxford (UK). His research focuses on vertebrate taphonomy, with a special emphasis on canids, and Pleistocene zooarchaeology. He was the former President of the Oxford University Archaeology Society (Michaelmas 2015) and was invited to join the WAC (World Archaeology Congress) Student Committee (2017).

He is also particularly interested in the socio-political contexts of heritage management and ownership, contemporary archaeological theory, and the histories of the academic disciplines of archaeology and anthropology as practical modes of inquiry. Gonzalo is convinced that more efforts are needed to transform the academic publishing landscape.}
%------University/Institution--------------
\def\affilone{Executive Editor, International Journal of Student Research in Archaeology
\\St Hugh’s College, University of Oxford}


\IJSRAopening%<---- don’t change this!
%-------
\lettrine{G}{ood} things are worth waiting for. I am thrilled to be presenting the fourth issue of the International Journal of Student Research in Archaeology, the global, free and independent student publication in our field. This new issue embodies our culture and our ethos: full of interesting features and innovative research articles by authors from four continents, from the buried to the gravitating, from the perishable to the monumental, and from the prehistoric to the contemporary.

The potential to transcend the powerful forces ruling our physical world, such as gravity, have long prevented access to outer space, becoming a frontier for any hominin species, including our own. However, outer space is now a daily sphere of human interaction littered with the traces of our activities. Through an encompassing view of archaeology, combining perspectives from the archaeology of the present and the future, astrophysics, as well as some insights that evoke Material Engagement Theory, outer space is now firmly grounded within the realm of the discipline, even if it is still somewhat peripheral in practice. \textbf{John Vandergugten} moderates a fascinating thematic interview forum with leading practitioners of the subdiscipline, exploring the potentials and limitations of archaeological research in a new domain of physical experience, and engaging with the astrosocial.

Surveying is an important dimension of archaeological discovery. It enables us to identify new sites and to characterise the type and density of archaeological features within a landscape. Surveys generate large quantities of data that often may go unpublished, thus depriving the wider community of invaluable information at different scales of resolution. Therefore, we are delighted to publish a multi-period survey report by \textbf{Bradley Husemann} for the Schaub Family Farm (Illinois, USA). Test pits provide information about stratigraphies and site integrity, and the study area is nicely illustrated using both historic maps and GIS. This work is a relevant contribution to the local history of Peoria County and the regional archaeology of the Eastern Woodlands.

For the purposes of discussing the remaining articles in the issue, I have grouped them in two broad themes: representation and exchange, which merge the cultural and the social aspects of past and present human experiences. \textbf{Ephraim Mwaita} brings to the fore the role of heritage in preserving traumatic memories of our recent past, embodied in material culture displayed in institutional settings. Robben Island (South Africa) hosts the prison where Mandela was kept captive, but the museum also documents the untold histories of cruel suffering that many other people experienced. Mwaita argues that there is still room for more explicit and personalised accounts beyond a \enquote*{triumphalist} narrative of overcoming general hardship.

The veneration of family ancestors is one of the fundamental dimensions of many societies, and it is the topic of two great papers, one on Greek funerary art and another on ship burials in Scandinavia. \textbf{Joseph Robson} discusses 14 plaster casts of Attic funerary stelai held at the Ashmolean Museum in Oxford, exploring the dialectics of the oikos and the polis, between life and death, through public presentation and representation of the nuclear family, the quintessential element of the private sphere in the Classical world. Thus, death becomes an opportunity for the living to imagine and reconfigure their past and their future. However, in certain circumstances, the passage is not always conceptually one-way, and the dead can use the same symbolic and material channels to re-engage with the living. \textbf{Cassandra Clark} discusses how ship-themed burials in Scandinavia, documented from the Mesolithic to the late Viking Age, are not merely symbolic vessels that transport the dead from one world to the next; instead, they may be seen as mechanisms that allow them to depart from and return to the realm of the living. In the context of this ritual ideology, fostered through feasting, the death would not be confined within the limits of a mound, as the burial ground granted them the possibility to travel through the cosmic sea.

Exchange is a paramount socio-economic mechanism, connecting the local and the global, and acting as agent of change while at the same time maintaining and renovating social ties. In the present article, we have two papers that explore the role of trade in territories located on the shores of the Indian Ocean, a dynamic arena of intercultural interaction. \textbf{Ashwini Lakshminarayanan} looks at the material evidence of Indo-Roman trade from the Arikamedu assemblage, the renowned port site studied by the UCL archaeologist Sir Mortimer Wheeler. Amphorae and beads evoke the long peripli of courageous sailors that navigated back and forth with the monsoonal tides, nodes of an extensive network that connected the Mediterranean world, the East African coast, the Arabian Peninsula, South Asia and beyond. In the context of this influx, rich polities developed in the Shashe-Limpopo basin and the Zimbabwe Plateau through the trade of gold, ivory, slaves, glass beads, and textiles, as well as other perishable materials.
\textbf{Humphrey Nyambiya} explores how these centres were involved in several spheres of interaction, from local exchanges with hunter-gatherer communities to long-distance trade; these redistributions of wealth fostered lavish burials and monumental stone structures way before the early colonial encounters from the sixteenth century. Nevertheless, the consequences of booming trade in urban centres are not universally welcomed. They are also associated with processes of gentrification and displacement of local communities, which in turn affects the goods on demand and the products on offer.
The anthropological significance of the heritage associated with these processes is discussed by \textbf{Zach Lindsey} in relation to the transformation that a traditional grocery store experienced in Austin, Texas, during the twentieth century. He also emphasises the role of oral and written sources for a multi-layered archaeology of the contemporary world, where multiple living actors can provide significant inputs in relation to the conceptualisation and interpretation of material culture.


This fourth issue also includes two conference reviews that took place during 2017: the ICOMOS International Committee on Archaeological Heritage Management (ICAHM) 2017 Annual Meeting and the 7th Annual Meeting of the European Society for the Study of Human Evolution, which took place in Leiden. The dates of these conferences highlight some of the delays associated with this particular issue, which was expected to appear before the summer of 2018. I would like to thank the unwavering support and commitment demonstrated by the formatting team, which day by day make this dream possible. The quality of the articles received continues to increase as the Journal establishes as one of the landmarks of student research in the field. Ahead of the next issue, and in order to better contribute to the growth and global expansion of this endeavour, we are redeveloping our reviewing system to improve submission to acceptance timeframes, by streamlining the submission system and the organisational logistics. Some of our reviewers are getting their doctorates, so there are going to be several openings in our reviewing team. A new publishing landscape, where neither authors nor readers have to pay to access scholarly research in archaeology, is possible if we take the right steps. The International Journal of Student Research in Archaeology is among the efforts pushing for greater accessibility of archaeological research. If you are interested in helping with the reviewing process of the journal or with managerial duties, please do not hesitate to contact us.


\booltrue{nobib}%in case there are no references cited
\IJSRAclosing%<---- don’t change this!
