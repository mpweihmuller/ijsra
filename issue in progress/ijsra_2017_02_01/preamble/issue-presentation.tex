\def\IJSRAidentifier{\currfilebase} %<---- don’t change this!
%-------Title | Email | Keywords | Abstract-------------
\def\shorttitle{Issue Presentation}
\def\maintitle{Presentation of the Second Issue of IJSRA}
\def\cmail{gonzalo.linaresmatas@st-hughs.ox.ac.uk}
%\def\keywords{Research, Archaeology, ...}
%\def\keywordname{}%<--- redefine the name “Keywords“ in needed language
%\def\abstract{In his paper Jon is showing ...}
%--------Author’s names------------
\def\authorone{Gonzalo Linares Matás}
%-------Biographical information-------------
\def\bioone{\authorone is a third-year undergraduate student reading the BA Archaeology \& Anthropology at St~Hugh’s College, University of Oxford (UK). 
He was the former President of the Oxford University Archaeology Society (Michaelmas 2015), and he has recently been invited to join the WAC (World Archaeology Congress) Student Committee. 
He is particularly interested in the socio-political contexts of heritage management and ownership, contemporary archaeological theory, and the histories of the academic disciplines of archaeology and anthropology as practical modes of inquiry. 
He is doing his undergraduate dissertation on the socio-economic dimensions of early bone technology, focusing on the late Early Pleistocene assemblage at the site of Cueva Negra del Estrecho del Río Quípar (Murcia, Spain). 
He is also very interested in transforming the academic publishing landscape.}
%------University/Institution--------------
\def\affilone{Executive Editor, International Journal of Student Research in Archaeology
\\St. Hugh’s College, University of Oxford}


\IJSRAopening%<---- don’t change this!
%-------
\lettrine{I}{am} very excited to present you the contents of our second issue, which has been the result of the joint efforts of all the members of our global IJSRA team. I am very grateful to our reviewers and authors for the trust and support given to our initiative.
We are delighted to have had the opportunity to talk with Prof. Chris Stringer (Natural History Museum, London). In his insightful conversation with our team member Hannah Ryan (Oxford), he explores how important is for disciplines such as archaeology and anthropology to communicate to prospective students about what insights they can offer. He discusses in our interview the importance of new fieldwork for palaeoanthropology, and archaeology in general, given the gaps existing in our knowledge base, but also how paramount it is to continue revisiting new collections through new interpretive lenses and techniques. He argues that scientists should communicate “beyond a purely academic audience”, and concludes by encouraging undergraduates and students in general to “get their formatives ideas out there”, but warns them to be ready for critiques, while also keeping an eye on networking with peers. “Because we are human beings, it’s sometimes hard to admit that we might have got it wrong, but as scientists it’s our duty to do so!”. We wholeheartedly thank Prof. Chris Stringer and his team for their time and attention.

Archaeology is discipline based on a practical mode of enquiry. The teaching of archaeology in academic settings sometimes favours theoretical education over learning how to conduct fieldwork, despite the paramount importance of this skill for the professional development of archaeology graduates. Jane Fyfe, Sean Winter (University of Western Australia), and Ashleigh Murszewski (La Trobe University) present in their paper how undergraduate and postgraduate students themselves, by means of their local archaeology societies, could attempt to overcome these educational limitations themselves. They describe the opportunities and challenges -particularly post-fieldwork report writing- for supervised and structured fieldwork they faced as part of the design and implementation of their initiative at the Archaeological Society of Western Australia. Within the constraints of regional legislations, and together with the ethical and regulatory issues associated with working on Indigenous sites, I believe that Archaeological Societies at Universities across the world have the potential to consider this model and learn from their experiences, which included the active engagement of students in contemporary and community archaeology projects, or supervised laboratory work.

Indeed, original fieldwork is a key source of information for archaeological interpretation, and provides an opportunity for students to develop crucial research insights about the processes involved in archaeological thinking through practice. At the same time, it is an irreversible process, and must be carried out carefully, recording thoroughly as many details as possible. Tim Forssman (University of the Witwatersrand) describes the occupation sequence of the Late Holocene site of Mafunyane Shelter, Eastern Botswana, and interprets the diversity of the assemblage, which includes stone tools, beads, animal remains, metal finds, and ceramics, as evidencing interactions between farming and forager communities in the Limpopo basin. The assessment of this hypothesis is carried out by means of the study of site utilisation patterns through spatial analysis. The paper also presents a revised chronological sequence of the site. At IJSRA we are looking forward to featuring more original research in archaeology from all over the world. 

Archaeological research and historical accounts offer insights into how our attitudes towards fundamental dimensions of human experience are plural and changing through time. Alba Pereda (University of Cambridge) combines material and documentary evidence to explore how the intersections between health and religion were perceived, instituted, and experienced in medieval England. The three case studies considered are fundamental sites for understanding the socio-economic role of medieval hospitals: St Leonard’s Hospital (York), one of the largest and wealthiest medieval hospitals in northern England; St Bartholomew’s Hospital (London) the oldest standing hospital in England; and St Mary Spital (London), one of the best-excavated ones. She argues that medieval hospitals were predominantly religious institutions associated with a church and the clergy, and healing had somatic and spiritual dimensions. Thus, medicine and worship shared the same rationality as part of a common explanatory framework of reality.

Archaeological remains can help create and consolidate these shared cognitive frameworks, and coins, for example, are particularly effective for conveying powerful ideological messages. Numismatic analyses need to consider the historical and cultural contexts in which these material currencies are developed. Rahul Raza (University of Oxford) argues that elephant imagery was invoked as an ideological representation of royal power in Bactria and India during the Hellenistic period. These kingdoms experienced the confluence of multiple conceptualisations of rulership, and had to device suitable platforms through which their message could reach the multiplicity of ethnic groups and social classes that composed their populations. He further argues how important it is for researchers to challenge the pervasive interpretive bias to overemphasise Hellenistic cultural practices, in the consolidation of the meaning and function of iconographic representations in these coins; local ruling traditions need to be taken more into account. Coinage, as an invaluable source of information, provides insights into the representations of the legitimising efforts of kings and elites to justify and preserve their privileged position of control and domination. 

The intersection between politics and heritage is therefore paramount. The interpretation and perception of the past is intrinsically linked with its conceptualisation in the present. National legislations form a corpus which embody the different public perceptions and definitions of what constitutes heritage. Rob Rownd (University of the Philippines) comparatively explores how prehistoric archaeological heritage is portrayed in the Republic of the Philippines’ National Cultural Heritage Act of 2009 and the Council of Europe’s Treaty of Valetta, formally the 1992 Convention on the Protection of Archaeological Heritage. Nonetheless, the author argues that heritage law is most effectively implemented following the boundaries of cultural landscapes, rather than through the frontiers of nation-states. The development of shared set of standards and expectations through region-wide policies, and particularly the universal definitions of supranational organisations such as UNESCO, ought to consider the diversity and the particularities of local experiences.
At the same time, the macro-scale nature of analysis based on processing large databases can provide interesting trends and patterns that may not be easily perceived from individual assemblages or case-studies. Dylan Davis (SUNY Binghamton) revisits Games in Culture by combining Big Data and worldwide evidence of chance games, thus offering interesting perspectives about cross-cultural dimensions of the histories of social engagements with the materiality of chance and fortune, and their association with morality, religiosity, and belief systems. The incorporation of ethnographic research and anthropological insights has the potential to help us better understand the nature and the diversity of archaeological remains and past human experiences.

\IJSRAseparator

\IJSRAsection{Literature Review}
A fundamental dimension of the research process is to critically assess the literature published about any given topic. Sarah Gyngell (Sydney University) reviews the debate around the notion of sedentism of Natufian communities in the Levant on the basis of the available material culture of housing and economic practices. The contributions of different authors to our understanding of the nature of the transition process from nomadic to sedentary evidence the existence of a whole mobility spectrum, with different levels of settlement permanence. Sarah suggests that the publications by authors such as Edwards and Shewan in particular support the conceptualisation of Natufian as relatively mobile communities, in contrast with the predominant position in favour of sedentism present in earlier studies. Sarah argues that an appreciation of the diversity of engagements and strategies of site occupation is fundamental for understanding and successfully undertaking further research.


\IJSRAseparator
\IJSRAsection{Conference Reviews}
The 2016 edition of the American Association of Physical Anthropologists took place in Atlanta (Georgia, USA), in April 2016. 
Devin Ward (University of Toronto), Michael Rivera \& Jaap Saers (University of Cambridge) focus on the student experiences at this large, international forum. Particularly interesting were the Undergraduate Research Symposium (URS) and the multiple student prizes awarded, which encourage student participation and engagement in the presentation of research at the AAPA Conference. Furthermore, they highlight how live-tweeting of presentations created many positive interactions and lively discussions beyond the realm of the Conference. In my opinion, this opening up of the Conference spaces to much wider audiences through social media seems to be one suitable channel for broadening public engagement with research-based disciplines. Besides, they note how important is the acknowledgement in such a forum of issues such as sexual harassment in academia, in order to devise strategies to tackle them more effectively. It is paramount that AAPA and other international public forums continue to support and encourage student participation.

For the past three years, the Centre for the Study of the Near East and Late Antiquity (CEPOAT) has been organising the International Congress of Young Researchers of the Ancient World (CIJIMA) at the University of Murcia (Spain). Consuelo Isabel Caravaca Guerrero \& Dámaris López Muñoz (University of Murcia) introduce the III edition of this forum: 125 people were in attendance, and 65 presentations were given over the two-day conference celebrated in April 2016. The Conference was organised around themed series about Egypt, Mesopotamia, Greece, the Iberian Iron Age, and Rome. A cross-cultural theme was the study of gender in the Ancient World. Most presentations were given in Spanish, but the organisers are looking forward to featuring even more international presentations in their future editions.

Sarah Scheffler, Andy Lamb, Rachel Wilkinson (University of Leicester) introduce the 19th Iron Age Research Student Symposium (IARSS). It was held at the University of Leicester between 19th–21st May 2016, in partnership with the Universities of Nottingham and Birmingham. They note that the IARSS series has become the longest running conference for Iron Age research students in the UK. Nonetheless, this is not incompatible with annual improvement and the addition of new features, and this edition, which gathered 53 delegates, included a program of workshops designed to introduce and develop skill sets fundamental for post-graduate researchers. Furthermore, the 2017 edition will be held at a continental institution. IJSRA is looking forward to working with IARSS in their future editions.

Rebekah Hawkins \& Francesca McMaster (University of Sydney) present a review of the 8th edition of the World Archaeological Congress, which was held in the Japanese city of Kyoto, during the 28th August – 2nd September 2016. They highlight how WAC President Koji Mizoguchi outlined the many barriers that non-English speaking people face when having to convey their presentations in English, sometimes eroding their confidence and research identity, emphasising the plurality of practice of our discipline. WAC aims to embody this academic commitment to inclusivity in terms of engagement with archaeology throughout the world, particularly Indigenous people. The global dimension of research featured prominently, and multiple voices resulted in a somewhat overwhelming number of sessions running at the same time. The next Congress (WAC-9) will take place in Prague, Czech Republic. The idea of a student-led review of WAC-8 was the result of a successful dialogue between IJSRA and the WAC Student Committee, with whom we are very excited to continue working in the future. I would like to thank Jacqueline Matthews, former Vice-Chair of WACSC, for her encouragement, and for considering our collaboration proposal.

The seventh edition of the Society for Medieval Archaeology Student Colloquium was held this past November 2016 in Brussels (Belgium), and was organised by Marit Van Cant (Free University of Brussels/the University of Sheffield). Sarah Kerr details how the SMA established the student colloquium “to create a platform for postgraduate students and early career archaeologists to present their research in a friendly environment, create networks with their peers across Europe, and learn about the wealth of research taking place in medieval archaeology”. The call for a new SMA Student Representative to organise the Colloquium will be opened in early 2017.  From IJSRA, we welcome and endorse these initiatives to showcase student research in any aspect of archaeology, and we are looking forward to engaging in future collaborations with the SMA Student Colloquium.

Our own Patricia Kenny (University College Dublin) details how the Irish Quaternary Association held its autumn symposium in Beggars Bush, Dublin (24th of November 2016). The title of the symposium was “Early Human Occupation of Ireland”, and the speakers covered a variety of topics relating to human occupation of Ireland from the Palaeolithic through to the Neolithic.

\IJSRAseparator

\IJSRAsection{Book Reviews}
Critical thinking when reading for assignments is a key dimension of the student learning experience. Offering the possibility of conducting book reviews for IJSRA opens up an important platform for encouraging students to engage in the exercise of this fundamental personal skill. Michael Rivera (Cambridge) transmits, in his Review of ‘A Companion to Paleopathology’, edited in 2012 by Anne Grauer (Loyola University of Chicago), a thorough awareness of the current issues contemporary paleopathological research faces. He notes how the books is divided in three parts: the first concerning the history and potential of paleopathology, the second focusing on its research methods and technologies, and the third series of chapters delve deeper into specific disease conditions. Michael Rivera notes how the volume highlights the importance of data recording consistency and comparability, and how holistic approaches should also aim to appreciate regional histories of disease in their environmental contexts.

\IJSRAseparator
As part of our constant efforts to reach as wide an audience as possible, we are always looking for people with diverse research interests to join our international team. If you are committed to improve the presence of excellent student scholarship in archaeology through our free, open-access, not-for-profit publication, please do get in touch!
\IJSRAclosing%<---- don’t change this!