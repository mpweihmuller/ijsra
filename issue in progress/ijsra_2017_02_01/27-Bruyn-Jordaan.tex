\documentclass[]{ijsra}
\def\IJSRAidentifier{\currfilebase} %<---- don’t change this!
%-------Title | Email | Keywords | Abstract-------------
\def\maintitle{Regional feature: Perspectives from southern African archaeology professionals}
\def\cmail{cherene.debruyn@yahoo.com}
\def\shorttitle{\maintitle}

%\def\keywords{Research, Archaeology, ...}
%\def\keywordname{}%<--- redefine the name “Keywords“ in needed language
%\def\abstract{In his paper Jon is showing ...}
%--------Author’s names------------
%--------Author’s names------------
\def\authorone{Cherene de Bruyn}  
\def\authortwo{Jacqueline Jordaan}%<---- comment or delete if you do not need a second author.
%-------Biographical information-------------
\def\bioone{Cherene de Bruyn is a 2016/2017 Chevening Scholar and a Masters student in Archaeology at the Institute of Archaeology, University College London. She was awarded her BA, BA Honours in Archaeology and BSc Honours in Physical Anthropology degrees from the University of Pretoria, SA. She has been part of teams working on various projects related to burials as well as Stone and Iron Age archaeology in SA, directed by local and international archaeologists.}
\def\biotwo{Jacqueline Jordaan is a PhD candidate in the department of Anthropology at the University of Manitoba. She received a M.A. from University of Pretoria, SA. Her studies concern southern African archaeology and heritage-making. She is involved with local and international efforts to promote student archaeological research and practice, as a past executive member of the Southern African Archaeology Student Council (SAASC) and as a current editorial board member for the International Journal of Student Research in Archaeology (IJSRA).}%<---- comment or delete if there is no second author.
%------University/Institution--------------
\def\affilone{Institute of Archaeology,
University College London, UK}
\def\affiltwo{Department of Anthropology, University of Manitoba}%<---- comment or delete if there is no second author.
%--------Mapping of authors to affiliations------------
%% authorone:--> * <--- copy/paste that symbol to \affiloneauthor etc. below
%% authortwo:--> † <--- copy/paste that symbol to \affiloneauthor etc. below
%% authorthree:--> ‡ <--- copy/paste that symbol to \affiloneauthor etc. below
%% authorfour: --> § <--- copy/paste that symbol to \affiloneauthor etc. below
%% authorfive: --> ¶ <--- copy/paste that symbol to \affiloneauthor etc. below
%-------------------------------------------------------------------------
\def\affiloneauthor{*}%<---- paste the symbol of the authors into {}
\def\affiltwoauthor{†}%<---- paste the symbol of the authors into {}

\begin{filecontents}{\IJSRAidentifier.bib}
@online{Pithouse2010,
author = {Pithouse, R.},
year = {2010},
title = {Universities for Bread and Roses},
organization  = {The South African Civil Society Information Service},
url = {http://www.sacsis.org.za/site/article/441.1},
urldate = {2017-04-07}
}

@article{Summers1995,
author = {Summers, R.},
year = {1995},
title = {A president looks back},
journaltitle = {South African Archaeological Bulletin},
volume = {50},
pages  = {175\psq}}
\end{filecontents}
\IJSRAopening
%-------
\lettrine{S}{outhern} African archaeology is diverse in topic and practice. From human origins to the historical period, archaeology in this region provides a kaleidoscope of life ways over a vast amount of time. From colonial roots, followed by apartheid restrictions, to current initiatives to decolonise the discipline, the practice of archaeology is undergoing transformation. Southern African archaeology is facing challenges. These challenges are protecting local heritage and sites, addressing the claims of various stakeholders, connecting with non-academic audiences, and developing and retaining archaeologists, by no means an exhaustive list or challenges entirely unique to southern Africa. We invited archaeologists to respond to a questionnaire about the state of archaeology in southern Africa, and the impact student led initiatives have on the discipline. 

These individuals have aimed to contribute to the discipline in different ways. Embedded in the tales of their achievements and experiences are some of the challenges faced by both young academic and professional career archaeologists in southern African archaeology. Jeanette Deacon, shares her initiation into the discipline, a story studded with key individuals in the establishment of southern African archaeology. Shadreck Chirikure, describes how he was seduced into archaeology, away from a more conventional degree with higher career employment, and with unyielding optimism shares his hope for the future of the discipline. Gavin Whitelaw, provides a realistic account of career prospects in the discipline and encourages extending our audience base. Ndukuyakhe Ndlovu, a strong advocate of transformation, developing and supporting of indigenous archaeologists and locally relevant research, shares his journey as an African practicing archaeology. On the other hand, Peter Mitchell, provides a well-informed outsider’s perspective to the challenges and practice of southern African archaeology. Lastly Kenneiloe Molopyane, a PhD candidate from the University of Witwatersrand provides a young professional perspective on southern African archaeology.

Archaeology is a changing discipline, with more academic journals, conferences, student-led research and public archaeology programs than a few decades ago. As part of this change it is also hoped that those conducting the research reflect the local stakeholders and interpretations of our shared human past. This feature presents personal accounts of people who followed their passion, though this is regionally framed, it is an experience shared across the archaeology community. As much as we endeavoured to represent a range of voices, we readily admit that the feature is dominated by male professional archaeologists. In this light, we propose future features that amend this bias and satisfy a broader audience, presenting the views of young career and professional archaeologists in both Academic and Cultural Resource Management environments. Thus we call upon students to confront the challenges faced by archaeologists in both local and global contexts, to create an open dialogue between academics and CRM professionals, making this discussion accessible through the IJSRA platform. This leads us to hope, that our diverse audience can find inspiration, solidarity, or some form of comfort in our shared experiences and challenges within Archaeology and that through engaging with these challenges we can contribute to a more diverse, transformed, and open archaeology.

\bigskip
\IJSRAsection{Abbreviations}
Commonly used Abbreviations:
\begin{labeling}{ASAPAXX}	
\item[SA] South Africa 
\item[ASAPA] Association of Southern African Professional Archaeologists
\item[CRM] Cultural Resources Management
\item[IJSRA] International Journal of Student Research in Archaeology
\item[SAHRA] South African Heritage Resources Agency
\item[UCT] University of Cape Town 
\item[WITS] University of Witwatersrand.
\end{labeling}

\IJSRAseparator
\IJSRAsection{Shadreck Chirikure}
{\sffamily Shadreck Chirikure, a PhD graduate from University College London, UK, is an associate professor at University of Cape Town, SA. He runs the Archaeology Materials Laboratory at UCT, the only facility dedicated to the study of African indigenous technology. He sits on the advisory boards of numerous journals and advocates practices that bring archaeological knowledge to the public (see his tedtalk on African indigenous technology).}

\begin{labeling}{IJSRA}	
\item[IJSRA] \emph{What drew you to archaeology and what path did you follow to become an archaeologist? }
	
\item[Shadreck Chirikure (SC)] I got into archaeology through chance. My desire was to work in finance but my A level points were not enough for me to embark on my dream career. I was then given a Bachelor of Arts programme specialising in Archaeology, History and Economic History. I remember liking Economic History a lot, largely because of the economic prefix! Things changed however that by my fourth year I was beginning to like Archaeology. I had to pull out of a competition to become a trainee banker because I got a scholarship to study a Master’s in Archaeology (Artefact Studies stream) at the Institute of Archaeology, University College London.


\item[IJSRA] \emph{Archaeologists usually have at least one story to tell around the campfire about their career or fieldwork experiences, do you have one to share with us?}
	
\item[SC] Mine is short. I liked finance to the extent that after graduating with a PhD, I enrolled for an honours in Finance. I then realised archaeology was much more than that, and so I remained in archaeology. I cannot imagine the boredom I would be experiencing.

\item[IJSRA] \emph{What are your thoughts on the state of archaeology in Southern Africa? What changes need to be addressed?}
	
\item[SC] Southern African archaeology is a very healthy discipline as shown by the very high levels of local and international interest. The strengths and weaknesses differ from country to country. To generalise a bit, there appears to be many researchers interested in the Stone Age but mostly working in South Africa and Tanzania. This somewhat gives the impression that the other areas had no human ancestors. 
I am thinking of countries like Zambia, Namibia, Mozambique, Angola, Zimbabwe, Malawi and so on. It will be important for young and upcoming archaeologists to consider research in these areas. In some countries such as Zimbabwe and SA, Iron Age research is very strong which is good. However, there is a need to broaden up areas of academic enquiry to include the deployment of more scientific techniques such as isotopes, DNA and other techniques that will help us understand the past better. 
Even in the Iron Age, more students are needed to engage with multiple theories and ideas. The problem with the domination of a sub-discipline by a few people and their interests (this happens elsewhere as well!) is that it appears as if the past was that limited when in fact it is only one’s imagination that can be limited. Whatever the case may be, we need more students and their energy to advance our understanding of the past. It is my hope that in a few years’ time southern African archaeology will have developed local competency in understanding plant economy, various aspects of archaeological science, computer modelling, demography and much more to create a diverse but vibrant discipline.


\item[IJSRA] \emph{What are your thoughts on the state of archaeology in Southern Africa? What changes need to be addressed?}
	
\item[SC] Funding, funding, and funding is one of the major stumbling blocks. Unlike some subjects such as accounting where one can have a successful career with an honours degree, archaeology is not like that. One has to do a Masters and even a PhD to get a reasonable job. Post-graduate studies require funding, which is either non-existent or not enough. My advice to aspiring archaeologists is that no matter what difficulties you encounter on the way up, it’s worth the while. Archaeology is a rewarding career that gives one an opportunity to think critically and independently and contribute to community development.


\item[IJSRA] \emph{Do you think the IJSRA and other student archaeology platforms and groups impact the discipline?}
	
\item[SC] Precisely! In fact, more support should be given to platforms that encourage student involvement in the discipline. Publishing is an important component of any archaeologist’s career and routine. It is important to publish where, as a student, you are given utmost care and encouragement. This does not happen with big international journals. Student journals make superstars, while established big journals cater for superstars, which explains the huge significance of the former. My own publishing career was helped by strong involvement with student journals!
\end{labeling}

\IJSRAseparator
\IJSRAsection{Peter Mitchell}
{\sffamily Peter Mitchell, a PhD graduate from University of Oxford, UK, is a professor of African Archaeology at University of Oxford and an honorary research fellow at the school Geography, Archaeology, and Environmental studies at the University of Witwatersrand. He is the co-editor for “Azania: Archaeological research in Africa” and is an editorial board member for several international journals. Over the years, he has conducted fieldwork in Lesotho and has raised awareness about the management of local heritage. }
\begin{labeling}{IJSRA}	

\item[IJSRA] \emph{What drew you to archaeology and what path did you follow to become an archaeologist?}
	
\item[Peter Mitchell (PM)] Probably like several other people of my generation, the arrival of the Tutankhamun exhibition in Britain in 1972 provided the first spark of interest, followed up by numerous holidays that took in ancient sites across the UK. Having got to Cambridge, I was fortunately able to take an option in African archaeology with David Phillipson to complement my main area of specialization, the Neolithic, Bronze Age and Iron Age of Europe. Captured by this, I decided to look around for postgraduate opportunities in African prehistory. Though as things turned out I went straight into doing a doctorate, I chose Oxford because it offered a two year course based Masters degree. Ray Inskeep, who was the Africanist at Oxford, had taught in Cape Town in the 1960s and early 1970s and, together with Pat Carter back in Cambridge, we worked out topic that might prove a viable DPhil thesis.

That thesis focused on the late Pleistocene Robberg industry from Pat’s Sehonghong site in Lesotho and it was to compare my observations of that with similar material from sites in SA that I first visited SA in 1985, principally being hosted in Stellenbosch by Hilary and Janette Deacon. Once I completed my DPhil I was lucky enough to get a post-doctoral research fellowship from the British Academy that allowed me to undertake my own fieldwork in Lesotho, followed by teaching experience and another postdoc in Cape Town before coming back to Britain in 1993 and returning to Oxford two years later to take up the post from which Ray had retired the year before.


\item[IJSRA] \emph{Archaeologists usually have at least one story to tell around the campfire about their career or fieldwork experiences, do you have one to share with us?}
	
\item[PM] 
Several come to mind, but only one that is obviously printable. Anyone whoever looks at the field notes from the site I excavated in Lesotho – Tloutle – may wonder what happened to context 128. The brief answer is that it was the subject of a disagreement between me and a local goat! The deposit was so damp that I had to sieve everything in the local stream and then dry it on plastic bags in the sun. This particular goat decided it wanted to investigate the plastic and despite trying my best to pull the bag back out of its mouth bag, stone flakes, bone fragments, and the rest of context 128 disappeared never to be seen again. I was more careful thereafter!


\item[IJSRA] \emph{What are your thoughts on the state of archaeology in Southern Africa? What changes need to be addressed?}
	
\item[PM] 
SA – and South African archaeology – are very different from when I first visited the country and it has been a privilege to see the country move out of the apartheid era and into a full democracy, however troubled. As far as the current state of its archaeology is concerned I think several things come to mind. First, the archaeological community, whether we think of students or professionals, is self-evidently still not reflective of the overall demographic make-up of the population as a whole. That has to be a serious matter both in terms of overall equity, but also in terms of what it may portend for the subject’s future in universities and more generally. Second, there seems, as far as I can gauge things, still to be a very serious shortfall in the efficient and thorough working of the relevant heritage bodies in many areas, with many provincial institutions , and perhaps SAHRA too, scarcely able to monitor threats to heritage resources or take pro-active measures to forestall them. The recent debacle over Canteen Kopje is a case in point. And finally, the CRM (Cultural Resources Management) field really needs, I think, to try and create jobs and training opportunities for more students – and for non-students who work within it – to build capacity among non-white sections of the South African population. None of these challenges are easy to meet, and I am very aware that many, many archaeologists have striven, and are striving, to make progress on them, but they need action if archaeology is to have a future within SA. And to achieve that too I think it goes without saying that it is absolutely essential that universities and museums are well funded and given the academic freedom necessary to undertake research that can continue to be of a world-class quality.

Some of these things are, of course, also relevant to other southern African states, but I also think it is important not to confuse matters. There is great variety here. ‘Transformation’ in the sense it is used in SA is simply not relevant in Botswana, for instance, and focusing on it in ASAPA (the Association of Southern African Professional Archaeologists) raises the difficult issue of whether ASAPA genuinely is a fully southern African organization or a narrowly South African one: there is a tension, hopefully a creative tension, here that is still not fully resolved. And of course, in some countries (Swaziland, Lesotho) there is no, or next to no, local archaeological community, making it exceptionally difficult to sustain any kind of research work or CRM presence.

\item[IJSRA] \emph{What difficulties do you think local students face in pursuing a career in archaeology? What advice would you give to aspiring archaeologists?}
	
\item[PM] As elsewhere, including the United Kingdom where I am in the midst of starting to select undergraduates for admission in 2017, one of the major problems is that students, and even more so perhaps their families and teachers, may perceive archaeology as a dead-end, a degree choice with no obvious future employment. 
That gains even more impact when thinking of moving into postgraduate study and it is easy to imagine the pressures on students when faced with the uncertain future of pursuing a career in archaeology in contrast to the allure of government jobs or the private sector. So, two things follow. 
First, academics need to do a really effective job of advertising and marketing their subject, stressing the intellectual rigour that it demands and the ‘transferable skills’ that it teaches so that potential students can realise that if they do well on an archaeology degree they have a product that will empower them in whatever they choose to do afterwards. 
And second, students from disadvantaged backgrounds in particular need to be nurtured and encouraged at every opportunity: this happens, but maybe SA, in particular, can learn here too from the kinds of strategies deployed in places like Tanzania, Zimbabwe, and Botswana in the 1980s and 1990s.

\item[IJSRA] \emph{Do you think the IJSRA and other student archaeology platforms and groups impact the discipline?}
	
\item[PM]I think they have to and they should. You – as editors, contributors and, I hope, readers, are the discipline’s future. Use IJSRA and every other opportunity you have within your universities and at conferences to convey what you think, what you value, and where you think archaeology – and archaeology teaching – should go. One of the great benefits of teaching where I do is that the very intimate, tutorial system we use gives me the opportunity to learn from undergraduates on a weekly basis as we discuss the essays they have written and the books and papers they have read. That kind of learning is exactly what IJSRA and other student archaeology platforms can offer the discipline as a whole.
\end{labeling}
\IJSRAseparator
\IJSRAsection{Janette Deacon}
{\sffamily Janette Deacon, a PhD graduate from the University of Cape Town, SA, is now retired. She was a lecturer, a journal editor, a Later Stone Age and rock art specialist, and the appointed archaeologist for the National Monuments Council. During her retirement, Janette has managed the Southern African Rock Art Project. The project, with assistance from various organisations, including the Getty Conservation Institute, has allowed for the training of  staff, at various Southern African world heritage and protected sites, as rock art tourist guides, assistants for management and conservation of rock art, and as contributors to the process of world heritage site nomination. In 2016 UCT awarded her an Honorary D.Litt. in recognition of her contribution to the field of archaeology.}
\begin{labeling}{IJSRA}	
\item[IJSRA] \emph{What drew you to archaeology and what path did you follow to become an archaeologist?}
	
\item[Janette Deacon (JD)]
I did not go to university with the intention of becoming an archaeologist, nor even of studying archaeology. I registered for Archaeology (then only a 2-year major at the University of Cape Town [UCT]) in 1959 as part of a BA in Geography, because students had to select at least one course from a list of several subjects. I chose Archaeology because it fitted into my timetable. Associate Professor A.J.H. (John) Goodwin was the only lecturer, and had been there for more than 30 years. By that time, he had been diagnosed with lung cancer and was often unable to complete a lecture.

In the middle of the year Goodwin went to Switzerland in the hopes of a cure. Glyn Isaac, a young graduate about to leave SA to study archaeology at Cambridge, took over the lectures and things began to make sense to me. He also took students into the field to look for handaxes, and encouraged us to attend lectures arranged by the South African Archaeological Society. I became friends with Carmel Schrire who introduced me to “Ginger” Townley Johnson, Hym Rabinowitz and Percy Sieff whose hobby was to search for and record rock paintings in the Cederberg. We went on two or three camping trips with them which I really enjoyed.

At the beginning of 1960, I had not made up my mind whether to do Archaeology II or another second year subject to complete my BA degree. When I went to register, I met Carmel who told me that Goodwin had died in December 1959 and that there was a rumour that the university would not offer archaeology courses in 1960. This would have been a problem for Carmel as she needed to complete her degree with Archaeology II that year in order to apply for entrance to the Cambridge tripos in 1961. She persuaded me to go with her to talk to the head of African Studies, Professor Monica Wilson, and convince her that both of us had to do Archaeology II that year in order to complete our degrees. Whether the rumour was true or not I cannot say, but Archaeology I was cancelled in 1960 and Archaeology II was not. During the first quarter we had lectures from staff in the Geology and Anatomy departments. In the second quarter, Brian Fagan, who at the time was employed as an Archaeologist at the Rhodes Livingstone Museum in Zambia (then Northern Rhodesia), was appointed as a temporary lecturer and patiently took us through a shortened version of the archaeology course he had recently completed at Cambridge. More importantly, he took us to the South African Museum to see the archaeological collections and taught us how to sort stone artefacts. I was given the task of drawing some of them and enjoyed that very much.

UCT had advertised the post of Senior Lecturer in Archaeology and Ray Inskeep, a Cambridge graduate, was appointed from 1 July 1960. His enthusiasm was infectious and he went out of his way to take us into the field and even participate in excavations at Elandsfontein near Hopefield with Professor Ronald Singer from the UCT Anatomy Department. It was this stimulating environment, and particularly the fieldwork, that gave me confidence.  

I had done well in Geography III and Professor William Talbot offered me a job as a research assistant for his work on Swartland agriculture in 1961. During that year I had valuable experience in mapping from aerial photographs, drawing maps and diagrams, and analysing weather patterns, while often joining Ray and his archaeology students (including Carmel Schrire, Garth Sampson and Renee Hirschon) on field trips. Towards the end of the year, the African Studies Department created a new position of Junior Lecturer in Archaeology and Ray persuaded me to apply, despite the fact that I did not have an Honours degree and my knowledge of archaeology was barely sufficient. During the summer vacation, Ray took Renee Hirschon and me on a road trip to visit museums and archaeologists between Cape Town and Livingstone on the Zambesi. It was a wonderful introduction to archaeologists and museum staff, some of whom became lifelong friends, and to key archaeological sites in southern Africa.

At the age of 22, I was duly appointed as Junior Lecturer, and also became Secretary of the Western Cape Branch of the South African Archaeological Society and was responsible for organising monthly lectures 
and excursions. Hilary Deacon, who had graduated with a BSc in Geology and Archaeology in 1955 and had worked as a field geologist
 in Tanzania, Kenya, Ghana and the UK, returned to UCT in 1962 to do an Honours degree in Archaeology. He admired my skills in drawing and organising field trips, and I admired his wide knowledge of geology, 
 skill in reading stratigraphy and the landscape, and repairing Land Rovers. We were married at the end of December 1962 and drove to Grahamstown where he had been appointed as Archaeologist at the Albany Museum.

We left Grahamstown in mid-1971 when Hilary was appointed to start the Department of Archaeology at Stellenbosch University. I had the opportunity to teach again at UCT from 1972 to 1975 when Ray Inskeep left for the Pitt Rivers Museum in Oxford. John Parkington and I were responsible for lecturing until Nikolaas van der Merwe was appointed as the first full Professor, and Andrew Smith joined the department in 1976. As a research assistant at Stellenbosch I was able to finish my PhD in 1982 and when the grant funds dried up, I joined the National Monuments Council as Archaeologist in 1989. As Editor of the \emph{The South African Archaeological Bulletin} from 1976 to 1993 I was able to keep up to date with publications and with developments in the discipline. This was especially useful during the drafting of the National Heritage Resources Act which replaced the National Monuments Act in 1999. I only became an archaeologist because I had encouragement from Ray Inskeep and Hilary Deacon. I am deeply grateful to them both for having faith in me.


\item[IJSRA] \emph{Archaeologists usually have at least one story to tell around the campfire about their career or fieldwork experiences, do you have one to share with us?}
	
\item[JD]
In the mid-1980s when we were excavating at Klasies River, I went one day with my 11 year-old daughter, Melissa, to do the shopping for the camp. We had an old Volkswagen kombi and although Hilary had reminded me that he had put his keys under the front bumper, I forgot and used my own keys. The farm track was narrow and bumpy for about 3\,km, then became a gravel road for 10\,km, and finally tarmac for another 20\,km. We despaired of finding the missing keys when I remembered the following day. We walked back up the hill to where the kombi was parked, but of course the keys were no longer under the bumper. Melissa and I walked the track searching for them, but to no avail. We had been talking about experimental archaeology around the campfire and Melissa suggested we use the same reasoning. If we put my bunch of keys (without the ignition key of course) under the bumper with a red ribbon, we should be able to see where they fell off, and there was a strong chance that the missing bunch would be found nearby. I drove the kombi (mini van) and she followed behind to see when they fell off. After about 1km, they were still in place so she got into the car with me and we stopped at regular intervals to check the bumper. At last, after about 2\,km, where the road dipped and then went up a short hill, the keys were no longer there. To our considerable relief, we found the red ribbon and my keys in the path, and the missing bunch was found within 1\,m of them. So, all other things being equal, archaeological experiments can be successful.



\item[IJSRA] \emph{What are your thoughts on the state of archaeology in Southern Africa? What changes need to be addressed?}
	
\item[JD]
The number of people with academic qualifications that enable them to be regarded as professional archaeologists in SA has grown exponentially over the past 5 decades, from fewer than 10 in the early 1960s to more than 200 today. Although the job opportunities at universities and museums have grown as well, and some archaeologists are able to generate income independently in the private sector, there are far more qualified people than there are jobs to sustain them. Productivity in the form of published research results has not kept pace with the increased number of archaeologists. Professionals in full-time positions are overworked and undervalued and only university staff members are rewarded for publishing research results. The newly-qualified tend to contribute to knowledge, but not to long-term development of archaeology in southern Africa. It is not only about money. The grants from the National Research Foundation are higher than they used to be, but the red-tape to apply for them is considerable and competition is stiff. Students need inspiration, moral support and role models, as well as remuneration.


\item[IJSRA] \emph{What difficulties do you think local students face in pursuing a career in archaeology? What advice would you give to aspiring archaeologists?}
	
\item[JD]
I get the impression that whereas 40-50 years ago it was relatively easy for a university lecturer or museum archaeologist to access a vehicle and fuel, take students into the field, and give them the opportunity to see archaeological sites and learn how to recognise and record them. It seems that the higher numbers of students today make it a more costly and complicated exercise and as a result there may be fewer opportunities for students to fully grasp the significance of archaeological method and techniques. I would advise them to go out of their way to gain experience on their own, either through voluntary work at museums, universities and archaeological heritage resource management organisations, through attending conferences, and/or joining the South African Archaeological Society and similar Non-Governmental Organisations.

Another major change has been an exponential increase in the number of published articles that a student who hopes to make a career in archaeology is expected to be familiar with. It is significantly easier to access publications and information through the internet, but finding time to read them is a challenge. I would advise students to read as widely as they can while they have access to sources available at universities.


\item[IJSRA] \emph{Do you think the IJSRA and other student archaeology platforms and groups impact the discipline?}
	
\item[JD]
Yes, I certainly believe they have the potential to do so, not only because of the experience they can obtain, but also to build social ties with their peers who they will work with in the future when contributing to the growth of the discipline.



\end{labeling}
\IJSRAseparator
\IJSRAsection{Ndukuyakhe Ndlovu}
{\sffamily Ndukuyakhe Ndlovu, a PhD graduate from Newcastle University, UK, is a senior lecturer in archaeology at the University of Pretoria, SA. He has over 15 years’ experience in heritage management in SA, having worked for both national and provincial heritage authorities. He is also the Editor-in-Chief for the \emph{The South African Archaeological Bulletin}, an executive member and a Secretary-elect of the World Archaeological Congress, and a Council member for the Association of Southern African Professional Archaeologists. 
}
\begin{labeling}{IJSRA}	
\item[IJSRA] \emph{What drew you to archaeology and what path did you follow to become an archaeologist?}
	
\item[Ndukuyakhe Ndlovu (NN)]
It was through lack of career guidance that I studied archaeology. Following the realization that the subjects I was registered for in my first year were not compatible with each other, I decided to seek advice. It was at this period that I was informed of a subject called archaeology. I registered for it in the following year, as there were no modules back then. I studied the first year of archaeology and thought I should continue to the second year. After finishing my third year, I did an honours in geography majoring in climatology, environmental studies, and palaeogeomorphology. This decision was heavily influenced by my thinking that I stand a better chance to secure employment, and that climatology had been an interest of mine from a younger age.

Following my successful attainment of my honours degree, I decided to go back and study archaeology. The main reason for this decision was that I wanted to challenge the notion that there were no Africans employed anywhere in SA as archaeologists. I got registered for a Masters degree in archaeology majoring in rock art studies. After finishing the coursework component, I secured my first employment which led to me not finishing my mini-dissertation. I registered for a Masters degree with the Department of Anthropology at Rhodes University. This decision was inspired by the notion that an anthropologist will have a better appreciation of the ideas I have for heritage management in SA. When I finished my studies, I then took a four-year break before registering for my PhD in Archaeology at Newcastle University. Since then I have worked in various capacities within the discipline of archaeology.

\item[IJSRA] \emph{Archaeologists usually have at least one story to tell around the campfire about their career or fieldwork experiences, do you have one to share with us?}
	
\item[NN]
I have had a wonderful time as an archaeologist in SA. What stands out for me is how, against my original fears, I was so welcomed by my white class mates when registered for my second and third years in archaeology. Besides that, also politically motivated, I will never forget how my dark skin meant that I was thought of as a domestic assistant to my classmates. In a few farms we visited as students, farm employees always spoke about how nice my employers were to me and how much they would appreciate me putting a good word for their daughters so that they also secure some employment.

Amongst the other experiences, three stories come to mind. First, during a survey for Iron Age sites in Limpopo, it got so hot we all pretty much ran out of anything to drink. I recall our fruitless attempts to get shade from anything: a small tree, under the bakkie, etc. Then, unannounced, one of the students quietly opened a can of Appletiser which he drank without sharing. No one was impressed because he knew how desperate we were for anything to drink. When we eventually reached a small shop selling drinks amongst other items, I bought and drank the coldest can of coke in SA on that day. Second, one of my classmates had a belief that he should not appear on photographs. He was of the view that photos take a part of his soul. We were always on alert so that no one should photograph him. I still don't understand this view but respected his belief. Third, while on a rock art field school, I ran out of water. I then asked my one classmate for something to drink. He was more than willing to share his water bottle with me. Other students laughed as I was about to take a sip but I did not read much into it. They laughed because they knew it was actually vodka coloured with some powdered orange juice. The taste said it all and everyone who witnessed me taking a sip laughed so hard. I could not believe that someone would be drinking vodka while on a field survey.



\item[IJSRA] \emph{What are your thoughts on the state of archaeology in Southern Africa? What changes need to be addressed?}
	
\item[NN]
Archaeology in Southern Africa has always enjoyed better access to funding, as SA has always been better resourced, compared to other regions in the continent. It is thus not surprising that, besides language issues, most of the archaeological publications have emanated out of Southern Africa. What I think is critical, going forward, is the following:
\begin{labeling}
\protect\item[a)] Transformation: we need to ensure that there are more Africans of South African descent entering and playing significant roles in the field. Currently, archaeology is unsurprisingly viewed as a white discipline.
\protect\item[b)] Transformation of a racial nature will have, I argue, further benefits beyond having the discipline being seen as transformed. Most importantly, transformation should also bring about new ways of interpreting the archaeological record, informed by theoretical approaches that are African defined in their origin.
\end{labeling}

\item[IJSRA] \emph{What difficulties do you think local students face in pursuing a career in archaeology? What advice would you give to aspiring archaeologists?}
	
\item[NN]
We are still teaching a curriculum that is focused on training archaeologists for the academic jobs even though they are not widely available. Most archaeology students secure employment in government and the contract archaeology sectors. I'm of the view that our curriculum does not fully prepare students for the two sectors. My advice would be that students must avail themselves to as much fieldwork opportunities as possible. Furthermore, I will advise that they secure internships even during their undergraduate studies to acquire meaningful experience which shall then prepare them for life after their degrees.

\item[IJSRA] \emph{Do you think the IJSRA and other student archaeology platforms and groups impact the discipline?}
	
\item[NN]
Indeed, I think such platforms have a critical role to play. For instance, as a student myself, I was an editorial assistant and later, became the editor of a student journal at Newcastle University in the U.K. The experience I acquired was immensely important. It served as a platform for me to learn important aspects about editing and other publishing aspects. I am today an Editor-in-Chief of a successful journal thanks to this opportunity. Another student journal we had at the university enabled me to publish a paper based on my PhD proposal which had been considered the best presentation made at a student conference. Such an opportunity gave me confidence that I could publish anything of significance. I used this platform to begin writing other journal articles and I have never looked back since.



\end{labeling}
\IJSRAseparator
\IJSRAsection{Gavin Whitelaw}
{\sffamily Gavin Whitelaw, is a chief curator in Human Sciences at KwaZulu-Natal Museum, SA. His research focus is on the nature of southern African farming societies in the first millennium\AD. He has presented southern African archaeology to the public through museum exhibits as well as through a travel nature series, called Shoreline.}
\begin{labeling}{IJSRA}	
\item[IJSRA] \emph{What drew you to archaeology and what path did you follow to become an archaeologist?}
	
\item[Gavin Whitelaw (GW)]
Since early childhood I have had a wide-ranging interest in the way the universe works. Natural history was my principal focus, but I was also fascinated by the past. Because of my natural history interest, I chose subjects at university that would prepare me for environmentally oriented work. But I got side-tracked. I had time, so I did archaeology as an extra subject and was very quickly swayed. It helped that I did sociology too, as the two subjects reinforced one another in terms of social theory. Sociology also gave that theory a special and immediate significance in early-1980s SA, because we covered topics such as Christian National Education and labour in the apartheid state. It became increasingly clear that archaeology had the capacity to disrupt and undermine dominant political narratives. For me, this realization mixed irresistibly with my feeling that SA then, and the University of Witwatersrand (Johannesburg) in particular, led the world in archaeological research.


\item[IJSRA] \emph{Archaeologists usually have at least one story to tell around the campfire about their career or fieldwork experiences, do you have one to share with us?}
	
\item[GW]
Campfire storytelling is not something I do. Perhaps the best response is to acknowledge the time I have spent with leading figures in archaeology, on the road, in the field, over meals and at the tea table: lecturers at the University of Witwatersrand who treated undergraduate (even first-year) students with respect and tolerance, who provided opportunities and encouragement and training; senior colleagues at the Natal Museum for their support and honing skills; and prominent foreign scholars for their humour and open friendship. Their (seeming) easy acceptance of me in their world was and is of immeasurable value to me.


\item[IJSRA] \emph{What are your thoughts on the state of archaeology in Southern Africa? What changes need to be addressed?}
	
\item[GW]
It is like it always has been and always will be – mixed. There is some good and excellent work and some that is of lesser value. I thought the presentations from southern Africa at the recent Society of Africanist Archaeologists conference in Toulouse, France were mostly very good indeed. I was similarly impressed at the last ASAPA conference I attended, in Gaborone, Botswana in 2013. Since many of these presentations were by students, I think there is a lot to be positive about. I wonder, though, where these students will find employment in the future. The main employers of archaeologists, universities and museums, are under severe pressure in a commoditized world that cares little about research that does not put bread on people’s tables. There is not anything we archaeologists can do to force large-scale structural changes to society (although our teaching might inspire challenges – after all, archaeology shows that different ways of being in the world must be possible). But perhaps we can do something about improving our communication of archaeological information to the wider public, a kind of educational enrichment in a variety of media. Archaeological research adds value to life and for that reason alone should be supported. 
Some years ago, Richard \textcite{Pithouse2010} quoted from a poem that Upton Sinclair wrote after a strike by textile workers in Massachusetts in 1912.
\begin{center}
\itshape	Our lives shall not be sweated from birth until life closes;\\
	Hearts starve as well as bodies; give us bread, but give us roses!
\end{center}
Archaeology provides the roses. It’s our job to convince the public of its sweet smell.

\item[IJSRA] \emph{What difficulties do you think local students face in pursuing a career in archaeology? What advice would you give to aspiring archaeologists?}
	
\item[GW]
Because I work at a museum, I do not deal regularly with students and I’m not familiar with the challenges they face, other than those general difficulties already alluded to – limited jobs in the formal sector, often with poor salaries. It’s been a long-term problem: in 1945 the archaeologist Leonard Woolley told Roger Summers, “Things are better nowadays, but it’s just as well not to be entirely dependent on your salary. Museums can be very mean at times.” So warned, Summers went on to spend more than 20 years in the museum and heritage sector in Rhodesia (now Zimbabwe), where he was initially the only professional archaeologist in the country \parencite{Summers1995}. 
Woolley’s warning sadly still carries weight, and the jobs, though more numerous now, still provide limited employment opportunities. Today, archaeological consultancy offers other openings, but it requires skills that can only really be acquired from experience in the field, so here again new graduates need positions and mentors. Students, I suspect, also suffer from the demands that universities today place on professional staff, with increased workloads meaning that less time can be dedicated to each student. From the outside, there seems to be a primary concern with numbers rather than quality.

Of the students I’ve met and worked with, the ones that have encouraged me most are those with an obvious, self-motivated enthusiasm for the discipline, for the daily work of recovering and processing artefacts, and for learning about them. I’m always happy to support these students.


\item[IJSRA] \emph{Do you think the IJSRA and other student archaeology platforms and groups impact the discipline?}
	
\item[GW]
It’s a difficult question to answer because I have never been part of any student group. They did not exist when I was student. Also, because I work at a museum I do not interact with students on a regular basis. So I cannot comment on the kind of support these groups might offer students or the role that they do or could play. I wonder, though, about the need for these student groups and the desire to create them. Why should students want to distinguish themselves formally from the professionals in the discipline? Who benefits?

Do the student groups have an impact on the discipline? My sense is no, or at least, none that I have experienced. For me, student groups are most visible on the programmes distributed prior to conferences, which place their activities firmly outside the normal conference programme. At most, we hear a brief report on these activities at conference business meetings.

\end{labeling}
\IJSRAseparator
\IJSRAsection{Keneiloe Molopyane}
{\sffamily Keneiloe Molopyane is an Associate lecturer and current PhD candidate of the University of Witwatersrand. She had been involved in an international collaborative project focused on the Trans-Atlantic Slave Trade. She completed her BHCS Heritage and Cultural Tourism and BA Honours degrees in Archaeology at the University of Pretoria and an MSc in Bioarchaeology at the University of York.}
\begin{labeling}{IJSRA}	
\item[IJSRA] \emph{What drew you to archaeology and what path did you follow to become an archaeologist?}
	
\item[Keneiloe Molopyane (KM)]
What drew me to archaeology? The most simple of questions, yet the hardest to answer. My first encounter with archaeology was actually from watching an episode of “The Adventures of Tintin: Cigars of the Pharaoh”. This was then followed by Indiana Jones movies (so cliché but true), and, and, and… I was the ripe young age of seven years when I decided this is what I wanted to do. I could not even pronounce the word “archaeology” until I was about nine or ten, do not get me started how old I was when I could actually spell the word. I have always been the curious kid and I guess the investigative properties of archaeology are what lured me in and kept me occupied for more than 30 minutes at a time.

Having graduated from highschool, I went straight to varsity to pursue a degree in archaeology. This was then followed by an honour’s in archaeology, an MSc in Bioarchaeology, and finally I find myself in the first of many final years of a PhD in anatomy.


\item[IJSRA] \emph{Archaeologists usually have at least one story to tell around the campfire about their career or fieldwork experiences, do you have one to share with us?}
	
\item[KM]
There are just so many stories to pick from, like the time I nearly fell into the campfire because I was laughing so hard at one of the many stories being told, or the time my classmates and I climbed up a cliff in an attempt to avoid a leopard so we could get back to the campsite after a long day of surveying, or the many times I suffered from heatstroke in the field and went all crazy. I guess the best stories I have to share are about my underwater adventures. Having dived on shipwrecks in the crazy “washing machine” like waters of Clifton Beach, Cape Town and Durban have been the best. I have lost so many dive knives over my underwater archaeology stint, it is ridiculous.


\item[IJSRA] \emph{What are your thoughts on the state of archaeology in Southern Africa? What changes need to be addressed?}
	
\item[KM]
An article on this very topic has been written by Bradfield (2016) in \emph{The South African Archaeological Bulletin}. A very somber read, but it does highlight a number of good points. I would like to see some young fresh blood, willing to take on new sectors of archaeology come through the system.  Archaeology is a relatively young discipline in Southern Africa and there are a range of fields one could slot into, it all depends on your skills. Reflecting on my years as an undergraduate sitting in archaeology lectures, I realised that in order to make it in this industry I needed to have a unique set of skills and be good at it, all whilst still having a firm handle on the general skills and knowledge sets of a “traditional” archaeologist, whoever they are. Specialist archaeologists seem to be all the rage these days, there is nothing wrong with that, but why limit yourself to one genre. By expanding your skill set you become adaptable and suddenly your perceptions of the state of archaeology in the country begin to change. No longer do you see a desolate and barren landscape, but rather one lush with opportunities if the right questions are asked. How many South African forensic archaeologists, or let’s just keep it simple, archaeologists that are familiar with human remains do we produce? What about local maritime and underwater archaeologists?  I may be a bit naïve in my response to this question, I blame my youth, but by stepping out of the traditional research spheres transformation of archaeology in Southern Africa can begin to take place.

\item[IJSRA] \emph{What difficulties do you think local students face in pursuing a career in archaeology? What advice would you give to aspiring archaeologists?}
	
\item[KM]
The most common question I get from both undergraduate and postgraduate students I teach is “what jobs are out there for archaeologists”. I then usually go on about jobs in academia, heritage management and the private sector…at this point their eyes are glazing over. Then there is the question of how much money will I be able to earn with a degree in archaeology? There seems to be a misconception about what a career in archaeology truly is. Not that I know the answer to what a career in archaeology is, but it is not an Indiana Jones type adventure all the time. The majority of my time has been spent sitting in a lab or in front of my computer researching or writing reports. When I do get to go “play” outside, the adventures just happen. I might not get paid as much as my peers who chose to go into economic or medical careers, but I do get to visit and work in an array of remote and sometimes exotic (depending on what your definition of exotic is) locations and get to do incredible things, that they may never get the opportunity to experience.

Any advice for aspiring archaeologists? Ask yourself if this is the career choice you want, and then ask yourself how badly you actually want it and how willing you are to put yourself out there. Then buckle up for the rollercoaster ride, because as much as there are ups, there are many downs. 

\item[IJSRA] \emph{Do you think the IJSRA and other student archaeology platforms and groups impact the discipline?}
	
\item[KM]
Such platforms are great initiatives that have been set up to get students to start networking and finding their way around archaeology to see where they “fit in”. It is a ‘Publish or Perish’ world in academia they say, so student journal platforms are a great way to get one started if a career in academia is intended.
 
\end{labeling}
\nocite{Summers1995,Pithouse2010}
\IJSRAclosing
