\def\IJSRAidentifier{\currfilebase} %<---- don’t change this!
%-------Title | Email | Keywords | Abstract-------------
\def\shorttitle{Issue Presentation}
\def\maintitle{Presentation of the third issue of IJSRA}
\def\cmail{gonzalo.linaresmatas@st-hughs.ox.ac.uk}
%--------Author’s names------------
\def\authorone{Gonzalo Linares Matás}
%-------Biographical information-------------
\def\bioone{\href{https://oxford.academia.edu/GonzaloLinaresMatás}{\authorone}  has just finished a BA Archaeology \& Anthropology at St Hugh’s College, University of Oxford (UK). He is starting a Research Master’s in Human Evolution and Palaeolithic Archaeology at the University of Leiden (The Netherlands), focusing on vertebrate taphonomy, early bone technologies, and faunal analysis. He was the former President of the Oxford University Archaeology Society (Michaelmas 2015), and he has recently been invited to join the WAC (World Archaeology Congress) Student Committee. 

He is particularly interested in the socio-political contexts of heritage management and ownership, contemporary archaeological theory, and the histories of the academic disciplines of archaeology and anthropology as practical modes of inquiry. He is also very interested in transforming the academic publishing landscape.}
%------University/Institution--------------
\def\affilone{Executive Editor, International Journal of Student Research in Archaeology
\\St. Hugh’s College, University of Oxford}


\IJSRAopening%<---- don’t change this!
%-------
\lettrine{I}{am} immensely proud to be presenting the third issue of the International Journal of Student Research in Archaeology. The wonderful community of editors, contributors, and readers is a constant source of inspiration and motivation to us, and we hope that through our work, our aspiration of a more inclusive publishing landscape for the promising students that are the present and future of our discipline will be a step closer.

We have been refining the publication layout, in pursuit of a distinctive visual style, and restructure the numbering sequence for the issues, which will follow a successive counter.

The two main themes for this issue are ‘heritage’ and ‘gender in archaeology’. The first two contributions tackle the issue of how heritage is imagined and constructed, and the various challenges facing preservation issues at the moment. The following two assess the extent to which contextualised osteoarchaeological evidence may inform us about attitudes to gendered persons and their bodies. Despite potential limitations, it will continue to grow as a fundamental avenue of enquiry into past human experiences.

As part of this issue, we have also incorporated an original feature, combining a series of interviews of leading and aspiring archaeologists in a comparative regional overview of the discipline. We are very grateful to Jeanette Deacon, Shadreck Chirikure, Gavin Whitelaw, Ndukuyakhe Ndlovu, Peter Mitchell, and Kenneiloe Molopyane for their insightful perspectives on what inspired them to become archaeologists, the challenges that the discipline is facing in Southern Africa and potential avenues for addressing these issues. This Q\&A session is seasoned with exhilarating anecdotes derived from many years of experiences in the field.

Featuring on our cover, Petra was a crossroad of desert caravan routes in ancient times. The reaped benefits afforded the Nabatean kings to carve monumental sandstone tombs and temples in their capital throughout the first century\BC. Valletta Verezen, in a lavishly illustrated paper, warns how risk- and damage-assessments of its geoarchaeological data are showing that this fragile wonder is rapidly deteriorating due to weathering; and firm and prompt actions are paramount for the safeguard of the region’s heritage.

Synthesising sources in English and Spanish, Frances Koziar and Camilo Gómez present a thorough and comprehensive review of the history of archaeological praxis in Mexico. A full range of diachronic changes are threaded together through ideological and political shifts in the conceptualisation of heritage and the indigenous past since the colonial period, particularly in the wake of post-colonial imaginations of the Mexican nation. They also consider recent attitudes to contemporary issues in archaeological practices, such as tourist-oriented conservation, and the role of the past in the consolidation of mestizo identities.

Dannielle Croucher delves into the remarkable osteoarchaeological databases of the Museum of London Archaeology (MOLA) to quantitatively assess changes in interpersonal violence patterns in Medieval and post-Medieval London (\nth{11}--\nth{19} centuries), through a gender-oriented perspective. The predominance of blunt and sharp injuries and the decrease in the number of sites with interpersonal violence is further supported by legal documents against (lethal) personal violence. The specific trend of increasing violence against males in post-Medieval London is at odds with broader patterns, highlighting the importance to consider specific local factors in the assessment of trends from osteoarchaeological data.

In ‘Gendering the Traces’, the elegant prose of Amanda Padoan provocatively illuminates the debate surrounding the study of gender in prehistoric archaeology, using the presence of women sacrifices at the elite male ‘Birdman’ burial in Cahokia Mound 72 as a main case study. She also engages with the broader question of what it means and entails to study gender in the past for the present and future of academic enquiry, and attempts to explore beyond the paradigm of “live women saving dead women from dead men”.

We also include in this issue reviews of both the NEBARSS Conference on Neolithic and Bronze Age archaeology, and the Australian Archaeological Association Conference, closing the volume with a book review on bone histology by Ariane Maggio.

In the chapter of future developments, we are preparing special issues for the publication of Conference Proceedings, and we welcome prospective organisers to enquire about the process. Furthermore, we are developing a Newsletter that will aim to further connect subscribers to the IJSRA team, providing additional archaeology content, information about world archaeology events and news, and opportunities to engage in research and outreach. We welcome feedback and suggestions for its successful implementation, as well as potential contents. If you have a fieldwork/lab picture that you would like to see featured in our website/social media, we would love to see it! We also welcome blog reviews of our issues/Journal: do send us an e-mail with any queries.
As part of our constant efforts to reach as wide an audience as possible, we are always looking for people with diverse research interests to join our international team. If you are committed to improve the presence of excellent student scholarship in archaeology through any aspect of our free, open-access, not-for-profit publication, please do get in touch!

\booltrue{nobib}%in case there are no references cited
\IJSRAclosing%<---- don’t change this!