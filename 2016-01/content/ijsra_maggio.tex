\openingarticle
\def\ppages{\pagerange{Maggio:firstpage}{Maggio:lastpage}}
\def\shorttitle{Constructing Identity from the Genetic Past}
\def\maintitle{\SIrange[range-phrase=--]{2}{3}{\percent} Neanderthal, \SI{5}{\percent} Denisovan, but \SI{100}{\percent} Human:  
	Constructing Identity from the Genetic Past}
\def\shortauthor{Ariane Maggio}
\def\authormail{akmaggio@outlook.com.au}
\def\affiliation{University of Western Australia}
\def\thanknote{\footnote{\href{https://uwa.academia.edu/ArianeMaggio}{Ariane Maggio} is an honours student at the University of Western Australia with a BA in Archaeology, a BSc in Genetics, and a Masters in Forensic Science (Anthropology). She has recently submitted her honours thesis, which focused on how the concept of the gene has been constructed, interpreted, and translated in modern human origins studies, using the example of Mitochondrial Eve as a case study. She also volunteers with a local youth archaeology group teaching archaeological practice to children.}}
%--------------------------------------------------------------
\mychapter{\maintitle}
\begin{center}
	{\Large\scshape\shortauthor\thanknote}\\[1em]
	\email \\
	\affiliation
\end{center}
\vspace{3em}
\midarticle
%--------------------------------------------------------------
\label{Maggio:firstpage}
\begin{myabstract}
There\marginnote{Abstract\\(in French see below)} has been a veritable explosion of archaeogenetic studies after the Human Genome was published in 2010. Among such studies, the sequencing of the Neanderthal genome provided new insight into human evolution and the past relationships between Homo sapiens and Homo neanderthalensis, even providing evidence for interbreeding. With an average of \SIrange[range-phrase=--]{2}{3}{\percent} of genes considered to originate from the Neanderthal line being present in Europeans, and current estimates of up to \SI{20}{\percent} Neanderthal contribution to the human genome, there has been a reconsideration of what makes us human. This paper presents a discussion of group and individual constructions of identity in an archaeogenetic world, using selected reactions to Neanderthal Genome Project and the Genographic Project by the general public as case studies.

\keywords[Keywords]{Identity, genetics, Neanderthal, archaeogenetics, genome.}
\end{myabstract}
	
%-----------------------------------------------	
%\section{Introduction}
\lettrine[lines=3,slope=-4pt,nindent=-4pt]{W}{hen} Linnaeus wrote the species definition for Homo sapiens in the \nth{18} century he simply put, ‘know thyself’ as the criteria \parencite{Ritvo_2009}. Thus, the concept of identity is explicit in the study of human origins; who am I; where do I come from; what makes me who I am? Now, nearly 300 years later, genomics is influencing research agendas in the humanities, particularly in archaeology, and affecting the way we see ourselves as individuals, social groups, populations and as a species, both in regards to past and present \parencite{Zwart_2009}. 
In short, knowing thyself has become a multidisciplinary and highly technologically advanced field; and yet the question has not been answered fully \parencite{Pollard_2009} Our identity as Homo sapiens remains unclear, we have yet to define what it is that makes us human and what separates us from our archaic ancestors. Even recently, human origins studies have been in the spotlight with the racism row over the Homo naledi finds \parencite{ABC_2015}. The misinterpretation of the archaeological findings as being racist by members of the wider general audience highlights the sensitive role of archaeological material and interpretation in the construction of identity. This paper will briefly introduce the concept of identity; how it is constructed, expressed, and how it has been interpreted within archaeology. The concept of identity will also be discussed in regards to genetics and genome studies, and how these relate to archaeological constructs of identity through the recent genome studies of Neanderthals (and by extension Denisovans). 
Finally, this essay will present the reaction to the results of the Neanderthal Genome Project and the Genographic Project as case studies; to demonstrate how genomic studies of archaic and modern humans can influence the construction of individual and group identity in the present.

%\section{What is Identity?}
\textcite{Wiley_1981} defines\marginnote{What is Identity?} identity as: ‘a quality that an entity possesses which is a by-product of its origin and its ability to remain distinct from other entities.’ This entity can be an individual or a population. Shared identity, or aspects thereof, can be the basis for socio-cultural groups or a national identity. Zeiler defines three broad concepts of identity: 
identity over possible worlds, identity as certain properties, and identity over time \parencite{Zeiler_2007}. Identity over possible worlds is the concept that identity is either the same or different over different possible worlds or universes \parencite{Zeiler_2007}.  
If you were born to different parents, for example, would you be you, or would you simply not exist? Identity as a sum of certain properties is the most common form of self-definition \parencite{Zeiler_2007}, 
the idea that I am me because of a list of properties that define me as an individual; that make me who I am. These can be properties such as hair and eye colour, gender, sexuality, education, language spoken, country of birth, ancestry or ethnicity. The final concept, identity over time, focuses on the question if an identity remains the same over time; for example that I either ascribed to the same identity in my past as I do in my future, or did not \parencite{Zeiler_2007}. In this concept, there arises the question if someone is the sum of their experiences, and therefore if their sense of identity changes with each new experience over time. Identity can be constructed from all three concepts of identity; it is flexible and it is subjective.

%\section{Identity in Archaeology}

Early \marginnote{Identity in Archaeology} in the history of the discipline of archaeology, in the Culture Historical period, archaeology was used to support burgeoning ideas of national identity and state formation in Europe \parencite{Meskell_2002}. Of particular importance was the use of archaeological material to bolster a national identity constructed by Nazi Germany, an identity that drew strict definitions of sameness and difference \parencite{Arnold_1990}. German archaeologist Kossinna introduced the idea that a superior human race, the Aryan race, was equivalent with the ancient Germans, thus arguing that Germany was the key to prehistory \parencite{Arnold_1992}. Since then identity in archaeology has become intrinsically linked to studies of class inequality, gender bias, sexual specificity, politics and nation, cultural heritage, and race \parencite{Fisher_2003}. 
With archaeology interested in the material culture and remains from past individuals and populations, archaeologists are faced with the illusion that the subjects of research, those whom identity we ascribe, are dead and buried \parencite{Meskell_2002}. Current definitions of identity coalesce around concepts of genealogy, biology, heritage, citizenship, sameness and difference \parencite{Meskell_2002}. Furthermore, the identity of living individuals, populations, and nations is in many ways still built on the archaeological knowledge of the past.

%\section{Identity in Genetics and Genomics}

Walter\marginnote{Identity in Genetics and Genomics} Gilbert once suggested that genetic knowledge would provide us with answers to the million dollar question, “Who are we?” \parencite{Zeiler_2007}. From our increasing knowledge of genetics we know now that our genome is unique \parencite{Zeiler_2007}. It is easy to see therefore, how this uniqueness can then lend our genome to define our identity. Our genome can be used to identify us in situations where other aspects of our identity are lost or unknown, such as in mass disasters \parencite{Hartman_2011}. There is the question of genetic identity; are we defined only by our genes, our base biology, or by our actions, environment and experiences? Epigenetics, wherein the expression of our genome is chemically altered by our environment and even our actions, has provided evidence that we are more than the base sum of our genes \parencite{Lock_2015}.

So while neither the genes nor collective genome determines the whole of an individual’s identity, they can form a layer within a multi-layered concept of identity that includes socio-cultural aspects, environment, and the sum of an individual’s life experiences \parencite{Zeiler_2007}. From a nationalistic or population identity perspective, genome studies can be ethically fraught. The issue of racism is a long standing one, with the ability to detect differences in the genome between population groups not aiding the endeavour \parencite{Carter_2007}. Particular genetic markers, can be more prevalent in particular population groups, and thus have been proposed for racial identification. On a more basic level, group identities that are formed on the basis of outward appearance, such as skin and hair colour, may be based on genomic similarities. The concepts of sameness and difference when applied to the genome may have some beneficial medical potential, but they also carry the looming shadow of application in eugenics \parencite{Garver_1994}.  

The idea of genetic/genomic identity is of particular relevance to archaeology, with the increase in ancient DNA studies \parencite{Matisoo-Smith_2012}. DNA and genome wide studies have become another tool with which archaeologists can extract information about past cultures, and the interaction of these past cultures with present \parencite{Matisoo-Smith_2012}. Many people strongly identify with past cultures, as part of their personal or national identity. The information that can be obtained by the genomic study of archaeological material can be of great importance therefore, not just to archaeologists, but to a wider public audience interested in the past. One such example are the comparative genomic studies of modern human origins, in particular the Neanderthal Genome Project \parencite{Green_2010} and sequencing of the Denisovan genome \parencite{Meyer_2012}, in an attempt to isolate what makes us unique and human.

%\section{Neanderthals and Denisovans: Human or not?}

In the \marginnote{Neanderthals and Denisovans: Human or not?} archaeological study of human ancestry, there has been much debate about the correct placement of Neanderthals (and later Denisovans) in the human tree of life. It must be acknowledged that there is a plethora of species definitions in existence, and the term ‘species’ itself is a fluid concept even for currently living taxa \parencite{DeQueiroz_2007}. Some placed Neanderthals as a separate distinct species to modern humans (for the sake of this paper, defined as genetically and anatomically modern), designating them Homo neanderthalensis \parencites{Harvati_2003}{White_2014}. Others placed them as a human subspecies, designating them Homo sapiens neanderthalensis \parencite{Churchill_2000}. Arguments for one or the other stemmed from a crisis of identity, a definition of where a Neanderthal fit in regards to human identity changing based on perceived sameness and difference \parencite{Harvati_2003}. Historically, Neanderthals were considered inferior or savage in comparison to both archaic and Modern Humans \parencites{Graves_1991}{King_1864}. The fossil remains of Neanderthals were visibly different to those of Archaic and Modern Humans, with a longer face, larger nasal cavity, thick brow ridges and a very robust stocky postcranial skeleton \parencite{White_2014}. Despite having a larger brain, with regards to volume, they were inferred to be less intelligent than humans due to the lack of ‘behaviourally modern’ artefacts associated with Neanderthal finds \parencite{Wynn_2008}. Finally, that they coexisted with Humans, and yet mysteriously went extinct, was considered to be the ultimate proof of their evolutionary inferiority \parencite{Graves_1991}. 

Archaeological studies have since provided evidence to suggest Neanderthals did display behaviourally modern traits; such as caring for their sick or infirm, burying their dead with some form of ritual, engaging in symbolic self-ornamentation, and engineering tools of both stone and other material, including manufacturing pitch to haft spear points \parencites{Spikins_2014}{Rendu_2014}{Villa_2010}. However, like with many aspects of archaeology, the evidence towards behavioural modernity for Neanderthals is highly debated \parencite{McGill_2015}. The discovery of a hyoid bone in a Neanderthal burial suggested that Neanderthal’s might have been capable of speech, a major component of behavioural modernity, but again this was contested \parencite{Lieberman_1993}. The potential for speech would later be supported by the identification of a gene variant known as FOXP2, a gene that in humans is linked to a vital protein for language development \parencite{Krause_2007}. The belief that Neanderthals were in some way inferior, or ‘less human’ that us still persisted, due to the relative scarcity of such artefacts that could be solely linked to Neanderthals and not the Humans who coexisted in the area \parencite{Mellars_2007}.

In 2010 the first draft of the Neanderthal genome was published \parencite{Green_2010}. It was later followed by the sequencing of further Neanderthal individuals and the Denisovan genome in 2013 \parencites{Wang_2013}{Meyer_2012}{Martinón-Torres_2011}. Comparison of the Neanderthal and Denisovan genomes indicated the two were closely related and interbreeding species, with approximately \SI{17}{\percent} of the Neanderthal genes present in the Denisovan genome \parencite{Reich_2010}. Comparison of both the Neanderthal and Denisovan genomes with the Human genome, both the published reference sample and with wider living populations, indicated that there had been multiple interbreeding events between Humans, Neanderthals and Denisovans over time \parencites{Sankararaman_2014}{Sankararaman_2012}. An average of \SIrange[range-phrase=--]{2}{3}{\percent} of the Neanderthal genome was present in individuals of European descent, and \SI{5}{\percent} of the Denisovan genome in individuals of Asian descent \parencites{Sankararaman_2014}{Reich_2010}. 
An estimated \SI{20}{\percent} of the genome is shared between the Neanderthal and the modern humans \parencite{Yong_2014}. 
Yet, no evidence was found of Neanderthal or Denisovan genes within the exclusively African individuals \parencite{Sankararaman_2012}. These results confirmed the theory that Neanderthals (and Denisovans) had interbred with humans, thus further confusing the argument of where they sat in the Hominid evolutionary tree; a separate species, or a subspecies \parencite{Mason_2011}.

A strictly biological concept of the species would imply that if the two groups were able to interbreed and have fertile young to pass on their genetic legacy, then they were not reproductively isolated and therefore were the same species \parencite{White_2014}. 
However,  there was a lack of Neanderthal and Denisovan specific genes on regions of the X chromosome and in mitochondrial DNA \parencite{Serre_2004}. This offered an exciting possible explanation; the offspring of such a pairing were inter-species hybrids, potentially with a Neanderthal father and a human mother \parencites{Serre_2004}{Mason_2011}. The alternative of a Neanderthal mother and a Human father might not theoretically result in children, although it has been known to occur in other species and thus cannot be truly discounted \parencite{Patterson_2006}. Male hybrids may have been potentially sterile \parencite{Patterson_2006}, and with further interbreeding of female hybrid offspring with human men, the percentage of Neanderthal DNA in the genome would be further reduced \parencite{Mason_2011}. Adding in natural selection against non-beneficial Neanderthal genes, and the higher variability of the human genome pool due to the later migration out of Africa, this interbreeding could account for the reduction of the Neanderthal genetic legacy to a percentage of the human genome in a fraction of the global population. However, the reduction of Neanderthal genetic material in the human genome may also be the result of population bottle necks, genetic drift, or even selective recombination events during fertilisation \parencites{Harvati_2007}{Awadalla_1990}. 
	
So the question of identity remains, if Neanderthals were supposedly inferior yet they are in part our genetic ancestors, what does that make us? Does an average of \SIrange[range-phrase=--]{2}{6}{\percent} of the genome render you Neanderthal or does it make you inferior to those who do not have any Neanderthal or Denisovan DNA? If you have the noted Neanderthal genes that are predicted to influence adaptive skin phenotypes \parencite{Vernot_2014}, are you superior to those who do not? The following case studies discuss two separate articles which feature reactions to the possibility of having Neanderthal (and Denisovan) DNA, and how this affects construction of identity. It should be noted here that both of these case studies are not scholarly peer reviewed studies, which was intentional as it is not the aim of this article to analyse how archaeologists have constructed identity from archaeogenetic studies, but rather to provide insight into how archaeogenetic information has been interpreted by the members of the general public.

%\section{Case Study: Reaction to the Neanderthal Genome Project}

In\marginnote{Case Study: Reaction to the Neanderthal Genome Project} an article published in GeneWatch, Terence Keel discusses the reactions of Christian Creationists and White Supremacists to the results of the Neanderthal Genome Project, and how both of these cultural subgroups have used the results of the project within their constructions of identity, both at the individual and shared level \parencite{Keel_2010}. The Creationists, Keel states, use the results to redraw the distinction between non-human and human, with several claiming that Neanderthals must then be fully human, despite anatomical and genetic differences. One listener of NPR’s coverage of the project, Keel states, went so far as to rethink the Christian Genesis creation mythos to include the results stating:

\begin{aquote}{\cite{Keel_2010}}
Adam and Eve were the first cognizant humans, their two offshoots- offspring were Neanderthal son Abel, and modern or Cro-Magnan son Cain. Abel the hunter, Cain the planter. And Cain killed off his brother Abel. It’s a terrible story of why there are no Neanderthals today.
\end{aquote}
In so doing this the listener has made Genesis a fable to represent Hominid evolution, thereby adapting the genetic data into their own sense of identity as a Christian Creationist. Neanderthals are considered human while still being ‘other’. They are human but they are not ‘us’. The fact that there is no evidence of Neanderthal DNA in purely African populations was not addressed.

In contrast, \textcite{Keel_2010} notes that the lack of Neanderthal DNA in purely African populations was a beacon for the white supremacist subgroup, which seized on the difference as a key part of their group and individual identity. Keel stated that the majority of commenters on the website \url{www.Stormfront.org}, a white supremacist blog/forum, related the Neanderthal DNA in Europeans as the reason behind their perceived supremacy and prowess over non-Europeans \parencite{Keel_2010}. Their believed physical superiority was the result of the physical strength of the Neanderthals, elevating the Neanderthals to a position of a supreme race. 
As one blogger stated:

\begin{aquote}{\cite{Keel_2010}}
[As] Neanderthal genes become more inundated with other racial mixes we have been evolving backwards [sic]. It may be that in a few hundred years so little will remain of these genes that we will be inseparable from the lower form of human (i.e. blacks).
\end{aquote}
Now it is clear that these individuals are using the genetic data from the Neanderthal Genome Project to attempt to legitimize their racist ideology, by emphasising the difference in genetic identity between a population group they view as inferior and themselves. 
This is much in the same vein as the use of prehistoric archaeology during the Culture Historical period by the Nazi party \parencite{Arnold_1992}. Interestingly, this runs in direct contrast to the historical perspective of Neanderthals as the inferior savage cousins to the superior humans \parencite{King_1864}. 

From this historical perspective, those with Neanderthal DNA, presumably like the white supremacist bloggers, would be biologically inferior to the exclusively human DNA of African's without European ancestry. This thought also runs in direct contrast to the historical view of privileged educated men during the \nth{18} and \nth{19} centuries; one where hunter-gatherer societies were considered savage while European societies were considered cultured \parencite{Ellingson_2001}. It can be wondered if the perspective of Neanderthals as savage and inferior would have changed in the past if they had known what we now know; that these educated men who thought themselves superior likely shared Neanderthal DNA.

Also inherent in the quote from the white supremacist blogger is an idea of identity changing over time. The blogger posits that in a matter of a few hundred years, genetic admixture will have diluted Neanderthal DNA to the point where people of European origin will no longer be able to identify as such genetically, and therefore in this racist argument, will no longer be biologically superior to African individuals. The argument itself is abhorrent, and the use of the genome data to form a shared and individual identity from a difference based and highly racist ideology, harkens back to the spectre of the construction of national identity in Nazi ideology. 

%\section{Case Study: An Individual Reaction to the Genographic Project}

This\marginnote{Case Study: An Individual Reaction to the Genographic Project} case study focuses on a blog style news article written by Gabrielle Jonas on \href{www.iScienceTimes.com}{iScienceTimes.com}, outlining her personal reaction to the results of her genetic contribution to the Genographic Project, run by the National Geographic \parencite{Jonas_2014}. Jonas relates that she was excited by the results which stated she was \SI{1.1}{\percent} Neanderthal and \SI{0.1}{\percent} Denisovan. 
More interesting to the discussion of individual identity however was Jonas’ blunt summary of how her mother would have reacted to the news:

\begin{aquote}{\cite{Jonas_2014}}
\ldots\ I think about what my mother’s response would have been to being told she was part Neanderthal. She wouldn’t have entertained such a notion, not even with an R.S.V.P… As for us being part Denisovan- it would have helped if the not-so dainty fossilized pinky had been adorned by a college ring, preferably Ivy-League.
\end{aquote}
This contrasts with the first case study, and with Jonas’ personal view, wherein the potential for Neanderthal contributions in the genome was accepted into their personal construction of identity. Jonas’ mother, she relates, would categorically deny being in anyway Neanderthal, it would not be part of her own constructed identity \parencite{Jonas_2014}. Like with the first case study, Jonas’ mother would define her self-identity as being different and in some way better, than an imagined Neanderthal identity. Reading between the lines; here Neanderthals are viewed here as lesser savages to the superior humans. If Humans and Neanderthals were the same species, and Neanderthals were a living species, then this ideological view could be considered as racist as the views of the white supremacists on \url{Stormfront.org}. This is not to say that Jonas’ mother is racist, rather her understanding is more likely influenced by the popular conception of the Neanderthal as a savage cave-man monster, rather than the academic perception of a human-like hominin. This understanding harkens back to the imperialist and colonial racist discourse regarding the concept of a savage; an individual that is considered less evolved than the person making the definition \parencite{Ellingson_2001}.  

When Jonas’ points out that her mother would be more likely to accept the knowledge that she had genes in common with the Denisovan fossil individual if that individual had been in an Ivy-League college, she is referring to the perceived difference in intelligence between Neanderthals and Humans in a tongue in cheek manner \parencite{Jonas_2014}. She would not accept being \SI{0.1}{\percent} Denisovan because in her mind, they are intellectually inferior to humans. Accepting this into her self-constructed identity would make her feel lesser or inferior, regardless of the fact that most Europeans share Neanderthal DNA and most Asians share Denisovan DNA, and there are predicted benefits of some of the identified genes \parencite{Vernot_2014}. Neanderthal was used as a slur, a way of calling someone un-evolved, as Jonas mentions; ‘despite the fun of being able to tell family members that I knew all along they were Neanderthals, the legacy of Neanderthals is not a LOL matter’ \parencite{Jonas_2014}. While Jonas goes on to acknowledge the potentially beneficial aspects to the inherited aspects of Neanderthal DNA as this legacy, in popular culture the legacy of Neanderthals is one of a lesser inferior savage cousin to humanity, which led to their resulting extinction.

%\section{Discussion}

Identity\marginnote{Discussion} is a highly subjective, multi-layered construct, composed of socio-cultural, environmental and biological aspects, such as genes, in addition to layers of individual and shared identities \parencite{Zeiler_2007}. Archaeology and identity have been intrinsically linked since the formation of the discipline, particularly in regards to national identity \parencite{Meskell_2002}. Genetic/Genomic identity is not simply restricted to the unique genome of an individual, but can also be used to construct group identity, particularly with regards to ethnic or racial identity though identification of particular differences between groups. However, this is ethically grey as it can lead to the support of racist ideologies \parencite{Carter_2007}. The data from the Neanderthal and Denisovan genome studies, provided evidence to support interbreeding between Humans, Neanderthals and Denisovans \parencites{Sankararaman_2012}{Sankararaman_2014}{Meyer_2012}. The small percentage of inherited Neanderthal and Denisovan DNA in modern humans has posited the question; if you have this DNA does it alter your perceived identity? The results thus challenged the historically constructed identity of Neanderthals and Denisovans as the lesser savage cousins of the superior modern humans. The case studies presented here similarly state an inherent perspective of individual and group superiority; one accepting the genome data as part of a constructed racial identity, with the other rejecting it on the basis of perceived inferiority. From this it is clear, that the analysis of archaeological material using genomic techniques can provide differing perspectives on identity, with regards to both the past cultural identity but also that of present living individuals. 

%\section{Conclusion}

In\marginnote{Conclusion} conclusion, genetic information, particularly that relating to human origins and the definition of what makes us human and unique, can be a minefield of sensitive issues when interpreted by the wider public. If a single gene is reported to have a beneficial effect over another, does that make those with that gene superior to those without? Does having genes that are linked to Neanderthals make you Neanderthal yourself? What percentage of genetic material do you need to have to identify as human or non-human? These are just some of the questions that can arise in the search to identify what makes us human, and what makes us (as individuals) unique. There is no right answer to any of this as identity is subjective and personal. It is something that is worth considering however, when undertaking such research, and when communicating that research outside of the archaeological sphere. Thus, we need to be wary of how our work will be interpreted by the general public. Archaeologists should take the opportunity to present their findings to the general public and thus challenge long held assumptions and prejudices.

\myseparator
The\marginnote{Acknowledgements} author thanks Associate Prof. \href{https://uwa.academia.edu/MartinPorr}{Martin Porr} for his comments on the first iteration of this manuscript.


\begin{myabstract}
\foreignlanguage{french}{Il y\marginnote{Abstract (French)} a eu une véritable explosion d’études d’archéogénétique, après la publication du génome humain en 2010. Parmi ses études, celle du séquençage du génome néanderthalien nous a permis de découvrir un autre aspect de l’évolution humaine et des relations qui existaient entre l’Homo sapiens et l’Homo neanderthalensis, et nous a même permis de découvrir des preuves de croisements. 
On considère qu’une moyenne de \SIrange[range-phrase=--]{2}{3}{\percent} des gènes présents chez les européens proviennent de la souche
néanderthalienne et on estime actuellement que la contribution du Neanderthal au génome humaine peut atteindre jusqu’à \SI{20}{\percent}. 
Ceci explique pourquoi il a fallu quelque peu reconsidérer le concept d’«humain». 
Cet exposé présente une discussion relative aux constructions d’identité, qu’elles soient individuelles ou de groupe, dans un monde archéogénétique. Pour ce faire, certaines réactions du grand public au Projet génome de Néanderthal et au Projet génographique sont choisies et développées sous la forme de d’études de cas.}



\keywords[Mots clés]{Identité, génétique, Néanderthal, archéogénétique, génome.}
\end{myabstract}
	%------------------------------------------------------------------------------
\printbibliography[heading=subbibnumbered] 
	\label{Maggio:lastpage}
%------------------------------------------------------------------------------
\closingarticle
