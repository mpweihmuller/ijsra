\documentclass[spanish]{ijsra}
\def\IJSRAidentifier{\currfilebase} %<---- don’t change this!
%-------Title | Email | Keywords | Abstract-------------
\def\shorttitle{Heritage in the Context of Nationalism}
\def\maintitle{The Rule of Law and the Power of Suggestion: Heritage in the Context of Nationalism}
\def\cmail{robrownd@gmail.com}
\def\keywords{Need keywords}
%\def\keywordname{}%<--- redefine the name “Keywords“ in needed language
\def\abstract{This \IJSRAsection{Abstract}paper discusses the concept of prehistoric archaeological heritage on a world scale by considering the intentions, functionality and language of the Republic of the Philippines’ Law 10066 \textit{(The National Cultural Heritage act of 2009}) and the Council of Europe (CoE)’s Treaty of Valetta (\textit{Convention on the Protection of the Archaeological Heritage-1969, 1985, 1992}). Both exist quite comfortably under the umbrella of the current UNESCO World Heritage initiatives. This paper acknowledges the necessity and benefits of Existing Political Units administrating and rationalizing the recovery process of artifacts found within their current borders. It also suggests that heritage law is most easily implemented and most effective when applied to a physical landscape rather than a contemporary political unit such as a nation state. In the case of countries such as the Philippines where the current political boundaries are the same as the boundaries of the physical landscape, a national law or policy is sufficient. In the case of a currently politically divided landscape such as the CoE, a region-wide policy that establishes a common set of standards and expectations as a buffer between the different national governments is a better option. }
%--------Author’s names------------
\def\authorone{Rob Rownd}
%\def\authortwo{}
%-------Biographical information-------------
\def\bioone{Get author bio}
%\def\biotwo{--}
%------University/Institution--------------
\def\affilone{Graduate Student at the Archaeological Studies Program, University of the Philippines\\ Professor, University of the Philippines’ Film Institute}
%\def\affiltwo{}%<---- comment or delete if there is no second author.
%\def\affilthree{}%<---- comment or delete if there is no third author.
%\def\affilfour{}%<---- comment or delete if there is no fourth author.
%\def\affilfive{}%<---- comment or delete if there is no fifth author.
%--------Mapping of authors to affiliations------------
%% authorone:--> * <--- copy/paste that symbol to \affiloneauthor etc. below
%% authortwo:--> † <--- copy/paste that symbol to \affiloneauthor etc. below
%% authorthree:--> ‡ <--- copy/paste that symbol to \affiloneauthor etc. below
%% authorfour: --> § <--- copy/paste that symbol to \affiloneauthor etc. below
%% authorfive: --> ¶ <--- copy/paste that symbol to \affiloneauthor etc. below
%-------------------------------------------------------------------------
%\def\affiloneauthor{}%<---- paste the symbol of the authors into {}
%\def\affiltwoauthor{}%<---- paste the symbol of the authors into {}
%\def\affilthreeauthor{}%<---- paste the symbol of the authors into {}
%\def\affilfourauthor{}%<---- paste the symbol of the authors into {}
%\def\affilfiveauthor{}%<---- paste the symbol of the authors into {}

\begin{filecontents}{\IJSRAidentifier.bib}
@online{UNESCO2015,
	title = {Intangible Cultural Heritage},
	date = {2015},
	url = {www.unesco.org/new/en/cairo/culture/intangible-cultural-heritage/},
	subtitle = {UNESCO Office in Cairo},
	OPTorganization = {UNESCO},
}

@MISC {councilofeurope2000,
author = "Council of Europe",
title  = "The official Gazette of the Council of Europe (1999-2000)",
publisher = {Council of Europe Publications},
location = {Strasbourg},
year   = "2000",
}

@MISC {councilforindependentarchaeology2001,
author       = "Council for Independent Archaeology",
title        = "Our reply to EH Position Statement",
howpublished = "Online",
year         = "2001",
note         = "http://www.independents.org.uk/the-valletta-report/english-heritage-position-statement/our-reply-to-eh-position-statement"
}

@BOOKLET {authorunlisted2003,
title        = "Philippine Cultural Heritage Law: Background",
howpublished = "Online",
address      = "http://www.unesco.org/culture/natlaws/media/pdf/philippines/ph_backgroundculthrtgelawinstit_engorof.pdf",
year         = "2003"
}

@URL {filipinocavedivers2014,
author = "Filipino Cave Divers",
year   = "2014",
series = "https://filipinocavedivers.com/guidelines/"
}

@ARTICLE {conardn2009,
author  = "Conard, N. and Malina, M. and Munzel, S.",
title   = "New flutes document the earliest musical tradition in southwestern Germany",
journal = "Nature",
year    = "2009",
volume  = "460",
pages   = "737--740"
}

@STANDARD {CESCEAP1992a,
title        = "The Treaty of Valetta",
organization = "European Union, Council of Europe",
institution  = "Council of Europe Select Committee of Experts on Archaeology and Planning",
year         = "1992",
month        = "jan",
url          = "www.conventions.coe.int/Treaty/EN/Treaties/Html/143.htm"
}

@STANDARD {CESCEAP1992b,
title        = "The European Convention of the Protection of the Archaeological Heritage: Plan",
organization = "European Union, Council of Europe",
institution  = "Council of Europe Select Committee of Experts on Archaeology and Planning",
year         = "1992",
month        = "jan",
url          = "http://conventions.coe.int/Treaty/EN/Reports/Html/143.htm",
note         = "Council of Europe Select Committee of Experts on Archaeology and Planning, The. European Convention of the Protection of the Archaeological Heritage: Plan. (Jan 16, 1992). (http://conventions.coe.int/Treaty/EN/Reports/Html/143.htm)"
}

@BOOKLET {groenewoudt2014,
title        = "Valletta Harvest: value for money",
author       = "Groenewoudt, Bert",
howpublished = "EAC Occasional paper",
num          = "10"
year         = "2014"

}

@ONLINE {potts2015,
author = "Potts, Lauren.",
title  = "Digging for treasure: Is 'nighthawking' stealing our past? BBC News.",
month  = "mar",
year   = "2015",
url    = "http://www.bbc.com/news/uk-england-31848684"
}

@STANDARD {RA4886,
title  = "RA 4886 Cultural Properties Preservation and Protection Act",
author = "Republic of the Philippines",
year   = "1966",
month  = "jun",
url   = "www.philippinelaw.info/statutes/bp4846.html"
}

@STANDARD {RA10066,
title  = "RA10066, The National Cultural Heritage Act",
author = "Republic of the Philippines",
year   = "2009",
}

@ONLINE {roeher2006,
author = "Roeher, Finlo",
title  = "Watching the detectorists. BBC News Magazine",
month  = "may",
year   = "2006",
url    = "http://news.bbc.co.uk/2/hi/uk_news/magazine/4966424.stm"
}

@ONLINE {taas2014,
author = "Ta-as, Apple",
title  = "MACTAN SEA CAVE SURPRISE: ëDoc Amoresí pushed for marine sanctuary. Cebu Daily News",
month  = "jul",
year   = "2014",
url    = "http://cebudailynews.inquirer.net/34991/mactan-sea-cave-surprise-doc-amores-pushed-for-marine-sanctuary"
}

@MISC {tankersley2014,
author       = "Tankersley, Clinton",
title        = "Historical Preservation in the Philippines",
howpublished = "Online",
month        = "jan",
year         = "2014",
url         = "www.preservelaw.com/2014/01/historic-preservation-philippines"
}

@ARTICLE {trotzieg1995,
author  = "Trotzieg, Gustov",
title   = "Archaeology as Part of the Swedish Support to Developing Countries",
journal = "Current Swedish Archaeology",
year    = "1995",
volume  = "3",
pages   = "139--144"
}

@ARTICLE {willemsw2007,
author  = "Willems, W. J. H.",
title   = "The work of making Malta: The Council of Europeís Archaeology and Planning Committee 1988-1996",
journal = "European Journal of Archaeology",
year    = "2007",
volume  = "10",
number  = "1",
pages   = "57--71"
}

@BOOKLET {young2001,
title   = "English Heritage Position Statement on the Valletta Convention",
author  = "Young, Chris",
address = "http://www.independents.org.uk/the-valletta-report/english-heritage-position-statement",
month   = "jul",
year    = "2001"
}

\end{filecontents}

\begin{document}
%\begin{otherlanguage}{spanish}
\IJSRAopening
%-------
\lettrine{A}{wareness} of the \IJSRAsection{Introduction}impact of contemporary cultural identities and contemporary notions of cultural difference on archaeological praxis and the concept of ‘heritage’ itself has matured considerably in the last 50 years. Despite this ever increasing self-knowledge, the relationship between those doing the digging and those being dug up exists in a negotiated, politicized and, therefore, temporary space. As such, heritage will always be subject to reinterpretation. Heritage law is concerned with assigning ownership of objects found \textit{in situ} and then using that sense of ownership to control and track those objects as they move down the modern scientific \textit{chaîne opératoire}  towards scholarly publication and public display. Typically, this is accomplished by designating a given government authority the right to control the granting of permissions to: look for, excavate, and then, retrieve objects to further study their potential archaeological value. 

Republic of the Philippines’ Act 10066, (\textit{The National Cultural Heritage Act of 2009 or RA10066}), and the CoE’s Treaty of Valetta (\textit{Convention on the Protection of the Archaeological Heritage-1992}) are the products of two contemporary political units with very different goals. The Philippines is both a young nation and a developing country still in the process of defining itself on its own terms. Filipino prehistoric archaeological heritage carries the weight of reaching back past colonial influences to touch something more uniquely or purely Filipino. 
The Council of Europe (CoE) contains the \num{28} member states of the European Union and 19 other countries that can best be described collectively as having a proverbial “\textit{toe hold}” in the European continent. 
These \num{19} additional countries include: \num{12} states that have recently been part of the Soviet Union or Yugoslavia, the 3 remaining principalities of southern Europe (Andorra, Monaco, and San Marino), Iceland, Norway, Switzerland, and Turkey.  With the exception of a few of the previously Communist countries, the CoE is comprised of relatively older nation states with more established collective sense of their individual national/ethnic identities. Founded in the decade after the Second World War by many of the same countries who would go on to found what became the EU, it is a truly modern assembly that is more organized around common ideas and shared values than a common tradition such as a religion, economy, ethnicity or location. Not surprisingly, European prehistoric archaeological heritage carries the weight of reaching back past current political and ethnic divisions to touch something more intrinsically universal (or, at least, more pan-European) in its deep past. 

While the ‘science’ from both locations follows the internal rules of objectivity prescribed by the scientific method, \textit{cultural heritage} is comprised of more than that. The pure or objective findings are always going to be contextualized in a meaning or series of meanings that is not controlled by either scientific institutions or scientists themselves. To put it simply, the residue of the past is always going to be political. This idea of scientific value being wrapped in something less tangible but equally important is at the heart of UNESCO’s Heritage Criteria. Being focused on finding and decoding the material traces of previous cultures, archaeology is a study of physical things. But even our very grounded, very quantifiable, very tangible discipline has expanded to include such things as “landscapes, industrial remains and other various forms with the notion of World Heritage or heritage common to humankind.  This particularly fragile form of heritage, often under threat of extinction, had not until now enjoyed sufficient sustained attention” 
\parencite{UNESCO2015}. 

The authors of the above UNESCO quote are specifically referencing the fragility of ‘Intangible Heritage’ such as traditional dance, music and a given culture’s relationship to the landscape, but for past \textit{tangible} heritage, the situation can be similarly challenging, as well as rewarding. 
For example, it is an amazing accomplishment to reconstruct a \num{30} thousand year old flute from \num{12} bone fragments, as excavators led by Nicholas Conrad did in Hohle Fels Cave, Germany, but the sense of wonder such a feat generates soon expands beyond questions about what kind of music was made with it and what purpose(s) it served \parencite{conard2009}. 
There is a real sense of surprise that music (and therefore musicians) and, at other sites, paintings (and therefore painters) have existed for so long. It is these conceptual wrappers of the physical objects recovered that really capture the public imagination and promote the development of a deeper public understanding of who we are as a species and how long we have been here. People have been doing the same things we do for a very long time. What does that say about us?

Despite the \IJSRAsection{Multiple levels of Public Policy moving toward a Common Goal}significant differences between the drafting bodies and the contemporary contexts, Filipino and Council of Europe heritage policies are quite similar on a functional level. The published planning documents for both of them tacitly acknowledge a common problem: private individuals removing artifacts from the ground and selling them to private collectors without allowing the public, including scientists, to be made aware of them, let alone study them %(CoE 1992b:2; RP 2009a:18-20). 
To combat this, heritage law moves artifacts into the public space before they are discovered and declares that the public or common right to know, to appreciate, and, to understand our past supersedes private ownership. Because these two policies were written by different types of agencies, (one a National government and one a Regional/International ‘authority’), they differ in tone while agreeing on this common goal and method. 

RA10066 bluntly \IJSRAsection{Heritage on a National Level: The Philippines
}assigns ownership of the moveable physical objects found in the ground within the borders of the country to the State. While RA100066 acknowledges private ownership of land and immobile structures on that land, even those considered to be \textit{cultural properties} and \textit{National Treasures}, it clearly separates ‘moveable objects’ from these “immovable cultural properties” %(Republic of the Philippines 2009a)
 and assigns their administration solely to the National Museum. Because there are currently no known ‘permanent structures’ considered to be prehistoric in the Philippines, the National Museum has effectively been assigned all authority over pre-literate sites and objects in the country including the ability to directly deputize the Armed Forces of the Philippines, the National Police and other local and national investigative agencies to enforce the act. 
This inclusive document is boilerplate legal writing that simply and clearly defines relevant terms, delegates authority and establishes processes \textit{including the legal penalties} to be doled out to violators. It is also a solidly nationalistic law that moves the decision making power over sites and artifacts away from the Local Government Organization (LGO) of an area and gives it to the National Museum. While in practice this consolidation of authority has frequently created administrative bottlenecks for archaeological workers, it is a necessary rationalization of the overall heritage process because it creates an easily understood framework that clearly assigns accountability for any phase of the process including any ‘accidental’ discoveries of artifacts and sites %(RP 2009a P.8).

As a final point, it is worth observing how RA10066 moves heritage away from all elected officials and into the hands of professionals and specialists.  In an excellent case study written last year, American Cultural Heritage Lawyer Clinton Tankersley traces the development of Philippines Heritage legislation from the early days of the American Occupation Period onwards. American Commission Act No. \num{243}, passed in 1901, mandated the installation of the Rizal Monument on the Luneta in Manila. 
It was the first time any government agency had issued a heritage decree of any kind in the country.  For the next \num{64} years, in both the Commonwealth and then the Republic, heritage legislation can best be described as a piece by piece and almost backhanded expansion of the concept of eminent domain into Philippine public space. 
The Philippines Historical Research and Markers Committee had its name shortened to the Philippines Historical Committee in 1937 but was never granted the authority to do more than identify sites of historical significance with a plaque and a speech. Even with the Cultural Properties Preservation and Protection Act of 1965 (CPPPA) 
creating a new category of Object, ‘the Cultural Treasure’ and formally declaring that the government would “preserve and protect the cultural properties of the nation” %(RP 1966 p. 1), 
the government only allowed itself the right to comment on a privately held ‘cultural treasure’ when it was being sold or going to be sent abroad for exhibition or study. When dealing with prehistoric material, the act only gave the Government the ability to require: 1) permits for excavation and; 2) permits for artifacts to leave the country but no power to police or track their movements within the country. Furthermore, while the CPPPA allowed for methodical cataloging, it was extremely limited and fuzzy in its classification system. For example, the act significantly limited the number of possible ‘Cultural Treasures’ by declaring that “[of] each kind or class of objects, only type and five best duplicates may be designated as ‘cultural treasures’. The remainder, if any, shall be treated as cultural property” %(RP 1966 p. 6). 
The significance of this is not only the limited number of samples provided with the full protection of the law but also that the act itself required almost instant classification of objects for the preservation protection to take effect. This is almost the opposite of a contemporary accessioning process where almost everything potentially interesting comes back to the lab for further study and clarification. The CPPPA was, in essence, an inventory control system but it was hardly scientific in its methodology.

Still, the CPPPA was a significant improvement over the legislation that existed before it. Ferdinand Marcos receives some credit as the father of Philippine Historical Preservation because he sponsored it and later added two presidential decrees %(PD 260 and PD 1505) 
to fill in some of its loopholes. The most significant impact of the presidential decrees forbids the modification or destruction of any identified “important historical edifice” without prior written permission %(Tankersley 2014). 
But on the whole and compared to RA10066, the Marcos era declarations are murky and ad hoc at best. For example, final authority for declaring whether an item was a regulated ‘Cultural Treasure’ or simply considered a “Cultural Property”, rested with a committee that consisted of the Director of the National Museum, the President of the Republic and two ‘experts’ appointed by the President. 

Whatever the reader’s opinion of Marcos as a leader may be, as an elected official rather than an independent historian or scientist, allowing him, or any other elected official to appoint the ‘experts’ who would then help define the ‘nature’ of a particular item kept the whole heritage process squarely in the short term realm of a given administration rather than allowing the process to develop a semblance of autonomy. As with most things during the Marcos era, the CPPPA was too tightly controlled and too personalized to be impartial, flexible or, ultimately, truly effective. By contrast, RA 10066 is a catchall document that only provides the frameworks, definitions, and processes needed to create a fair, efficient, impartial system that is flexible enough to adapt to change. There is no mention of specific scientific or classification methods and the deliberate exclusion of that level of detail allows the intent of the law to be adapted to changes and refinements of those methods as they occur over time. It is a step back from charismatic, situational, decision-making and a step towards a more impartial professionalism and transparency.

To date (August 31, 2016), this fresh legislation has not been tested in a court of law. However, in the five and half years since it took effect, RA 10066 has been used by both the state and private organizations to begin to ‘turf’ the idea of a common national heritage (both intangible and solid) that should supersede notions of private interests and personal ownership. Signed into law in March of 2010
 and broken down into an Implementation document (the working papers used by lawyers and administrators) in 2012, 
 RA 10066 is now being used to confront and contain existing violations of common heritage as they happen. Sometimes this involves creating a separation from open access public land and a place where is access is restricted because of the fragility of the heritage elements found there. For example, the private group, Filipino Cave Divers (FDC) began to promote the new law as a way to protect a freshwater cave system that was discovered in the heart of an Industrial Development Zone on Mactan Island in 2002. 

Mactan is a \SI{75}{\kilo\metre\squared}island with a population of \num{430000}, a large international airport and a bustling, tax-incentive driven manufacturing district. 
It is located just a few kilometers off the shore of the second largest metropolis in the country, Cebu City. It is also an uplifted chunk of Karstic limestone that was probably formed in a shallow sea sometime during the late Pliocene to early Pleistocene. (Aleta, Tomita, Aleta, Lupo, Kawano, 2003). With discovery of the Pawod Underwater Cave System in 2002, modern life and geological time converged a few meters off a road that will take you past Lapu-Lapu City Hall in less than 30 minutes.

Beside the road is a cenote about one third of the size of a basketball court that has been used as a recreational swimming hole as long as people can remember. The freshwater is still reasonably clean but far past drinkable and the rocks surrounding what was a hole in a cave roof are littered with plastic bags, styrofoam takeout containers and other detritus of modern convenience living. One gets the sense that a lot of people have enjoyed themselves here without giving any thought to who owns the land and pool or who takes care of them. This fallow land looks like a vacant lot in need of a good cleaning but there are quite a few undefined spaces like it in the Philippines and they tend to blend together.

Behind this picnic area, and out of the afternoon sun, is a smaller pool that doesn’t get as much foot traffic.  In 2001, at the bottom of this pool, Dr. Alphonso Amores found a chute that lead to an unknown, pristine, submerged cavern system that is (currently) “the only [known] freshwater underwater cave system in an urban setting” %(Ta-as 2014). 
\textit{Doc boy}, as he was known, was a FilAm (Filipino American) retired Reconstructive Surgeon and avid scuba diver who learned to cave dive in Florida. Coming home to Philippines after retiring, the good doctor discovered the Pawod Cave System and was a founding member of the private group Filipino Cave Divers (FCD) that now use it as a training ground for new cave divers. 

Cave diving is an extremely dangerous and technically demanding activity, so it is not surprising that it appeals to very meticulous, patient people who pay great attention to details. Indeed, the online roster of the FCD is full of scientists, cinematographers, and training directors who had already amassed a slew of advanced diving certifications before they entered a wet cave. One detail that has not escaped their attention is the gradual infiltration of the residue of the less structured human activity in larger pool into the previously untouched environment of the cave. Plastic and other detritus have been found a full \SI{5}{\metre} inside the chute and divers, unaffiliated with the FCD, have damaged the walls of the cave by dragging cinderblocks into the cavern to use as anchors for guidelines. 

The FCD seeks to protect this pristine environment from further abuse by promoting the idea of responsible use or stewardship:
\begin{displayquote}
	Stewardship comes with the privilege of diving the underwater caves. The environment is fragile at best. Guidelines are established and laws are promulgated in order to preserve these unique natural resources. %(FCD 2015)
	\end{displayquote}
	

In a coordinated campaign to heighten awareness of this fragile site located just a few meters away from a publically used swimming hole and to establish the legality of their own activities within the space, the FCD are lobbying the local municipal government to consolidate existing environmental regulations under the umbrella of RA 10066. Sighting the presence of a significant number of marine fossils and possible geological clues to the origins of Mactan Island that have been found in the caves, the FCD is evoking the archaeological, cultural heritage zone and systemic research into natural history sections of RA 10066 to define the Pawod Cave System as an area to be treated differently than the unregulated public pool above it. %(FDC 2015)

As a developing nation that has only had two-three generations to get used to the idea of actually owning its own land, trying to declare any resources, even those as unique as the Pawod Cave System off limits to unregulated use is very problematic. The sense of squatting or temporarily using something or someplace that will ultimately be taken away from you works against the idea of responsible use or stewardship. The FDC’s creative use of RA 10066 as an organizing or consolidating structure to create a sense of common heritage is a work in progress and it is still unknown how successful it will ultimately be.

While RA10066 \IJSRAsection{Heritage on a Regional Level: The Council of Europe}is a law, a document generated by the state that is able to define and even dictate behavior to institutions and individuals, the Treaty of Valetta is an agreement between established equals (the individual nation states) who share common goals but use different established methods and practices to achieve them. The Treaty is a beautifully written piece of hopeful diplomacy that reflects this. It has to be because there is no mechanism for the CoE to enforce compliance. The CoE is only effective when it convinces nations to willingly adhere to its treaties and enforce them within their own boarders at their own expense. It accomplishes this through treaties which constantly remind all involved of their common interests and values while acknowledging their differing methods of pursuing those values and pointing out the potential benefits of continuing peaceful collaboration. Still, it is unfair to label the CoE as simply a Public Relations machine or suggest that it is toothless. Despite lacking an equivalent of the UN’s blue helmeted peace keeping forces, it has been extremely effective at promoting its agreed upon values and goals simply by monitoring member states compliance with its treaties.   

It is worth remembering at this point that, the 1980’s and 1990’s were a very positive time for the EU. The political unit changed its name twice: first, from the Common Market to The European Community, and then, to the European Union (EU). Each one of these name changes reflected further quantifiable integration of its Labor Markets, Currencies and Industrial Systems. It became easier and easier for people and money to move about within its borders. Today when it comes to commerce, those internal borders have all but disappeared. However, the EU itself is expressly forbidden from extending itself into the area of cultural affairs by its own treaty of origin. While the more inclusive CoE is a completely separate entity, some of the original EU member states remain the core of CoE and continue to provide a majority of its funding %(CoE 2000). 
Whether this is a sinister arrangement or a pragmatic arrangement is a matter of ideological debate irrelevant to this paper but it is certainly clear that a pattern of mutual beneficence and common interests has been followed over the last five decades. The CoE tends to push forward the cultural side of the EU’s economic agenda. It promotes international integration and seeks to establish Council wide standards for individual rights and responsibilities. With archaeology, the CoE began to standardize the goals and outputs of the discipline by defining a common meaning of Cultural Heritage for its members. The Valetta Treaty of 1992 is the culmination of a 30-year dialog between interested parties. It is the third, and at this point, latest revision of those agreed upon definitions and goals. 

While RA10066 focuses on definitions, conditions and processes, Valetta focuses on the definitions and intentions that lie underneath them. However, just like RA10066, there is little to no discussion of the details of any method that will be used in preservation work or scientific excavation and research. Instead, those methods are to be determined by the individual nation states who sign the document and agree to collaborate. It is stated that each country involved has the right to use whatever methods they see fit provided that they comply with the sense of common ownership and responsible behavior the treaty is espousing. The specifics of archaeological praxis are just not discussed and, as with RA10066, this allows the document to be more flexible and effective. 

Two clear examples of this can be found in the First and Eighth Articles of the Treaty\footnote{Some text has been italicized by the author.}:

\begin{displayquote}
	Article 1- \textit{Definition of the archaeological heritage}
	\begin{enumerate}
		\item The aim of this (revised) Convention is to protect the archaeological heritage as a source of the \textit{European collective memory} and as an instrument for historical and scientific study.
		\item To this end, {the following} shall be considered to be elements of the archaeological heritage, all remains and objects and any other traces of mankind from past epochs:
		\begin{enumerate}
			\item \textit{the preservation and study of which help to retrace the history of mankind and its relation with the natural environment};
			\item for which excavations or discoveries and other methods of research into mankind and the related environment are the main sources of information; and
			\item which are located in any area within the jurisdiction of the Parties. %(EU 1992 p. 2).  
		\end{enumerate}
	\end{enumerate}
	Article 8- \{\textit{Promise to Collaborate and Openly Publish}\}\\
	Each Party undertakes:
	\begin{enumerate}
		\item to facilitate the national and international exchange of elements of the archaeological heritage for professional scientific purposes while taking appropriate steps to ensure that such circulation in no way prejudices the cultural and scientific value of those elements;
		\item to promote the pooling of information on archaeological research and excavations in progress and to contribute to the organization of international research programs. %(EU 1992 p. 4).    
	\end{enumerate}
\end{displayquote}

Notice that each of these articles is worded in a way that promotes the idea that this is a newly defined common meaning for heritage which does not impinge on or threaten a member nation’s sovereignty over its own past or its current sense of identity. Instead, Archaeological Heritage found anywhere in Europe is (also) a “\textit{source of the European collective memory}” \textit{and} “the preservation and study of which help to \textit{retrace the history of mankind and its relation with the natural environment}”. 
The matter-of-fact rhetoric of this treaty is a combination of low key common sense and objective idealism that extends to Article 8’s discussion of collaboration. Here again the simple premise presented is that the heritage belongs equally to everyone collaborating (aka anyone agreeing to follow the treaty) and everyone who signs on the dotted line should have equal access to all of it. All in, for a degree bearing professional archaeology worker, there is little to take issue with in the Valetta Treaty. It is comprised of one common sense idea after another. In an article written ten years after the treaty was approved, Professor Willem Willems who served as the Dutch representative to the treaties council of Experts between 1988 and 1996, is critical of some aspects of the treaty’s implementation but stresses that:

\begin{displayquote}
	We can be quite certain that the adoption, at Malta, of the Valletta Convention will, in future, be considered to have been a watershed in the development of European archaeology. The Valletta Convention defines a standard for the way in which European states \textit{should} manage their archaeological heritage and also provides a frame of reference in this regard for countries outside Europe. It has placed archaeology – \textit{which used to be, in the main, an academic discipline} – firmly \{\textit{and pragmatically}\} in the world of spatial planning, contracting and public decision-making, \textit{unsurprisingly to the distress of some of its practitioners}. %(Willem 2007)
	\end{displayquote}

Willems’ article is written from the point of view of a true insider connected to both academia and government. In addition to being an established scientist, he has been a University Dean, a government official in The Netherlands and served as the president of the European Archaeological Association and other International Organizations. The last sentence in the quote above suggests that in his activist agenda, archaeology and heritage should be more than merely academic concerns. As such, they are best served by practitioners who proactively engage with developers and elected officials. Willems concludes the article with the observation that the ideals of the Valetta agreement would be better served if the EU itself were to agree to the Convention (Treaty) because the discipline of archaeology would then have a voice in both the EU government and the CoE. While that would very likely be a benefit to archaeology and heritage in the region, it seems extremely unlikely that there would be that significant a change in the constitution of the EU any time. As we can appreciate in the two final examples below, while the member states of the EU share a common economic system, they still have significant cultural differences. 

In any \IJSRAsection{Information Overload: The Netherlands
}case, Valetta has been very effective in influencing archaeological policy and law in the CoE member states, but the degree of effectiveness has varied considerably from state to state. The degree of effectiveness seems to be directly proportional to the length of time the member nation took to ratify the treaty. Willems’ home country of Holland, for example, was an early adopter of the Treaty. All together, \num{20} of the then \num{43} CoE member nations signed the document within a year and implemented laws that followed its guidelines within two. 
In the \num{13} years since ratification, Dutch developers have been so compliant with laws based on the models in Valetta that academic archaeologists in Holland are suffering from the happy problem of having to wade through piles and piles of field reports from excavations that were legally required but not motivated by a research question. This inversion of the traditional archaeological process left the Dutch with over \num{7000} site reports of varying quality to mine for data to turn into knowledge. 
Even with over half of these reports (\num{4000}) being dismissed due to methodological and/or technical issues, the sheer volume of synthesizing work proved to be too much for universities to handle by themselves. Given the availability of ample sources of public funding, the demand for the synthesizing services has lead to an unusual situation where there are now more archaeologists with PhDs working outside the academic realm of the Netherlands than inside it %(Groenewoudt 2014 p. 93).  

Coping with this brave new world of field work and data coming before research questions and having to source this enormous volume of work to a mixed field of qualified vendors both inside and outside of academia eventually forced Dutch Heritage (DH) to develop a new process more similar to bidding out creative work or even crowd sourcing than it is to traditional academic inquiry. DH solved this by ranking all of the reports based on location, quality of fieldwork and which potential research questions were most likely be fully or partially answered by synthesizing the research. Universities, Private Archaeological Firms and Municipal Archaeological Services then bid the work on a ‘per project basis’ in the form of a brief that listed price, schedule and personnel to be involved including project leaders. The first batch of contracts were awarded in 2013 based on the ‘best’ bid for a given question with the potential quality of the finished work considered more important than price. This first round of ‘contracts’ will be completed in 2016 in the form of dissertations and published academic articles. DH is already happy enough with the preliminary submissions that they plan to continue the program by offering a new round of contracts for bidding in 2015 %(Groenewoudt, 2014). 
It is highly unlikely that anybody foresaw this sudden bonanza of unfocused field data as a byproduct of adopting the models in the Valetta Treaty. Its will be interesting to see how good the science that comes out it will be. 

While the \IJSRAsection{Resistance to Professionalization: England}Netherlands actively embraced Valetta from the start and incorporated its ideas into its scientific processes and legal system, the British were far more reserved and did not implement the Treaty until 2001. According to both Willems and a British organization that calls itself Council for Independent Archeologists, the main sticking point and cause for this delay was the inclusion of Article 3 of the Treaty\footnote{Some text has been italicized by the author.}.

\begin{displayquote}
	Article 3\\
	To preserve the archaeological heritage and guarantee the scientific significance of archaeological research work, each Party undertakes:
	\begin{enumerate}
		\item to apply procedures for the \textit{authorization and supervision of excavation} and other archaeological activities in such a way as:  
		\begin{enumerate}
			\item to prevent any illicit excavation or removal of elements of the archaeological heritage; 
			\item to ensure that \textit{archaeological excavations and prospecting are undertaken in a scientific manner} and provided that:
			\begin{enumerate}
				\item non-destructive methods of investigation are applied wherever possible;
				\item the elements of the archaeological heritage are not uncovered or left exposed during or after excavation without provision being made for their proper preservation, conservation and management
			\end{enumerate}
		\end{enumerate}
		\item to ensure that excavations and other potentially destructive techniques are carried out \textit{only by qualified, specially authorized persons};
	\end{enumerate}
\end{displayquote}

Willems, like all academic archaeologists, lives in a world where archaeology and cultural heritage have the potential to increase knowledge and improve our understanding of ourselves. But he is also part of a select club that actively excludes others. The Valetta Treaty, and RA10066, are both documents that reinforce that exclusion. While they are precise and candid about heritage being publicly owned and emphasize the public’s right to view and understand it, they also actively restrict the role the public is allowed to play in all phases of the process aside from spectatorship.  Both of the documents share a common premise that archaeology is a true profession which requires organized training, peer certification, and government authorizations to control survey work, excavations and the recovery of artifacts. In this world there is no place for the careful and enthusiastic but uncertified amateur aside from volunteering to work on an authorized and supervised dig. Most of the world agrees with them. But not some of the British.

Amateur Archaeology has a long and productive tradition throughout the UK. The British Museum in particular and the academic units of the discipline in general had benefited from some extraordinary finds by amateurs and had learned to live with some of the sloppy documentation and technique to preserve the greater good which came from having a no cost work force scouting for sites to share. Currently there are dozens of active self-proclaimed Amateur Archaeological Societies and an estimated \num{30000} 
Metal Detectorists spread throughout the UK \parencites{finlo2006}{potts2015}%(BBC 2006 and 2015). 
While metal detectors are a technology developed to locate bombs during WW2, the roots of some of the societies stretch back to a time well before Mortimer Wheeler. Quite a few of both have absolutely no interest in being supervised by anybody. This is not to suggest that any of them have illicit intentions or are even in it for more than the thrill of the find. Instead they were and still are true amateurs who do archaeology in their spare time because they love doing it. The problem with the introduction of the Valetta treaty was that their view of both archaeology and heritage was at odds with the more professional science that was supplanting them. Back on the continent they were perceived as a curious and backward lot. That the gentle language of Article 3 would offend them as it did was a complete surprise to the Valetta’s Committee of experts because most of the delegates had wanted include a statement in Article 3 to directly “ban the unlicensed use of metal detectors” period and only agreed to present what they considered watered down text when the British representative to the committee, trained archaeology Geoff Wainwright convinced them that the ban would never been accepted in the UK \parencite[62]{willems2007}. 

Even after the \num{8} year lapse that occurred before English Heritage sanctioned the Treaty, the British amateurs were perfectly willing to tell the politicians and university professors exactly where they thought they were wrong. And nearly \num{25} years after that, the UK is the sole CoE member state that has struck a working truce with its collection of feisty amateurs rather than move on to a purely professionalized work mode. In recent years, this truce has begun to work for to the British Governments benefit as dedicated amateurs, especially metal detectorists, have evolved into an amateur policing and early warning system constantly on the lookout for nighthawks (the British version of professional treasure hunters) operating on private and public lands. 

Yet, however happy and functional this resolution this, it is nearly the opposite of the situation that has evolved \SI{300}{\kilo\metre} further east in The Netherlands. 
As such it is the best argument for an informal buffer such as the CoE’s Valetta Treaty between Nations with different internal politics rather than a common set of binding laws. 

While profit \IJSRAsection{Conclusions}driven thieves such as the treasure hunters of the Philippines and nighthawks in the UK are a serious problem and most places with a rich archaeological heritage are suffering from one form of organized criminal pilfering or another, the importance of preserving the physical traces of our collective memory and studying them to learn about our species is beginning to become apparent to most people. Both the Rule of Law \parencite{RA10066} and the Power of Suggestion (Valetta) are proving to be useful tools for the preservation of Cultural Heritage. By creating a space to preserve archaeological data, the concept of heritage also creates a context for it to be more deeply respected and understood. With our postmodern mindsets, we would not dare to claim to have created the only context for anything that comes out of the ground. Multiple meanings, contradictory or not, and multiple values are an accepted part of our everyday professional lives as they are an accepted part of our everyday social lives, gradually growing the idea of how interconnected all humans through identifying, studying and presenting our common heritage



%\IJSRAsection{small headline}

%\IJSRAseparator


\IJSRAclosing
%\end{otherlanguage}
\end{document}