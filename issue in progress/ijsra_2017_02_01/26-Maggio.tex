\documentclass[%
	%draft
	]{ijsra}
\def\IJSRAidentifier{\currfilebase} %<---- don’t change this!
%-------Title | Email | Keywords | Abstract-------------
\def\shorttitle{Review of \emph{Bone Histology}}
\def\maintitle{Review of \emph{Crowder, C. \& Stout, S. D. (eds.) 2012. Bone Histology: An Anthropological Perspective. Boca Raton: CRC Press}}
\def\cmail{ariane.maggio@research.uwa.edu.au}
\def\keywords{Bone histology, Histomorphology. Histomorphometry, Light microscopy, Bone biology}
%--------Author’s names------------
\def\authorone{Ariane Maggio} 
%-------Biographical information-------------
\def\bioone{Ariane Maggio is a PhD student at the Centre for Forensic Anthropology at the University of Western Australia. Her current research examines histological approaches to forensic identification using human and non-human bone in an Australian context.}
%------University/Institution--------------
\def\affilone{Centre for Forensic Anthropology, University of Western Australia}

\begin{filecontents}{\IJSRAidentifier.bib}
@book{An2003,
   author = {An, Yuehuei H. and Martin, Kylie L.},
   title = {Handbook of Histology Methods for Bone and Cartilage},
   publisher = {Humana Press},
   location = {Totowa},
   year = {2003},
	 }
@book{Crowder2005,
author = {Crowder, C.},
year = {2005},
title = {Evaluating the use of quantitative bone histology to estimate adult age at death},
publisher = {ProQuest Dissertations Publishing}
}

@article{DeBoer2016,
author = {De Boer, H. H. and van der Merwe, A. E.},
year = {2016},
title = {Diagnostic dry bone histology in human paleopathology},
journal = {Clin Anat}, 
volume = {29},
pages = {831-843}
}

@proceedings{Grupe2012,
author = {Grupe, G. and Garland, A. N.},
year  = {2012},
title = {Histology of Ancient Human Bone},
subtitle = {Methods and Diagnosis},
eventtitle = {Proceedings of the “Palaeohistology Workshop”},
eventdate = {1990-10-03/1990-10-05},
venue ={Göttingen}, 
location = {Berlin},
publisher = {Springer},
}

@article{Hillier2007,
author = {Hillier, M. L. and Bell, L. S.},
year = {2007},
title = {Differentiating Human Bone from Animal Bone},
subtitle = {A Review of Histological Methods},
journaltitle = {Journal of Forensic Sciences},
volume = {52}, 
pages = {249-263},
}

@article{Hollund2012,
author = {Hollund, H. I. and  Jans, M. M. E. and  Collins, M. J. and Kars, H. and  Joosten, I. and Kars, S. M.},
year = {2012},
title = {What Happened Here?},
subtitle = {Bone Histology as a Tool in Decoding the Postmortem Histories of Archaeological Bone from Castricum, The Netherlands},
journaltitle = {International Journal of Osteoarchaeology}, 
volume = {22}, 
pages = {537-548},
}

@book{Horocholyn2013,
author = {Horocholyn, K.},
year = {2013},
title = {Comparative Histology of Burned Mammals Using Light Microscopy},
subtitle = {Examining Heat-Induced Changes in Femoral Samples of Deer, Pig and Cow},
publisher = {ProQuest Dissertations Publishing},
}

@book{Recker1983,
editor = {Recker, R. R.},
year = {1983},
title = {Bone histomorphometry},
subtitle = {Techniques and interpretation},
location = {Boca Raton},
publisher = {CRC Press}
}
\end{filecontents}
\begin{document}
\IJSRAopening%<---- don’t change this!
%-------
\lettrine{H}{istological} analysis of bone can be a useful tool in the physical anthropologist or archaeologist’s arsenal. Analyses of bone microstructure using conventional light microscopy or scanning electron microscopy can be used for human versus non-human species identification \parencite{Hillier2007}, analysis of osteogenesis, age-at-death estimation \parencite{Crowder2005}, in-depth analysis of pathological conditions affecting bone \parencite{DeBoer2016}, in addition to examining taphonomical bone alterations of bone, such as burning \parencite{Horocholyn2013,Hollund2012}. This volume provides a useful introduction to bone histology for anthropologists, from bone biology and development to histomorphology (which can refer to the morphology itself but also includes qualitative analyses of said morphology), and histomorphometry (quantitative measurement of histomorphology). The volume also provides an overview on how to prepare histological sections of bone based on the experience of the contributing author. This volume complements other books on bone histology such as Handbook of Histology Methods for Bone and Cartilage \parencite{An2003}, Bone Histomorphometry: Techniques and Interpretation \parencite{Recker1983} and Histology of Ancient Human Bone: Methods and Diagnosis \parencite{Grupe2012}.

The volume is edited by Dr. Christian Crowder, a forensic anthropologist and the Director of Forensic Anthropology at Harris County Institute of Forensic Sciences, and Dr. Samuel Stout, a physical anthropologist and professor at Ohio State University. Both editors have an extensive publication history on bone biology, histology and histomorphometry. The volume consists of 15 chapters, and the various contributors to the volume include specialists in anthropology, biomechanics, cell biology, palaeopathology, and orthopaedics. Despite being comprehensive, the volume is relatively easy to read and even offers step by step instructions that can be followed by not only specialists but novices and students alike.

Chapter 1 is written by the editors and provides an introduction to bone remodelling, histomorphology, and histomorphometry. This chapter considers how the microstructure of bone can be considered in relation to how bone is remodelled, using the example of histomorphological features of bone, such as osteons. They provide a solid groundwork for how the relationship between histomorphology and bone remodelling provides the basis for age estimation and to infer bone remodelling activity using histomorphometry. Chapters 2 and 3 build on this framework and present an overview of longitudinal bone growth, regulatory systems, biomechanics, and the difference between bone modelling and bone remodelling. Diverting from bone, Chapter 4 discusses the utility of dental hard tissue in bone histology, with specific reference to the record of growth used in histological examination of the enamel and dentine.

Of particular relevance to archaeologists and anthropologists alike, Chapter 5 discusses the use of histology to differentiate human from non-human in fragmentary and otherwise non-diagnostic skeletal material. Chapter 6 discusses the most commonly used histomorphometrical methods for age estimation. Chapters 7 and 8 both discuss the biomechanics of bone. The former considers how histomorphology can be used to interpret signs of skeletal strain that may be associated with load history, such as through occupational use.  The latter considers biomechanics in terms of fracture risk, the material composition of bone, and how bone fracture patterns can be used to predict the context surrounding the cause of the fracture.

Chapter 9 delves into the realms of taphonomy, providing a brief history of histotaphonomy, and demonstrates how post-mortem microstructural change can be used to examine taphonomy by giving examples of characteristic changes. Chapter 10 considers the use of histology for the classification of pathological conditions from a palaeopathological perspective. This chapter also provides some very interesting examples of histological analyses of pathophysiology of bony tissues. Chapter 11 presents a consideration of the histological study of archaeological bone, and explores the usefulness of histomorphometry to determine whether an element is human versus non-human, age-at-death, and skeletal health and disease. The authors of this chapter propose that bone histology should be included in osteological training. 
The next two chapters discuss two collections of bone samples with known demographic information, which are important resources for hard tissue research. Chapter 12 provides a history and description of the collections of the Anatomical Division of the National Museum of Health and Medicine at the Armed Forces Institute of Pathology; the collection includes more than 10,000 slides of stained bone and joint specimens. Chapter 13 describes the Melbourne Femur Collection at the University of Melbourne in collaboration with the Victorian Institute of Forensic Medicine. These chapters demonstrate the importance of developing and maintaining hard tissue collections for scientific study, which can be used for comparison with archaeological material.
The final two chapters discuss practical aspects of hard tissue histology. Chapter 14 outlines the practical considerations of a bone histology laboratory, with instructions on preparing samples for histological analysis, including embedding, cutting sections, grinding and polishing sections, mounting on slides and staining. This chapter is a useful practical resource for those considering performing histological analyses, and also outlines the principles of light microscopy for the uninitiated. Chapter 15 builds on this and considers the usefulness of 3D imaging and micro-computed tomography (micro-CT) in anthropological contexts, such as on the field of palaeopathology, where it has potential to assess bone porosity. 

Histological analyses can provide a great deal of information about an individual; whether through osteobiographical considerations such as species identification, age, occupational strain indicators, and pathology. In addition, histology can be used to consider post-mortem taphonomical effects such as burning and microbial activity. All of these elements can be of interest to archaeologists studying skeletal material; species identification of bone fragments may aid in subsistence studies, while age, strain indicators, and pathology can provide important information regarding composition, disease events, and behaviours of past populations. This volume provides a solid background on histology, histomorphology, and histomorphometry and is an essential resource for anthropologists and archaeologists who are interested in the histological examination of skeletal material – be it contemporary forensic material, or archaeological. 

\IJSRAclosing%<---- don’t change this!
\end{document}