\documentclass[%
	%draft
	]{ijsra}
\def\IJSRAidentifier{\currfilebase} %<---- don’t change this!
%-------Title | Email | Keywords | Abstract-------------
\def\maintitle{Regional feature: Perspectives from southern African archaeology professionals}
\def\cmail{cherene.debruyn@yahoo.com}
\def\shorttitle{\maintitle}

%\def\keywords{Research, Archaeology, ...}
%\def\keywordname{}%<--- redefine the name “Keywords“ in needed language
%\def\abstract{In his paper Jon is showing ...}
%--------Author’s names------------
%--------Author’s names------------
\def\authorone{Cherene de Bruyn}
\def\authortwo{Jacqueline Jordaan}%<---- comment or delete if you do not need a second author.
%-------Biographical information-------------
\def\bioone{Cherene de Bruyn is a 2016/2017 Chevening Scholar and a Masters student in Archaeology at the Institute of Archaeology, University College London. She was awarded her BA, BA Honours in Archaeology and BSc Honours in Physical Anthropology degrees from the University of Pretoria, South Africa. She has been part of teams working on various projects related to burials as well as Stone and Iron Age archaeology in South Africa, directed by local and international archaeologists.}
\def\biotwo{MISSING}%<---- comment or delete if there is no second author.
%------University/Institution--------------
\def\affilone{Institute of Archaeology
University College London, UK}
\def\affiltwo{MISSING}%<---- comment or delete if there is no second author.
%--------Mapping of authors to affiliations------------
%% authorone:--> * <--- copy/paste that symbol to \affiloneauthor etc. below
%% authortwo:--> † <--- copy/paste that symbol to \affiloneauthor etc. below
%% authorthree:--> ‡ <--- copy/paste that symbol to \affiloneauthor etc. below
%% authorfour: --> § <--- copy/paste that symbol to \affiloneauthor etc. below
%% authorfive: --> ¶ <--- copy/paste that symbol to \affiloneauthor etc. below
%-------------------------------------------------------------------------
\def\affiloneauthor{*}%<---- paste the symbol of the authors into {}
\def\affiltwoauthor{†}%<---- paste the symbol of the authors into {}


\begin{document}
\IJSRAopening
%-------
\lettrine{S}{outhern} African archaeology is diverse in topic and practice. From human origins to the historical period, archaeology in this region provides a kaleidoscope of life ways over a vast amount of time. From colonial roots, followed by apartheid restrictions, to current initiatives to decolonise the discipline, the practice of archaeology is undergoing transformation. Southern African archaeology is facing challenges. These challenges are protecting local heritage and sites, addressing the claims of various stakeholders, connecting with non-academic audiences, and developing and retaining archaeologists, by no means an exhaustive list or challenges entirely unique to southern Africa. We invited archaeologists to respond to a questionnaire about the state of archaeology in southern Africa, and the impact student led initiatives have on the discipline. 

These individuals have aimed to contribute to the discipline in different ways. Embedded in the tales of their achievements and experiences are some of the challenges faced by both young academic and professional career archaeologists in southern African archaeology. Jeanette Deacon, shares her initiation into the discipline, a story studded with key individuals in the establishment of southern African archaeology. Shadreck Chirikure, describes how he was seduced into archaeology, away from a more conventional degree with higher career employment, and with unyielding optimism shares his hope for the future of the discipline. Gavin Whitelaw, provides a realistic account of career prospects in the discipline and encourages extending our audience base. Ndukuyakhe Ndlovu, a strong advocate of transformation, developing and supporting of indigenous archaeologists and locally relevant research, shares his journey as an African practicing archaeology. On the other hand, Peter Mitchell, provides a well-informed outsider’s perspective to the challenges and practice of southern African archaeology. Lastly Kenneiloe Molopyane, a PhD candidate from the University of Witwatersrand provides a young professional perspective on southern African archaeology.

Archaeology is a changing discipline, with more academic journals, conferences, student-led research and public archaeology programs than a few decades ago. As part of this change it is also hoped that those conducting the research reflect the local stakeholders and interpretations of our shared human past. This feature presents personal accounts of people who followed their passion, though this is regionally framed, it is an experience shared across the archaeology community. As much as we endeavoured to represent a range of voices, we readily admit that the feature is dominated by male professional archaeologists. In this light, we propose future features that amend this bias and satisfy a broader audience, presenting the views of young career and professional archaeologists in both Academic and Cultural Resource Management environments. Thus we call upon students to confront the challenges faced by archaeologists in both local and global contexts, to create an open dialogue between academics and CRM professionals, making this discussion accessible through the IJSRA platform. This leads us to hope, that our diverse audience can find inspiration, solidarity, or some form of comfort in our shared experiences and challenges within Archaeology and that through engaging with these challenges we can contribute to a more diverse, transformed, and open archaeology.

\bigskip
\IJSRAsection{Abbreviations}
Commonly used Abbreviations:
\begin{labeling}{ASAPAXX}	
\item[SA] South Africa 
\item[ASAPA] Association of Southern African Professional Archaeologists
\item[CRM] Cultural Resources Management
\item[SAHRA] South African Heritage Resources Agency
\item[UCT] University of Cape Town 
\item[WITS] University of Witwatersrand.
\end{labeling}

\IJSRAseparator
\IJSRAsection{Shadreck Chirikure}
{\sffamily Shadreck Chirikure, a PhD graduate from University College London, UK, is an associate professor at University of Cape Town, SA. He runs the Archaeology Materials Laboratory at UCT, the only facility dedicated to the study of African indigenous technology. He sits on the advisory boards of numerous journals and advocates practices that bring archaeological knowledge to the public (see his tedtalk on African indigenous technology).}

\begin{labeling}{IJSRA}	
\item[IJSRA (International Journal of Student Research in Archaeology)] \emph{What drew you to archaeology and what path did you follow to become an archaeologist? }
	
\item[Shadreck Chirikure (SC)] I got into archaeology through chance. My desire was to work in finance but my A level points were not enough for me to embark on my dream career. I was then given a Bachelor of Arts programme specialising in Archaeology, History and Economic History. I remember liking Economic History a lot, largely because of the economic prefix! Things changed however that by my fourth year I was beginning to like Archaeology. I had to pull out of a competition to become a trainee banker because I got a scholarship to study a Master’s in Archaeology (Artefact Studies stream) at the Institute of Archaeology, University College London.


\item[IJSRA] \emph{Archaeologists usually have at least one story to tell around the campfire about their career or fieldwork experiences, do you have one to share with us?}
	
\item[SC] Mine is short. I liked finance to the extent that after graduating with a PhD, I enrolled for an honours in Finance. I then realised archaeology was much more than that, and so I remained in archaeology. I cannot imagine the boredom I would be experiencing.

\item[IJSRA] \emph{What are your thoughts on the state of archaeology in Southern Africa? What changes need to be addressed?}
	
\item[SC] Southern African archaeology is a very healthy discipline as shown by the very high levels of local and international interest. The strengths and weaknesses differ from country to country. To generalise a bit, there appears to be many researchers interested in the Stone Age but mostly working in South Africa and Tanzania. This somewhat gives the impression that the other areas had no human ancestors. I am thinking of countries like Zambia, Namibia, Mozambique, Angola, Zimbabwe, Malawi and so on. It will be important for young and upcoming archaeologists to consider research in these areas. In some countries such as Zimbabwe and South Africa, Iron Age research is very strong which is good. However, there is a need to broaden up areas of academic enquiry to include the deployment of more scientific techniques such as isotopes, DNA and other techniques that will help us understand the past better. Even in the Iron Age, more students are needed to engage with multiple theories and ideas. The problem with the domination of a sub-discipline by a few people and their interests (this happens elsewhere as well!) is that it appears as if the past was that limited when in fact it is only one’s imagination that can be limited. Whatever the case may be, we need more students and their energy to advance our understanding of the past. It is my hope that in a few years’ time southern African archaeology will have developed local competency in understanding plant economy, various aspects of archaeological science, computer modelling, demography and much more to create a diverse but vibrant discipline.


\item[IJSRA] \emph{What are your thoughts on the state of archaeology in Southern Africa? What changes need to be addressed?}
	
\item[SC] Funding, funding, and funding is one of the major stumbling blocks. Unlike some subjects such as accounting where one can have a successful career with an honours degree, archaeology is not like that. One has to do a Masters and even a PhD to get a reasonable job. Post-graduate studies require funding, which is either non-existent or not enough. My advice to aspiring archaeologists is that no matter what difficulties you encounter on the way up, it’s worth the while. Archaeology is a rewarding career that gives one an opportunity to think critically and independently and contribute to community development.


\item[IJSRA] \emph{Do you think the IJSRA and other student archaeology platforms and groups impact the discipline?}
	
\item[SC] Precisely! In fact, more support should be given to platforms that encourage student involvement in the discipline. Publishing is an important component of any archaeologist’s career and routine. It is important to publish where, as a student, you are given utmost care and encouragement. This does not happen with big international journals. Student journals make superstars, while established big journals cater for superstars, which explains the huge significance of the former. My own publishing career was helped by strong involvement with student journals!
\end{labeling}

\IJSRAseparator
\IJSRAsection{Peter Mitchell}
{\sffamily Peter Mitchell, a PhD graduate from University of Oxford, UK, is a professor of African Archaeology at University of Oxford and an honorary research fellow at the school Geography, Archaeology, and Environmental studies at the University of Witwatersrand. He is the co-editor for “Azania: Archaeological research in Africa” and is an editorial board member for several international journals. Over the years, he has conducted fieldwork in Lesotho and has raised awareness about the management of local heritage. }
\begin{labeling}{IJSRA}	

\item[IJSRA (International Journal of Student Research in Archaeology)] \emph{What drew you to archaeology and what path did you follow to become an archaeologist?}
	
\item[Peter Mitchell (PM)] Probably like several other people of my generation, the arrival of the Tutankhamun exhibition in Britain in 1972 provided the first spark of interest, followed up by numerous holidays that took in ancient sites across the UK. Having got to Cambridge, I was fortunately able to take an option in African archaeology with David Phillipson to complement my main area of specialization, the Neolithic, Bronze Age and Iron Age of Europe. Captured by this, I decided to look around for postgraduate opportunities in African prehistory. Though as things turned out I went straight into doing a doctorate, I chose Oxford because it offered a two year course based Masters degree. Ray Inskeep, who was the Africanist at Oxford, had taught in Cape Town in the 1960s and early 1970s and, together with Pat Carter back in Cambridge, we worked out topic that might prove a viable DPhil thesis.

That thesis focused on the late Pleistocene Robberg industry from Pat’s Sehonghong site in Lesotho and it was to compare my observations of that with similar material from sites in South Africa that I first visited South Africa in 1985, principally being hosted in Stellenbosch by Hilary and Janette Deacon. Once I completed my DPhil I was lucky enough to get a post-doctoral research fellowship from the British Academy that allowed me to undertake my own fieldwork in Lesotho, followed by teaching experience and another postdoc in Cape Town before coming back to Britain in 1993 and returning to Oxford two years later to take up the post from which Ray had retired the year before.


\item[IJSRA] \emph{Archaeologists usually have at least one story to tell around the campfire about their career or fieldwork experiences, do you have one to share with us?}
	
\item[PM] 
Several come to mind, but only one that is obviously printable. Anyone whoever looks at the field notes from the site I excavated in Lesotho – Tloutle – may wonder what happened to context 128. The brief answer is that it was the subject of a disagreement between me and a local goat! he deposit was so damp that I had to sieve everything in the local stream and then dry it on plastic bags in the sun. This particular goat decided it wanted to investigate the plastic and despite trying my best to pull the bag back out of its mouth bag, stone flakes, bone fragments, and the rest of context 128 disappeared never to be seen again. I was more careful thereafter!


\item[IJSRA] \emph{What are your thoughts on the state of archaeology in Southern Africa? What changes need to be addressed?}
	
\item[PM] 
South Africa – and South African archaeology – are very different from when I first visited the country and it has been a privilege to see the country move out of the apartheid era and into a full democracy, however troubled. As far as the current state of its archaeology is concerned I think several things come to mind. First, the archaeological community, whether we think of students or professionals, is self-evidently still not reflective of the overall demographic make-up of the population as a whole. That has to be a serious matter both in terms of overall equity, but also in terms of what it may portend for the subject’s future in universities and more generally. Second, there seems, as far as I can gauge things, still to be a very serious shortfall in the efficient and thorough working of the relevant heritage bodies in many areas, with many provincial institutions , and perhaps SAHRA (South African Heritage Resources Agency) too, scarcely able to monitor threats to heritage resources or take pro-active measures to forestall them. The recent debacle over Canteen Kopje is a case in point. And finally, the CRM (Cultural Resources Management) field really needs, I think, to try and create jobs and training opportunities for more students – and for non-students who work within it – to build capacity among non-white sections of the South African population. None of these challenges are easy to meet, and I am very aware that many, many archaeologists have striven, and are striving, to make progress on them, but they need action if archaeology is to have a future within South Africa. And to achieve that too I think it goes without saying that it is absolutely essential that universities and museums are well funded and given the academic freedom necessary to undertake research that can continue to be of a world-class quality.

Some of these things are, of course, also relevant to other southern African states, but I also think it is important not to confuse matters. There is great variety here. ‘Transformation’ in the sense it is used in South Africa is simply not relevant in Botswana, for instance, and focusing on it in ASAPA (the Association of Southern African Professional Archaeologists) raises the difficult issue of whether ASAPA genuinely is a fully southern African organization or a narrowly South African one: there is a tension, hopefully a creative tension, here that is still not fully resolved. And of course, in some countries (Swaziland, Lesotho) there is no, or next to no, local archaeological community, making it exceptionally difficult to sustain any kind of research work or CRM presence.

\item[IJSRA] \emph{What difficulties do you think local students face in pursuing a career in archaeology? What advice would you give to aspiring archaeologists?}
	
\item[PM] As elsewhere, including the United Kingdom where I am in the midst of starting to select undergraduates for admission in 2017, one of the major problems is that students, and even more so perhaps their families and teachers, may perceive archaeology as a dead-end, a degree choice with no obvious future employment. That gains even more impact when thinking of moving into postgraduate study and it is easy to imagine the pressures on students when faced with the uncertain future of pursuing a career in archaeology in contrast to the allure of government jobs or the private sector. So, two things follow. First, academics need to do a really effective job of advertising and marketing their subject, stressing the intellectual rigour that it demands and the ‘transferable skills’ that it teaches so that potential students can realise that if they do well on an archaeology degree they have a product that will empower them in whatever they choose to do afterwards. And second, students from disadvantaged backgrounds in particular need to be nurtured and encouraged at every opportunity: this happens, but maybe South Africa, in particular, can learn here too from the kinds of strategies deployed in places like Tanzania, Zimbabwe, and Botswana in the 1980s and 1990s.

\item[IJSRA] \emph{Do you think the IJSRA and other student archaeology platforms and groups impact the discipline?}
	
\item[PM]I think they have to and they should. You – as editors, contributors and, I hope, readers, are the discipline’s future. Use IJSRA and every other opportunity you have within your universities and at conferences to convey what you think, what you value, and where you think archaeology – and archaeology teaching – should go. One of the great benefits of teaching where I do is that the very intimate, tutorial system we use gives me the opportunity to learn from undergraduates on a weekly basis as we discuss the essays they have written and the books and papers they have read. That kind of learning is exactly what IJSRA and other student archaeology platforms can offer the discipline as a whole.
\end{labeling}
\IJSRAseparator
\IJSRAsection{Janette Deacon}
{\sffamily Janette Deacon, a PhD graduate from the University of Cape Town, SA, is now retired. She was a lecturer, a journal editor, a Later Stone Age and rock art specialist, and the appointed archaeologist for the National Monuments Council. During her retirement, Janette has managed the Southern African Rock Art Project. The project, with assistance from various organisations, including the Getty Conservation Institute, has allowed for the training of  staff, at various Southern African world heritage and protected sites, as rock art tourist guides, assistants for management and conservation of rock art, and as contributors to the process of world heritage site nomination. In 2016 UCT awarded her an Honorary D.Litt. in recognition of her contribution to the field of archaeology.}
\begin{labeling}{IJSRA}	
\item[IJSRA (International Journal of Student Research in Archaeology)] \emph{What drew you to archaeology and what path did you follow to become an archaeologist?}
	
\item[JD]


\item[IJSRA] \emph{Archaeologists usually have at least one story to tell around the campfire about their career or fieldwork experiences, do you have one to share with us?}
	
\item[JD]



\item[IJSRA] \emph{What are your thoughts on the state of archaeology in Southern Africa? What changes need to be addressed?}
	
\item[JD]


\item[IJSRA] \emph{What difficulties do you think local students face in pursuing a career in archaeology? What advice would you give to aspiring archaeologists?}
	
\item[JD]


\item[IJSRA] \emph{Do you think the IJSRA and other student archaeology platforms and groups impact the discipline?}
	
\item[JD]

\end{labeling}
\IJSRAseparator
\IJSRAsection{Janette Deacon}
{\sffamily Janette Deacon, a PhD graduate from the University of Cape Town, SA, is now retired. She was a lecturer, a journal editor, a Later Stone Age and rock art specialist, and the appointed archaeologist for the National Monuments Council. During her retirement, Janette has managed the Southern African Rock Art Project. The project, with assistance from various organisations, including the Getty Conservation Institute, has allowed for the training of  staff, at various Southern African world heritage and protected sites, as rock art tourist guides, assistants for management and conservation of rock art, and as contributors to the process of world heritage site nomination. In 2016 UCT awarded her an Honorary D.Litt. in recognition of her contribution to the field of archaeology.}
\begin{labeling}{IJSRA}	
\item[IJSRA (International Journal of Student Research in Archaeology)] \emph{What drew you to archaeology and what path did you follow to become an archaeologist?}
	
\item[JD]


\item[IJSRA] \emph{Archaeologists usually have at least one story to tell around the campfire about their career or fieldwork experiences, do you have one to share with us?}
	
\item[JD]



\item[IJSRA] \emph{What are your thoughts on the state of archaeology in Southern Africa? What changes need to be addressed?}
	
\item[JD]


\item[IJSRA] \emph{What difficulties do you think local students face in pursuing a career in archaeology? What advice would you give to aspiring archaeologists?}
	
\item[JD]


\item[IJSRA] \emph{Do you think the IJSRA and other student archaeology platforms and groups impact the discipline?}
	
\item[JD]

\end{labeling}
\IJSRAseparator
\IJSRAsection{Janette Deacon}
{\sffamily Janette Deacon, a PhD graduate from the University of Cape Town, SA, is now retired. She was a lecturer, a journal editor, a Later Stone Age and rock art specialist, and the appointed archaeologist for the National Monuments Council. During her retirement, Janette has managed the Southern African Rock Art Project. The project, with assistance from various organisations, including the Getty Conservation Institute, has allowed for the training of  staff, at various Southern African world heritage and protected sites, as rock art tourist guides, assistants for management and conservation of rock art, and as contributors to the process of world heritage site nomination. In 2016 UCT awarded her an Honorary D.Litt. in recognition of her contribution to the field of archaeology.}
\begin{labeling}{IJSRA}	
\item[IJSRA (International Journal of Student Research in Archaeology)] \emph{What drew you to archaeology and what path did you follow to become an archaeologist?}
	
\item[JD]


\item[IJSRA] \emph{Archaeologists usually have at least one story to tell around the campfire about their career or fieldwork experiences, do you have one to share with us?}
	
\item[JD]



\item[IJSRA] \emph{What are your thoughts on the state of archaeology in Southern Africa? What changes need to be addressed?}
	
\item[JD]


\item[IJSRA] \emph{What difficulties do you think local students face in pursuing a career in archaeology? What advice would you give to aspiring archaeologists?}
	
\item[JD]


\item[IJSRA] \emph{Do you think the IJSRA and other student archaeology platforms and groups impact the discipline?}
	
\item[JD]

\end{labeling}
\IJSRAseparator
\IJSRAsection{Janette Deacon}
{\sffamily Janette Deacon, a PhD graduate from the University of Cape Town, SA, is now retired. She was a lecturer, a journal editor, a Later Stone Age and rock art specialist, and the appointed archaeologist for the National Monuments Council. During her retirement, Janette has managed the Southern African Rock Art Project. The project, with assistance from various organisations, including the Getty Conservation Institute, has allowed for the training of  staff, at various Southern African world heritage and protected sites, as rock art tourist guides, assistants for management and conservation of rock art, and as contributors to the process of world heritage site nomination. In 2016 UCT awarded her an Honorary D.Litt. in recognition of her contribution to the field of archaeology.}
\begin{labeling}{IJSRA}	
\item[IJSRA (International Journal of Student Research in Archaeology)] \emph{What drew you to archaeology and what path did you follow to become an archaeologist?}
	
\item[JD]


\item[IJSRA] \emph{Archaeologists usually have at least one story to tell around the campfire about their career or fieldwork experiences, do you have one to share with us?}
	
\item[JD]



\item[IJSRA] \emph{What are your thoughts on the state of archaeology in Southern Africa? What changes need to be addressed?}
	
\item[JD]


\item[IJSRA] \emph{What difficulties do you think local students face in pursuing a career in archaeology? What advice would you give to aspiring archaeologists?}
	
\item[JD]


\item[IJSRA] \emph{Do you think the IJSRA and other student archaeology platforms and groups impact the discipline?}
	
\item[JD]

\end{labeling}
\IJSRAclosing
\end{document}