\#Abstract Poggio Civitate is the location of an ancient Etruscan
settlement in present-day Tuscany which, in the Archaic period
(c.580-c.480 BC), was home to a monumental courtyard complex, its roof
adorned with a series of terracotta acroteria. These acroteria took the
form of seated and standing men and women, which form the basis of this
study, as well as fantastic and real animals. The question addressed
will be the appropriate identification of these figures, regarding which
there is currently no scholarly agreement. The matter is contentious
because of the uncertainty about the function of the Archaic Building
itself. This paper argues that the Archaic Building most likely
functioned as a domestic complex, based on an analysis of the plan of
the building and its relationship to the previous phase. It is then
argued that the arbitrary categories of ``mortal'' and ``divine'' often
applied to the acroteria are an anachronism and that the acroteria
should in fact be identified more broadly as status markers: symbols of
the wealth and power of the Archaic Building's elite occupants.

\textbf{Introduction}

Poggio Civitate is located in northern inland Etruria, near to the River
Ombrone and the modern \emph{comune} of Murlo, an hour's drive south of
the major city of Siena (see \textbf{Fig.1}). It is situated in a
prominent, highly visible position (Meyers 2003: 114) on the eastern
portion of a hilltop plateau bordering central Tuscany (Donati 2000:
324; Meyers 2003: 72). Occupied since the Iron Age, the site was first
monumentalized in the Orientalizing period (c.720-c.580 BC) (Haynes
2000: 115). After the Orientalizing Period complex was destroyed in the
late seventh century BC (Turfa \& Steinmayer 2002: 3), Poggio Civitate
was rebuilt in the Archaic Period (Damgaard Andersen 1990: 79). It was
upon the Archaic Building that the rooftop acroteria considered in this
paper once stood.

A descriptive overview of the rooftop acroteria from the site will be
given in Part One. To interpret these acroteria, one must first decide
upon the likely function of the Archaic Building, something that is
still disputed. Therefore, Part Two of this paper is dedicated to
demonstrating precisely how the varied scholarly interpretations of the
site's function impact the understanding of the acroteria. From this
discussion, the conclusion is drawn that the Archaic Building functioned
as a domestic structure. Finally, in Part Three, the implications that
this proposed function has on the interpretation of the acroteria are
explored. All dates given will be BC.

\textbf{Fig.1:} Map showing the location of Poggio Civitate within wider
Etruria.\textbf{\\
Source:} Plan by Renate Sponer Za for Winter, Symbols of Wealth and
Power, 2009, page LII.

\textbf{Part One: Descriptive Overview}

The Orientalizing phase of occupation at Poggio Civitate
(\textbf{Fig.2}) is dated from 650 to around 610/600BC, when a fire
destroyed the site (Edlund-Gantz 1972: 198). The Orientalizing complex
comprised three buildings; OC1, a residence; OC2, an industrial
workshop; and OC3, a tripartite structure which has been interpreted as
a religious building (Haynes 2000: 115).

\textbf{Fig.2:} Plan of the Orientalizing phase at Poggio Civitate.

\textbf{Source:} Drawing by Renate Sponer Za for Winter, Symbols of
Wealth and Power, 2009, Plan 8.

The Archaic Building (\textbf{Fig.3}) was constructed between 600-575BC
atop the remains of the Orientalizing complex (Haynes 2000: 117). It
consisted of a monumental four-winged structure measuring 60 x 61.8m
(Meyers 2012: 4), enclosing a colonnaded courtyard (O'Donoghue 2013:
269), surrounded by four wings of continuous rooms (Meyers 2003: 70). On
the western side, a defensive structure extended the complex a further
30m (Tuck et al.~2010: 93).

\textbf{Fig.3:} Plan of the Archaic Building at Poggio Civitate.

\textbf{Source:} Drawing by Renate Sponer Za for Winter, Symbols of
Wealth and Power, 2009, Plan 9.

The acroteria were part of a decorative programme which encompassed the
whole terracotta roof. \textbf{The} roof was also decorated with
sculpted female heads, feline waterspouts, rosettes, and gorgon
antefixes (Edlund-Gantz 1972: 203), as well as relief sculpted revetment
plaques, bearing images of a banquet, a procession, an assembly, and a
horse race (Rathje 1994: 95). Stylistic analyses suggest that all of the
roof elements were conceived as one sculptural programme (Winter 2009:
159). \textbf{Fig.4} provides a guide to where these elements would have
been situated on the roof.

\textbf{Fig.4:} Generic roof with names of architectural elements.

\textbf{Source:} Drawing by Renate Sponer Za for Winter, Symbols of
Wealth and Power, 2009, page 6.

Now that the context of the Archaic Building at Poggio Civitate has been
defined, the general features of the acroteria can be described, before
the specifics of each category of sculpture from the roof are given. All
of the acroteria are made of terracotta (Phillips 1992: 17), sculpted in
the round, and would originally have been painted (Winter 2009: 511).
They were located on top of the tiles used to protect the ridgepole, the
horizontal beam that formed the apex of the roof. This location is
identified by the fact that the feet of some of the acroteria are still
connected to ridge tiles (Winter 2009: 198). Although around 30\% of the
building's entire decorative elements survive (Tuck 2016: 111), no
single acroterion is whole, but there are over 200 fragments
(Edlund-Berry 1993: 117).

There are five categories of acroteria: seated figures, striding
figures, helmeted heads, and real and fantastic animals. Fragments from
at least ten life-size seated males are known (Winter 2009: 155).
\textbf{\emph{The}} bearded males (\textbf{Fig.5}) wear wide-brimmed,
high-crowned hats, long tunics, and pointed boots (Haynes 2000, 120;
Winter 2009: 196). They are seated on stools.

\textbf{Fig.5:} Reconstruction drawing of a seated male acroterion from
the Archaic Building. This reconstruction is now known to comprise
elements from two different figures but it nevertheless gives a general
idea of the original appearance of this class of sculpture.

\textbf{Source:} Drawing by Renate Sponer Za for Winter, Symbols of
Wealth and Power, 2009, Ill. 3.8.1 3.E.1.a.

The seated female sculptures (\textbf{Fig.6}) are estimated to number at
least nine (Winter 2009: 155) and are slightly smaller than the males.
The seated women, also positioned on stools (Winter 2009: 198), wear
long skirts, medallions, and boots with curled toes (Haynes 2000: 120).
The positioning of the hands of both the male and female seated figures
indicates that they once held attributes, but these are now lost (Tuck
2016: 110). The formation of the ridge tile fragments below the
acroteria suggest that the seated statues were positioned across the
axis of the roof, either facing into or out of the courtyard
(Edlund-Berry 1993: 119). In addition to this, at least four figures
represented in a striding pose are known from the roof (Winter 2009:
155). Only the ball of the foot and the toes of these figures touches
the ridge tile, indicating a striding posture (Winter 2009: 199). At
least six helmeted heads have been discovered, which likely belong to
these striding figures (Winter 2009: 200).

\textbf{Fig.6:} Reconstruction drawing of a seated female acroterion
from the Archaic Building.

\textbf{Source:} Drawing by Renate Sponer Za for Winter, Symbols of
Wealth and Power, 2009, Ill. 3.8.2. 3.E.1.b.

An estimated minimum of thirty-two animal statues have been recovered
from Poggio Civitate (Winter 2009: 157). Usually, these are smaller than
the human figures (Donati 2000: 324). The fantastic animals include
sphinxes, centaurs, griffins, and a hippocamp (Winter 2009: 157), as
well as real animals like felines, boars, rams, horses, and hippopotami
(Tuck 2016: 111). The animal acroteria, like the striding figures, were
oriented along the axis of the roof (Edlund-Berry 1993: 119).

\textbf{Fig.7} Reconstruction drawing of a male sphinx, part of the
acroterion. From the Archaic Building

\textbf{Source:} Drawing by Renate Sponer Za for Winter, Symbols of
Wealth and Power, 2009, Ill. 3.8.3. 3.E.2.c.

The exact location of the sculptures along the ridgepole is unknown
(Edlund-Berry 1993: 119) because the remains of the acroteria are often
not found in the position where they would have fallen from the roof
(Edlund-Gantz 1972: 209). Fragments have been discovered in four main
locations across the site: alongside the building's north wing; in the
fossa ditch which runs along the building's west wing; within the
so-called ``dump'' past the fossa; and inside the courtyard
(Edlund-Gantz 1972: 210). An assemblage of fragments belonging to the
human figures has been discovered on the northern flank, which might
indicate their original placement here (Winter 2009: 157). The animal
acroteria may have occupied the roofs of the other three wings (Winter
2009: 157). \textbf{Fig.8} indicates how the acroteria may have looked
on the roof.

\textbf{Fig.8:} Line drawing reconstructing the appearance of the
acroteria on the roof of the Archaic building.

\textbf{Source:} Drawing by Renate Sponer Za for Winter, Symbols of
Wealth and Power, 2009, Ill. Roof 3-8. 3. Ridge.

\textbf{Part Two: Reconciling the} \textbf{Building Function}

The human acroteria figures that form the main subject of this paper
have been variously interpreted in scholarship as divinities
(Edlund-Berry 1993; Turfa \& Steinmayer 2002; Rathje 2007), heroized
ancestors (O'Donoghue 2013), and symbols of the surrounding community
(Tuck 2016). These interpretations are problematic for a number of
reasons. First, the fragmentary nature of the remains and their
scattered find-spots makes a full reconstruction of the figures
difficult. Second, they lack attributes that might aid their
identification. Third, the understanding of the acroteria is tied up
with the interpretation of the function of the Archaic Building as a
whole (Edlund-Gantz 1972: 206), but there is little scholarly agreement
about the use of the complex. Problems with interpreting the function of
the Archaic Building stem from the fact that it was emptied of many of
its goods prior to its abandonment (Tuck 2016: 111) and that many of the
small finds that were discovered were recorded in a manner that would be
deemed imprecise by modern standards (Rathje 2004: 59). There is also
disagreement about the destruction of the building. Furthermore, few
comparanda for the Archaic Building exist, save for one similar complex
at Acquarossa, near Viterbo (Meyers 2013: 46). The effect that the
proposed function of the building has on the interpretation of the
acroteria can be shown through examining the various scholarly debates.

\emph{Sanctuary}

Scholars previously believed that all Etruscan buildings decorated with
architectural terracottas were religious (Phillips 1992: xv). The
interpretation of the Archaic Building as a religious complex might lead
one to identify the figures on the roof as gods. Figures of gods as
acroteria are indeed found on sanctuary buildings throughout Etruria.
Examples include the fragmentary acroteria thought to represent Herakles
and Athena from a temple at Satricum, dated 540-520BC (Winter 2009:
466), the central acroterial group representing Herakles and Athena from
the second-phase of the Temple of Matuta at S. Ombono in Rome, dated
530BC (Winter 2009: 379), and the terracotta acroteria representing
Herakles and Apollo from the Portonaccio Temple at Veii, dated 510BC
(Neils 2008: 40). These acroteria are identified as deities by the
presence of attributes such as the helmet of Athena and lion skin of
Herakles.

Arguments for reading Poggio Civitate's Archaic Building as religious
are made on the grounds of artistic comparanda and the supposed ritual
destruction of the site. Artistic comparanda can be found in painted
pottery and bronze reliefs from the Orientalizing period. Here,
divinities are frequently represented alongside fantastic animals, and
the presence of these two motifs together on the roof at Poggio Civitate
has been taken to indicate that the seated figures are divine (Phillips
1992: 47). Another stylistic argument is made on the grounds that the
seated acroteria are shown in a similar manner to seated figures on the
assembly revetment plaques from the building (\textbf{Fig.9})
(Edlund-Berry 1993: 120). These figures are often interpreted as gods
(Rathje 2007: 180).

\textbf{Fig.9} Line drawing of one of the terracotta revetment plaques
from the Archaic Building, illustrating an assembly scene. The
participants in this scene are commonly interpreted as gods.

\textbf{Source:} Drawing by Renate Sponer Za for Winter, Symbols of
Wealth and Power, 2009, Ill. 3.7.3. 3.D.5.c.

The supposed ritual destruction of the Archaic Building may also
indicate a religious function. Scholars see evidence for ritual
destruction at Poggio Civitate in the burial of some of the
architectural terracottas in the fossa ditch, 25m away from the building
(Edlund-Berry 1993: 17). This has been taken to indicate the deliberate
removal of these artefacts. There is comparative evidence from Etruscan
religious contexts for the burial of sacred images. For example, the
temple terracottas at Veii underwent a similar deposition (De Grummond
1997: 33). Other evidence for ritual destruction comes from a well on
the site, which contained large quantities of roof tile, all dated to
the Archaic period (Tuck et al.~2010: 97). The excavator has concluded
that the tiles were deliberately deposited there as part of the ritual
destruction (Tuck et al.~2010: 102). Ritual destruction of a site is
thought to be an indicator of sanctity (Edlund-Berry 1993: 121), so this
evidence has brought about the conclusion that Poggio Civitate was a
sanctuary, and that the acroteria represent divinities.

However, there are many counterarguments to this theory. First, the plan
of the Archaic Building is nothing like any other known religious
structure (Phillips 1992: xv). There are no votives, altars, or
religious inscriptions at Poggio Civitate, which are indicative of a
sanctuary (Damgaard Andersen 1990: 79). The stylistic evidence is also
problematic: that the figures on the assembly plaque are divinities is
contentious (Damgaard Andersen 1990: 79); the assembly scene may equally
represent mortal banqueters (Rathje 2007: 180). Furthermore, unlike the
comparable examples of acroteria from Veii, Caere, and S. Ombono, which
bear attributes of the gods and goddesses that they represent, no extant
iconographic feature of the acroteria at Poggio Civitate suggests that
they are divinities (De Grummond 1997: 36).

Additionally, the claim that the site was ritually destroyed is not
universally accepted. Most arguments regarding the ritual destruction
rest on the fact that the acroteria were ritually buried. But not all
the terracottas from the Archaic building were found in the so-called
ritual burial area; some were found where they are likely to have fallen
during the building's collapse (De Grummond 1997: 33). If ritual burial
of the acroteria was the intention, the job was incomplete (De Grummond
1997: 33). It can be argued that the building was instead abandoned.
This is indicated by the fact that many of the contents of the Archaic
Building were removed prior to destruction, suggesting an intentional
abandonment rather than catastrophic demolition (Tuck 2016: 111).
Additionally, the erosion of the Archaic Building's eastern wing
indicates that the destruction was a drawn out process (De Grummond
1997: 34). Due to the refutation of the ritual destruction evidence and
the absence of votives and other religious material from the site, the
theory that the Archaic Building had an exclusively religious function
is now largely abandoned in scholarship.

\emph{Communal Meeting-Place}

That the Archaic Building functioned as a sanctuary is not the only
possible conclusion that can be drawn from the ritual destruction
argument. Other proponents of the ritual destruction theory, working
from the same evidence detailed above, state that the Archaic Building
was thus destroyed because it was a powerful political center (Phillips
1992: 49). Phillips argued that the time, effort, and expense involved
in ritual destruction indicate that Poggio Civitate was a center of
political importance, destroyed as it posed a threat to neighbouring
communities (Phillips 1992: 49). However, Phillips leaves the exact
nature of the powerful political organization he believed to be based at
Poggio Civitate obscure. There are two available theories: an
organization of either local or regional significance.

Firstly, the building may well have served as a gathering place and
mercantile center for the local community. Festivals, exchange, and
trade necessitated a gathering of the population, and the current
excavator suggests that Poggio Civitate was the place where this
occurred (Tuck 2016: 112). The plan of the building has elements which
might be suited to public use, including the large, two storied rooms
within the Archaic complex's northern flank (Donati 2000: 324; Winter
2009: 153). The smaller rooms within the western and southern flanks may
have been appropriate for use as shops (Turfa \& Steinmayer 2002: 9).
The complex thus might be seen as a venue for community interaction.

Conversely, others argue that the central location of Poggio Civitate
within northern Etruria means that it served as a meeting place not for
the local community but for an Etruscan league of northern cities.
Dionysius of Halicarnassus (\emph{Roman Antiquities} 3.51) refers to an
alliance between the Etruscan towns of Volterra, Arezzo, Chiusi,
Vetuloni, and Rusellae, and the site of Poggio Civitate was ideally
placed to connect all of these urban centers (Edlund-Gantz 1972: 198).
The sizeable Archaic building might have accommodated members of the
league (Damgaard Andersen 1990: 79). If either of these interpretations
about the public function of the Archaic Building hold true, the
acroteria might thus be interpreted as symbols of these collective
entities, either the local community, or the northern league (Tuck 2016:
112).

\textbf{But again, these theories are problematic. Livy attests to the
existence of an Etruscan League (4.23.5, 4.25.7, 4.61.2, 5.17.6, 6.2.2),
which is believed to have met at the \emph{Fanum Voltumnae} near Orvieto
(Haynes 2000: 135; Stopponi 2014: 632), but nothing from Poggio Civiate
suggests that such a league met at this site (O'Donoghue 2013: 273).}
\textbf{Additionally, for the complex to be a gathering space for the
local community, one would expect the Archaic Building to be the only
structure of its type in the area. However, this is not the case. Poggio
Civitate was not isolated; there are thought to be several other
courtyard complexes in the near vicinity (Acconcia 2012: 96-105).
Consequently, that Poggio Civitate functioned as a central gathering
place either for the surrounding community or for a league of cities is
doubtful, as is the interpretation of the acroteria as emblems of these
associations.}

\emph{Elite Residence}

Another theory is that the Archaic Building functioned as a domestic
structure, the residence of an elite Etruscan family, on the basis of
which, the acroteria are seen as either ancestors of this aristocratic
dynasty (O'Donoghue 2013: 268), or divinities associated with them
(Turfa \& Steinmayer 2002: 9). Those who propose such a theory base
their arguments on the few small finds from the Archaic Building and the
elaborate architectural decoration.

That people lived on the site of Poggio Civitate is suggested by the
discovery of domestic pottery and courseware (Damgaard Andersen 1990:
79). The size of the building, and the luxurious small finds of ivory
and bronze, and imported Greek pottery (Edlund-Gantz 1972: 201) have
been considered evidence that the building served as a residence for an
elite family (Edlund-Berry 1994: 16). Although not peculiar to
residential contexts, the aristocratic themes such as banqueting which
are illustrated on the revetment plaques from the Archaic Building have
also been taken to indicate that it functioned as a domestic complex
(O'Donoghue 2013: 272). However, \textbf{the} original excavator wrote
that it was ``out of the question'' for the building to be considered a
domestic complex (Phillips 1992: xvi). Nevertheless, this paper,
building on the above arguments made by O'Donoghue (2013) and Turfa and
Steinmayer (2002), argues that an analysis of the plan of the Archaic
Building confirms its residential function.

The plan of the Archaic Building at Poggio Civitate may well indicate
its use as a domestic structure. Although there are few well-excavated
parallels in Etruscan archaeology for a structure like the Archaic
Building, there is comparative evidence from Zone F in Acquarossa
(\textbf{Fig.10}) (Edlund-Gantz 1972: 199).

\textbf{Fig.10:} Plan of the monumental residence in Zone F at
Acquarossa. Dated to the second quarter of the sixth century.

\textbf{Source}: Drawing by Renate Sponer Za for Winter, Symbols of
Wealth and Power, 2009, Plan 2.2

At Acquarossa, a monumental residence was constructed in the mid
6\textsuperscript{th} century, comprising two L-shaped buildings and a
forecourt (Haynes 2000: 139). Like the complex at Poggio Civitate,
smaller rooms were arranged around the central courtyard, which was
porticoed on two sides. Small finds, including cooking stoves, hearths,
and wool-working equipment, suggest that the Acquarossa building was
domestic (Haynes 2000: 141). Although the complex at Acquarossa is
smaller than the Archaic Building at Poggio Civitate (Meyers 2012: 5),
their similarities in plan and decoration (the roof at Acquarossa was
also tiled and bore terracotta adornments (Damgaard Andersen 2001: 256))
mean that they may well have functioned similarly.

However, there are some notable differences between the two structures.
For instance, the Acquarossa building is located in an urban area,
unlike Poggio Civitate (Meyers 2013: 43). Furthermore, the large
northern rooms in the Archaic Building are not paralleled at Acquarossa.
This may shed doubt on the suggestion that the buildings functioned in
the same manner. Nevertheless, evidence from the Orientalizing period
domestic structure at Poggio Civitate provides further parallels which
link the Archaic Building with a domestic function.

At Poggio Civitate, Structure OC1 from the Orientalizing phase is almost
universally identified as a residential building. Its identification is
relatively secure, thanks to the numerous small finds. These include
cooking equipment and inlays for furniture as well as banqueting
equipment and precious metal jewellery (Tuck 2016: 107). Supporting
evidence of a domestic function for OC1 comes in the form of faunal
remains, which indicate the consumption of beef on the site, associated
with elite banqueting (Kansa and MacKinnon 2014: 85). OC1 is comparable
in size and plan to the large rooms in the Archaic Building's northern
flank. There is also evidence that OC1 was two-storied (Haynes 2000:
115), like the Archaic Building's northern rooms. The argument that the
northern part of the Archaic Building served a public function is based
on its size, but the domestic use of the similarly sized Orientalizing
period structure on the site suggests that this is not necessarily the
case. Such large quarters could evidently serve a domestic purpose in
the Orientalizing period and this may have continued in the Archaic
phase.

The strong sense of continuity between the Orientalizing and the Archaic
phases (Meyers 2013: 41) might add further weight to the argument that
the later building continued the functions of the earlier complexes. The
placement and orientation of the Archaic Building closely follows the
Orientalizing structures OC1 and OC3 (Turfa \& Steinmayer 2002: 3), seen
when comparing the two plans in \textbf{Figs.2-3}. As well as the
similarities in orientation and plan, the short lapse of time between
the destruction of the Orientalizing complex and the construction of the
Archaic Building has led some to conclude that the latter was built as a
direct replacement for buildings OC1 and OC3 from the earlier phase
(Edlund-Gantz 1972: 198). OC1 is confidently identified as a residence,
but OC3 has prompted more debate. Its tripartite layout has led scholars
to interpret it as a religious structure, because many later Etruscan
temples took this form (Tuck 2009: 95). However, contemporary domestic
structures with a tripartite plan have been found at Acquarossa (Potts
2011: 319). Therefore, OC3 may equally be interpreted as a domestic
building. This could lend itself to the conclusion that the Archaic
building was the successor to the domestic complexes OC1 and OC3, and
that it fulfilled the same role. However, this alone is insufficient
evidence to conclude that the Archaic Building served a residential
function. Functions of buildings can change through different phases of
occupation, as shown globally through examples such as the reuse of
Neolithic-era chamber tombs as dwellings in late Bronze Age/early Iron
Age Scotland (Hingley 1996); the repurposing of former Forum spaces as
domestic complexes in mid-fifth century AD Emerita (Osland 2016); and
the changing function of buildings used successively for habitation,
storage, and livestock enclosures in Iran (Cameron 1991, 44-5).
Nevertheless, the Archaic Building may still be interpreted as a
domestic area, based on the arrangement of rooms in the western and
eastern flanks of the building.

\textbf{\emph{Prima facie}, the layout of} the western and eastern
flanks \textbf{may resemble that of shops. However,} many of these
\textbf{communicate with each other through internal doors (Fig.11),
which is more indicative of a domestic structure. Rooms surrounding
courtyards in domestic complexes are often connected in this manner, as
seen in the plan of the eastern wing of the monumental residence from
Acquarossa (Haynes 2000: 139). Competing vendors in neighbouring shops
would be unlikely to have an interest in their premises being connected
in this way. Therefore, it seems likely that the rooms in the western
and southern flanks of the Archaic Building were also domestic.}

\textbf{Fig.11 Plan of the Archaic Building at Poggio Civitate. Arrows
added by author to indicate internal doorways within the western and
southern flanks.}

\textbf{Source: Drawing by Renate Sponer Za for Winter, Symbols of
Wealth and Power, 2009, Plan 9.}

In addition to this, the Archaic Building's courtyard structure may well
have functioned in the same manner as the atrium in the Roman
\emph{domus}. Diodorus Siculus (\emph{Library of History} 5.40) records
that in domestic architecture, the Romans adopted the Etruscan invention
of the peristyle courtyard,\footnote{In line with the Loeb translation
  of this passage, ``Τυρρηνῶν'' (literally ``Tyrrheni'') has here been
  read as ``Etruscans''. That ``Tyrrheni'' refers to the Etruscans is
  established by Strabo \emph{Geography} 5.2: ``Οἱ Τυρρηνοὶ τοίνυν παρὰ
  τοῖς Ῥωμαίοις Ἑτροῦσκοι καὶ Τοῦσκοι προσαγορεύονται'' / ``The
  Tyrrheni, then, are called among the Romans `Etrusci' and `Tusci.'\,''}
a structure used to contain the homeowner's entourage. The Roman form of
this structure is known as the atrium (Prayon 2009: 60), the reception
space where the Roman paterfamilias could receive clients for the
\emph{salutatio} ritual \textbf{\emph{(Platts 2016: 47). The anecdote
from Diodorus Siculus suggests that the Etruscan courtyard was used for
similar purposes, which may mean that the courtyard at Poggio Civitate
functioned as a semi-public space within a domestic complex, like the
Roman atrium. However,}} the size of the 40m² courtyard (Winter 2009:
153) might make this proposal seem unlikely. Yet \textbf{\emph{there is
strong evidence that the Orientalizing building OC1, which is of
comparable size to the Archaic Building, served a domestic purpose. This
means that the size of the complex need not discount the interpretation
of the Archaic Building having a residential function.}}

If the Archaic Building did function as a domestic complex, it would be
in keeping with trends seen in wider Etruria. Studies of general
settlement patterns in Rome's hinterland show a growth in the total
number of new sites during the Orientalizing period, with a twofold
increase in the Archaic period (Fulminante 2014: 141). This has been
interpreted as an aristocratic migration into the countryside
(Fulminante 2014: 142). The theory finds support in the succession of
so-called proto-villas, dating back to the 8\textsuperscript{th}
century, that have been discovered on Rome's Palatine; similar
settlement patterns are also known at Poggio dei Cavallari at Satricum,
beginning in the late sixth century (Fulminante 2014: 142). Therefore,
if we choose to read the Archaic Building as a domestic complex, it
would not be an anomaly in the Italian countryside, as other elite
countryside residences exist in the region in the Archaic period.
Overall, it seems highly likely that the Archaic Building at Poggio
Civitate functioned as a domestic complex.

\textbf{Part Three: Interpreting the Acroteria}

In Part Two, we reached the conclusion that the Archaic Building at
Poggio Civitate is likely to have served as an aristocratic residence.
Now the implications that this function has on the interpretation of the
acroteria can be ascertained.

Most scholars who view the Archaic Building as a domestic structure view
the figural acroteria as heroized ancestors of the elite family, or
deities associated with them. Advocates of the deity theory cite the
sculpture's position on the roof as evidence for their status as gods,
as they walk the sky in the manner of the divine (Warden 2009: 207). In
addition to this, divine figures are very often accompanied by fantastic
animals in Orientalizing art, and the presence of similar motifs on the
roof at Poggio Civitate has led to the conclusion that the acroteria
represent deities (Phillips 1992: 47). Stylistic parallels are cited as
evidence of the acroteria representing aristocratic ancestors (Turfa \&
Steinmayer 2002: 9). These parallels include the seated figures from the
Tomb of the Five Chairs at Cerveteri (\textbf{Fig.12}), dated 30 years
earlier (O'Donoghue 2013: 276). The seated figures from this chamber
tomb are thought to represent the ancestors of those interred within the
complex (Haynes 2000: 93). These figures have much in common with the
acroteria from the Archaic Building in terms of form and positioning,
which has led scholars to interpret similar meanings into the Poggio
Civitate statuary.

\textbf{Fig.12} Photograph of two of the terracotta seated ancestor
statues from the Tomb of the Five Chairs in Cerveteri, dated to the
seventh century BC (author's picture).

Straightforwardly interpreting the figural acroteria as divinities,
however, is problematic. While distinction between sacred and secular is
very clear in the modern realm (Edlund-Berry 1993: 121), Warden argues
that such a definite divide was not applicable for an Etruscan audience
(Warden 2009: 209-10). He theorises that for the Etruscans, divinity and
mortality were not hard and fast categories but two points on a single
spectrum, along which it was possible to move; he cites evidence for an
Etruscan belief that one's status could move from human to immortal
through the consumption of sacrificial meat (Warden 2009: 205). In
addition to this, the elite of Etruscan society was theocratic, meaning
that religious and secular powers were inseparable (Warden 2009: 198).
This is illustrated in the texts from the Zagreb mummy wrappings, one of
the few extant examples of substantial Etruscan literature. There are a
series of prayers are preserved where the divinities are consistently
referred to as noble (Warden 2009: 209). This emphasizes the conflation
of the noble and the divine for the Etruscans (Rix 1997: 393). It
therefore seems an anachronism to enforce the modern separate categories
of ``ancestor'' and ``god'' onto the acroteria. It may be more
appropriate, and closer to the original Etruscan perception of the
sculptures, to consider the acroteria from Poggio Civitate as neither
gods nor mortals.

This paper instead proposes that might be more appropriate to interpret
the acroteria by considering their prominence and visibility, and the
expenditure that their production necessitated. The acroteria were
highly visible in the landscape; they were positioned two stories above
ground level (Turfa \& Steinmayer 2002: 9), on a hilltop building that
would have been easily seen from the surrounding hills, and from the
complex's southeastern approach (Meyers 2013: 52). They were handmade
(Winter 2009: 509) with an incredibly high level of detail (Meyers 2012:
6) and their manufacture would have taken months, necessitating a
substantial financial outlay (Turfa \& Steinmayer 2002: 18). Firing the
terracottas would also have been costly due to the expense of the
charcoal required (Turfa \& Steinmayer 2002: 18). This conspicuous
consumption may well indicate that the acroteria should be interpreted
as status markers, prominent symbols which conveyed the wealth of their
owners.

In fact, the entire Archaic Building can be read as a public display of
elite status and prestige. The huge area occupied by the building speaks
of the elite residents' ability to control the landscape. The acroteria
complement this by communicating of an ability to command resources and
engage in conspicuous consumption. This interpretation also fits with
the animal acroteria found on the roof. Although the original excavator,
believing the seated figures to be gods, concluded that the fantastic
animals served to augment the status of the divine figures on the roof
(Phillips 1992: 24), there is no reason to believe that the same
interpretation cannot apply when the figural acroteria are more broadly
considered as status indicators. They can still be understood as
enhancing the prestige of the roof, as it seems reasonable to assume
that the real-life animals featured were chosen because they were held
in high regard by the Etruscans (Harrison 2013: 1091). In Etruscan art
from the Archaic period in general, domestic animals often accompanied
the main images (Harrison 2013: 1091), and the acroteria fit with this
overall trend. As for the fantastic animal acroteria, beasts of this
type are usually interpreted as having an apotropaic role: they are seen
as protective figures on temples and tombs (Harrison 2013: 1099). This
interpretation may also be applicable to the mythological animals seen
on the Archaic Building roof; they might be read as protectors of the
aristocratic occupants within. Therefore, the interpretation of the
animal acroteria is in-keeping with that of the rest of the statuary
from the roof, as collectively, the building itself, the figural
acroteria, and the animal acroteria all convey the same message of
wealth, power, and elite command over the landscape.

\textbf{Conclusion}

Overall, this paper has argued that the function of the Archaic Building
at Poggio Civitate was domestic, based on the internal layout of the
building and its relationship to the preceding Orientalizing phase. As a
consequence, this paper advocates the interpretation of the acroteria as
status markers, symbols of the wealth and power wielded by the
aristocrats who inhabited the Archaic Building. This approach is
prioritized over that seen in existing scholarship, which favours the
arbitrary categories of ``mortal'' or ``divine''.
