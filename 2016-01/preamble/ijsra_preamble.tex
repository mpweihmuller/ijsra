\usepackage[%
%oldstyle,%
%sflining%
]{biolinum}
\usepackage[%
lining,			%lining: keine tiefergelegte Zahlen Commands 				\oldstylenums{...} and \liningnums{...} are defined to allow for local use of old-style figures or lining figures, respectively.
]{ebgaramond} 
\setkomafont{partnumber}{\normalfont\Huge}
\setkomafont{part}{\normalfont\scshape\Huge}
\setkomafont{chapter}{\normalfont\LARGE}%\uppercase
\setkomafont{section}{\normalfont\Large}%\uppercase
\setkomafont{subsection}{\normalfont}%\uppercase
\setkomafont{subsubsection}{\normalfont}%\uppercase
\setkomafont{paragraph}{\normalfont\scshape}%\uppercase
\addtokomafont{descriptionlabel}{\normalfont\sffamily}%\textsf\large}%\uppercase
\addtokomafont{sectioning}{\linespread{1}\selectfont} %einfacher Zeilenabstand in Überschriften
%-----------------------------
\usepackage{babel}
%\usepackage[newcommands]{ragged2e}%Flattersatz mit Trennungen
%\linespread{1.05} % Line spacing - Palatino needs more space between lines
\usepackage[%
final,% final - enable microtype; use "draft" to disable
stretch=10,% stretch=10, shrink=10 - reduce stretchability/shrinkability (default is 20/20)
shrink=10
]{microtype}

\usepackage{xspace}
\usepackage{colortbl}
\usepackage{abbrevs}
\usepackage{alertmessage}
\robustify{\DateMark} % after having loaded abbrevs
%-------------------------------------------
\usepackage[ 
	headsepline, 
%	plainheadsepline,
%	footsepline,
%	plainfootsepline, 
%markcase=upper, 
%automark, 
draft=false,
]{scrlayer-scrpage} 
\addtokomafont{headsepline}{\color{mygold}}
\KOMAoptions{headsepline=3pt}


\pagestyle{scrheadings}
\clearscrheadfoot


\lohead[\normalfont\sffamily\footnotesize International Journal of Student Research in Archaeology (IJSRA) \newline 
\normalfont\sffamily\footnotesize \issuemonth\ \issueyear\ $\bullet$ Vol. \issuevolume, No. \issuenumber \ppages]{\normalfont\sffamily\footnotesize \shorttitle\ \ppages}
\rehead{\normalfont\sffamily\footnotesize \shortauthor}
\rohead[{{\vspace{-3em}\includegraphics[width=2cm]{figures/ijsra_logo}}}]{\normalfont\textbf{\sffamily\thepage}}
\lehead{\normalfont\textbf{\sffamily\thepage}}
\refoot{\normalfont\sffamily\footnotesize International Journal of Student Research in Archaeology}
\lofoot{\normalfont\sffamily\footnotesize \issuemonth\ \issueyear\ $\bullet$ Vol. \issuevolume, No. \issuenumber}
\automark[subsection]{section}  
%\renewcommand{\sectionmark}[1]{\markright{#1}}

%-----------------------------------------
\usepackage[%
%flushmargin, %
			marginal,
ragged,%
hang, %
bottom%
]{footmisc} %Fussnoten
\raggedbottom
\addtolength{\skip\footins}{.5\baselineskip} % Abstand Text <->
% Fußnote
%		\setlength{\footnotemargin}{1em}
\deffootnote{0em}{1em}{{\sffamily\textbf\thefootnotemark}\ }%Ausgabe der Fußnotenziffer in normal
%--------------------------------------
\newcommand{\mychapter}[1]{
	\setcounter{chapter}{1}
%	\setcounter{section}{0}
	\chapter*{#1}
	\addcontentsline{toc}{chapter}{\shortauthor\newline \maintitle}
}
\usepackage{chngcntr} %Voraussetzung fuer Fussnoten durch alle Kapitel durchnummerieren
\counterwithout{footnote}{chapter} %Fussnoten durch alle Kapitel durchnummerieren
\counterwithout{figure}{chapter}
\counterwithout{table}{chapter}
%--------------------------------------
%\usepackage{titling}
%\settowidth{\thanksmarkwidth}{*}
%\setlength{\thanksmargin}{.07cm}
%\usepackage[noblocks]{authblk}
%\renewcommand\Authfont{\scshape}
%\renewcommand\Affilfont{\itshape}
%\renewcommand\Authands{,  }
\usepackage{metalogo}
\usepackage{etoolbox}
\usepackage{graphicx}
\usepackage{wrapfig} % Paket zur Positionierung einbinden
\usepackage{booktabs}%für schönere Tabellen
\usepackage{multirow}
\usepackage{xcolor}
\usepackage{pdfpages}
\clubpenalty=10000				% prevent single lines at the beginning of a paragraph (Schusterjungen)
\widowpenalty=10000				% prevent single lines at the end of a paragraph (Hurenkinder)
\displaywidowpenalty=10000		%
%-----------------------------------------
\usepackage[					% page layout modifications
paper=a4paper,					% 	- use A4 paper size
head=4\baselineskip,							% 	- no header
foot =4\baselineskip,
bindingoffset=0.5cm,			% 	- binding correction
top=3cm,						% 	- total body: top margin
left=2cm,					% 	- total body: left margin (odd pages)
right=5cm,					% 	- total body: right margin (odd pages)
bottom=5cm,					% 	- total body: bottom margin
marginparwidth=4cm,			% 	- width for side note
%	marginparsep=10pt,				% 	- space between notes and body text (content)
%	footskip=2cm,					% 	- footer skip size
]{geometry}
%\usepackage{canoniclayout}
\usepackage{multicol} % Used for the two-column layout of the document
\usepackage[hang, small,labelfont=bf,up,textfont=it,up]{caption} % Custom captions under/above floats in tables or figures
\usepackage{subcaption}
\usepackage{setspace}			% for line spacing, e.g. \onehalfspacing
\usepackage{booktabs} % Horizontal rules in tables
\usepackage{float} % Required for tables and figures in the multi-column environment - they need to be placed in specific locations with the [H] (e.g. \begin{table}[H])
\usepackage{nth}

\usepackage[					% advanced quotes
strict=true,					% 	- warning are errors now
style=english,					% 	- german quotes
%german=guillemets		muss für \MakeAutoQuote (s.u.) deaktiviert sein
]{csquotes}
%\MakeAutoQuote{»}{«}		%% macht aus « und » deutsche Anführungszeichen
%\MakeAutoQuote*{›}{‹}		%% macht aus › und ‹ deutsche einfache Anführungszeichen
%\defineshorthand{">}{»} \defineshorthand{"<}{«}
\usepackage{siunitx} %Supreme typesetting of units
\sisetup{%
	detect-all, %Zahlen werden in der aktuellen Schrift angezeigt
	exponent-to-prefix  			= true,
	round-mode          				= places, 
	round-precision     			= 2,
	group-minimum-digits 	= 4, % Für "Tausenderpunkt" --> 1.234 anstatt 1234
	group-separator={,},% für "12.345" statt "12 345"
	%  scientific-notation = engineering, % Use multiples of 3 as exponent
	%	locale=DE, % Typeset numbers and units the German way
	range-phrase ={$\times$},%
	zero-decimal-to-integer,%aus "2.0" wird "2"
	range-units=single,  % --> 2 x 2 m, - auskommentieren für 2 m x 2 m
}


\usepackage{lettrine} % The lettrine is the first enlarged letter at the beginning of the text
\usepackage{paralist} % Used for the compactitem environment which makes bullet points with less space between them

\usepackage{marginnote}
\renewcommand{\marginfont}{%\noindent\rule{0pt}{0.7\baselineskip}
	\normalfont\small\sffamily}
\newenvironment{myabstract}%
{\list{}{\rightmargin\leftmargin}%
	\footnotesize\itshape\item\relax}
{\endlist}




%\usepackage{abstract} % Allows abstract customization
%\renewcommand{\abstractnamefont}{\normalfont\bfseries} % Set the "Abstract" text to bold
%\renewcommand{\abstracttextfont}{\normalfont\small\itshape} % Set the abstract itself to small italic text






%\definecolor{myblue}{RGB}{6,2,52}
\definecolor{myblue}{rgb}{0,0,51}
\definecolor{mygold}{RGB}{204,153,51}
%\definecolor{mygold}{cmyk}{0,25,75,20}


%%%%%%%%%%%%%   biblatex   %%%%%%%%%%%%%%%%%%
\usepackage[					%% use  for bibliography
backend=biber,
style=authoryear,
isbn=false,
giveninits=true,
%dashed=false,
]{biblatex}
\renewcommand{\labelnamepunct}{\addcolon\space}
\renewcommand*{\bibpagespunct}{\addcolon\space}
\renewcommand*{\nameyeardelim}{\addcomma\addspace}
\renewcommand*{\postnotedelim}{\addcolon}
\DeclareFieldFormat{postnote}{#1}
\DeclareFieldFormat{pages}{#1}
\DeclareNameAlias{sortname}{family-given}
\DeclareNameAlias{default}{family-given}
\DeclareNameAlias{editor}{sortname}

\DeclareFieldFormat{multipostnote}{#1}
\DeclareFieldFormat[article,inbook,incollection,inproceedings,patent,thesis,unpublished]{citetitle}{#1\isdot}
\DeclareFieldFormat[article,inbook,incollection,inproceedings,patent,thesis,unpublished]{title}{#1\isdot}
%\renewbibmacro{in:}{%
%  \ifentrytype{article}{}{\setunit{\addcomma\addspace }\printtext{\bibstring{in}\addspace}}}
  
\usepackage{xpatch}
\xpatchbibmacro{date+extrayear}{%
  \printtext[parens]%
}{%
  \setunit{\space}%
  \printtext%
}{}{}

\renewbibmacro*{in:}{
\ifentrytype{incollection}{%
%  \DeclareNameAlias{editor}{first-last}
\setunit{\addcomma\addspace }\printtext{\bibstring{in}\addspace}%
  \ifnameundef{editor}
    {}
    {\printnames{editor}%
     \addspace 
     \mkbibparens{\usebibmacro{editorstrg}}
     \setunit{\addcomma\addspace}% 
    }%
  \usebibmacro{maintitle+booktitle}
  \clearfield{maintitle}
  \clearfield{booktitle}
  \clearfield{volume}
  \clearfield{part}
  \clearname{editor}
  }
  {%
  \ifentrytype{inproceedings}{%
%  \DeclareNameAlias{editor}{first-last}
\setunit{\addcomma\addspace }\printtext{\bibstring{in}\addspace}%
  \ifnameundef{editor}
    {}
    {\printnames{editor}%
     \addspace 
     \mkbibparens{\usebibmacro{editorstrg}}
     \setunit{\addcomma\addspace}% 
    }%
  \usebibmacro{maintitle+booktitle}
  \clearfield{maintitle}
  \clearfield{booktitle}
  \clearfield{volume}
  \clearfield{part}
  \clearname{editor}
  }{%
  \ifentrytype{article}{}{\setunit{\addcomma\addspace }\printtext{\bibstring{in}\addspace}}}}}%

\renewbibmacro*{volume+number+eid}{%
  \printfield{volume}%
%  \setunit*{\adddot}% DELETED
%  \setunit*{\addnbspace}% NEW (optional); there's also \addnbthinspace
  \printfield{number}%
  \setunit{\addcomma\space}%
  \printfield{eid}}
\DeclareFieldFormat[article]{number}{\mkbibparens{#1}}

%----------------------------------------------------------------------------------------
%	TITLE SECTION
%----------------------------------------------------------------------------------------

\usepackage{hyperxmp}
\usepackage{hyperref}
\hypersetup{					% setup the hyperref-package options
	pdftitle={International Journal of Student Research in Archaeology (IJSRA)},	% 	- title (PDF meta)
	pdfsubject={Issue 1},% 	- subject (PDF meta)
	pdfauthor={},	% 	- author (PDF meta)
	pdfauthortitle={},
	pdfcopyright={Copyright (c) \the\year\ IJSRA. All rights reserved.},
	pdfhighlight=/N,
	pdfdisplaydoctitle=true,
	pdfdate={\the\year-\the\month-\the\day}
	pdflang={de},
	pdfcaptionwriter={Lukas C. Bossert},
	pdfkeywords={International Journal, Archaeology, Undergraduates},
	pdfproducer={XeLaTeX},
	pdflicenseurl={http://creativecommons.org/licenses/by-nc-nd/4.0/},
	plainpages=false,			% 	- 
  colorlinks   = true, %Colours links instead of ugly boxes
  urlcolor     =  myblue, %Colour for external hyperlinks
  linkcolor    = myblue, %Colour of internal links
  citecolor   = myblue, %Colour of citations
  linktoc=page,
  	pdfborder={0 0 0},			% 	-
	breaklinks=true,			% 	- allow line break inside links
	bookmarksnumbered=true,		%
	bookmarksopenlevel=2,
	bookmarksopen=false,		%
	final=true	% = true, nur bei web-Dokument!! (wichtig!!)
}