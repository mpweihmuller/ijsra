\openingarticle
\def\ppages{\pagerange{editorial:firstpage}{editorial:lastpage}}
\def\shorttitle{IJSRA Editorial}
\def\maintitle{A Student Perspective on the Present of Archaeology: IJSRA Editorial}
\def\shortauthor{Gonzalo Linares Matás}
\def\authormail{gonzalo.linaresmatas@st-hughs.ox.ac.uk}
\def\affiliation{St. Hugh's College, University of Oxford}
\def\thanknote{\footnote{\href{https://oxford.academia.edu/GonzaloLinaresMat\%C3\%A1s}{Gonzalo Linares Matás} is a second-year undergraduate studying the BA in Archaeology \& Anthropology at St. Hugh’s College, University of Oxford. He is particularly interested in archaeological theory, the socio-political contexts of heritage interpretation, the representation of the past and archaeological practice, and human interaction both in the recent and distant past (cultural contact, colonisation, resistance, conflict resolution, environmental management) from a historical and anthropological perspective. He has published in a number of international undergraduate journals, such as \textit{Trowel} (UCD) or \textit{Dig It} (Flinders University, Australia). His undergraduate dissertation focuses on the experimental and use-wear analysis of Palaeolithic bone remains, which may turn out to be the oldest bone tools so far discovered in Europe. He has undertaken fieldwork in Murcia, Spain (sites of Cueva Negra, Sima de las Palomas, La Almoloya), England (Dorchester) and Greece (Knossos Gypsades, Crete). He is the former President of the Oxford University Archaeological Society, Coordinator of the IV OUAS National Undergraduate Conference (March 2016), member of the \textit{Murcian Association for the study of Paleoanthropology and the Quaternary} and Founder and Executive Editor of the \textit{International Journal of Student Research in Archaeology}.}
}
%--------------------------------------------------------------
\mychapter{\maintitle}
\begin{center}
{\Large\scshape\shortauthor \thanknote}\\[1em]
\email \\%[.5em]
\affiliation
\end{center}
\vspace{3em}
\midarticle
%--------------------------------------------------------------
\label{editorial:firstpage}
%--------------------------------------------------------------
\lettrine[nindent=0em,lines=3]{S}{tudents} play a fundamental role in the present and future development of any academic discipline. With their idealism, passion for improving, and fresh and innovative perspectives, they are uniquely placed to foster changes in the academic landscape, particularly concerning the visibility and relevance of their work. Archaeology is not yet economically self-sustainable. Therefore, it depends on public funding and development to maintain its research activity. In the age of publish or perish, where research and impact has to be measurable and quantified, funding availability is widening the gap between the opportunities available for student research and publication throughout the world. Some eminent researchers have argued that there may not be a need for another publication. In fact, I agree that the publishing sector is already oversaturated with low quality publications aiming to profit from the anxieties of those students in their early researching careers who, due to limited institutional support or funding, face serious limitations to present their innovative research and hypotheses to the broader academic community through an international publication. The student need and enthusiasm for an independent international publication run on a voluntary basis for and by students has been shown by the multiple messages of support and encouragement received over the past year, and by the considerable number of submissions received. We have only been able to publish less than half of the total number of submissions in this first issue. We are very positively overwhelmed by the way in which the Journal has been received. I believe that international and cooperative ventures are the way forward in a context of increasing nationalisation and politicisation of research perspectives.
		The academic discipline of archaeology has been maturing for over a century. The initial interests for undertaking archaeological fieldwork in order to prove metanarratives of cultural evolution and supremacy discourses, or for the antiquarian pleasure to collect and classify antiquities, are being increasingly questioned. 
        As \textcite{Trigger_2006} argues, archaeology is not a universal or self-evident activity, and, as such, it is important to be reflexive about why we are still conducting research, and for whom such research is relevant. Archaeologists may claim that their work is politically neutral, but the embodied memories of archaeological remains are symbolically imbued with political significance \parencite{Meskell_2012a}. 
        Thus, archaeology can easily become politicised -- used as part of a propagandistic agenda. Archaeology is capable of legitimating certain views on the past \parencite{Alexandri_2002}; thus, by controlling its practice, a politically correct historical narrative of a nation can be constructed. At the same time, in the current context of globalisation, the control of the archaeological discipline becomes an accepted stance through the conceptualisation of a ‘national heritage’ \parencite{Özdogan_1998}. 
        A country can assert its political independence through claims to heritage ownership: “\textit{we} study \textit{our} past”. This is what I call here the “vernacularisation of archaeological praxis”, what I see as a rising phenomenon in the twenty-first century. 
        Illustrative examples of this concept are the availability of state funding for excavations by local archaeologists in developing countries (China, Kazakhstan, Ukraine, Russia, Brazil, etc.). In addition, the independent presence in a certain nation-state of foreign archaeologists is increasingly being perceived, at least informally, as a potential ‘threat’, because they could challenge newly-constructed political narratives of a national past. This phenomenon is particularly relevant in post-colonial nation-states.
        Therefore, I would argue that this vernacularisation is a natural attempt from nation-state governments to control the internationalisation of national research throughout the world. State control is relevant in countries where heritage is ubiquitous (such as Greece, Turkey, China or South Africa, amongst many others), and it may also be initially intended to preserve the integrity of this ‘national heritage’, although this strategy is not always successful. Cuts on state funding availability (on infrastructure, bureaucracy and surveillance), derived from austerity measures \parencite{Phillips_2012}, a limited increase of new thorough scientific excavations, and a decrease in foreign academic funding due to an increase of state control and dependence for conducting excavations, are present trends in the landscape of archaeological practice. I argue that they are the central causes behind the lack of both adequate publishing of archaeological research, and the inadequate maintenance and protection of heritage sites in many countries.
		
		Undoubtedly, the potential of archaeology to explore questions about our past and about ourselves is an attractive dimension of its practice; however, the quasi intimate and subjective nature of this enterprise should make us realise that past remains of material culture and heritage are not equally perceived by everyone. We have realised that the personal perspectives that our interpretations of the past as humans, not just as students of archaeology, at times raise complicated ethical dilemmas \parencite{Scarre_2006}. One example is the profit-driven looting of archaeological contexts. Whereas most researchers and a large number of the public and individual citizens conceive archaeological materials as worthy of respect and consideration for their embodied memories and the past they represent, archaeological objects are frequently ascribed an economic value. This commodification of past material culture, fostered by the demand of art dealers and private collectors of antiquities, is a serious issue in the preservation and recording of the present presence of past material culture \parencite{Proulx_2013}. The legal framework around heritage protection is not proving enough to stop organised criminal activities capable of transcending national boundaries. At the same time, the socio-economic profile of looting is changing, particularly in crisis-struck countries, such as Greece, or those facing a war/post-war context (Irak, Syria, Lybia, etc.). Some of these irregular interventions are becoming subsistence activities, given the rise in applications for metal-detector permits and the number of first-offender cases concerning their inappropriate use \parencite{Skoumas_2015}. Together with the decrease in state-funding for the protection of archaeological sites, the integrity of archaeological collections and, particularly, unexcavated remains in Greece are now facing serious threats. Lieutenant Monovasios \parencite[in][]{Skoumas_2015} argues that although the effects of the crisis will be gone one day, the impact that looting will have on the identities of future generations will endure. Without denying the importance of heritage, it is imperative to consider that the initial elements of the looting chain (farmers, construction workers, etc.) are facing poverty and deprivation, and looting offers them a temporary solution. I would argue that policy-makers, archaeologists and officers need to raise further awareness of the illegality of looting archaeological contexts, but also to contribute towards the resolution of the deeper issues that are causing lay people to engage in these destructive activities.

		The purposeful defacing and destruction of heritage by IS and other extreme religious or political groups is a further dimension of the differential perception of the value of past material culture. Heritage can be relevant to the local population, who may demonstrate their importance frequently or occasionally, in special circumstances, when heritage items play a paramount role as embodied memories of past events worth commemorating. The appropriation of their meaning, and the destruction of the material representation of their significance, have been, throughout history, ubiquitous features in the repertoire of those using force to exert their claimed cultural and political superiority. The apparently radically different political strategies of Saddam Hussein being depicted as an Assyrian ruler, and destruction of Iraqi Assyrian heritage, conducted under the name of IS, are acts which exemplify the political exploitation of archaeology. Nevertheless, a novel feature of the globalised reality of the twenty-first century is that local destruction of heritage has become a powerful political statement at an international level, associated with the tragic loss of human lives in armed conflicts –Khaled Asaad, Syrian archaeologist in Palmira, in memoriam–. This politisation of heritage destruction is possible because of the intellectual historic background of democratic, late capitalist societies tends to consider archaeological remains in a universalistic sense, as if they represented the shared past of all living humans \parencite{Holtorf_2006}. Furthermore, heritage sites represent valuable economic assets, given their potential to attract tourists \parencite{Meskell_2002a}, particularly if they are famous and spectacular. IS and related organisations are not necessarily aiming to destroy the past, but to alter the socio-political shape of the present. With the destruction of sites, they are not only purposefully erasing pre-Islamic heritage and ways of experiencing life, they are also, strategically and in relatively inexpensive ways, challenging the international authority of Western societies, who established legal frameworks for the protection of heritage in armed contexts \parencite{Gerstenblith_2009}, but are not being able to enforce them. I would argue that archaeological sites are valuable provided that we accord value to the personal and collective memories that they embody and symbolise, not because of their material existence –as ruins, broken pots or burnt bones. Symbols of memory do not need to be labelled World Heritage Sites and be made of gold, silver or marble in order to be worth destroying, or protecting. Furthermore, and given the paramount role that material culture plays in the processes of cultural self-identification, the destruction of heritage in ethnic conflicts, such as the Mostar Bridge during the Yugoslavian war \parencite{Chapman_1994}, goes beyond military strategy, as it seeks to attack the material roots of certain communities’ past, as a way of ideologically destabilising them. Thus, archaeologists must engage in the communication and the sharing of knowledge in ways comprehensible for local communities and wider audiences, in order to ensure that these memories are understood, and respected.
		
		Another thoroughly-explored example is the existence of different approaches towards ancestral human remains by indigenous people and researchers. The implementation of the Native American Graves Protection and Repatriation Act (NAGPRA), and discoveries such as Kennewick Man, have sparked relevant debates regarding the ownership and control of the narratives about the past \parencite{Bruning_2006}, further exemplifying the political dimension of archaeological practice. Archaeologists tend to consider their research relevant for broadening our understanding of the socio-economic context and the demographic history of American populations. On the other hand, indigenous communities tend to place greater emphasis on their oral histories as a reliable source of ancestral knowledge, and perceive archaeological practices an attempt to control their heritage and sense of history. They further argue that exhumation of physical remains for scientific studies is disrespectful to their ancestors, and it could have negative impact on their descendants. These legitimate claims have to be considered by post-colonial archaeologies, who should cooperate with indigenous groups and fully engaging the communities \parencite{Hodder_2010}, listening their voices, their conceptualisations of the past, and considering the value of their traditional knowledge for a more complete interpretative framework \parencites{Wilcox_2010}[cf. for perspectives on archaeology in Africa:][] {Pikirayi_2015}. A collaborative, respectful enterprise needs mutual commitment in order to make research an enriching learning experience for all the people involved, from the beginnings of the process of knowledge creation to publication.
		
		Publication is a fundamental dimension of archaeological practice. Publishing is not just a record of the processes and methodologies, it frequently involves the interpretation of the embodied memories of what we excavate. Some of the most prestigious publishing brands and journals follow a traditional, subscription-based (or pay-per-view) model of publishing. This model limits the impact of the research of authors by preventing unsubscribed readers access to the contents of the articles. Besides, each article represents a very high cost to the academic community, even though most university students, and virtually everyone outside academic, will not be able to access the research. In the age of digital information, the existence of online repositories has made this model hard to justify. In addition, it preserves and even increases academic inequalities between those institutions and individuals who can afford the exorbitant subscriptions and those who cannot. Authors agree to give away their research and limit the readership of their papers because these journals are reputed brands which will look good on individual CVs. Open Access is an alternative approach. “Golden” Open Access journals follow a different, pay-per-publish, model. This profitable invention promises an increase in readership and public impact for research. In exchange, it is now the funding agencies of the author who pays the publishing costs. It is certainly a laudable movement towards the democratisation of global access to research, a for-profit compromise of traditional and Open Access approaches. Nonetheless, predatory journals have unprofessionally exploited this business model merely to gain profit, and without the aim of furthering scholarship. Traditional publishers have been slowly embracing a mixed model in response, combining a “Golden” Open-Access policy, charging author processing fees (APF), with the subscription approach. Publishers have thus been increasing their profits, but the costs of institutional access to research of the institutions continues to increase in this way \parencite{Matthews_2015}, as they have to pay both subscription and publishing charges.
		
Traditional access obstacles for readers may become new limitations to authors. I would argue that authors are paying a fictitious price for a service they should not actually need. Why should authors pay large fees to publishing houses (most of which is pure profit) for making their research available to the public, when much of the peer-reviewing process these journals are credited for tends to be done mostly as free service by academics? Because they, as individual researchers, have no choice at the moment if they want their published papers to be fully available, without committing the ‘crime’ of making their research material, copyrighted by traditional publishers, open to readers. A convenient dichotomy between researchers and publishers, due to conformity, lack of knowledge or experience, is having clear consequences. Whereas authors are having increasing trouble to find ways to fund \textit{their} research, publishers –as intermediaries–, are getting the material revenue of \textit{their} work. The situation of the academic publishing landscape is economically unsustainable for institutions \parencite{Sample_2012}, researchers, and students.  Archaeologists, and researchers in general, need to join efforts to act cooperatively: peer-reviewing is already being done for free, we just need top researchers to coordinate the editorial and publishing process of the new generation of leading journals on a voluntary basis. Generosity and cooperation are the way forward to debunk a for-profit industry which is channelling research to media impact in order to furthering profits instead of advancing science and knowledge. It is going to be a long, slow and demanding process, but it promises to decisively alter the future of the academic landscape for the better.

The twenty-first century is full of opportunities for archaeology students. Increasing opportunities for mobility of students and faculty is opening a greater array of work and research possibilities, thus creating an environment which improves knowledge transference and provides access to world-class institutions. The multicultural environment of modern universities adds a further enriching dimension to the learning experience. Moreover, the development of technology has facilitated access to vast quantities of information and has fostered international collaborations between scholars. At the same time, there are some challenges to overcome. In contrast to the early years of the profession, when the number of trained archaeologists was relatively low, the number of new archaeology graduates is increasingly rising, perhaps at a faster pace than existing demand within the field. This is the case in countries, such as Spain, the US, Canada or the UK, where uncontrolled growth fostered an over-dependence of a development-led archaeological focus \parencite{Aitchison_2009}. The collapse of the real estate bubble and the sub-prime mortgage market, and the subsequent decline in the number of opportunities, both in terms of jobs and projects, revealed the instability within the field. The short-term perspective of most of the development-led archaeology companies, focused on immediate gain rather than on economic sustainability, will need rethinking if the private archaeological sector in countries such as Spain is to be reconstructed with solid foundations \parencite{Hamilakis_2015}. I argue that the solution to the problems that students are facing to find employment should not be tackled through a perspective of competition. The mentality of our society, and thus, of most of its members, emphasises individual self-development, individual gain and individual success. Nevertheless, a forum within a cooperative approach, where students interested in improving our social reality, coming from different backgrounds, can share their ideas and discuss solutions would be a far more beneficial learning experience. The technology of the twenty-first century offers resolved people the possibility of becoming a team with the ability to make positive impacts in the field of archaeology, offering a suitable environment where to present and discuss innovative solutions to the challenges facing our discipline. I want to be part of this team, and I invite you to join us.

We are the \textit{International Journal of Student Research in Archaeology} (IJSRA). We are the first independent, unaffiliated and markedly international journal focused exclusively on student academic research in archaeology. Our aim is to become a global reference point, a free, open-access, international forum for the exchange of excellent student scholarship in a context of constructive dialogue and inclusiveness. We are developing a platform and a methodology to ensure that this forum makes a real impact in the realm of student research in archaeology. This \textit{Journal} seeks to enhance the academic experience of students worldwide by publishing their quality research, review articles, perspectives about the state of the field and any additional material useful for students and anyone interested in any aspect of archaeology.

This peer-reviewed \textit{Journal} shows that young researchers have the ability to apply their knowledge in innovative and successful ways, incorporating locally relevant narratives into wider archaeological discourses, and thus addressing the issue of the isolating over\-fragmentation of national and corporate research agendas \parencite{Mizoguchi_2015}. 
We are run by students on a voluntary, not-for-profit basis. We believe that getting involved in the publication process, both in its author and editor aspects, is a great opportunity for university students to develop their writing, reviewing and publishing skills. 
Our \textit{Journal} values and encourages diversity. 
It aims to foster global participation and to attract the submission of the best student research in archaeology, regardless of academic institution, nationality, gender, ethnicity or religion, in order to enhance international cooperation and mutual understanding. If you are interested in supporting our philosophy, please get in touch. With that said, I am proud to present you our first issue. 

\printbibliography[heading=subbibnumbered] 

\alertinfo{	\textbf{\sffamily Disclaimer:} The views expressed here are my current personal perspectives, and do not necessarily represent the views of any past, present or future institution I may be affiliated to, nor it necessarily represent the personal views of the Board members of this Journal, now or in the future.
}
%----------------------------------------------------------------------------------------
\label{editorial:lastpage}
\closingarticle